\documentclass[11pt]{article}
\usepackage{amssymb}
\usepackage{latexsym}
\usepackage{amsmath}
\usepackage{amsthm}
\usepackage{stmaryrd}
\usepackage{fancyhdr}
\pagestyle{headings}
\usepackage{dsfont}
\usepackage{pifont}
\usepackage{mathtools}
\usepackage{natbib}
\usepackage{tikz-cd}
\usepackage{pgfplots}
\usepackage{enumitem} 
\usepackage{hyperref}
\usepackage{geometry}
\geometry{left=4cm,right=4cm}
\pgfplotsset{every axis/.append style={
                    axis x line=middle,    % put the x axis in the middle
                    axis y line=middle,    % put the y axis in the middle
                    axis line style={<->}, % arrows on the axis
                    xlabel={$x$},          % default put x on x-axis
                    ylabel={$y$},          % default put y on y-axis
                    ticks=none,
                    }}
%\usepackage[urw-garamond]{mathdesign}
%\usepackage{cmbright}
%\usepackage{concmath}
%\usepackage{sansmathfonts}
%\renewcommand*\familydefault{\sfdefault} %% Only if the base font of the document is to be sans serif

%\usepackage{pdfrender,xcolor,scrpage2}
%\pdfrender{StrokeColor=black,TextRenderingMode=2,LineWidth=1pt}
\tikzset{
  subseteq/.style={
    draw=none,
    edge node={node [sloped, allow upside down, auto=false]{$\subseteq$}}},
  Subseteq/.style={
    draw=none,
    every to/.append style={
      edge node={node [sloped, allow upside down, auto=false]{$\subseteq$}}}
    },
    Subsetneq/.style={
    draw=none,
    every to/.append style={
      edge node={node [sloped, allow upside down, auto=false]{$\subsetneq$}}}
    },
  Supseteq/.style={
    draw=none,
    every to/.append style={
      edge node={node [sloped, allow upside down, auto=false]{$\supseteq$}}}
  }
}

\hypersetup{
    colorlinks,
    citecolor=blue,
    filecolor=blue,
    linkcolor=blue,
    urlcolor=blue
}
\newtheorem{thm}{Theorem}[section]
\newtheorem{prop}[thm]{Proposition}
\newtheorem{lemma}[thm]{Lemma}
\newtheorem{cor}[thm]{Corollary}
\newtheorem{dfn}[thm]{Definition}
\newtheorem{axiom}[thm]{Axiom}
\newtheorem{rmk}[thm]{Remark}
\newtheorem{rmkt}[thm]{Remark by TeXer}
\newtheorem{ex}[thm]{Example}
\newtheorem{nex}[thm]{Non-example}
\newtheorem{exercise}[thm]{Exercise}
\newtheorem{question}[thm]{Question}
\newtheorem{problem}[thm]{Problem}
\newtheorem{dfn/thm}[thm]{Definition/Theorem}
\renewcommand{\baselinestretch}{1.1}
\renewcommand{\hom}{\text{ Hom}}
\newcommand{\affn}{\mathbb A}
\newcommand{\proj}{\mathbb P}
\newcommand{\reals}{\mathbb R}
\newcommand{\cplx}{\mathbb C}
\newcommand{\intg}{\mathbb Z}
\newcommand{\bbf}{\mathbb F}
\newcommand{\ratl}{\mathbb Q}
\newcommand{\torus}{\mathbb T}
\newcommand{\sca}{{\mathfrak a}}
\newcommand{\scb}{{\mathfrak b}}
\newcommand{\scc}{{\mathfrak c}}
\newcommand{\scm}{{\mathfrak m}}
\newcommand{\scn}{{\mathfrak n}}
\newcommand{\scp}{{\mathfrak p}}
\newcommand{\scq}{\mathfrak q}
\newcommand{\frakg}{{\mathfrak g}}
\newcommand{\frakd}{{\mathfrak d}}
\newcommand{\calf}{{\cal F}}
\newcommand{\calg}{{\cal G}}
\newcommand{\cala}{{\cal A}}
\newcommand{\calb}{{\cal B}}
\newcommand{\calc}{{\cal C}}
\newcommand{\cale}{{\cal E}}
\newcommand{\cali}{{\cal I}}
\newcommand{\call}{{\cal L}}
\newcommand{\caln}{{\cal N}}
\newcommand{\calo}{{\cal O}}
\newcommand{\calr}{{\cal R}}
\newcommand{\mathbold}{\bf}
\newcommand{\cinf}{C^{\infty}}
\newcommand{\row}[2]{#1_1,\dots ,#1_{#2}}
\newcommand{\dbyd}[2]{{\partial #1\over\partial #2}}
\newcommand{\Space}{{\bf Space}}
\newcommand{\alg}{{\mathbold Alg}}
\newcommand{\notsubset}{\not \subset}
\newcommand{\notsupset}{\not \supset}
\newcommand{\pois}{{\mathbold Pois}}
\newcommand{\pitilde}{\tilde{\pi}}
\newcommand{\rta}{\rightarrow}
\newcommand{\Lrta}{\Longrightarrow}
\newcommand{\lrta}{\longrightarrow}
\newcommand{\llrta}{\longleftrightarrow}
\newcommand{\Llta}{\Longleftarrow}
\newcommand{\Llrta}{\Longleftrightarrow}
\newcommand{\lgl}{\langle}
\newcommand{\rgl}{\rangle}
\newcommand{\inj}{\hookrightarrow}
\newcommand{\surj}{\twoheadrightarrow}
\newcommand{\cmark}{\ding{51}}%
\newcommand{\xmark}{\ding{55}}%
\newcommand{\downmapsto}{\rotatebox[origin=c]{-90}{$\scriptstyle\mapsto$}\mkern2mu}
\renewcommand{\qedsymbol}{$\square$}
\bibliographystyle{plain}
\title{\bf Lecture Notes for Algebraic Geomtry I}
\author{Lecture delivered by Emmanuel Kowalski\\
Notes by Lin-Da Xiao}
\date{2018 ETH} %\thanks{Research partially supported by NSF Grant DMS-96-25122 and the Miller Institute for Basic Research in Science.}
\begin{document}
\maketitle
\tableofcontents
\newpage
\section{Feb 27th: Algebraic sets and morphisms}
\href{https://imaginary.org/programs}{https://imaginary.org/programs}

Recall: $V(I)\subset \affn^n =\{x|\forall f\in I, f(x)=0\}$.

\begin{thm}\label{thm:equivalence_of_categories_algebraic_sets_K_algebras}
Let $Y_1\subset \affn^n, X_1,...,X_n$, $Y_2\subset \affn^m, T_1,,..., T_m$ affine algebraic sets. There are bijections 
$$
\begin{aligned}
&\hom_{K-alg}(\calo(Y_2),\calo(Y_1))
\\
&\overset{(*)}{\llrta}\{(f_1,...,f_m)\in K[X]^m|\forall x\in Y_1,(f_1(x),...,f_m(x))\in Y_2) \}\\
&\overset{(**)}{\llrta} \{f:Y_1\lrta Y_2|\forall \varphi\in \calo(Y_2),\varphi\circ f\text{ is  in }  \calo(Y_1)\}
\end{aligned}
$$
\end{thm}
\begin{proof}
\underline{Key observation}

To give $(f_1,...,f_m)\in K[X]^m$ is ``the same '' as giving a ring morphism $g_0:K[T]\lrta K[X]: T_i\mapsto f_i$, which gives by composition $g_1=\pi_1\circ g_0$, where $\pi_1: K[X]\lrta \calo(Y_1)$ is the canonical projection.
$$
g_1: K[T]\lrta \calo(Y_1)
$$
which has a factorization
\[
\begin{tikzcd}
K[T] \arrow[r,"g_1"] \arrow[d,"\pi_2"]  & \calo(Y_1)  \\
   \calo(Y_2) \arrow[ur,"g"] & 
\end{tikzcd}
\]
iff $g_1(I(Y_2))=0$, which means iff 
$$
g_1(\varphi)=\text{ ``replace $T_i$ by $f_i$ in $\varphi$''}
$$
belongs to $I(Y_1)$ if $\varphi\in I(Y_2)$, which means if $x\in Y_1$, then $g_1(\varphi)(x)=0$. That means $\varphi(f_1(x),...,f_m(x))=0$ for $\varphi\in I(Y_2),$ i.e., $(f_1(x),...,f_m(x))\in Y_2$. If $x\in Y_1$. In the statement, this gives the $(*)$ bijection. Any $k$-algebra morphism $\calo(Y_1)\lrta\calo(Y_2)$  comes from $K[T]\lrta \calo(Y_1)$ s.t. it vanishes on $I(Y_2)$.


For the bijection $(**)$, suppose 
$$
g:Y_1\overset{g}{\lrta} Y_2\overset{\varphi}{\lrta} K
$$
sends $\varphi(Y_2)$ to $\varphi\circ g\in\calo(Y_1)$. Then we get 
$$
\begin{aligned}
& \calo(Y_2)\lrta \calo(Y_1)\\
&\varphi\longmapsto \varphi\circ g,
\end{aligned}
$$
which is a $K$-algebra morphism.

As for the reverse direction, given $g$. From $\calo(Y_2)\lrta \calo(Y_1)$ to get a $g:Y_1\lrta Y_2$. We get a $\tilde{g}:Y_1\lrta Y_2$ in the second set
$$
\tilde{g}(x)=(f_1(x),...,f_m(x))
$$
then we have $\varphi\circ g\in\calo(Y_1)$ for $\varphi\in \calo(Y_2)$. One checks that this shows that the first and third sets are the same.
\end{proof}

Define morphism $Y_1\lrta Y_2$ by the second(and third) set.  Composition in the obvious way and identity is a morphism.
$\Lrta$ get a category $(Alg_K)$ of affine algebraic sets over $K$.

\begin{cor}
$Y\mapsto \calo(Y)$, $g\mapsto [\varphi\mapsto \varphi\circ g]$ is a functor: $(Alg_K)\lrta (K-Alg)^{opp}$.
\end{cor}

\underline{Facts}: The ``image'' of this functor is the category of finitely generated $K$-algebras which are reduced.
\begin{proof}
$A$ finitely generated reduced $K$-algebra. ($\exists n\geq 1$, so that $K[X_1,...,X_n]/I\cong A$). Then ``$A$ is reduced''$\Llrta$ $I$ is radical ideal.
$\Lrta A=\calo(V(I))$, where $V(I)\subset \affn^n$.
\end{proof}

\begin{cor}
There is a equivalence of categories between 
$$
(Alg_K)\llrta (\text{fin. gen. reduced.} K-Algs.)
$$
\end{cor}

\begin{ex}\ 
\begin{enumerate}[label=(\arabic*)]
\item $\affn^1\lrta V(Y^2-X^3-X^2)\subset \affn^2$, $t\mapsto (t^2-1,t(t^2-1))$
\item $\affn^1\lrta V(Y^2-X^3)\subset \affn^2$: $t\longmapsto (t^2,t^3)$ is a bijection but \underline{Not} an isomorphism.
\item Assume $K$ with characteristic $p>0$, $K\supset \mathbb{F}_p$. $Y=V(f_1,...,f_m)$ where $f_i\in \mathbb{F}_p[X]\subset K[X]$. Consider the morphism:
$$
\begin{aligned}
&Y\lrta Y\\
& (x_1,..,x_n)\longmapsto (x_1^p,...,x_n^p).
\end{aligned}
$$
It is bijective and homeomorphism but not an isomorphism if $dim(Y)\geq 1$.
\end{enumerate}
\end{ex}
\begin{prop}
$Y=V(I)\subset \affn^n$
\begin{enumerate}[label=(\arabic*)]
\item The points of $Y$ are in bijection with maximal ideals $I\subset \calo(Y)$ by 
$$
Y\ni x\longmapsto \{f\in \calo(Y)|f(x)=0\}
$$
\item We have a bijection 
$$
\calo(Y)\llrta \hom_{Alg_K}(Y,\affn^1)
$$
\end{enumerate}
\end{prop}
\begin{proof}
(1) $I_x:=Ker(\calo(Y)\lrta K)$, $f\mapsto f(x)$, since the evaluation map is surjective $[1\mapsto 1]$, we get an isomorphism 
$$
\calo(Y)/I_x\overset{\sim}{\lrta} K,
$$
so $I_x$ is maximal in $\calo(Y)$.

Conversely, if $I\subset \calo(Y)$ is maximal, we get $I=I'/I(Y)$ for $I'\subset K[X]$ maximal. 

Nullstellensatz says $\exists (x_1,...,x_n)\in\affn^n$ s.t., $I'=(X_1-x_1,...,X_n-x_n)$. 

Since $I'\supset I(Y)$, we get $(x_1,..,x_n)\in Y$. Then we check that $\calo(Y)\lrta\calo(Y)/I\cong K$ is just given by $f\mapsto f(x_1,...,x_n)$. That means $I=I_x$.

(2) We saw in~\ref{thm:equivalence_of_categories_algebraic_sets_K_algebras}, that there is a bijection between sets
$$
\hom_{Alg_k}(Y,\affn^1)\llrta \hom_{K-alg}(\calo(\affn^1),\calo(Y)).
$$
But $\hom_{K-alg}(\calo(\affn^1),\calo(Y))=\hom_{K-alg}(K[X],\calo(Y))\cong \calo(Y)$ (by $g:\calo(\affn^1)\lrta \calo(Y)$, $g\mapsto g(X)$)
\end{proof}
\section*{Projective Algebraic sets}

Projective sets can have a good notion of ``compactness''.

N.B. Any $Y\in (Alg_K)$ is \textbf{quasi-compact}( open cover have a finite subcover).

\begin{dfn}
$\proj^n_K=\proj^n$ can be either defined as 

``the set of  lines in $\affn^{n+1}$ that pass through the origin''

or

``the equivalence classes of points in $K^{n+1}\backslash \{0\}$ with the equivalence relation $x\sim y$ iff $x=\lambda y$ for some $\lambda \in K$'' and we use the notion $[x_0:...:x_n]$ for the equivalence class of $(x_0,..,x_n)$
\end{dfn}

These two definitions are equivalent: 

Given a line $l\in \affn^1\llrta $ hyperplane in $K^{n+1}$, corresponds to a equation
$$
a_0X_0+....+a_n X_n=0
$$
with at least one of $a_i$ non-zero.

Conversely, from $[x_0:..:x_n]$, we  we get the corresponding hyperplane/line trivially.


Notes the following fact:

$$
\proj^n=\cup_{0\leq i\leq n} H_i,
$$
where $H_i=\{[x_0,...,x_n]|x_i\neq 0\}$ and there is a bijection 
$$
\begin{aligned}
&H_i\lrta K^n\\
&[x_0:...:x_n]\longmapsto\left(\frac{x_0}{x_i},...,\widehat{\frac{x_i}{x_i}},..,\frac{x_n}{x_i}\right)\\
&
[y_1:...:y_{i-1}:1:y_{i}:...:y_n]\mapsfrom(y_1,...,y_n)
\end{aligned}
$$
We define from linear algebra some notions in $\proj^n$ a line in $\proj^n$ is the image by the projection $K^{n+1}\backslash \{0\}\lrta \proj^n$ of the two dimensional affine subspace.

\begin{ex}
$l_1,l_2\subset \proj^2$ lines $\l_1\cap l_2$ is a line if $l_1$ and $l_2$ are identical and would be a single point otherwise.
\end{ex}

Observation: If $f\in K[X_0,...,X_{n+1}]$ is homogeneous, then for $x\in \proj^n$, it makes no sense to speak of ``$f(x)\in K$'', but the zero-loci or the set where $f(x)\neq 0$ does make sense.

\begin{dfn}
 A  \textbf{projective algebraic set} $S\subset \proj^n$ is 
 $$
  S=\{ x\in \proj^n|f_1(x)=...=f_m(x)=0\},
 $$
 where $f_1,...,f_m$ are homogeneous of some degrees.
\end{dfn}

\underline{Notation}: $V(f_1,..,f_n)$

\begin{ex}
$V(Y^2 Z-X^3-X Z^2)\subset \proj^2$. 
\end{ex}
Let $0\leq i\leq n$, then $S\cap H_i=\{[x_0:...:x_n]\in S| x_i\neq 0\}$ is , via the bijection  $H_i\lrta K^n$, in bijection with an affine algebraic set $S_1\subset \affn^n$ given by $\tilde{f_1}(y)=...=\tilde{f}_m(y)=0$, where $\tilde{f}_i(y_1,..,y_n)=f_i(y_1,...,y_{i-1},1, y_i,...,y_n)$




\end{document}