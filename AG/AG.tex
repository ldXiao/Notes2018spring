\documentclass[11pt]{article}
\usepackage{amssymb}
\usepackage{latexsym}
\usepackage{amsmath}
\usepackage{amsthm}
\usepackage{stmaryrd}
\usepackage{fancyhdr}
\pagestyle{headings}
\usepackage{dsfont}
\usepackage{pifont}
\usepackage{mathtools}
\usepackage{natbib}
\usepackage{tikz-cd}
\usepackage{pgfplots}
\usepackage{enumitem} 
\usepackage{hyperref}
\usepackage{geometry}
\geometry{left=4cm,right=4cm}
\pgfplotsset{every axis/.append style={
		axis x line=middle,    % put the x axis in the middle
		axis y line=middle,    % put the y axis in the middle
		axis line style={<->}, % arrows on the axis
		xlabel={$x$},          % default put x on x-axis
		ylabel={$y$},          % default put y on y-axis
		ticks=none,
}}
%\usepackage[urw-garamond]{mathdesign}
%\usepackage{cmbright}
%\usepackage{concmath}
%\usepackage{sansmathfonts}
%\renewcommand*\familydefault{\sfdefault} %% Only if the base font of the document is to be sans serif

%\usepackage{pdfrender,xcolor,scrpage2}
%\pdfrender{StrokeColor=black,TextRenderingMode=2,LineWidth=1pt}
\tikzset{
	subseteq/.style={
		draw=none,
		edge node={node [sloped, allow upside down, auto=false]{$\subseteq$}}},
	Subseteq/.style={
		draw=none,
		every to/.append style={
			edge node={node [sloped, allow upside down, auto=false]{$\subseteq$}}}
	},
	Subsetneq/.style={
		draw=none,
		every to/.append style={
			edge node={node [sloped, allow upside down, auto=false]{$\subsetneq$}}}
	},
	Supseteq/.style={
		draw=none,
		every to/.append style={
			edge node={node [sloped, allow upside down, auto=false]{$\supseteq$}}}
	}
}

\hypersetup{
	colorlinks,
	citecolor=blue,
	filecolor=blue,
	linkcolor=blue,
	urlcolor=blue
}
\newtheorem{thm}{Theorem}[section]
\newtheorem{prop}[thm]{Proposition}
\newtheorem{lemma}[thm]{Lemma}
\newtheorem{cor}[thm]{Corollary}
\newtheorem{dfn}[thm]{Definition}
\newtheorem{axiom}[thm]{Axiom}
\newtheorem{rmk}[thm]{Remark}
\newtheorem{rmkt}[thm]{Remark by TeXer}
\newtheorem{ex}[thm]{Example}
\newtheorem{nex}[thm]{Non-example}
\newtheorem{exercise}[thm]{Exercise}
\newtheorem{question}[thm]{Question}
\newtheorem{problem}[thm]{Problem}
\newtheorem{dfn/thm}[thm]{Definition/Theorem}
\renewcommand{\baselinestretch}{1.1}
\renewcommand{\hom}{\text{ Hom}}
\newcommand{\affn}{\mathbb A}
\newcommand{\proj}{\mathbb P}
\newcommand{\reals}{\mathbb R}
\newcommand{\cplx}{\mathbb C}
\newcommand{\intg}{\mathbb Z}
\newcommand{\bbf}{\mathbb F}
\newcommand{\ratl}{\mathbb Q}
\newcommand{\torus}{\mathbb T}
\newcommand{\sca}{{\mathfrak a}}
\newcommand{\scb}{{\mathfrak b}}
\newcommand{\scc}{{\mathfrak c}}
\newcommand{\scm}{{\mathfrak m}}
\newcommand{\scn}{{\mathfrak n}}
\newcommand{\scp}{{\mathfrak p}}
\newcommand{\scq}{\mathfrak q}
\newcommand{\frakg}{{\mathfrak g}}
\newcommand{\frakd}{{\mathfrak d}}
\newcommand{\calf}{{\cal F}}
\newcommand{\calg}{{\cal G}}
\newcommand{\cala}{{\cal A}}
\newcommand{\calb}{{\cal B}}
\newcommand{\calc}{{\cal C}}
\newcommand{\cale}{{\cal E}}
\newcommand{\cali}{{\cal I}}
\newcommand{\call}{{\cal L}}
\newcommand{\caln}{{\cal N}}
\newcommand{\calo}{{\cal O}}
\newcommand{\calr}{{\cal R}}
\newcommand{\mathbold}{\bf}
\newcommand{\cinf}{C^{\infty}}
\newcommand{\row}[2]{#1_1,\dots ,#1_{#2}}
\newcommand{\dbyd}[2]{{\partial #1\over\partial #2}}
\newcommand{\Space}{{\bf Space}}
\newcommand{\alg}{{\mathbold Alg}}
\newcommand{\notsubset}{\not \subset}
\newcommand{\notsupset}{\not \supset}
\newcommand{\pois}{{\mathbold Pois}}
\newcommand{\pitilde}{\tilde{\pi}}
\newcommand{\rta}{\rightarrow}
\newcommand{\Lrta}{\Longrightarrow}
\newcommand{\lrta}{\longrightarrow}
\newcommand{\llrta}{\longleftrightarrow}
\newcommand{\Llta}{\Longleftarrow}
\newcommand{\Llrta}{\Longleftrightarrow}
\newcommand{\lgl}{\langle}
\newcommand{\rgl}{\rangle}
\newcommand{\inj}{\hookrightarrow}
\newcommand{\surj}{\twoheadrightarrow}
\newcommand{\cmark}{\ding{51}}%
\newcommand{\xmark}{\ding{55}}%
\newcommand{\downmapsto}{\rotatebox[origin=c]{-90}{$\scriptstyle\mapsto$}\mkern2mu}
\renewcommand{\qedsymbol}{$\square$}
\bibliographystyle{plain}
\title{\bf Lecture Notes for Algebraic Geomtry I}
\author{Lecture delivered by Emmanuel Kowalski\\
	Notes by Lin-Da Xiao}
\date{2018 ETH} %\thanks{Research partially supported by NSF Grant DMS-96-25122 and the Miller Institute for Basic Research in Science.}
\begin{document}
	\maketitle
	\tableofcontents
	\newpage
	\section{Feb 27th: Algebraic sets and morphisms}
	\href{https://imaginary.org/programs}{https://imaginary.org/programs}
	
	Recall: $V(I)\subset \affn^n =\{x|\forall f\in I, f(x)=0\}$.
	
	\begin{thm}\label{thm:equivalence_of_categories_algebraic_sets_K_algebras}
		Let $Y_1\subset \affn^n, X_1,...,X_n$, $Y_2\subset \affn^m, T_1,,..., T_m$ affine algebraic sets. There are bijections 
		$$
		\begin{aligned}
		&\hom_{K-alg}(\calo(Y_2),\calo(Y_1))
		\\
		&\overset{(*)}{\llrta}\{(f_1,...,f_m)\in K[X]^m|\forall x\in Y_1,(f_1(x),...,f_m(x))\in Y_2) \}\\
		&\overset{(**)}{\llrta} \{f:Y_1\lrta Y_2|\forall \varphi\in \calo(Y_2),\varphi\circ f\text{ is  in }  \calo(Y_1)\}
		\end{aligned}
		$$
	\end{thm}
	\begin{proof}
		\underline{Key observation}
		
		To give $(f_1,...,f_m)\in K[X]^m$ is ``the same '' as giving a ring morphism $g_0:K[T]\lrta K[X]: T_i\mapsto f_i$, which gives by composition $g_1=\pi_1\circ g_0$, where $\pi_1: K[X]\lrta \calo(Y_1)$ is the canonical projection.
		$$
		g_1: K[T]\lrta \calo(Y_1)
		$$
		which has a factorization
		\[
		\begin{tikzcd}
		K[T] \arrow[r,"g_1"] \arrow[d,"\pi_2"]  & \calo(Y_1)  \\
		\calo(Y_2) \arrow[ur,"g"] & 
		\end{tikzcd}
		\]
		iff $g_1(I(Y_2))=0$, which means iff 
		$$
		g_1(\varphi)=\text{ ``replace $T_i$ by $f_i$ in $\varphi$''}
		$$
		belongs to $I(Y_1)$ if $\varphi\in I(Y_2)$, which means if $x\in Y_1$, then $g_1(\varphi)(x)=0$. That means $\varphi(f_1(x),...,f_m(x))=0$ for $\varphi\in I(Y_2),$ i.e., $(f_1(x),...,f_m(x))\in Y_2$. If $x\in Y_1$. In the statement, this gives the $(*)$ bijection. Any $k$-algebra morphism $\calo(Y_1)\lrta\calo(Y_2)$  comes from $K[T]\lrta \calo(Y_1)$ s.t. it vanishes on $I(Y_2)$.
		
		
		For the bijection $(**)$, suppose 
		$$
		g:Y_1\overset{g}{\lrta} Y_2\overset{\varphi}{\lrta} K
		$$
		sends $\varphi(Y_2)$ to $\varphi\circ g\in\calo(Y_1)$. Then we get 
		$$
		\begin{aligned}
		& \calo(Y_2)\lrta \calo(Y_1)\\
		&\varphi\longmapsto \varphi\circ g,
		\end{aligned}
		$$
		which is a $K$-algebra morphism.
		
		As for the reverse direction, given $g$. From $\calo(Y_2)\lrta \calo(Y_1)$ to get a $g:Y_1\lrta Y_2$. We get a $\tilde{g}:Y_1\lrta Y_2$ in the second set
		$$
		\tilde{g}(x)=(f_1(x),...,f_m(x))
		$$
		then we have $\varphi\circ g\in\calo(Y_1)$ for $\varphi\in \calo(Y_2)$. One checks that this shows that the first and third sets are the same.
	\end{proof}
	
	Define morphism $Y_1\lrta Y_2$ by the second(and third) set.  Composition in the obvious way and identity is a morphism.
	$\Lrta$ get a category $(Alg_K)$ of affine algebraic sets over $K$.
	
	\begin{cor}
		$Y\mapsto \calo(Y)$, $g\mapsto [\varphi\mapsto \varphi\circ g]$ is a functor: $(Alg_K)\lrta (K-Alg)^{opp}$.
	\end{cor}
	
	\underline{Facts}: The ``image'' of this functor is the category of finitely generated $K$-algebras which are reduced.
	\begin{proof}
		$A$ finitely generated reduced $K$-algebra. ($\exists n\geq 1$, so that $K[X_1,...,X_n]/I\cong A$). Then ``$A$ is reduced''$\Llrta$ $I$ is radical ideal.
		$\Lrta A=\calo(V(I))$, where $V(I)\subset \affn^n$.
	\end{proof}
	
	\begin{cor}
		There is a equivalence of categories between 
		$$
		(Alg_K)\llrta (\text{fin. gen. reduced.} K-Algs.)
		$$
	\end{cor}
	
	\begin{ex}\ 
		\begin{enumerate}[label=(\arabic*)]
			\item $\affn^1\lrta V(Y^2-X^3-X^2)\subset \affn^2$, $t\mapsto (t^2-1,t(t^2-1))$
			\item $\affn^1\lrta V(Y^2-X^3)\subset \affn^2$: $t\longmapsto (t^2,t^3)$ is a bijection but \underline{Not} an isomorphism.
			\item Assume $K$ with characteristic $p>0$, $K\supset \mathbb{F}_p$. $Y=V(f_1,...,f_m)$ where $f_i\in \mathbb{F}_p[X]\subset K[X]$. Consider the morphism:
			$$
			\begin{aligned}
			&Y\lrta Y\\
			& (x_1,..,x_n)\longmapsto (x_1^p,...,x_n^p).
			\end{aligned}
			$$
			It is bijective and homeomorphism but not an isomorphism if $dim(Y)\geq 1$.
		\end{enumerate}
	\end{ex}
	\begin{prop}
		$Y=V(I)\subset \affn^n$
		\begin{enumerate}[label=(\arabic*)]
			\item The points of $Y$ are in bijection with maximal ideals $I\subset \calo(Y)$ by 
			$$
			Y\ni x\longmapsto \{f\in \calo(Y)|f(x)=0\}
			$$
			\item We have a bijection 
			$$
			\calo(Y)\llrta \hom_{Alg_K}(Y,\affn^1)
			$$
		\end{enumerate}
	\end{prop}
	\begin{proof}
		(1) $I_x:=Ker(\calo(Y)\lrta K)$, $f\mapsto f(x)$, since the evaluation map is surjective $[1\mapsto 1]$, we get an isomorphism 
		$$
		\calo(Y)/I_x\overset{\sim}{\lrta} K,
		$$
		so $I_x$ is maximal in $\calo(Y)$.
		
		Conversely, if $I\subset \calo(Y)$ is maximal, we get $I=I'/I(Y)$ for $I'\subset K[X]$ maximal. 
		
		Nullstellensatz says $\exists (x_1,...,x_n)\in\affn^n$ s.t., $I'=(X_1-x_1,...,X_n-x_n)$. 
		
		Since $I'\supset I(Y)$, we get $(x_1,..,x_n)\in Y$. Then we check that $\calo(Y)\lrta\calo(Y)/I\cong K$ is just given by $f\mapsto f(x_1,...,x_n)$. That means $I=I_x$.
		
		(2) We saw in~\ref{thm:equivalence_of_categories_algebraic_sets_K_algebras}, that there is a bijection between sets
		$$
		\hom_{Alg_k}(Y,\affn^1)\llrta \hom_{K-alg}(\calo(\affn^1),\calo(Y)).
		$$
		But $\hom_{K-alg}(\calo(\affn^1),\calo(Y))=\hom_{K-alg}(K[X],\calo(Y))\cong \calo(Y)$ (by $g:\calo(\affn^1)\lrta \calo(Y)$, $g\mapsto g(X)$)
	\end{proof}
	\section*{Projective Algebraic sets}
	
	Projective sets can have a good notion of ``compactness''.
	
	N.B. Any $Y\in (Alg_K)$ is \textbf{quasi-compact}( open cover have a finite subcover).
	
	\begin{dfn}
		$\proj^n_K=\proj^n$ can be either defined as 
		
		``the set of  lines in $\affn^{n+1}$ that pass through the origin''
		
		or
		
		``the equivalence classes of points in $K^{n+1}\backslash \{0\}$ with the equivalence relation $x\sim y$ iff $x=\lambda y$ for some $\lambda \in K$'' and we use the notion $[x_0:...:x_n]$ for the equivalence class of $(x_0,..,x_n)$
	\end{dfn}
	
	These two definitions are equivalent: 
	
	Given a line $l\in \affn^1\llrta $ hyperplane in $K^{n+1}$, corresponds to a equation
	$$
	a_0X_0+....+a_n X_n=0
	$$
	with at least one of $a_i$ non-zero.
	
	Conversely, from $[x_0:..:x_n]$, we  we get the corresponding hyperplane/line trivially.
	
	
	Notes the following fact:
	
	$$
	\proj^n=\cup_{0\leq i\leq n} H_i,
	$$
	where $H_i=\{[x_0,...,x_n]|x_i\neq 0\}$ and there is a bijection 
	$$
	\begin{aligned}
	&H_i\lrta K^n\\
	&[x_0:...:x_n]\longmapsto\left(\frac{x_0}{x_i},...,\widehat{\frac{x_i}{x_i}},..,\frac{x_n}{x_i}\right)\\
	&
	[y_1:...:y_{i-1}:1:y_{i}:...:y_n]\mapsfrom(y_1,...,y_n)
	\end{aligned}
	$$
	We define from linear algebra some notions in $\proj^n$ a line in $\proj^n$ is the image by the projection $K^{n+1}\backslash \{0\}\lrta \proj^n$ of the two dimensional affine subspace.
	
	\begin{ex}
		$l_1,l_2\subset \proj^2$ lines $\l_1\cap l_2$ is a line if $l_1$ and $l_2$ are identical and would be a single point otherwise.
	\end{ex}
	
	Observation: If $f\in K[X_0,...,X_{n+1}]$ is homogeneous, then for $x\in \proj^n$, it makes no sense to speak of ``$f(x)\in K$'', but the zero-loci or the set where $f(x)\neq 0$ does make sense.
	
	\begin{dfn}
		A  \textbf{projective algebraic set} $S\subset \proj^n$ is 
		$$
		S=\{ x\in \proj^n|f_1(x)=...=f_m(x)=0\},
		$$
		where $f_1,...,f_m$ are homogeneous of some degrees.
	\end{dfn}
	
	\underline{Notation}: $V(f_1,..,f_n)$
	
	\begin{ex}
		$V(Y^2 Z-X^3-X Z^2)\subset \proj^2$. 
	\end{ex}
	Let $0\leq i\leq n$, then $S\cap H_i=\{[x_0:...:x_n]\in S| x_i\neq 0\}$ is , via the bijection  $H_i\lrta K^n$, in bijection with an affine algebraic set $S_1\subset \affn^n$ given by $\tilde{f_1}(y)=...=\tilde{f}_m(y)=0$, where $\tilde{f}_i(y_1,..,y_n)=f_i(y_1,...,y_{i-1},1, y_i,...,y_n)$
	
	
	\section{Mar 2nd: Projective algebraic sets and regular functions}
	Recall: $\proj^n_K=K^{n-1}-\{0\}/\sim$, and $H_i:=\{[x_0:...:x_n]|x_i\neq 0\}$ is in bijection with $\affn^n$. $V(f_1,...,f_m)=\{x\in \proj^n|\forall i,f_i(x)=0\}$, where $f_1,..f_m$ are homogeneous.
	
	More generally, we can define 
	$$
	V(I)=V(\text{homogeneous element of $I$=})=V(\cup_{d\geq 0} I_d)
	$$
	where $I$ is an homogeneous ideal of $K[X_0,...,X_n]$ that is  $I=\oplus_{d\geq 0} I_d$, $I_d$ the the degree $d$ piece of $K[X_0,...,X_n]$.
	
	Conversely, given $S\subset \proj^n$, we can define 
	$$
	I(S):=\text{ideal generated by homogeneous polynomials that vanishes on $S$}
	$$
	\begin{lemma}
		This is a homogeneous ideal
	\end{lemma}
	\begin{proof}
		$f\in I(s)\Lrta f=\sum_{i\in I}g_i f_i$, where $f_i$ is homogeneous and vanishes on $S$. We can expand each $g_i$ as $\sum_j g_{ij}$, where each $g_{ij}$ is homogeneous in $I(S)$. Then we know $f\in \otimes I(S)_d$ and the converse is clear.
	\end{proof}
	\begin{lemma}
		The projective sets $V(I)$ where $I$ is homogeneous form the closed sets of a topology. It is called the Zariski topology (same name for the induced topology on projective sets).
	\end{lemma}
	\begin{ex}
		$H_0\subset \proj^n$ and $\sigma: H_0\cong \affn^n$. Under this bijection, the Zariski topologies correspond $\sigma $ is a homeomorphism
		$$
		f\in K[X_0,...,X_n]\text{ homogeneous} \leadsto V(f)\subset \proj^n
		$$
		$$
		\tilde{f}=f(1,X_1,...,X_n)\in K[X_1,..,X_n]\leadsto V(\tilde{f})\subset \affn^n
		$$
		and $\sigma(V(f))=V(\tilde{f})$.
	\end{ex}
	\begin{dfn}
		$Y\subset \proj^n$ projective $S(Y)=K[X_0,...,X_n]/I(Y)$, \textbf{homogeneous coordinate ring}
	\end{dfn}
	\underline{Note} elements in $S(Y)$ are not functions on $Y$. The geometric meaning of $S(Y)$ will be explained latter with the language of schemes.
	
	We now want to define morphisms of  projective algebraic sets. We have to look at it more carefully because we can not simply copy the affine definition.
	\begin{dfn}
		$Y\subset \proj^n$ projective, let $V\subset Y $ be an open subsets of $Y$.
		\begin{enumerate}[label=(\arabic*)]
			\item $f: V\lrta K$ continuous is called \textbf{regular} on $Y$ if $\forall x\in Y$, $\exists U$ open $x\in U$, $\exists f_1,f_2\in K[X_0,...,X_n]$ homogeneous of same degree such that $f_2(x)\neq 0$ for all $x\in U$ and $f(x)=\frac{f_1(x)}{f_2(x)}$ for $x\in U\cap Y$
			\item $Y_1,Y_2$ are projective sets in $\proj^n,\proj^m$, $f: Y_1\lrta Y_2$ is a \textbf{morphism} if $f$ is continuous and for any $U\subset Y_1$ open and any $\varphi:U\lrta K$ regular, the composite $\varphi\circ f: f^{-1}(U)\lrta K$ is regular.
		\end{enumerate}
	\end{dfn}
	\underline{Note}: IN $(2)$, one can not restrict to $\varphi$ regular on $Y_2$ because often the space of such function is reduced to $K$
	\begin{prop}
		For $\proj^n$, the space of functions regular on $\proj^n$ is $K$.
	\end{prop}
	\begin{proof}
		The case $n=1$ implies the general case: if $f:\proj^n\lrta K$ regular, and $x\neq y$ in $\proj^n$, the line joining $x$ to $y$ in $\proj^n$ is ``isomorphic'' to $\proj^1$ and $f|_L$ is regular so constant, hence $f(x)=f(y)$.
		
		For $n=1$, suppose $x$, $y$ are arbitrary points and let $U \ni x$, $V\ni y$ be open neighbourhoods such that $f|_U = \frac{f_1(x)}{f_2(x)}$ and $f|_V = \frac{g_1(x)}{g_2(x)}$ where $f_1$, $f_2$, $g_1$, $g_2$ are homogeneous polynomials and $f_1$, $f_2$ have the same degree as well as $g_1$, $g_2$. We may assume that $f_1$ and $f_2$ are coprime and also $g_1$, $g_2$ are coprime. Hence on $U \cap V$,
		$$
		f_1 g_2 = g_1 f_2.
		$$
		We know that $U \cap V$ is infinite so this implies $f_1 = g_1$ and $ f_2 = g_2$. Since $x$ and $y$ were arbitrary points we conclude that $f = \frac{f_1(x)}{f_2(x)}$ on all of $\proj^1$ hence $f$ is a constant.
	\end{proof}
	\underline{Concretely}: To say that $f: Y_1\subset \proj^n\lrta Y_2\subset \proj^m$  is a morphism of projective algebraic sets. It reduces to 
	$\forall x\in Y_1,\exists U$ open containing $x$ s.t. there exists $f_0,...,f_m\in K[X_0,...,X_{n+1}]$
	homogeneous of same degree, with no common zero in $U$, such that
	$\forall y\in U\cap Y_1$, $f(y)=[f_0(y):...:f_m(y)]$. It is easy to see that if $f$ is of this form, then it is a morphism.
	
	The converse is left as an exercise.
	
	\begin{ex}
		\begin{enumerate}[label=(\arabic*)]
			\item Let $g\in Gl_n(K), n\geq 1$. Define 
			$$
			f_g:\proj^n\lrta \proj^n
			$$
			$$
			[x_0:...:x_n]\longmapsto [g(x_0,...,x_n)]
			$$
			is a morphism. In fact, it is an isomorphism. $f^{-1}_g=f_{g^{-1}}$.
			It also has some other properties: $f_{g}=f_{\lambda g},\lambda\neq 0$ and we get an induced group morphism
			\[
			\begin{tikzcd}
			PGL_{n+1}(K) \arrow[d] \arrow[r, equal] & GL_{n+1}(K)/K^\times \\
			Aut_{proj}(\proj^n) & 
			\end{tikzcd}
			\]
			which is an isomorphism. A special case is $Aut_{hol}(\cplx \proj^1)=PGL_2(\cplx)$
			$$
			g\longmapsto \left[z\mapsto \frac{az+b}{cz+d}\right]
			$$
			\item  $K=\cplx$. One can do  holomorphic geometry (using holomorphic functions instead of polynomials). IN $\cplx^n$,we get a  much more complicated picture [e.g. $V(\sin z)$] is a an infinite sets in $\proj^n_\cplx$, however Chow proved that the holomorphic sets and the projective algebraic sets are the same (Serre ``GAGA'' principle compares many different invariant of both categories.)
			\item Consider the map $S:=V(Y^2Z-X^3-XZ^2)\overset{f}{\lrta} \proj^1$, $[x:y:z]\mapsto [y:z]$. \\
			\underline{Claim}, this is a morphism of projective sets.
			\\
			This means that there is no solution to $Y^2Z-X^3-XZ^2=0$ with $Y=Z=0$. (But $[x:y:z]\mapsto [x:z]$ is not a morphism because $[0:1:0]\in S$). $f$ is surjective but not injective $[x:y:z]$ and $[x:-y:z]$ have same image. This works in field $k$ with $\text{char} k\neq 2$.
			\item $\proj^1\overset{v}{\lrta \proj^2}$, $[x:y]\mapsto [x^2:xy:y^2]$ (special case of Veronese embedding). This is a morphism. The image of $v$ is equal to $[y_0:y_1:y_2], \proj^2$. $S=V(Y_1^2-Y_0Y_2)$.  In fact, $\sigma$ gives an isomorphism
			$\sigma: \proj^1\lrta S$ with inverse given by 
			$$
			\tau: S\lrta \proj^1
			$$
			$$
			[y_0:y_1:y_2]\mapsto\left\{\begin{matrix*}
			&[Y_1:Y_2] \text{ if } Y_2\neq 0\\
			&[Y_0:Y_1]\text{ if } Y_0\neq 0
			\end{matrix*}
			\right.
			$$
			$\tau$ is a morphism defined on all of $S$, because if $[y_0:y_1:y_2]\in S$ satisfies $y_0=y_2=0$, it would implie $y^2_1=y_0y_2=0\Lrta y_1=0$ 
			$$
			\begin{aligned}
			\tau\circ \sigma([x:y])&=\tau([x^2:xy:y^2])
			&=\left\{\begin{matrix*}
			&[xy:y^2]=[x:y], y\neq 0
			\\
			&[x^2:xy]=[x:y], x\neq 0
			\end{matrix*}\right.
			\end{aligned}
			$$
			therefore $\tau\circ \sigma= id_{\proj^1}$ and $\sigma\circ \tau=id_S$ can proved similarly
			
			One can not find $f_0:f_1$ in $K[Y_0,Y_1,Y_2]$ s.t. $\tau([y_0:y_1:y_2]=[f_0(Y):f_1(Y)]$ for all $Y\in S$
		\end{enumerate}
	\end{ex}
	
	\section*{Rational/birational maps}
	
	$Y\subset \affn^n$ algebraic if $Y$ is irreducible, then $\calo(Y)$ is an integral domain. Let $K(Y)$ be its quotient field. What is the geometric meaning of $K(Y)$? It is called the function field of $Y$.
	
	We will see
	\begin{thm}
		For $Y_1,Y_2$ affine varieties (irreducible) $K(Y_1)\cong K(Y_2)$ as fields
		
		$\Llrta$ $\exists U_1\subset Y_1$ open dense subset and $\exists U_2\subset Y_2$ open dense subset such that $U_1$ and $U_2$ are isomorphic.
	\end{thm}
	
	
	
	
\end{document}