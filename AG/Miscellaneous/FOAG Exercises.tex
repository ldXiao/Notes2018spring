\documentclass[11pt]{article}
\usepackage{amssymb}
\usepackage{latexsym}
\usepackage{amsmath}
\usepackage{amsthm}
\usepackage{mathtools}
\usepackage{natbib}
\usepackage{tikz-cd}
\usepackage{enumitem} 
\usepackage{hyperref}
\hypersetup{
    colorlinks,
    citecolor=blue,
    filecolor=blue,
    linkcolor=blue,
    urlcolor=blue
}
\newtheorem{thm}{Theorem}[section]
\newtheorem{prop}[thm]{Proposition}
\newtheorem{lemma}[thm]{Lemma}
\newtheorem{cor}[thm]{Corollary}
\newtheorem{dfn}[thm]{Definition}
\newtheorem{axiom}[thm]{Axiom}
\newtheorem{rmk}[thm]{Remark}
\newtheorem{ex}[thm]{Example}
\newtheorem{exercise}[thm]{Exercise}
\newtheorem{question}[thm]{Question}
\newtheorem{problem}[thm]{Problem}
\newtheorem{dfn/thm}[thm]{Definition/Theorem}
\renewcommand{\baselinestretch}{1.05}
\newcommand{\reals}{\mathbb R}
\newcommand{\cplx}{\mathbb C}
\newcommand{\intg}{\mathbb Z}
\newcommand{\bbk}{\mathbb K}
\newcommand{\bbf}{\mathbb F}
\newcommand{\ratl}{\mathbb Q}
\newcommand{\torus}{\mathbb T}
\newcommand{\sca}{{\mathfrak a}}
\newcommand{\scb}{{\mathfrak b}}
\newcommand{\scc}{{\mathfrak c}}
\newcommand{\scm}{{\mathfrak m}}
\newcommand{\scn}{{\mathfrak n}}
\newcommand{\scp}{{\mathfrak p}}
\newcommand{\scq}{\mathfrak q}
\newcommand{\frakg}{{\mathfrak g}}
\newcommand{\frakd}{{\mathfrak d}}
\newcommand{\calf}{{\cal F}}
\newcommand{\calg}{{\cal G}}
\newcommand{\cala}{{\cal A}}
\newcommand{\calb}{{\cal B}}
\newcommand{\calc}{{\cal C}}
\newcommand{\cale}{{\cal E}}
\newcommand{\call}{{\cal L}}
\newcommand{\caln}{{\cal N}}
\newcommand{\calo}{{\cal O}}
\newcommand{\calr}{{\cal R}}
\newcommand{\mathbold}{\bf}
\newcommand{\cinf}{C^{\infty}}
\newcommand{\row}[2]{#1_1,\dots ,#1_{#2}}
\newcommand{\dbyd}[2]{{\partial #1\over\partial #2}}
\newcommand{\Space}{{\bf Space}}
\newcommand{\alg}{{\mathbold Alg}}
\newcommand{\notsubset}{\not \subset}
\newcommand{\notsupset}{\not \supset}
\newcommand{\pois}{{\mathbold Pois}}
\newcommand{\pitilde}{\tilde{\pi}}
\newcommand{\rta}{\rightarrow}
\newcommand{\Lrta}{\Longrightarrow}
\newcommand{\lrta}{\longrightarrow}
\newcommand{\llrta}{\longleftrightarrow}
\newcommand{\Llta}{\Longleftarrow}
\newcommand{\Llrta}{\Longleftrightarrow}
\newcommand{\lgl}{\langle}
\newcommand{\rgl}{\rangle}
\newcommand{\inj}{\hookrightarrow}
\newcommand{\downmapsto}{\rotatebox[origin=c]{-90}{$\scriptstyle\mapsto$}\mkern2mu}
\renewcommand{\qedsymbol}{$\square$}
\bibliographystyle{plain}
\title{\bf Solution Manual to Ravi Vakil's FOAG}
\author{by Vector\_Cat} %\thanks{Research partially supported by NSF Grant DMS-96-25122 and the Miller Institute for Basic Research in Science.}
\begin{document}
\maketitle
\section{Chap 1}
\section{Chap 2}
\section{Chap 3}
\begin{dfn}
The set $\text{Spec}\cala$ is the set of  prime ideals of $\cala$. The prime ideals $\scp$ of $\cala$ when considered as an element of $\text{Spec}\cala$ will be denoted $[\scp]$, to avoid confusion. Element $a\in\cala$ will be called \textbf{functions on} $\text{Spec}\cala$, and their \textbf{value} at the point $[\scp]$ will be $a\mod \scp$.

If $\cala$ is generated over a base field or base ring by element $x_1,...,x_r,$ the elements are often called \textbf{coordinates}
\end{dfn}

Afterwards, we will interpret functions on $\text{Spec}\cala$ as global sections of the structure sheaf, i.e., as a function on a ringed space.
\begin{ex}\ 
\begin{itemize}
\item $\text{Spec }\cplx[x]=\mathbb{A}^1_\cplx$
\item $\text{Spec }k[x]=\mathbb{A}^1_k$, where $k$ is an algebraically closed field. This is called the affine line over $k$.
\item $\text{Spec }\intg$. Isomorphic to the set of prime numbers union with $\{(0)\}$
\end{itemize}
\end{ex}
\begin{exercise}
aaa
\end{exercise}
\begin{exercise}
 IMPORTANT EXERCISE FOR THOSE WITH A LITTLE EXPERIENCE WITH MANIFOLDS. Suppose that $\pi: X\lrta Y$ is a continuous map of differentiable manifolds (as topological spaces — not a priori differentiable). Show that π is differentiable if differentiable functions pull back to differentiable functions, i.e., if pullback by π gives a map $\calo_Y\rta \pi_*\calo_X$. (Hint: check this on small patches. Once you figure out waht your trying to show you will find realize that the resuilt is immediate.)
\end{exercise}
\begin{proof}
aaa
\end{proof}
\end{document}