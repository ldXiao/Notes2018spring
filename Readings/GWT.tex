%%%%%%%%%%%%%%%%%%%%%%%%%%%%%%%%%%%%%%%%%
% Tufte-Style Book (Documentation Template)
% LaTeX Template
% Version 1.0 (5/1/13)
%
% This template has been downloaded from:
% http://www.LaTeXTemplates.com
%
% Original author:
% The Tufte-LaTeX Developers (tufte-latex.googlecode.com)
%
% License:
% Apache License (Version 2.0)
%
% IMPORTANT NOTE:
% In addition to running BibTeX to compile the reference list from the .bib
% file, you will need to run MakeIndex to compile the index at the end of the
% document.
%
%%%%%%%%%%%%%%%%%%%%%%%%%%%%%%%%%%%%%%%%%

%----------------------------------------------------------------------------------------
%	PACKAGES AND OTHER DOCUMENT CONFIGURATIONS
%----------------------------------------------------------------------------------------

\documentclass{tufte-book} % Use the tufte-book class which in turn uses the tufte-common class

\hypersetup{colorlinks} % Comment this line if you don't wish to have colored links

\usepackage{microtype} % Improves character and word spacing

\usepackage{lipsum} % Inserts dummy text

\usepackage{booktabs} % Better horizontal rules in tables

\usepackage{graphicx} % Needed to insert images into the document
\graphicspath{{graphics/}} % Sets the default location of pictures
\setkeys{Gin}{width=\linewidth,totalheight=\textheight,keepaspectratio} % Improves figure scaling

\usepackage{fancyvrb} % Allows customization of verbatim environments
\fvset{fontsize=\normalsize} % The font size of all verbatim text can be changed here

\newcommand{\hangp}[1]{\makebox[0pt][r]{(}#1\makebox[0pt][l]{)}} % New command to create parentheses around text in tables which take up no horizontal space - this improves column spacing
\newcommand{\hangstar}{\makebox[0pt][l]{*}} % New command to create asterisks in tables which take up no horizontal space - this improves column spacing

\usepackage{xspace} % Used for printing a trailing space better than using a tilde (~) using the \xspace command

\newcommand{\monthyear}{\ifcase\month\or January\or February\or March\or April\or May\or June\or July\or August\or September\or October\or November\or December\fi\space\number\year} % A command to print the current month and year

\newcommand{\openepigraph}[2]{ % This block sets up a command for printing an epigraph with 2 arguments - the quote and the author
\begin{fullwidth}
\sffamily\large
\begin{doublespace}
\noindent\allcaps{#1}\\ % The quote
\noindent\allcaps{#2} % The author
\end{doublespace}
\end{fullwidth}
}

\newcommand{\blankpage}{\newpage\hbox{}\thispagestyle{empty}\newpage} % Command to insert a blank page

\usepackage{units} % Used for printing standard units

\newcommand{\hlred}[1]{\textcolor{Maroon}{#1}} % Print text in maroon
\newcommand{\hangleft}[1]{\makebox[0pt][r]{#1}} % Used for printing commands in the index, moves the slash left so the command name aligns with the rest of the text in the index 
\newcommand{\hairsp}{\hspace{1pt}} % Command to print a very short space
\newcommand{\ie}{\textit{i.\hairsp{}e.}\xspace} % Command to print i.e.
\newcommand{\eg}{\textit{e.\hairsp{}g.}\xspace} % Command to print e.g.
\newcommand{\na}{\quad--} % Used in tables for N/A cells
\newcommand{\measure}[3]{#1/#2$\times$\unit[#3]{pc}} % Typesets the font size, leading, and measure in the form of: 10/12x26 pc.
\newcommand{\tuftebs}{\symbol{'134}} % Command to print a backslash in tt type in OT1/T1

\providecommand{\XeLaTeX}{X\lower.5ex\hbox{\kern-0.15em\reflectbox{E}}\kern-0.1em\LaTeX}
\newcommand{\tXeLaTeX}{\XeLaTeX\index{XeLaTeX@\protect\XeLaTeX}} % Command to print the XeLaTeX logo while simultaneously adding the position to the index

\newcommand{\doccmdnoindex}[2][]{\texttt{\tuftebs#2}} % Command to print a command in texttt with a backslash of tt type without inserting the command into the index

\newcommand{\doccmddef}[2][]{\hlred{\texttt{\tuftebs#2}}\label{cmd:#2}\ifthenelse{\isempty{#1}} % Command to define a command in red and add it to the index
{ % If no package is specified, add the command to the index
\index{#2 command@\protect\hangleft{\texttt{\tuftebs}}\texttt{#2}}% Command name
}
{ % If a package is also specified as a second argument, add the command and package to the index
\index{#2 command@\protect\hangleft{\texttt{\tuftebs}}\texttt{#2} (\texttt{#1} package)}% Command name
\index{#1 package@\texttt{#1} package}\index{packages!#1@\texttt{#1}}% Package name
}}

\newcommand{\doccmd}[2][]{% Command to define a command and add it to the index
\texttt{\tuftebs#2}%
\ifthenelse{\isempty{#1}}% If no package is specified, add the command to the index
{%
\index{#2 command@\protect\hangleft{\texttt{\tuftebs}}\texttt{#2}}% Command name
}
{%
\index{#2 command@\protect\hangleft{\texttt{\tuftebs}}\texttt{#2} (\texttt{#1} package)}% Command name
\index{#1 package@\texttt{#1} package}\index{packages!#1@\texttt{#1}}% Package name
}}

% A bunch of new commands to print commands, arguments, environments, classes, etc within the text using the correct formatting
\newcommand{\docopt}[1]{\ensuremath{\langle}\textrm{\textit{#1}}\ensuremath{\rangle}}
\newcommand{\docarg}[1]{\textrm{\textit{#1}}}
\newenvironment{docspec}{\begin{quotation}\ttfamily\parskip0pt\parindent0pt\ignorespaces}{\end{quotation}}
\newcommand{\docenv}[1]{\texttt{#1}\index{#1 environment@\texttt{#1} environment}\index{environments!#1@\texttt{#1}}}
\newcommand{\docenvdef}[1]{\hlred{\texttt{#1}}\label{env:#1}\index{#1 environment@\texttt{#1} environment}\index{environments!#1@\texttt{#1}}}
\newcommand{\docpkg}[1]{\texttt{#1}\index{#1 package@\texttt{#1} package}\index{packages!#1@\texttt{#1}}}
\newcommand{\doccls}[1]{\texttt{#1}}
\newcommand{\docclsopt}[1]{\texttt{#1}\index{#1 class option@\texttt{#1} class option}\index{class options!#1@\texttt{#1}}}
\newcommand{\docclsoptdef}[1]{\hlred{\texttt{#1}}\label{clsopt:#1}\index{#1 class option@\texttt{#1} class option}\index{class options!#1@\texttt{#1}}}
\newcommand{\docmsg}[2]{\bigskip\begin{fullwidth}\noindent\ttfamily#1\end{fullwidth}\medskip\par\noindent#2}
\newcommand{\docfilehook}[2]{\texttt{#1}\index{file hooks!#2}\index{#1@\texttt{#1}}}
\newcommand{\doccounter}[1]{\texttt{#1}\index{#1 counter@\texttt{#1} counter}}

\usepackage{makeidx} % Used to generate the index
\makeindex % Generate the index which is printed at the end of the document

% This block contains a number of shortcuts used throughout the book
\newcommand{\vdqi}{\textit{VDQI}\xspace}
\newcommand{\ei}{\textit{EI}\xspace}
\newcommand{\ve}{\textit{VE}\xspace}
\newcommand{\be}{\textit{BE}\xspace}
\newcommand{\VDQI}{\textit{The Visual Display of Quantitative Information}\xspace}
\newcommand{\EI}{\textit{Envisioning Information}\xspace}
\newcommand{\VE}{\textit{Visual Explanations}\xspace}
\newcommand{\BE}{\textit{Beautiful Evidence}\xspace}
\newcommand{\TL}{Tufte-\LaTeX\xspace}
\usepackage{amssymb}
\usepackage{latexsym}
\usepackage{amsmath}
\usepackage{amsthm}
%\usepackage{stmaryrd}
\newcommand{\Swarrow}{\mathbin{\rotatebox[origin=c]{45}{$\Leftarrow$}}}
\usepackage{fancyhdr}
\pagestyle{headings}
\usepackage{dsfont}
\usepackage{pifont}
\usepackage{mathtools}
\usepackage{natbib}
\usepackage{tikz-cd}
\usepackage{pgfplots}
\usepackage{enumitem}
\tikzset{
  subseteq/.style={
    draw=none,
    edge node={node [sloped, allow upside down, auto=false]{$\subseteq$}}
    },
  Subseteq/.style={
    draw=none,
    every to/.append style={
      edge node={node [sloped, allow upside down, auto=false]{$\subseteq$}}}
    },
    Subsetneq/.style={
    draw=none,
    every to/.append style={
      edge node={node [sloped, allow upside down, auto=false]{$\subsetneq$}}}
    },
  Supseteq/.style={
    draw=none,
    every to/.append style={
      edge node={node [sloped, allow upside down, auto=false]{$\supseteq$}}}
  }
}
\newcounter{dummy} 
\numberwithin{dummy}{section}
\newtheorem{thm}{Theorem}[dummy]
\newtheorem{prop}[thm]{Proposition}
\newtheorem{lemma}[thm]{Lemma}
\newtheorem{cor}[thm]{Corollary}
\newtheorem{dfn}[thm]{Definition}
\newtheorem{axiom}[thm]{Axiom}
\newtheorem{rmk}[thm]{Remark}
\newtheorem{rmkt}[thm]{Remark by TeXer}
\newtheorem{ex}[thm]{Example}
\newtheorem{nex}[thm]{Non-example}
\newtheorem{exercise}[thm]{Exercise}
\newtheorem{question}[thm]{Question}
\newtheorem{problem}[thm]{Problem}
\newtheorem{dfn/thm}[thm]{Definition/Theorem}
\renewcommand*{\thethm}{\thesection.\arabic{thm}}
\renewcommand{\thefootnote}{\arabic{footnote}}

\makeatletter
\@addtoreset{footnote}{section}
\makeatother

%------------------------Colors----------------------------
\definecolor{mygray}{gray}{0.6}

%-------------------------Short cuts----------------------
\renewcommand{\baselinestretch}{1.1}
\newcommand{\mor}{{\textnormal Mor\,}}
\renewcommand{\hom}{{\textnormal Hom}}
\newcommand{\spec}{\textnormal{Spec}\,}
\newcommand{\reals}{\mathbb R}
\newcommand{\cplx}{\mathbb C}
\newcommand{\intg}{\mathbb Z}
\newcommand{\bbf}{\mathbb F}
\newcommand{\ratl}{\mathbb Q}
\newcommand{\torus}{\mathbb T}
\newcommand{\sca}{{\mathfrak a}}
\newcommand{\scb}{{\mathfrak b}}
\newcommand{\scc}{{\mathfrak c}}
\newcommand{\scm}{{\mathfrak m}}
\newcommand{\scn}{{\mathfrak n}}
\newcommand{\scp}{{\mathfrak p}}
\newcommand{\scq}{\mathfrak q}
\newcommand{\frakg}{{\mathfrak g}}
\newcommand{\frakd}{{\mathfrak d}}
\newcommand{\pd}{{\partial}}
\newcommand{\proj}{{\mathbb P}}
\newcommand{\calf}{{\cal F}}
\newcommand{\calg}{{\cal G}}
\newcommand{\cala}{{\cal A}}
\newcommand{\calb}{{\cal B}}
\newcommand{\calc}{{\cal C}}
\newcommand{\cald}{{\cal D}}
\newcommand{\cale}{{\cal E}}
\newcommand{\cali}{{\cal I}}
\newcommand{\call}{{\cal L}}
\newcommand{\calm}{{\cal M}}
\newcommand{\caln}{{\cal N}}
\newcommand{\calo}{{\cal O}}
\newcommand{\calr}{{\cal R}}
%\newcommand{\mathbold}{\bf}
\newcommand{\cinf}{C^{\infty}}
\newcommand{\row}[2]{#1_1,\dots ,#1_{#2}}
\newcommand{\dbyd}[2]{{\partial #1\over\partial #2}}
\newcommand{\Space}{{\bf Space}}
\newcommand{\alg}{{\mathbold Alg}}
\newcommand{\notsubset}{\not \subset}
\newcommand{\notsupset}{\not \supset}
\newcommand{\pois}{{\mathbold Pois}}
\newcommand{\pitilde}{\tilde{\pi}}
\newcommand{\rta}{\rightarrow}
\newcommand{\llta}{\longleftarrow}
\newcommand{\Lrta}{\Longrightarrow}
\newcommand{\lrta}{\longrightarrow}
\newcommand{\llrta}{\longleftrightarrow}
\newcommand{\Llta}{\Longleftarrow}
\newcommand{\Llrta}{\Longleftrightarrow}
\newcommand{\lgl}{\langle}
\newcommand{\rgl}{\rangle}
\newcommand{\inj}{\hookrightarrow}
\newcommand{\surj}{\twoheadrightarrow}
\newcommand{\cmark}{\ding{51}}%
\newcommand{\xmark}{\ding{55}}%
\newcommand{\downmapsto}{\rotatebox[origin=c]{-90}{$\scriptstyle\mapsto$}\mkern2mu}
\renewcommand{\qedsymbol}{$\square$}
\newcommand{\ssh}{\mathsf{Sh}}
\newcommand{\bfp}{\mathbf{P}}
%----------------------------------------------------------------------------------------
%	BOOK META-INFORMATION
%----------------------------------------------------------------------------------------

\title{Mini-Course on Gromov-Witten Theory} % Title of the book

\author{Lecture dilivered by Chiu-Chu Melissa Liu} % Author

%\publisher{Publisher of This Book} % Publisher

%----------------------------------------------------------------------------------------

\begin{document}

\frontmatter

%----------------------------------------------------------------------------------------
%	EPIGRAPH
%----------------------------------------------------------------------------------------

%\thispagestyle{empty}
%\openepigraph{The public is more familiar with bad design than good design. It is, in effect, conditioned to prefer bad design, because that is what it lives with. The new becomes threatening, the old reassuring.}{Paul Rand, {\itshape Design, Form, and Chaos}}
%\vfill
%\openepigraph{A designer knows that he has achieved perfection not when there is nothing left to add, but when there is nothing left to take away.}{Antoine de Saint-Exup\'{e}ry}
%\vfill
%\openepigraph{\ldots the designer of a new system must not only be the implementor and the first large-scale user; the designer should also write the first user manual\ldots If I had not participated fully in all these activities, literally hundreds of improvements would never have been made, because I would never have thought of them or perceived why they were important.}{Donald E. Knuth}

%----------------------------------------------------------------------------------------

\maketitle % Print the title page

%----------------------------------------------------------------------------------------
%	COPYRIGHT PAGE
%----------------------------------------------------------------------------------------

\newpage
\begin{fullwidth}
~\vfill
\thispagestyle{empty}
\setlength{\parindent}{0pt}
\setlength{\parskip}{\baselineskip}
Copyright \copyright\ \the\year\ \thanklessauthor

\par\smallcaps{Published by \thanklesspublisher}

\par\smallcaps{tufte-latex.googlecode.com}

\par Licensed under the Apache License, Version 2.0 (the ``License''); you may not use this file except in compliance with the License. You may obtain a copy of the License at \url{http://www.apache.org/licenses/LICENSE-2.0}. Unless required by applicable law or agreed to in writing, software distributed under the License is distributed on an \smallcaps{``AS IS'' BASIS, WITHOUT WARRANTIES OR CONDITIONS OF ANY KIND}, either express or implied. See the License for the specific language governing permissions and limitations under the License.\index{license}

\par\textit{First printing, \monthyear}
\end{fullwidth}

%----------------------------------------------------------------------------------------

\tableofcontents % Print the table of contents

%----------------------------------------------------------------------------------------

%\listoffigures % Print a list of figures

%----------------------------------------------------------------------------------------

%\listoftables % Print a list of tables

%----------------------------------------------------------------------------------------
%	DEDICATION PAGE
%----------------------------------------------------------------------------------------

%\cleardoublepage
%~\vfill
%\begin{doublespace}
%\noindent\fontsize{18}{22}\selectfont\itshape
%\nohyphenation
%Dedicated to those who appreciate \LaTeX{} and the work of \mbox{Edward R.~Tufte} and %\mbox{Donald E.~Knuth}.
%\end{doublespace}
%\vfill
%\vfill

%----------------------------------------------------------------------------------------
%	INTRODUCTION
%----------------------------------------------------------------------------------------

\cleardoublepage
\chapter{Introduction} % The asterisk leaves out this chapter from the table of contents

This sample book discusses the design of Edward Tufte's books\cite{Tufte2001,Tufte1990,Tufte1997,Tufte2006} and the use of the \doccls{tufte-book} and \doccls{tufte-handout} document classes.

%----------------------------------------------------------------------------------------

\mainmatter

%----------------------------------------------------------------------------------------
%	CHAPTER 1
%----------------------------------------------------------------------------------------
\chapter{Lecture 1}
\section{Gromov-Witten Theory: an Overview}
There are two approaches developed in $1990'$s towards Gromov-Witten theory.

\underline{Symplectic Approach}:
Given a compact manifold, $(X,\omega, J)$, with compatible symplectic structure $\omega$ and complex structure $J$, i.e. a K{\"a}hler manifold.

Naively, Gromov-Witten (GW) invariants of $X$ count parametrized homolomorphic curves in $X$:
$$
f:C\overset{hol}{\lrta} X
$$
with $C$ a compact Riemann surfaces. 

More generally, given $(X,\omega, J)$ with $J$ being only an almost complex structure that compatible with symplectic structure $\omega$.
By compatibility,  we mean the almost complex structure
$$
J:T_X\lrta T_X, J^2=-id_{T_X}
$$
satisfies
$$
\omega(Ju,Jv)=\omega(u,v)
$$
and
$$
\omega(u,Ju)>0
$$
for $u\neq 0$.

Naively, GW invariants of $X$ count \textbf{$J$-holomorphic }curves in $X$.
\begin{dfn}
A \textbf{$J$-holomorphic } curve means a smooth map $f$ 
$$
f:(C,j)\lrta (X,J)
$$
where $(C,j)$ is a compact Riemann surface that satisfies the Cauchy-Riemann equation
$$
J\circ df=df\circ j
$$
\end{dfn}
The classic reference of the symplectic approach includes
\begin{itemize}
\item M. Gromov, ``Pseudoholomorphic curves in Symplectic manifolds''
\item D.McDuff, D. Salamon, ``$J$-holomorphic curves and symplectic topoogy''
\end{itemize}
\underline{Algebraic Approach}

Let $X$ be  a smooth projective variety over $\cplx$. $X\inj \proj^N$, by Kodaira's embedding theorem, $X$ must be a compact K{\"a}hler manifold. Naively, GW invariants of $X$ count \textbf{parametrized algebraic curves} in $X$.
\begin{dfn}

\end{dfn}
\subsection*{Reminders on topological manifolds}
\begin{dfn}
\begin{enumerate}
  \item A \textbf{topological manifold} is a topological space $X$, which has an open cover $\{U_i\}_{i\in I}$ such that for each $i\in I$, there exists a homeomorphism between $U_i$ and an open subset in $\reals^{n_i}$ 

  \item the category of topological manifold is a subcategory of $\mathsf{Top}$. It is denoted as $\mathsf{VarTop}$. 
\end{enumerate}
\end{dfn}

\subsection{Smooth and etale morphisms}
\section{Lecture 4: Schemes and algebraic spaces II}
\section{Lecture 4 1/2: Complements on schemes and algebraic spaces}
\chapter{Stacks}
\section{Lecture 5: Stacks I}
\section{Lecture 6: Stacks II}
\section{Lecture 7: Stacks III}
\section{Lecture 8: Stacks IV}
\section{Lecture 9: Some exercises}
\chapter[On the Use of the tufte-book Document Class]{On the Use of the \texttt{tufte-book} Document Class}
\label{ch:tufte-book}

\begin{dfn}
adas
\end{dfn}
The \TL document classes define a style similar to the style Edward Tufte uses in his books and handouts. Tufte's style is known for its extensive use of sidenotes, tight integration of graphics with text, and well-set typography. This document aims to be at once a demonstration of the features of the \TL document classes and a style guide to their use.

%------------------------------------------------

\section{Page Layout}\label{sec:page-layout}
\subsection{Headings}\label{sec:headings}\index{headings}
This style provides \textsc{a}- and \textsc{b}-heads (that is, \Verb|\section| and \Verb|\subsection|), demonstrated above.

If you need more than two levels of section headings, you'll have to define them yourself at the moment; there are no pre-defined styles for anything below a \Verb|\subsection|. As Bringhurst points out in \textit{The Elements of Typographic Style},\cite{Bringhurst2005} you should ``use as many levels of headings as you need: no more, and no fewer.''

The \TL classes will emit an error if you try to use \linebreak\Verb|\subsubsection| and smaller headings.

\newthought{In his later books},\cite{Tufte2006} Tufte starts each section with a bit of vertical space, a non-indented paragraph, and sets the first few words of the sentence in \textsc{small caps}. To accomplish this using this style, use the \doccmddef{newthought} command:
\begin{docspec}
\doccmd{newthought}\{In his later books\}, Tufte starts\ldots
\end{docspec}

%------------------------------------------------

\section{Sidenotes}\label{sec:sidenotes}
One of the most prominent and distinctive features of this style is the extensive use of sidenotes. There is a wide margin to provide ample room for sidenotes and small figures. Any \doccmd{footnote}s will automatically be converted to sidenotes.\footnote{This is a sidenote that was entered using the \texttt{\textbackslash footnote} command.} If you'd like to place ancillary information in the margin without the sidenote mark (the superscript number), you can use the \doccmd{marginnote} command.\marginnote{This is a margin note. Notice that there isn't a number preceding the note, and there is no number in the main text where this note was written.}

The specification of the \doccmddef{sidenote} command is:
\begin{docspec}
\doccmd{sidenote}[\docopt{number}][\docopt{offset}]\{\docarg{Sidenote text.}\}
\end{docspec}

Both the \docopt{number} and \docopt{offset} arguments are optional. If you provide a \docopt{number} argument, then that number will be used as the sidenote number. It will change of the number of the current sidenote only and will not affect the numbering sequence of subsequent sidenotes.

Sometimes a sidenote may run over the top of other text or graphics in the margin space. If this happens, you can adjust the vertical position of the sidenote by providing a dimension in the \docopt{offset} argument. Some examples of valid dimensions are:
\begin{docspec}
\ttfamily 1.0in \qquad 2.54cm \qquad 254mm \qquad 6\Verb|\baselineskip|
\end{docspec}
If the dimension is positive it will push the sidenote down the page; if the dimension is negative, it will move the sidenote up the page.

While both the \docopt{number} and \docopt{offset} arguments are optional, they must be provided in order. To adjust the vertical position of the sidenote while leaving the sidenote number alone, use the following syntax:
\begin{docspec}
\doccmd{sidenote}[][\docopt{offset}]\{\docarg{Sidenote text.}\}
\end{docspec}
The empty brackets tell the \Verb|\sidenote| command to use the default sidenote number.

If you \emph{only} want to change the sidenote number, however, you may completely omit the \docopt{offset} argument:
\begin{docspec}
\doccmd{sidenote}[\docopt{number}]\{\docarg{Sidenote text.}\}
\end{docspec}

The \doccmddef{marginnote} command has a similar \docarg{offset} argument:
\begin{docspec}
\doccmd{marginnote}[\docopt{offset}]\{\docarg{Margin note text.}\}
\end{docspec}

%------------------------------------------------

\section{References}
References are placed alongside their citations as sidenotes, as well. This can be accomplished using the normal \doccmddef{cite} command.\sidenote{The first paragraph of this document includes a citation.}

The complete list of references may also be printed automatically by using the \doccmddef{bibliography} command. (See the end of this document for an example.) If you do not want to print a bibliography at the end of your document, use the \doccmddef{nobibliography} command in its place. 

To enter multiple citations at one location,\cite[-3\baselineskip]{Tufte2006,Tufte1990} you can provide a list of keys separated by commas and the same optional vertical offset argument: \Verb|\cite{Tufte2006,Tufte1990}|. 
\begin{docspec}
\doccmd{cite}[\docopt{offset}]\{\docarg{bibkey1,bibkey2,\ldots}\}
\end{docspec}

%------------------------------------------------

\section{Figures and Tables}\label{sec:figures-and-tables}
Images and graphics play an integral role in Tufte's work. In addition to the standard \docenvdef{figure} and \docenvdef{tabular} environments, this style provides special figure and table environments for full-width floats.

Full page--width figures and tables may be placed in \docenvdef{figure*} or \docenvdef{table*} environments. To place figures or tables in the margin, use the \docenvdef{marginfigure} or \docenvdef{margintable} environments as follows (see figure~\ref{fig:marginfig}):


\begin{docspec}
\textbackslash begin\{marginfigure\}\\
\qquad\textbackslash includegraphics\{helix\}\\
\qquad\textbackslash caption\{This is a margin figure.\}\\
\qquad\textbackslash label\{fig:marginfig\}\\
\textbackslash end\{marginfigure\}\\
\end{docspec}

The \docenv{marginfigure} and \docenv{margintable} environments accept an optional parameter \docopt{offset} that adjusts the vertical position of the figure or table. See the ``\nameref{sec:sidenotes}'' section above for examples. The specifications are:
\begin{docspec}
\textbackslash{begin\{marginfigure\}[\docopt{offset}]}\\
\qquad\ldots\\
\textbackslash{end\{marginfigure\}}\\
\mbox{}\\
\textbackslash{begin\{margintable\}[\docopt{offset}]}\\
\qquad\ldots\\
\textbackslash{end\{margintable\}}\\
\end{docspec}

Figure~\ref{fig:fullfig} is an example of the \docenv{figure*} environment and figure~\ref{fig:textfig} is an example of the normal \docenv{figure} environment.


As with sidenotes and marginnotes, a caption may sometimes require vertical adjustment. The \doccmddef{caption} command now takes a second optional argument that enables you to do this by providing a dimension \docopt{offset}. You may specify the caption in any one of the following forms:
\begin{docspec}
\doccmd{caption}\{\docarg{long caption}\}\\
\doccmd{caption}[\docarg{short caption}]\{\docarg{long caption}\}\\
\doccmd{caption}[][\docopt{offset}]\{\docarg{long caption}\}\\
\doccmd{caption}[\docarg{short caption}][\docopt{offset}]%
\{\docarg{long caption}\}
\end{docspec}
A positive \docopt{offset} will push the caption down the page. The short caption, if provided, is what appears in the list of figures/tables, otherwise the ``long'' caption appears there. Note that although the arguments \docopt{short caption} and \docopt{offset} are both optional, they must be provided in order. Thus, to specify an \docopt{offset} without specifying a \docopt{short caption}, you must include the first set of empty brackets \Verb|[]|, which tell \doccmd{caption} to use the default ``long'' caption. As an example, the caption to figure~\ref{fig:textfig} above was given in the form
\begin{docspec}
\doccmd{caption}[Hilbert curves...][6pt]\{Hilbert curves...\}
\end{docspec}

Table~\ref{tab:normaltab} shows table created with the \docpkg{booktabs} package. Notice the lack of vertical rules---they serve only to clutter the table's data.

\begin{table}[ht]
\centering
\fontfamily{ppl}\selectfont
\begin{tabular}{ll}
\toprule
Margin & Length \\
\midrule
Paper width & \unit[8\nicefrac{1}{2}]{inches} \\
Paper height & \unit[11]{inches} \\
Textblock width & \unit[6\nicefrac{1}{2}]{inches} \\
Textblock/sidenote gutter & \unit[\nicefrac{3}{8}]{inches} \\
Sidenote width & \unit[2]{inches} \\
\bottomrule
\end{tabular}
\caption{Here are the dimensions of the various margins used in the Tufte-handout class.}
\label{tab:normaltab}
\end{table}

\newthought{Occasionally} \LaTeX{} will generate an error message:\label{err:too-many-floats}
\begin{docspec}
Error: Too many unprocessed floats
\end{docspec}
\LaTeX{} tries to place floats in the best position on the page. Until it's finished composing the page, however, it won't know where those positions are. If you have a lot of floats on a page (including sidenotes, margin notes, figures, tables, etc.), \LaTeX{} may run out of ``slots'' to keep track of them and will generate the above error.

\LaTeX{} initially allocates 18 slots for storing floats. To work around this limitation, the \TL document classes provide a \doccmddef{morefloats} command that will reserve more slots.

The first time \doccmd{morefloats} is called, it allocates an additional 34 slots. The second time \doccmd{morefloats} is called, it allocates another 26 slots.

The \doccmd{morefloats} command may only be used two times. Calling it a third time will generate an error message. (This is because we can't safely allocate many more floats or \LaTeX{} will run out of memory.)

If, after using the \doccmd{morefloats} command twice, you continue to get the \texttt{Too many unprocessed floats} error, there are a couple things you can do.

The \doccmddef{FloatBarrier} command will immediately process all the floats before typesetting more material. Since \doccmd{FloatBarrier} will start a new paragraph, you should place this command at the beginning or end of a paragraph.

The \doccmddef{clearpage} command will also process the floats before continuing, but instead of starting a new paragraph, it will start a new page.

You can also try moving your floats around a bit: move a figure or table to the next page or reduce the number of sidenotes. (Each sidenote actually uses \emph{two} slots.)

After the floats have placed, \LaTeX{} will mark those slots as unused so they are available for the next page to be composed.

%------------------------------------------------

\section{Captions}
You may notice that the captions are sometimes misaligned. Due to the way \LaTeX's float mechanism works, we can't know for sure where it decided to put a float. Therefore, the \TL document classes provide commands to override the caption position.

\paragraph{Vertical alignment} To override the vertical alignment, use the \doccmd{setfloatalignment} command inside the float environment. For example:

\begin{fullwidth}
\begin{docspec}
\textbackslash begin\{figure\}[btp]\\
\qquad \textbackslash includegraphics\{sinewave\}\\
\qquad \textbackslash caption\{This is an example of a sine wave.\}\\
\qquad \textbackslash label\{fig:sinewave\}\\
\qquad \hlred{\textbackslash setfloatalignment\{b\}\% forces caption to be bottom-aligned}\\
\textbackslash end\{figure\}
\end{docspec}
\end{fullwidth}

\noindent The syntax of the \doccmddef{setfloatalignment} command is:

\begin{docspec}
\doccmd{setfloatalignment}\{\docopt{pos}\}
\end{docspec}

\noindent where \docopt{pos} can be either \texttt{b} for bottom-aligned captions, or \texttt{t} for top-aligned captions.

\paragraph{Horizontal alignment}\label{par:overriding-horizontal}
To override the horizontal alignment, use either the \doccmd{forceversofloat} or the \doccmd{forcerectofloat} command inside of the float environment. For example:

\begin{fullwidth}
\begin{docspec}
\textbackslash begin\{figure\}[btp]\\
\qquad \textbackslash includegraphics\{sinewave\}\\
\qquad \textbackslash caption\{This is an example of a sine wave.\}\\
\qquad \textbackslash label\{fig:sinewave\}\\
\qquad \hlred{\textbackslash forceversofloat\% forces caption to be set to the left of the float}\\
\textbackslash end\{figure\}
\end{docspec}
\end{fullwidth}

The \doccmddef{forceversofloat} command causes the algorithm to assume the float has been placed on a verso page---that is, a page on the left side of a two-page spread. Conversely, the \doccmddef{forcerectofloat} command causes the algorithm to assume the float has been placed on a recto page---that is, a page on the right side of a two-page spread.

%------------------------------------------------

\section{Full-width text blocks}

In addition to the new float types, there is a \docenvdef{fullwidth} environment that stretches across the main text block and the sidenotes area.

\begin{Verbatim}
\begin{fullwidth}
Lorem ipsum dolor sit amet...
\end{fullwidth}
\end{Verbatim}

\begin{fullwidth}
\small\itshape\lipsum[1]
\end{fullwidth}

%------------------------------------------------

\section{Typography}\label{sec:typography}

\subsection{Typefaces}\label{sec:typefaces}\index{typefaces}
If the Palatino, \textsf{Helvetica}, and \texttt{Bera Mono} typefaces are installed, this style will use them automatically. Otherwise, we'll fall back on the Computer Modern typefaces.

\subsection{Letterspacing}\label{sec:letterspacing}
This document class includes two new commands and some improvements on existing commands for letterspacing.

When setting strings of \allcaps{ALL CAPS} or \smallcaps{small caps}, the letter\-spacing---that is, the spacing between the letters---should be increased slightly.\cite{Bringhurst2005} The \doccmddef{allcaps} command has proper letterspacing for strings of \allcaps{FULL CAPITAL LETTERS}, and the \doccmddef{smallcaps} command has letterspacing for \smallcaps{small capital letters}. These commands will also automatically convert the case of the text to upper- or lowercase, respectively.

The \doccmddef{textsc} command has also been redefined to include letterspacing. The case of the \doccmd{textsc} argument is left as is, however. This allows one to use both uppercase and lowercase letters: \textsc{The Initial Letters Of The Words In This Sentence Are Capitalized.}

%------------------------------------------------

\section{Document Class Options}\label{sec:options}

\index{class options|(}
The \doccls{tufte-book} class is based on the \LaTeX\ \doccls{book} document class. Therefore, you can pass any of the typical book options. There are a few options that are specific to the \doccls{tufte-book} document class, however.

The \docclsoptdef{a4paper} option will set the paper size to \smallcaps{A4} instead of the default \smallcaps{US} letter size.

The \docclsoptdef{sfsidenotes} option will set the sidenotes and title block in a \textsf{sans serif} typeface instead of the default roman.

The \docclsoptdef{twoside} option will modify the running heads so that the page number is printed on the outside edge (as opposed to always printing the page number on the right-side edge in \docclsoptdef{oneside} mode). 

The \docclsoptdef{symmetric} option typesets the sidenotes on the outside edge of the page. This is how books are traditionally printed, but is contrary to Tufte's book design which sets the sidenotes on the right side of the page. This option implicitly sets the \docclsopt{twoside} option.

The \docclsoptdef{justified} option sets all the text fully justified (flush left and right). The default is to set the text ragged right. The body text of Tufte's books are set ragged right. This prevents needless hyphenation and makes it easier to read the text in the slightly narrower column.

The \docclsoptdef{bidi} option loads the \docpkg{bidi} package which is used with \tXeLaTeX\ to typeset bi-directional text. Since the \docpkg{bidi} package needs to be loaded before the sidenotes and cite commands are defined, it can't be loaded in the document preamble.

The \docclsoptdef{debug} option causes the \TL classes to output debug information to the log file which is useful in troubleshooting bugs. It will also cause the graphics to be replaced by outlines.

The \docclsoptdef{nofonts} option prevents the \TL classes from automatically loading the Palatino and Helvetica typefaces. You should use this option if you wish to load your own fonts. If you're using \tXeLaTeX, this option is implied (\ie, the Palatino and Helvetica fonts aren't loaded if you use \tXeLaTeX). 

The \docclsoptdef{nols} option inhibits the letterspacing code. The \TL\ classes try to load the appropriate letterspacing package (either pdf\TeX's \docpkg{letterspace} package or the \docpkg{soul} package). If you're using \tXeLaTeX\ with \docpkg{fontenc}, however, you should configure your own letterspacing. 

The \docclsoptdef{notitlepage} option causes \doccmd{maketitle} to generate a title block instead of a title page. The \doccls{book} class defaults to a title page and the \doccls{handout} class defaults to the title block. There is an analogous \docclsoptdef{titlepage} option that forces \doccmd{maketitle} to generate a full title page instead of the title block.

The \docclsoptdef{notoc} option suppresses \TL's custom table of contents (\textsc{toc}) design. The current \textsc{toc} design only shows unnumbered chapter titles; it doesn't show sections or subsections. The \docclsopt{notoc} option will revert to \LaTeX's \textsc{toc} design.

The \docclsoptdef{nohyper} option prevents the \docpkg{hyperref} package from being loaded. The default is to load the \docpkg{hyperref} package and use the \doccmd{title} and \doccmd{author} contents as metadata for the generated \textsc{pdf}.

\index{class options|)}

%----------------------------------------------------------------------------------------
%	CHAPTER 3
%----------------------------------------------------------------------------------------


\backmatter

%----------------------------------------------------------------------------------------
%	BIBLIOGRAPHY
%----------------------------------------------------------------------------------------

\bibliography{bibliography} % Use the bibliography.bib file for the bibliography
\bibliographystyle{plainnat} % Use the plainnat style of referencing

%----------------------------------------------------------------------------------------

\printindex % Print the index at the very end of the document

\end{document}