\documentclass[11pt]{article}
\usepackage{amssymb}
\usepackage{latexsym}
\usepackage{amsmath}
\usepackage{amsthm}
%\usepackage{stmaryrd}
\newcommand{\Swarrow}{\mathbin{\rotatebox[origin=c]{45}{$\Leftarrow$}}}
\usepackage{fancyhdr}
\pagestyle{headings}
\usepackage{dsfont}
\usepackage{pifont}
\usepackage{mathtools}
\usepackage{natbib}
\usepackage{tikz-cd}
\usepackage{pgfplots}
\usepackage{enumitem} 
\usepackage{hyperref}
\usepackage{geometry}
\geometry{left=4cm,right=4cm}
\pgfplotsset{every axis/.append style={
                    axis x line=middle,    % put the x axis in the middle
                    axis y line=middle,    % put the y axis in the middle
                    axis line style={<->}, % arrows on the axis
                    xlabel={$x$},          % default put x on x-axis
                    ylabel={$y$},          % default put y on y-axis
                    ticks=none,
                    }}
%\usepackage[urw-garamond]{mathdesign}
%\usepackage{cmbright}
%\usepackage{concmath}
%\usepackage{sansmathfonts}
%\renewcommand*\familydefault{\sfdefault} %% Only if the base font of the document is to be sans serif

%\usepackage{pdfrender,xcolor,scrpage2}
%\pdfrender{StrokeColor=black,TextRenderingMode=2,LineWidth=1pt}
\tikzset{
  subseteq/.style={
    draw=none,
    edge node={node [sloped, allow upside down, auto=false]{$\subseteq$}}
    },
  Subseteq/.style={
    draw=none,
    every to/.append style={
      edge node={node [sloped, allow upside down, auto=false]{$\subseteq$}}}
    },
    Subsetneq/.style={
    draw=none,
    every to/.append style={
      edge node={node [sloped, allow upside down, auto=false]{$\subsetneq$}}}
    },
  Supseteq/.style={
    draw=none,
    every to/.append style={
      edge node={node [sloped, allow upside down, auto=false]{$\supseteq$}}}
  }
}

\hypersetup{
    colorlinks,
    citecolor=blue,
    filecolor=blue,
    linkcolor=blue,
    urlcolor=blue
}
\newtheorem{thm}{Theorem}[section]
\newtheorem{prop}[thm]{Proposition}
\newtheorem{lemma}[thm]{Lemma}
\newtheorem{cor}[thm]{Corollary}
\newtheorem{dfn}[thm]{Definition}
\newtheorem{axiom}[thm]{Axiom}
\newtheorem{rmk}[thm]{Remark}
\newtheorem{rmkt}[thm]{Remark by TeXer}
\newtheorem{ex}[thm]{Example}
\newtheorem{nex}[thm]{Non-example}
\newtheorem{exercise}[thm]{Exercise}
\newtheorem{question}[thm]{Question}
\newtheorem{problem}[thm]{Problem}
\newtheorem{dfn/thm}[thm]{Definition/Theorem}
\renewcommand{\baselinestretch}{1.1}
\newcommand{\mor}{{\textnormal Mor\,}}
\renewcommand{\hom}{{\textnormal Hom}}
\newcommand{\reals}{\mathbb R}
\newcommand{\cplx}{\mathbb C}
\newcommand{\intg}{\mathbb Z}
\newcommand{\bbf}{\mathbb F}
\newcommand{\ratl}{\mathbb Q}
\newcommand{\torus}{\mathbb T}
\newcommand{\sca}{{\mathfrak a}}
\newcommand{\scb}{{\mathfrak b}}
\newcommand{\scc}{{\mathfrak c}}
\newcommand{\scm}{{\mathfrak m}}
\newcommand{\scn}{{\mathfrak n}}
\newcommand{\scp}{{\mathfrak p}}
\newcommand{\scq}{\mathfrak q}
\newcommand{\frakg}{{\mathfrak g}}
\newcommand{\frakd}{{\mathfrak d}}
\newcommand{\pd}{{\partial}}
\newcommand{\calf}{{\cal F}}
\newcommand{\calg}{{\cal G}}
\newcommand{\cala}{{\cal A}}
\newcommand{\calb}{{\cal B}}
\newcommand{\calc}{{\cal C}}
\newcommand{\cald}{{\cal D}}
\newcommand{\cale}{{\cal E}}
\newcommand{\cali}{{\cal I}}
\newcommand{\call}{{\cal L}}
\newcommand{\calm}{{\cal M}}
\newcommand{\caln}{{\cal N}}
\newcommand{\calo}{{\cal O}}
\newcommand{\calr}{{\cal R}}
\newcommand{\mathbold}{\bf}
\newcommand{\cinf}{C^{\infty}}
\newcommand{\row}[2]{#1_1,\dots ,#1_{#2}}
\newcommand{\dbyd}[2]{{\partial #1\over\partial #2}}
\newcommand{\Space}{{\bf Space}}
\newcommand{\alg}{{\mathbold Alg}}
\newcommand{\notsubset}{\not \subset}
\newcommand{\notsupset}{\not \supset}
\newcommand{\pois}{{\mathbold Pois}}
\newcommand{\pitilde}{\tilde{\pi}}
\newcommand{\rta}{\rightarrow}
\newcommand{\llta}{\longleftarrow}
\newcommand{\Lrta}{\Longrightarrow}
\newcommand{\lrta}{\longrightarrow}
\newcommand{\llrta}{\longleftrightarrow}
\newcommand{\Llta}{\Longleftarrow}
\newcommand{\Llrta}{\Longleftrightarrow}
\newcommand{\lgl}{\langle}
\newcommand{\rgl}{\rangle}
\newcommand{\inj}{\hookrightarrow}
\newcommand{\surj}{\twoheadrightarrow}
\newcommand{\cmark}{\ding{51}}%
\newcommand{\xmark}{\ding{55}}%
\newcommand{\downmapsto}{\rotatebox[origin=c]{-90}{$\scriptstyle\mapsto$}\mkern2mu}
\renewcommand{\qedsymbol}{$\square$}
\bibliographystyle{plain}
\title{\bf Mastere course on algebraic stacks}
\author{\\
Translated by Lind Axiao}
\date{2018 ETH} %\thanks{Research partially supported by NSF Grant DMS-96-25122 and the Miller Institute for Basic Research in Science.}
\begin{document}
\maketitle
\tableofcontents
\newpage
\section{Lecture 1: Reflections on the notions of space I}
The goal of the first course is to understand the notion of manifold in different contexts (topological, differentiable, analytic...) We will start by looking at the case of topological manifolds.
\subsection{Reminders on topological manifolds}
\begin{dfn}
\begin{enumerate}
  \item A \textbf{topological manifold} is a topological space $X$, which has an open cover $\{U_i\}_{i\in I}$ such that for each $i\in I$, there exists a homeomorphism between $U_i$ and an open subset in $\reals^{n_i}$ 

  \item the category of topological manifold is a subcategory of $\mathsf{Top}$. It is denoted as $\mathsf{VarTop}$. 
\end{enumerate}
\end{dfn}

Let $X$ be a topological manifold and $\{U_i\}_{i\in I}$ is an open cover as in the definition above. We put, for $i$ and $j$ in $I$, $U_{i,j}:=U_i\cap U_j$. We have a diagram of topological spaces
$$
\coprod_{(i,j)\in I^2}U_{i,j}\rightrightarrows \coprod_{i\in I} U_{i},
$$
where the first morphism sends the component $U_{i,j}$ into $U_i$ by the inclusion $U_{i,j}\inj U_i$, and the second morphism sends $U_{i,j}$ into $U_j$ by the inclusion $U_{i,j}\inj U_j$. There also exists  a morphism
$$
\coprod_{i\in I} U_i\lrta X
$$
which restricts to inclusion $U_i\inj X$ for all $i$, which equalizes the above two morphisms. We obtain also a well-defined morphism from the coequalizer of the first diagram to $X$
$$
\text{Colim}\left(\coprod_{(i,j)\in I^2}U_{i,j}\rightrightarrows \coprod_{i\in I}U_i\right)\lrta X.
$$
What makes it important is the following lemma.
\begin{lemma}
The morphism 
$$
\textnormal{Colim}\left(\coprod_{(i,j)\in I^2}U_{i,j}\rightrightarrows \coprod_{i\in I}U_i\right)\lrta X
$$
is an isomorphism.
\end{lemma}
\begin{proof}
The lemma says that for a topological space $Y$, and a given morphism $f:X\lrta Y$ is the same as giving for each $i\in I$ a morphism $f_i:U_i\lrta Y$ so that $(f_i)|_{U_{i,j}}=(f_j)|_{U_{i,j}}$ for all $(i,j)\in I^2$. (This is a direct translation of the universal property of the coequalizer. Which is true by the gluing lemma of continuous maps)
\end{proof}
The above lemma has to be interpreted in the following way: all topological manifold is obtained from a colimit of a diagram of open sets in $\reals^n$ (for $n$ variable). we can draw from it the following principle:

\textit{The category $\mathsf{TopMfd}$ of the topological manifolds can be constructed from the category of open sets in $\reals^n$ (with morphism continuous maps).}

We would explain the principle in the following section.
\subsection{Manifold and sheaves}
Let $\calc$ be the full subcategory of $\mathsf{TopMfd}$, of which the objects are open sets in $\reals^n$ for some $n$. We denote $\mathsf{Pr}(\calc)$ the category of presheaves of sets on $\calc$, (also denoted as $\widehat{\calc}$ ). We consider Yoneda embedding in the case of $\calc$
$$
\begin{aligned}
h\_: \mathsf{TopMfd}&\lrta \mathsf{Pr}(\calc)\\
X&\longmapsto h_X,
\end{aligned}
$$
where the presheaf $h_X$ is defined by 
$$
h_X(Y):=\hom_{\mathsf{TopMfd}}(Y,X)
$$
for all $Y\in \calc\subset \mathsf{TopMfd}$.

\begin{lemma}\label{lem:yoneda_full_faithful}
The functor $h\_$ above is full and faithful.
\end{lemma}
\begin{proof}
The functor is faithful: for two morphisms $f,g:X\lrta X'$, we consider an open cover $\{U_i\}$ of $X$ so that each $U_i\in\calc$ this exists because $X$ is a manifold). If $h_f=h_g$, for every $i\in I$, the two maps 
$$
h_f(U_i)=h_g(U_i):\hom(U_i, X)=h_X(U_i)\lrta \hom(U_i, X')=h_{X'}(U_i)
$$
are equal. This means that $f|_{U_i}=g|_{U_i}$, for every $i$, hence that $f=g$. 

The functor is full: Let $X$ and $X'$ be two topological manifolds and $u:h_X\lrta h_{X'}$ is a morphism of in $\mathsf{Pr}(\calc)$. Let $\{U_i\}$ be an open cover of $X$ with $U_i\in \calc$. For all $i$, the morphism $u$ induces a map 
$$
h_X(U_i)=\hom(U_i,X)\lrta h_{X'}(U_i)=\hom(U_i,X').
$$
This map send the 
inclusion $U_i\subset X$ to morphisms $f_i:U_i\lrta X'$ for all $i$.  For all $i$ and $j$ in $I$, the elements $(f_i)|_{U_{i,j}}\in h_{X'}(U_{i,j})$ agree because they are both images of the inclusion $U_{i,j}\inj X$ because the morphism of presheaves $u$ is compatible with the restriction maps. There the morphisms $f_i:U_i\lrta X'$ give a continuous map $f:X\lrta X'$. By construction $h_f=u$.
\end{proof}
The Lemma~\ref{lem:yoneda_full_faithful} is a good point remark, we know $\mathsf{TopMfd}$ can be identified with a full subcategory of $\mathsf{Pr}(\calc)$. We are now seeking to characterize the subcategory.

We start by making $\calc$ a Grothendieck site. We say a collection of morphism $\{U_i\lrta U\}_{i\in I}$ in $\calc$ is a \textbf{covering family} if each morphism $U_i\lrta U$ is an open immersion and if the map $\coprod_{i\in I}U_i\lrta U$ is surjective. This define a pretopology on $\calc$, and we denote the associated topology $\tau$. 
\begin{exercise}
Verify it indeed induces a  pretopology.
\end{exercise}
\begin{proof}[Sol]
Check the covering family defined above satisfies \href{https://ncatlab.org/nlab/show/Grothendieck+pretopology}{the axioms} of Grothendieck pretopology. And the Grothendieck topology $\tau$ on $\calc$ is generated by union of covering families (or, the covering family gives a basis of topology $\tau$)
\end{proof}
\begin{lemma}\label{lem:sheaf_on_site}
For all $X\in \mathsf{TopMfd}$ the presheaf $h_X\in\mathsf{Pr}(\calc)$ is a sheaf with respect to the topology $\tau$.
\end{lemma}
\begin{proof}
See the \href{https://stacks.math.columbia.edu/tag/00VM}{definition} of a sheaf on a site.
It is another way of saying for each topological manifold $Y$ and an open cover $\{U_i\lrta Y\}_{i\in I}$, to give a continuous map from $Y$ to $X$ is the same as to give a a collection of continuous map $f_i:U_i\lrta X$ such that $f_i$ and $f_j$ coincide on $U_i\cap U_j$. 
\end{proof}
As a result, the Lemma~\ref{lem:sheaf_on_site} implies that the there is a fully faithful functor 
$$
h\_: \mathsf{TopMfd}\lrta\mathsf{Sh}(\calc,\tau).
$$
A sheaf isomorphic to $h_X$ is called \textbf{representable} by $X$. In a general way, we identify the category of $\mathsf{TopMfd}$ with its image in $\mathsf{Sh}(\calc,\tau)$.

To characterize the image, we put the following definition
\begin{dfn}\ 
\begin{enumerate}
\item A morphism $f: F\lrta G$ in $\mathsf{Sh}(\calc,\tau)$ is a \textbf{local homeomorphism} if for all $X\in\calc$ and each morphism $h_X\lrta G$, the sheaf $F\times_G h_X$ is representable by $Y\in \mathsf{TopMfd}$, and the induced morphism $Y\lrta X$ by the projection $F\times_G h_X\simeq h_Y\lrta h_X$ is a local homeomorphism as morphism in $\mathsf{Top}$. 
\item A morphism in $\mathsf{Sh}(\calc,\tau) $ is an \textbf{open immersion} if it is a monomorphism and a local homeomorphism.
\end{enumerate}
\end{dfn}
It is easy to check that the open immersions in $\mathsf{Sh}(\calc,\tau)$ are stable under composition. We can also verify that the local homeomorphisms  are stable under composition, but it requires corollary~\ref{cor:criterion_representable} below. We also show that a morphism of topological manifolds $X\lrta Y$ is a local homeomorphism of topological spaces if and only if the morphism of sheaves $h_X\lrta h_Y$ is a local homeomorphism as defined above.

We therefore have the proposition below.
\begin{prop}\label{lem:criterion_representable_TopMfd}
A sheaf $F\in \mathsf{Sh}(\calc,\tau)$ is representable by one topological manifold iff there exists a family of objects $\{U_i\}_{i\in I}$ in $\calc$, and a morphism of sheaves
$$
p:\coprod_{i\in I}h_{U_i}\lrta F,
$$ 
that satisfy the following two conditions
\begin{enumerate}
\item The morphism $p$ is an epimorphism of sheaves.
\item For each $i\in I$, the morphism $h_{U_i}\lrta F$ is an open immersion.
\end{enumerate}
\end{prop} 
\begin{proof}
We start by supposing that $F$ is representable by one topological manifold $X$. We choose an open cover $\{U_i\}_{i\in I}$ of $X$ with $U_i\in \calc$ and we consider the morphism
$$
p:\coprod_{i\in I}h_{U_i}\lrta F\simeq h_X
$$
induced by the inclusion $U_i\subset X$. Explicitly, we have
$$
p(W):\left(\coprod_{i}h_{U_i}\right)(W)=\coprod_i h_{U_i}(W)\lrta h_X(W).
$$ 
For $Y\in \calc$ and $f:Y\lrta X$ an element in $h_X(Y)=\hom(Y,X)$, we regard $\{f^{-1}(U_i)\}_{i\in I}$ as an open cover of $Y$. Furthermore, for each $i\in I$, there exists a commutative diagram
$$
\begin{tikzcd}
f^{-1}(U_i) \arrow[d,"g"] \arrow[r] & Y \arrow[d,"f"] \\
U_i \arrow[r] & X
\end{tikzcd}
$$
which show that $f$ is locally in the image of $p$ in the sense that $f|_{f^{-1}(U_i)}$ is in the image of $p(U_i)$. By \href{http://stacks.math.columbia.edu/tag/00WL}{Tag 00WL, stacks-project},  this implies that $p$ is an epimorphism of sheaves. Moreover, for $Y\in \calc$ and for all morphism $h_Y\lrta h_X$, corresponding to a morphism $f:Y\lrta X$, we have
$$
h_{U_i}\times_{h_{X}}h_Y\simeq h_{U_i\times_X Y}=h_{f^{-1}(U_i)},
$$
where the induced map $f^{-1}(U_i)\lrta Y$ is the plain inclusion hence must be local homeomorphism in $\mathsf{Top}$,
which means $h_{U_i}\lrta F$ is a local homeomorphism by definition.
As $f^{-1}(U_i)\lrta Y$ is an open immersion, we observe that every morphism $h_{U_i}\lrta F$ is an open immersion. ($h$ is fully faithful, therefore preserves limits and colimits, thus gives us the equality above. For the same reason $h$ preserves monomorphisms, hence $h_{f^{-1}(U_i)}\lrta h_Y$ is a monomorphism. Take the special case $Y=X$, $h_{U_i}\lrta F$ is a monomorphism.)

Conversely, suppose $F$ is a sheaf satisfying the two conditions in the proposition. We construct the topological space $X$ in the following way: let $\{U_i\}_{i\in I}$ be a family of objects in the category $\calc$ and $p:\coprod h_{U_i}\lrta F$ is a morphism in the statement of the proposition. We set $U=\coprod_i U_i\in \mathsf{TopMfd}$. We remark that the morphism
$$
\coprod h_{U_i}\lrta h_U
$$
is an isomorphism in $\mathsf{Sh}(\calc,\tau)$ (Exercise, verify this.) We consider the two projections
$$
h_U\times_F h_U\rightrightarrows h_U.
$$
By hypothesis, we have
$$
h_U\times_F h_U= \coprod_{i} h_{U_i}\times_F\coprod_{j} h_{U_j}\simeq\coprod_{i,j} h_{U_{i,j}}\simeq h_R,
$$
where $R=\coprod_{i,j}U_{i,j}$, with $h_{U_{i,j}}\simeq h_{U_i}\times_F h_{U_j}$. ($U_{i,j}$ is the representing object of $h_{U_i}\times_F h_{U_j}$, is isomorphic to an open set in $U_i$ and in $U_j$). The second isomorphism above is from the fact that finite limit commutes with filtered colimit. By Lemma~\ref{lem:yoneda_full_faithful} and Lemma~\ref{lem:sheaf_on_site} the diagram 
$$
h_R\rightrightarrows h_U
$$
is image by $h$ of the diagram of topological manifolds $R\rightrightarrows U$. We set the 
$$
X:=\textnormal{colim}(R\rightrightarrows U),
$$
where the colimit is taken in the category $\mathsf{Top}$. Note that $R$ defines an equivalence relation on $U$ and that $X$ is the quotient space.

We further remark that $X$ is a topological manifold. For it, observe by definition the morphism $U\lrta X$ is surjective. More over, $U_i\lrta X$ is an open immersion. Indeed, from the fact that $h_{U_i}\lrta F$ is a monomorphism , we have $U_{i,i}=U_i$, which implies that $U_i\lrta X$ is injective (Exercise, verify this). Moreover, a subset $V\subset X$ is open iff its preimage in $U$ by the projection $U\lrta X$ is open. But the inverse image of $U_i\subset X$ by the projection is the subset $\coprod_j U_{i,j}\subset U$ which is indeed an open set. This shows that $X$ is covered by the opens $U_i\in\calc$, and therefore is a topological manifold.

It remains to show that $F$ is isomorphic to $h_X$. There exists a morphism of sheaves
$$
\textnormal{colim}(h_R\rightrightarrows h_U)\lrta h_X.
$$
Because $h_U\lrta F$ is an epimorphism and that \underline{the epimorphism of sheaves are} \href{https://ncatlab.org/nlab/show/effective+epimorphism}{  effective epimorphisms}. By the definition of effective epimorphism
$$
\textnormal{colimit}(h_U\times_Fh_U\rightrightarrows h_U)\simeq F
$$
and because $h_R\simeq h_U\times_F h_U$ as described above, we have
$$
F\simeq \textnormal{colimit}(h_R\rightrightarrows h_U).
$$
It then remains verify that $\textnormal{colim}(h_R\rightrightarrows h_U)\simeq h_X$. $h_U\lrta h_X$ is also an epimorphism of sheaves (Exercise, verify this), it remains to show that the morphism
$$
h_R\lrta h_U\times_{h_X} h_U
$$
is an isomorphism. Recall that $h$ is a fully faithful functor, it suffices to verify that the morphism
$$
R\lrta U\times_X U
$$ is an isomorphism, which is true because the morphism $U_{i,j}\lrta U_i\times_X U_j$ is an isomorphism. Explicitly, $U_i\times_X U_j=\{(u,v)\in U_i\coprod U_j: \iota_i(u)=\iota_j(v)\}$
$$
\begin{tikzcd}
U_{i,j} \arrow[rdd, "\alpha_{i,j}"'] \arrow[rrd, "\beta_{i,j}"] \arrow[rd, dashed] &  &  \\
 & U_i\times_X U_j \arrow[d] \arrow[r] & U_j \arrow[d, "\iota_j"] \\
 & U_i \arrow[r, "\iota_i"'] & X
\end{tikzcd}
$$
the dashed arrow is given by $z\longmapsto (\alpha_{i,j}(z),\beta_{i,j}(z))$, it is injective because $\alpha_{i,j},\beta_{i,j}$ are injective. It is surjective because $X:=\coprod_i U_i/\sim$, where $\iota_i(u)=\iota_j(v)$ iff
$(u,v)=(\alpha_{i,j}(z),\beta_{i,j}(z))$ for some $z\in U_{i,j}$. And by hypothesis, $h_(U_i)\lrta F$ is open immersion therefore is a local homeomorphism, we know $\alpha_{i,j},\beta_{i,j}$ are local homeomorphism. Altogether we know $z\mapsto (\alpha_{i,j}(z),\beta_{i,j}(z))$ is a bijective local homeomorphism hence a homeomorphism.
\end{proof}
\begin{cor}\label{cor:criterion_representable}
Let $X\in\mathsf{TopMfd}$, and $F\lrta X$ a morphism of shaves. If there exists an open covering $\{U_i\}_{i\in I}$ of $X$ so that for all $i\in I$ the sheaf $F\times_{h_X} h_{U_i}$ is representable by a topological manifold, the sheaf $F$ is representable by a topological manifold.
\end{cor}
\begin{proof}
For each $i\in I$, we choose $\{V_{i,j}\}_{j\in J}$ and $\coprod_j h_{V_{i,j}}\lrta F\times_{h_X} h_{U_i}$ from the above proposition~\ref{lem:criterion_representable_TopMfd}. We verify then
$$
\coprod_{i,j}h_{V_{i,j}}\lrta F
$$
is a morphism from the proposition~\ref{lem:criterion_representable_TopMfd}(Exercise, verify this)
\end{proof}
\subsection{Quotient manifolds}
\subsection{Remarks on manifolds}
\end{document}
