\documentclass[11pt]{article}
\usepackage{amssymb}
\usepackage{latexsym}
\usepackage{amsmath}
\usepackage{amsthm}
%\usepackage{stmaryrd}
\newcommand{\Swarrow}{\mathbin{\rotatebox[origin=c]{45}{$\Leftarrow$}}}
\usepackage{fancyhdr}
\pagestyle{headings}
\usepackage{dsfont}
\usepackage{pifont}
\usepackage{mathtools}
\usepackage{natbib}
\usepackage{tikz-cd}
\usepackage{pgfplots}
\usepackage{enumitem} 
\usepackage{hyperref}
\usepackage{geometry}
\geometry{left=4cm,right=4cm}
\pgfplotsset{every axis/.append style={
                    axis x line=middle,    % put the x axis in the middle
                    axis y line=middle,    % put the y axis in the middle
                    axis line style={<->}, % arrows on the axis
                    xlabel={$x$},          % default put x on x-axis
                    ylabel={$y$},          % default put y on y-axis
                    ticks=none,
                    }}
%\usepackage[urw-garamond]{mathdesign}
%\usepackage{cmbright}
%\usepackage{concmath}
%\usepackage{sansmathfonts}
%\renewcommand*\familydefault{\sfdefault} %% Only if the base font of the document is to be sans serif

%\usepackage{pdfrender,xcolor,scrpage2}
%\pdfrender{StrokeColor=black,TextRenderingMode=2,LineWidth=1pt}
\tikzset{
  subseteq/.style={
    draw=none,
    edge node={node [sloped, allow upside down, auto=false]{$\subseteq$}}
    },
  Subseteq/.style={
    draw=none,
    every to/.append style={
      edge node={node [sloped, allow upside down, auto=false]{$\subseteq$}}}
    },
    Subsetneq/.style={
    draw=none,
    every to/.append style={
      edge node={node [sloped, allow upside down, auto=false]{$\subsetneq$}}}
    },
  Supseteq/.style={
    draw=none,
    every to/.append style={
      edge node={node [sloped, allow upside down, auto=false]{$\supseteq$}}}
  }
}

\hypersetup{
    colorlinks,
    citecolor=blue,
    filecolor=blue,
    linkcolor=blue,
    urlcolor=blue
}
\newtheorem{thm}{Theorem}[section]
\newtheorem{prop}[thm]{Proposition}
\newtheorem{lemma}[thm]{Lemma}
\newtheorem{cor}[thm]{Corollary}
\newtheorem{dfn}[thm]{Definition}
\newtheorem{axiom}[thm]{Axiom}
\newtheorem{rmk}[thm]{Remark}
\newtheorem{rmkt}[thm]{Remark by TeXer}
\newtheorem{ex}[thm]{Example}
\newtheorem{nex}[thm]{Non-example}
\newtheorem{exercise}[thm]{Exercise}
\newtheorem{question}[thm]{Question}
\newtheorem{problem}[thm]{Problem}
\newtheorem{dfn/thm}[thm]{Definition/Theorem}
\renewcommand{\baselinestretch}{1.1}
\newcommand{\mor}{{\text Mor\,}}
\renewcommand{\hom}{{\text Hom}}
\newcommand{\reals}{\mathbb R}
\newcommand{\cplx}{\mathbb C}
\newcommand{\intg}{\mathbb Z}
\newcommand{\bbf}{\mathbb F}
\newcommand{\ratl}{\mathbb Q}
\newcommand{\torus}{\mathbb T}
\newcommand{\sca}{{\mathfrak a}}
\newcommand{\scb}{{\mathfrak b}}
\newcommand{\scc}{{\mathfrak c}}
\newcommand{\scm}{{\mathfrak m}}
\newcommand{\scn}{{\mathfrak n}}
\newcommand{\scp}{{\mathfrak p}}
\newcommand{\scq}{\mathfrak q}
\newcommand{\frakg}{{\mathfrak g}}
\newcommand{\frakd}{{\mathfrak d}}
\newcommand{\pd}{{\partial}}
\newcommand{\calf}{{\cal F}}
\newcommand{\calg}{{\cal G}}
\newcommand{\cala}{{\cal A}}
\newcommand{\calb}{{\cal B}}
\newcommand{\calc}{{\cal C}}
\newcommand{\cald}{{\cal D}}
\newcommand{\cale}{{\cal E}}
\newcommand{\cali}{{\cal I}}
\newcommand{\call}{{\cal L}}
\newcommand{\calm}{{\cal M}}
\newcommand{\caln}{{\cal N}}
\newcommand{\calo}{{\cal O}}
\newcommand{\calr}{{\cal R}}
\newcommand{\mathbold}{\bf}
\newcommand{\cinf}{C^{\infty}}
\newcommand{\row}[2]{#1_1,\dots ,#1_{#2}}
\newcommand{\dbyd}[2]{{\partial #1\over\partial #2}}
\newcommand{\Space}{{\bf Space}}
\newcommand{\alg}{{\mathbold Alg}}
\newcommand{\notsubset}{\not \subset}
\newcommand{\notsupset}{\not \supset}
\newcommand{\pois}{{\mathbold Pois}}
\newcommand{\pitilde}{\tilde{\pi}}
\newcommand{\rta}{\rightarrow}
\newcommand{\llta}{\longleftarrow}
\newcommand{\Lrta}{\Longrightarrow}
\newcommand{\lrta}{\longrightarrow}
\newcommand{\llrta}{\longleftrightarrow}
\newcommand{\Llta}{\Longleftarrow}
\newcommand{\Llrta}{\Longleftrightarrow}
\newcommand{\lgl}{\langle}
\newcommand{\rgl}{\rangle}
\newcommand{\inj}{\hookrightarrow}
\newcommand{\surj}{\twoheadrightarrow}
\newcommand{\cmark}{\ding{51}}%
\newcommand{\xmark}{\ding{55}}%
\newcommand{\downmapsto}{\rotatebox[origin=c]{-90}{$\scriptstyle\mapsto$}\mkern2mu}
\renewcommand{\qedsymbol}{$\square$}
\bibliographystyle{plain}
\title{\bf Notes of B-V formalism in derived settings}
\author{\\
Notes by Lind Axiao}
\date{2018 ETH} %\thanks{Research partially supported by NSF Grant DMS-96-25122 and the Miller Institute for Basic Research in Science.}
\begin{document}
\maketitle
\tableofcontents
\newpage

\section{Recap on BV formalism}
$$
\int_X e^{i S_0(X)/\hbar} f(x) dx
$$
If $S_0$ is a Morse function on a finite dimensional manifold.

But usually we would have to work on infinite dimensional space.

\underline{idea}: 
\begin{enumerate}
\item embed $X$ into a larger graded manifold $V$ and extend $S_0(X)$ to a function on $S(x)$ on $V$ and express the initial integral as
$$
\int_{V\subset T^*[-1]V} e^{i S(y)/\hbar}f(y) dy
$$
then deform $V$ as a Lagrangian inside the odd cotangent bundle. In order to make the integral invariant, the new $S$ has to satisfies the quantum master equation $QME$. At the oder $\hbar=0$, QME reduces to the classical master equation
$$
[S_{0},S_{0}]=0
$$
\end{enumerate}

We first given a heuristic version of BV-formalism of quantum field theory on ``points'', where the moduli space is finite dimensional.

Let $M$ be a finite dimensional smooth manifold or affine variety.
Let $S:M\lrta \mathbb{A}^1$ be a smooth function. The critical locus of $S$
$$
Crit(S)=graph(dS)\times_{T^*M}M
$$
the fibered product
$$
\begin{tikzcd}
Crit(S) \arrow[d] \arrow[r] & M \arrow[d, "dS_0"] \\
M \arrow[r, "zero\ sections"] & T^*M
\end{tikzcd}
$$
is the intersection of graph of $dS$ and the zero section inside $T^*M$. 

Sometimes this intersection could be non-transitive, we want to define a derived version.

Traditionally, the BV-BRST complex of Lagrangian field theory is obtained in three steps
\begin{enumerate}
\item Find a Koszul-Tate complex to resolve the critical locus;
\item find a BRST complex to encode the gauge invariance
\item apply the homological perturbation theory to find a unified BV-differential.
$$
s_{BV}=s_{KT}+s_{BRST}+...
$$
\end{enumerate}
Choosing the derived critical locus is equivalent to inverting the Koszul-Tate resolution.



\section{Derived functor, homotopy pushout}
We skip here the introduction of model category but only remember that given a model category $\calc$, the homotopy category $\gamma:\calc\lrta Ho(\calc)$ exists, which is the localization of $\calc$ w.r.t the weak equivalence. Any functor $G:\calc\lrta \calb$ which sends weak equivalences to isomorphisms would factor through $\gamma$.

Given three categories, $\cala,\calb,\calc$ and two functors $X,F$. 
$$
\begin{tikzcd}
\cala \arrow[d, "X"'] \arrow[r, "F"] & \calc \\
\calb \arrow[ru, "R"', dashed] & 
\end{tikzcd}
$$
The \textbf{right Kan extension of $X$ along $F$} consists of a functor $R:\calb\lrta \calc$ and a natural transformation $\eta:RF\lrta X$ which is \textbf{couniversal} with respect to the specification, in the sense that for any functor $M:\calb\lrta \calc$ and a natural transformation $\mu: MF\lrta X$, a unique natural transformation $\delta: M\lrta R$ is defined and the diagram of functors commutes
$$
\begin{tikzcd}
RF \arrow[d, "\eta"'] &  \\
X & MF \arrow[l, "\mu"] \arrow[lu, "\delta_F"', dashed]
\end{tikzcd}
$$
where $\delta_F(a)=\delta(F(a))\lrta RF(a)$ for any object $a$ of $\cala$.

Similarly, we have a dual notion of \textbf{left Kan extension}.


\begin{dfn}
Let $\calc$ be a model category and let $F:\calc\lrta \calb$ by any functor. We call the right Kan extension of $F$ along $\gamma:\calc\lrta Ho(\calc)$ the left derived functor of $F$. We will denote it by $(\mathbf{L}F,\eta)$, where $\eta$ is a defining natural transformation in Kan extension.

Dually, the left Kan extension of $F$ along $\gamma:\calc\lrta Ho(\calc)$ the right derived functor of $F$. 
\end{dfn}

In the case $F=(co)lim:\calc^\cald\lrta \calc$, where $\cald$ is a diagram. We can define the homotopy limits and homotopy colimits:
$\mathbf{R}lim$ and $\mathbf{L}colim$.

It would be long story to introduce the model structure on $\calc^{\cald}$, which we suppress here.

If $\cald$ is chosen to be 
$$
\begin{tikzcd}
\bullet \arrow[r] & \bullet & \bullet \arrow[l],
\end{tikzcd}
$$
the $\mathbf{R}lim$ is the \textbf{homotopy pullback}

The derived critical locus is now defined to be a homotopy pullback in the category $caga_{\leq0}^{op}$
$$
\begin{tikzcd}
dCrit(S) \arrow[rr] \arrow[dd] &  & M \arrow[dd, "dS"] \\
 & \Swarrow &  \\
M \arrow[rr, "\underline{0}"] &  & T^*M
\end{tikzcd}
$$
The detailed construction is discussed in the next section.

\section{Derived everything}
\subsection{Recap on spaces, functor of point}
First, we recall the definition of a topological manifold. 

A topological variety is an topological space together with an open cover $\{U_i\}_{i\in I}$ such that for each $i\in I$ there exists a homeomorphism from $U_i$ to an open set in $\reals^{n_i}$, each integer $n_i\geq 0$ depends on $i$.

We can give a fancy definition of topological manifolds

Consider the coequilizer
$$
Colim\left(\coprod_{(i,j)\in I^2} U_{i,j}\rightrightarrows \coprod_{i\in I}U_i\right),
$$
where the upper morphism is induced by $U_{i,j}\lrta U_i$ while the second morphism is induced by the morphism $U_{i,j}\lrta U_j$.
The morphism from $\coprod_i U_i\lrta X$ would factorize through 
$$
Colim\left(\coprod_{(i,j)\in I^2}U_{i,j}\rightrightarrows \coprod_{i\in I}U_i\right)\lrta X
$$
\begin{lemma}
The morphism 
$$
Colim\left(\coprod_{(i,j)\in I^2}U_{i,j}\rightrightarrows \coprod_{i\in I}U_i\right)\lrta X
$$
is an isomorphism.
\end{lemma}
\begin{proof}
Consider $Y$ another topological space with a morphism
$$
Colim\left(\coprod_{(i,j)\in I^2}U_{i,j}\rightrightarrows \coprod_{i\in I}U_i\right)\overset{f}{\lrta} Y
$$
$f_i:=f|_{U_{i}}$ and $f_i|_{U_{i,j}}=f_j|_{U_{i,j}}$. We can define an map $g$ from $X$ to $Y$ such that they agree pointwisely. $g|_{U_i}=f_i$. The restriction to each open set in the open cover is continuous, we therefore know $g$ is itself continuous. 

Hence, there is a morphism from $X$ to $Y$ and continuous map $g$ is unique because it has to agree with $f$ pointwisely.

$X$ satisfies the universal property of coequalizer hence is isomorphic to the coequalizer.
\end{proof}

We can consider it in an even fancier way. Use $\calc$ to denote the full subcategory of topological manifold. One consider the Yoneda embedding from the category $\calc$ to the category of presheaves over $\calc$, where $PSh(\calc)$ denote the functor category $[\calc^{op},Sets]$
$$
h_{\_}: TopMfd\lrta PSh(\calc)
$$
$$
X\longmapsto h_X
$$
where $h_X(Y):=\hom_{TopMfd}(Y,X)$ for al $Y\in \calc\subset TopMfd$
\begin{lemma}
The functor $h_\_$ defined above is fully faithful. 
\end{lemma}
\begin{proof}
refer to any proof of Yoneda lemma.
\end{proof}
(This functor is not necessarily essentially surjective)
These lemma means $TopMfd$ is equivalent to a subcategory of presheaves over $\calc$. We all now trying to characterize this subcategory.

We start by making $\calc$ a Grothendieck site.
\begin{dfn}
We can specify that certain collections of maps with a common codomain should cover their codomain. A family of morphisms \{$U_i\lrta U\}_{i\in I}$ is called a covering family is each morphism $U_i\lrta U$ is an open immersion and the induced morphism on the coproduct $\coprod_{i\in I}U_i\lrta U$ is surjective. This definition gives the neighborhood system for a pretopology (\textbf{Grothendieck pretopology}), we denote the associated topology $\tau$
\end{dfn}
\begin{lemma}
For every $X\in TopMfd$, the presheaf $h_X\in PSh(\calc)$ is a sheaf with the specified topology $\tau$
\end{lemma}

\begin{dfn}
We say a functor $F:\calc\lrta Sets $ is \textbf{representable} if it is naturally isomorphic to $h_X$ for some object $X\in\calc$.

A morphism $f:F\lrta G$  of $Sh(\calc,\tau)$ is a local homeomorphism if for each $X\in \calc$ and all morphism $h_X\lrta G$, the sheaf $F\times_G\times h_X$ is representable by $Y\in TopMfd$, and the induced morphism $Y\lrta X$ by projection $F\times_G h_X\cong h_Y\lrta X$ is a local homeomorphism of topological spaces.

A morphism of sheaves $Sh(\calc,\tau)$ is an \textbf{open immersion} if it is a monomorphism and a local homeomorphism.
\end{dfn}

\begin{prop}
A sheaf $F\in Sh(\calc,\tau)$ is representable by a topological manifold if there exists a family of objects $\{U_i\}_{i\in I}$ in $\calc$ and a morphism of sheaves
$$
p:\coprod_{i\in I}h_{U_i}\lrta F
$$
such that the following two conditions holds
\begin{enumerate}
\item $p$ is an epimorphism.
\item For all $i\in I$, the morphism $U_i\lrta F$ is an open immersion.
\end{enumerate}
\end{prop}

Generally, we can identifies the category of $TopMfd$ with its image in $Sh(\calc,\tau)$

In context of algebraic geometry, things are more famous.

Schemes can be characterized as representable sheaves $Sh(Aff,\tau)$, where $\tau$ is the canonical Grothendieck topology.



\begin{dfn}
A \textbf{derived scheme} is a pair $(X,\calo)$ consisting of topological space and a sheaf $\calo$ of commutative ring spectra on $X$ such that the
\begin{enumerate}
\item pair $(X,\pi_0\calo)$ is a scheme and 
\item each $\pi_k\calo$ is a quasi-coherent $\pi_0\calo$-module.
\end{enumerate}
\end{dfn}

From the homotopical point of view, we note that a derived scheme $X$ defines a functor
$$
h_X: dAff^{op}=cdga_{\leq 0}\lrta Sets
$$
Furthermore,  we have the following lemma
\begin{lemma}
$h_X$ sends each quasi-isomorphism of $cdga_{\leq 0}$ to an isomorphism in $Sets$.
\end{lemma}

Recall the model structure on $cdga_{\leq}$, the weak equivalence are just the quasi-isomorphisms, i.e., the functor $h_X$ factors through the homotopy category $Ho(cdga_{\leq 0})$. Following the spirit of functor of points, we can regard $X$ as 
a locally representable sheaf in $Sh(Ho(cdga_{\leq 0}))$.
\subsection{Stacks, derived stacks}
Roughly speaking, a Stack is a sheaf that takes values in categories rather than sets.
\begin{dfn}
A category $\calb$ with a functor $F$ to a category $\calc$ is called a \textbf{fibered category over $\calc$} if for any morphism $G:X\lrta Y$ in $\calc$ and any object $y\in\calb$ s.t. $F(y)=Y$, there is a poullback $g: x\lrta y$ of $y$ by $F$, i.e. $F(g)=G$.
\end{dfn}

\begin{dfn}
The category $\calb$ is called a \textbf{prestack} over a category $\calc$ with a Grothendieck topology if it is fibered over $\calc$ and 

for any object $U\in\calc$ and object $x,y\in\calb$ with image $U$, the functor from objects over $U$ to sets taking $[F:V\lrta U]$ to $\hom(F^*x, F^*y)$ is a sheaf.

The category $\calb$ is called a \textbf{stack} over the category $\calc$ with a Grothendieck topology if it is a prestack over $\calc$ and every descent datum is effective.

 A \textbf{descent datum} consists roughly of a covering of an object $V$ of $C$ by family $V_i$, elements $x_i$ in the fiber over $V_i$ and morphism $f_{ji}$ between the restrictions of $x_i$ and $x_j$ to $V_{ij}=V_i\times_V V_{j}$ satisfying the compatibility condition $f_{ki}=f_{kj}f_{ji}$. The descent datum is called effective if the elements $x_i$ are essentially the pullbacks of an element $x$ with image $V$.
\end{dfn}
\textbf{The descent condition here is just a derived version of the usual sheaf axioms.} The fiber functor $F:\calb\lrta \calc$ can be regarded a sheaf on $\calb$ with value in $\calc$.

For example if we tak $\calc=Grpds$ the stack is called $1$-stack.

We will jump through the story of $n$-stacks and go directly to $\infty-stacks$.

\subsection{Derived critical locus}
$$
\begin{tikzcd}
dCrit(S) \arrow[rr] \arrow[dd] &  & M \arrow[dd, "dS"] \\
 & \Swarrow &  \\
M \arrow[rr, "\underline{0}"] &  & T^*M
\end{tikzcd}
$$
Mostly, it would be easier to analyze it in terms of functions on it. We will mostly work in the affine case only to convince ourselves.

Let $R$ be a commutative $k$-algebra, and $P$ a projective $R$-module of finite type. Let $S:=Sym_R(P^\vee)$ the symmetric algebra on on the $R$-dual $P^\vee$. $S$ is a commutative $R$-algebra. Consider $\wedge^\bullet P^\vee$ be the exterior algebra of $P^\vee$ as an $R$-module and we can construct a non-positively graded $S$-module $S\otimes_R\wedge^\bullet P^\vee$ graded by 
$$
(S\otimes_R \wedge^\bullet P^\vee)_m:=S\otimes_R\wedge^{-m}P^\vee
$$
.
This $S\otimes_{R}P^\vee$ is naturally a graded commutative $S$-algebra and can be endowed with a degree $1$ differential $d$.
The differential $d$ is induced by a homomorphism
$$
\begin{aligned}
h:& R\lrta Hom_R(P,P)\cong P^\vee\otimes P\\
& 1_R\longmapsto \sum_i{\alpha_i}\otimes x_i
\end{aligned}
$$
$$
d(a\otimes(\beta\wedge\cdots\wedge \beta_{n+1}))=\sum_{j} a\cdot \alpha_{j}\otimes \sum_{k}(-1)^k\beta_k(x_j)(\beta_{1}\wedge\cdots \wedge \hat{\beta}_k\wedge \cdots \wedge \beta_{n+1})
$$
 $(S\otimes_R,d)$ is commutative differential non-positively graded algebra over $S$. We call it Koszul cdga and denote it by $K(R;P)$.
\begin{prop}
The cohomology of Koszul cdga $K(R;P)$  is zero in degrees $\leq 0$, and $H^0(K(R;P))\cong R$. 
\end{prop}
\begin{proof}
See for example \href{https://math.berkeley.edu/~ogus/Math_250B-2016/Notes/koszul.pdf}{this notes}
\end{proof}

\section{The BRST on $dCrit(S)$ $=$ the BV-BRST on $Crit(S)$}

\end{document}