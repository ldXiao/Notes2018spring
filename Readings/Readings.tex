\documentclass[11pt]{article}
\usepackage{amssymb}
\usepackage{latexsym}
\usepackage{amsmath}
\usepackage{amsthm}
\usepackage{stmaryrd}
\usepackage{fancyhdr}
\pagestyle{headings}
\usepackage{dsfont}
\usepackage{pifont}
\usepackage{mathtools}
\usepackage{natbib}
\usepackage{tikz-cd}
\usepackage{pgfplots}
\usepackage{enumitem} 
\usepackage{hyperref}
\usepackage{geometry}
\geometry{left=4cm,right=4cm}
\pgfplotsset{every axis/.append style={
                    axis x line=middle,    % put the x axis in the middle
                    axis y line=middle,    % put the y axis in the middle
                    axis line style={<->}, % arrows on the axis
                    xlabel={$x$},          % default put x on x-axis
                    ylabel={$y$},          % default put y on y-axis
                    ticks=none,
                    }}
%\usepackage[urw-garamond]{mathdesign}
%\usepackage{cmbright}
%\usepackage{concmath}
%\usepackage{sansmathfonts}
%\renewcommand*\familydefault{\sfdefault} %% Only if the base font of the document is to be sans serif

%\usepackage{pdfrender,xcolor,scrpage2}
%\pdfrender{StrokeColor=black,TextRenderingMode=2,LineWidth=1pt}
\tikzset{
  subseteq/.style={
    draw=none,
    edge node={node [sloped, allow upside down, auto=false]{$\subseteq$}}
    },
  Subseteq/.style={
    draw=none,
    every to/.append style={
      edge node={node [sloped, allow upside down, auto=false]{$\subseteq$}}}
    },
    Subsetneq/.style={
    draw=none,
    every to/.append style={
      edge node={node [sloped, allow upside down, auto=false]{$\subsetneq$}}}
    },
  Supseteq/.style={
    draw=none,
    every to/.append style={
      edge node={node [sloped, allow upside down, auto=false]{$\supseteq$}}}
  }
}

\hypersetup{
    colorlinks,
    citecolor=blue,
    filecolor=blue,
    linkcolor=blue,
    urlcolor=blue
}
\newtheorem{thm}{Theorem}[section]
\newtheorem{prop}[thm]{Proposition}
\newtheorem{lemma}[thm]{Lemma}
\newtheorem{cor}[thm]{Corollary}
\newtheorem{dfn}[thm]{Definition}
\newtheorem{axiom}[thm]{Axiom}
\newtheorem{rmk}[thm]{Remark}
\newtheorem{rmkt}[thm]{Remark by TeXer}
\newtheorem{ex}[thm]{Example}
\newtheorem{nex}[thm]{Non-example}
\newtheorem{exercise}[thm]{Exercise}
\newtheorem{question}[thm]{Question}
\newtheorem{problem}[thm]{Problem}
\newtheorem{dfn/thm}[thm]{Definition/Theorem}
\renewcommand{\baselinestretch}{1.1}
\newcommand{\mor}{{\text Mor\,}}
\newcommand{\reals}{\mathbb R}
\newcommand{\cplx}{\mathbb C}
\newcommand{\intg}{\mathbb Z}
\newcommand{\bbf}{\mathbb F}
\newcommand{\ratl}{\mathbb Q}
\newcommand{\torus}{\mathbb T}
\newcommand{\sca}{{\mathfrak a}}
\newcommand{\scb}{{\mathfrak b}}
\newcommand{\scc}{{\mathfrak c}}
\newcommand{\scm}{{\mathfrak m}}
\newcommand{\scn}{{\mathfrak n}}
\newcommand{\scp}{{\mathfrak p}}
\newcommand{\scq}{\mathfrak q}
\newcommand{\frakg}{{\mathfrak g}}
\newcommand{\frakd}{{\mathfrak d}}
\newcommand{\pd}{{\partial}}
\newcommand{\calf}{{\cal F}}
\newcommand{\calg}{{\cal G}}
\newcommand{\cala}{{\cal A}}
\newcommand{\calb}{{\cal B}}
\newcommand{\calc}{{\cal C}}
\newcommand{\cale}{{\cal E}}
\newcommand{\cali}{{\cal I}}
\newcommand{\call}{{\cal L}}
\newcommand{\caln}{{\cal N}}
\newcommand{\calo}{{\cal O}}
\newcommand{\calr}{{\cal R}}
\newcommand{\mathbold}{\bf}
\newcommand{\cinf}{C^{\infty}}
\newcommand{\row}[2]{#1_1,\dots ,#1_{#2}}
\newcommand{\dbyd}[2]{{\partial #1\over\partial #2}}
\newcommand{\Space}{{\bf Space}}
\newcommand{\alg}{{\mathbold Alg}}
\newcommand{\notsubset}{\not \subset}
\newcommand{\notsupset}{\not \supset}
\newcommand{\pois}{{\mathbold Pois}}
\newcommand{\pitilde}{\tilde{\pi}}
\newcommand{\rta}{\rightarrow}
\newcommand{\llta}{\longleftarrow}
\newcommand{\Lrta}{\Longrightarrow}
\newcommand{\lrta}{\longrightarrow}
\newcommand{\llrta}{\longleftrightarrow}
\newcommand{\Llta}{\Longleftarrow}
\newcommand{\Llrta}{\Longleftrightarrow}
\newcommand{\lgl}{\langle}
\newcommand{\rgl}{\rangle}
\newcommand{\inj}{\hookrightarrow}
\newcommand{\surj}{\twoheadrightarrow}
\newcommand{\cmark}{\ding{51}}%
\newcommand{\xmark}{\ding{55}}%
\newcommand{\downmapsto}{\rotatebox[origin=c]{-90}{$\scriptstyle\mapsto$}\mkern2mu}
\renewcommand{\qedsymbol}{$\square$}
\bibliographystyle{plain}
\title{\bf Notes of Readings in Topology and Geometry}
\author{\\
Notes by Lin-Da Xiao}
\date{2018 ETH} %\thanks{Research partially supported by NSF Grant DMS-96-25122 and the Miller Institute for Basic Research in Science.}
\begin{document}
\maketitle
\tableofcontents
\newpage

\section*{About the Notes:}
This notes is a summary of personal reading  of Topology and geometry.
\section{Complex manifold}
\subsection{Complex structure, almost complex structure}
\begin{dfn}
A complex valued function $f:\cplx^m\lrta\cplx$ is \textbf{holomorphic} if  $f=f_1+i f_2$ it satisfies the \textbf{Cauchy-Riemann relations} for $z^\mu=x^\mu+i y^\mu$,
$$
\frac{\pd f_1}{\pd x^\mu}=\frac{\pd f_2}{\pd y^{\mu}}
$$
$$
\frac{\pd f_2}{\pd x^\mu}=-\frac{\pd f_1}{\pd y^{\mu}}
$$
\end{dfn}

\section{Spectral sequences}
\begin{dfn}
An \textbf{exact couple} is an exact sequence of Abelian groups of the form
\begin{center}
\begin{tikzcd}
A \arrow[rr, "i"] &  & A \arrow[ld, "j"] \\
 & B \arrow[lu, "k"] & 
\end{tikzcd}
\end{center}
where $i,j,k$ are group homomorphisms. Define $d:B\lrta B$ by $d=j\circ k$. Then $d^2=j(jk)k=0$, so the homology group $H(B)=ker(d)/im(d)$ is wel-defined Abelian group. 
\end{dfn}
Out of the exact couple, we can construct a \textbf{derived couple}
\begin{center}
\begin{tikzcd}
A' \arrow[rr, "i'"] &  & A' \arrow[ld, "j'"] \\
 & B' \arrow[lu, "k'"] & 
\end{tikzcd}
\end{center}
by setting
\begin{enumerate}[label=(\roman*)]
\item $A':=i(A)$; $B':=H(B)$.
\item $i'$ induced from $i$, i.e., $i'(i(a))=i(i(a))$
\item If $a'=i(a)\in A'$ with $a\in A$, then $j'(a'):=[j(a)]\in H(B)$.
\item $k'$ is induced from $k$. Consider a comology calss $[b]$, $d b=0\Llrta jkb=0$, then $kb\in i(A)$. Define $k'[b]:=kb\in i(A)$.
\end{enumerate}
\begin{proof}
We can check that $j'$ is wel-defined : $ia=i a_1\Lrta [j(a)]=[j(a_1)]$. Indeed $i(a-a_1)=0\Lrta (a-a_1)\in im(k)$. $\exists z\in B$ s.t. $k(z)=a-a_1\Lrta [j(a-a_1)]=[jk(z)]=[dz]$. Also $k'$ is well defined: $[b]=[b_1]\Lrta b-b_1=d z\Lrta k(b-b_1)=k j kz=0$.

The derived sequence is indeed exact:

{Exactness at $\overset{k'}{\lrta}A'\overset{i'}{\lrta}$}: $i'\circ k'[b]=i'kb=ikb=0$, we know $ker(i')\supseteq im(k')$. For the reverse inclusion, $i(a)\in ker(i')\Lrta iia=0$ then $ia\in im(k)$ because the original exact couple is exact. $ia=kb\Lrta k'[b]$, hence $im(k')\supseteq ker(i')$.

Exactness at $\overset{i'}{\lrta}A'\overset{j'}{\lrta}$: $j'\circ i'(ia)=j'(iia)=[jia]=0$. For the reverse inclusion, consider $ia\in ker(j'), j'(ia)=[ja]=0\Lrta ja=db\in B\Lrta j(a-kb)=0\Lrta a-kb=i(a_1)\Lrta i(a-kb)=ia=ii(a_1)=i'(ia_1)$.

Exactness at $B'$: $k'j'(ia)=k'[ja]=kja=0\Lrta ker(k')\supseteq im(j')$. 
For the reverse inclusion, we can pick $[b]\in ker(k')\Lrta k'[b]=kb=0\Lrta b=ja$ for some $a\in A$. $[b]=j'(ia)\in im(j')$.

\end{proof}

\section{Fundamental groups}
\begin{dfn}
Let $X$ and $Y$ be topological spaces and $f, g:X\lrta Y$ continuous maps. A \textbf{homotopy} from $f$ to $g$ is a continuous map
$$
H:X\times [0,1]\lrta Y, (x,t)\longmapsto H(x,t)=H_t(x)
$$
such that $f(x)=H(x,0)$ and $g(x)=H(x,1)$ $\forall x\in X$. $f=H_0$ and $g=H_1$, $f\simeq g$
\end{dfn}
The homotopy relation is an equivalence relation on the set o f continuous maps $X\lrta Y$. Given two homotopy $K:f\simeq g$ and $L:g\simeq h$, the product homotopy $K*L$
$$
(K*L)(x,t)=\left\{
\begin{matrix*}
K(x,2t), & 0\leq t\leq 1/2,\\
L(x,2t-1), & 1/2\leq t\leq 1,
\end{matrix*}
\right.
$$
and shows $f\simeq h$.

The inverse homotopy is defined to be $H^{-}(x,t):=H(x,1-t)$. Notice that product of homotopy and inverse homotopy is not constant homotopy.

The equivalence class of $f$ is denoted $[f]$   and called the homotopy class of $f$ . We denote by $[X,Y]$  the set of homotopy classes $[f]$  of maps $f : X \lrta Y$. A homotopy $H_t:  X \lrta  Y$ is said to be relative to $A\subset X$   if the restriction $H_t|_A $ does not dependont(is constant on $A$).We use the notation $H:f \lrta g(rel A)$in this case.

Quotient category means we identify some of the morphism. For each $\mor (X,Y)$, we quotient a  relation $R_{X,Y}$.
\begin{dfn}
Topological spaces and homotopy classes of maps form a quotient category of \textit{Top}, which is called \textbf{homotopy category}, denoted \textit{h-Top}. The composition of homotopy class is induced by composition of representing maps. The isomorphism in this category is homotopy equivalence.
\end{dfn}

In the category of \textit{h-Top}. Consider the $Hom$-functors. Given $f:X\lrta Y$.
$$
Hom(Z,\_)(f)=f_*:[Z,X]\lrta [Z,Y]: g\longmapsto f g
$$
$$
Hom(\_,Z)(f)=f^*:[Z,X]\lrta [Z,Y]: h\longmapsto  hf
$$
$f$ is h-equivalence (isomorphism in the category \textit{h-Top}) iff $Hom(\_,Z)(f)$ is always bijective. (Yoneda Lemma), similarly for $Hom(Z,\_)(f)$. Because we know for $f_*,g_*,g_*f_*$, 2 of the three maps are bijective implies that the third is bijective. This can be translated into homotopy category, where $f,g,fg$ two of the three homotopy class being homotopy equivalence implies the third is also a homotopy equivalence.

Let $P$ be a point. A map $P\lrta Y$ can be identified as its image and a homotopy can be identified with path. Then the Hom-functor $[P,\_]$ can be identified as $\pi_0$

\begin{prop}
The product of paths has the following properties:
\begin{enumerate}[label=(\roman*)]
\item Let $\alpha:I\lrta I$ be continuous and $\alpha(0)=0$, $\alpha(1)=1$. Then $u\simeq u\alpha$.
\item $u_1*(u_2*u_3)=(u_1*u_2)*u_3$
\item $u_1\simeq u_1'$ and $u_2\simeq u_2'$ implies $u_1*u_2\simeq u_1'*u_2'$.
\item $u*u^-$ is always defined and homotopic to the constant path.
\item $k_{u(0)}*u\simeq u\simeq u*k_{u(1)}$.
\end{enumerate}
\begin{proof}
\begin{enumerate}[label=(\roman*)]
\item Let $H:(s,t)\mapsto u(s(1-t)+t\alpha(s))$ is homotopy from $u$ to $u\alpha$.
\item choose 
$$
\alpha(t)=\left\{
\begin{matrix*}
2t, & t\leq \frac{1}{4}\\
t+\frac{1}{4} & \frac{1}{4}\leq t\leq \frac{1}{2}\\
\frac{t+1}{2}, & \frac{1}{2}\leq t\leq 1
\end{matrix*}
\right.
$$
we have $u_1*(u_2*u_3)\alpha=(u_1*u_2)*u_3$, then we can apply (i)
\item Given $F_i:u_i\simeq U_i'$, then we can define the homotopy $G:u_1*u_2\simeq u_1'u_2'$ 
$$
G(s,t)=\left\{
\begin{matrix*}
F_1(2s,t), & 0\leq t\leq \frac{1}{2}\\
F_2(2s-1,t) & \frac{1}{2}\leq t\leq 1\\
\end{matrix*}
\right.
$$
\item The map $F: I \times  I \lrta X$ defined as 
$$
F(s,t)=\left\{
\begin{matrix*}
u(2s(1-t)), & 0\leq t\leq \frac{1}{2}\\
u(2(1-s)(1-t)) & \frac{1}{2}\leq t\leq 1\\
\end{matrix*}
\right.
$$
is the homotopy from $u*u^-$ to the constant path.
\item use (i) again.
\end{enumerate}
\end{proof}
\end{prop}

This basically says that the homotopy class of path with a fixed point is a group.

From homotopy classes of paths in $X$, we obtain again a category denote $\Pi(X)$. The objects are the points of $X$. A morphism from $x$ to $y$ is a homotopy class of paths. It is called \textbf{Fundamental groupoid} of $X$. The automorphism group of the object $x$ in this category is the fundamental group of $X$ with base point $x$.

\begin{prop}
Let $H : X  \times I \lrta Y$ be a homotopy from $f$ to $g$. Each $x \in X$ yields the path $H^x:t\mapsto H(x,t)$ and the morphism $[H^x]$ in $\Pi(Y)$ from $f(x)$ to $g(x)$. The $H^x$ constitute a natural transformation $\Pi(H)$
\end{prop}
\begin{prop}
 Let $f : X \lrta Y$ be a homotopy equivalence. Then the functor $\Pi(f):\Pi(X)\lrta \Pi(Y)$ is an equivalence of categories. The induced maps between morphism sets $f_*:\Pi X(x,y)\lrta \Pi Y(fx,fy)$ are bijections. IN particular,
 $$
 \pi_1(f):\pi_1(X,x)\lrta \pi_1(Y,f(x)), [\omega]\mapsto [f\omega]
 $$
 is an isomorphism for each $x\in X$
\end{prop}

\begin{thm}
(R. Brown). Let $X_0$ and $X_1$ be a subspace of $X$ such that the interiors cover $X$. Let $i_\nu:X_{01}\inj X_\nu$ and $j_\nu:X_\nu\inj X$ be the inclusions. Then 
\[
\begin{tikzcd}
\Pi(X_{01}) \arrow[r, "\Pi(i_0)"] \arrow[d, "\Pi(i_1)"'] & \Pi(X_0) \arrow[d, "\Pi(j_0)"] \\
\Pi(X_1) \arrow[r, "\Pi(j_1)"] & \Pi(X)
\end{tikzcd}
\]
 is a pushout (fibered coproduct) in the category of groupoids. Or $\Pi(X)$  is the colimit of diagrams indexed by $X_0\supset X_{01}\subset X_1$
 \end{thm}


\subsection{Spectral sequences of filtered complexes}
Now, we want to construct the spectral sequence of a filtered complex.
\begin{dfn}
Let $K$ be differential complex with differential operator $D$. A subcomplex $K'$ is a subgroup in $K$ such that $DK'\subseteq K'$. A sequence of subcomplex 
$$K=K_0\supseteq K_1\supseteq K_2...$$
is called a filtration of $K$. A differential complex with specified filtration is called a filtered complex with  
\textbf{associated graded complex}
$$
Gr_{\bullet} K=\oplus^\infty_{p=0} K_p/K_{p+1}
$$
\end{dfn} 

For filtered complex $K$, let $A$ be the group
$$
A=\oplus_{p\in\intg} K_p.
$$
$A$ is again a differential complex with differential operator $\oplus D$. Define $\iota$ to be the inclusion $A\inj A$ induced by $K_{p+1}\inj K_p$. Then we have a short exact sequence
$$
0\lrta A\overset{\iota}{\lrta} A\overset{j}{\lrta} B:=Gr_\bullet K\lrta 0.
$$
If $A$, $K$ are themselves graded chain complex (A different grading from the associated grading w.r.t. to filtration, we use upper index to distinguish it from the filtration index), we have a long exact sequence of cohomology groups
$$
\lrta H^k(A^{\bullet})\overset{i}{\lrta} H^{k}(A)\overset{j_1}{\lrta}H^{k}(B)\overset{k_1}{\lrta}H^{k+1}(A)\lrta..
$$
Consider that $H(A)=\oplus_{k}H^{k}(A)$, we have the exact couple
\begin{center}
\begin{tikzcd}
H(A) \arrow[rr, "i"] &  & H(A) \arrow[ld, "j_1"] \\
 & H(B) \arrow[lu, "k_1"] & 
\end{tikzcd}:=
\begin{tikzcd}
A_1 \arrow[rr, "i"] &  & A_1 \arrow[ld, "j_1"] \\
 & B_1 \arrow[lu, "k_1"] & 
\end{tikzcd}
\end{center}
From now on we will suppress the subscript of $i_n$ because by definition, $i_n(i_{n-1}...(i(a)))=i^n(a)$. Even if they are not graded, we can still artificially construct the short exact sequence of chain complex
\begin{center}
\begin{tikzcd}
0\ar[r] & A \arrow[r, "i"]\arrow[d, "D"] & A\arrow[d, "D"]\arrow[r, "j"] & B \arrow[d, "D"]\ar[r] & 0
\\
0\ar[r] & A \arrow[r, "i"]\arrow[d, "D"] & A\arrow[d, "D"]\arrow[r, "j"] & B \arrow[d, "D"]\ar[r] & 0
\\
0\ar[r] & \vdots \arrow[r, "i"] & \vdots\arrow[r, "j"] & \vdots \ar[r] & 0
\end{tikzcd}
\end{center}
And it still gives the above exact couple. 

Then we have all the derived exact couples and label it  with $r$ meaning it is the $r$-th derived exact couple of the first one:
\begin{center}
\begin{tikzcd}
A_r \arrow[rr, "i"] &  & A_r \arrow[ld, "j_r"] \\
 & B_r \arrow[lu, "k_r"] & 
\end{tikzcd}.
\end{center}
Consider the special case where 
the filtration terminates after $K_3$.
$$
K_{-1}=K_0\supseteq K_1\supseteq K_2\supseteq K_3\supseteq 0
$$
\begin{center}
\begin{tikzcd}
A_1:= & H(K_0)\ar[d, equal] \arrow[r, phantom, "\oplus"] &  H(K_1)\ar[l, bend right, "i"'] \arrow[r, phantom, "\oplus"] &  H(K_2) \ar[l, bend right, "i"'] \arrow[r, phantom, "\oplus"] &  H(K_3) \ar[l, bend right, "i"']\\
A_2:=i(A_1)= & i H(K_0)\ar[d, equal] \arrow[r, phantom, "\oplus"] &  iH(K_1)\ar[l, bend right, "\supseteq"']\ar[d, equal] \arrow[r, phantom, "\oplus"] &  iH(K_2)\ar[l, bend right, "i"'] \arrow[r, phantom, "\oplus"] &  iH(K_3)\ar[l, bend right, "i"'] \\
A_3:=i(A_2)= & i^2 H(K_0)\ar[d, equal] \arrow[r, phantom, "\oplus"] &  i^2H(K_1)\ar[d, equal]\ar[l, bend right, "\supseteq"'] \arrow[r, phantom, "\oplus"] &  i^2H(K_2)\ar[d, equal]\ar[l, bend right, "\supseteq"'] \arrow[r, phantom, "\oplus"] &  i^2H(K_3)\ar[l, bend right, "i"'] \\
A_4:=i(A_3)= & i^3 H(K_0) \arrow[r, phantom, "\oplus"] &  i^3H(K_1)\ar[l, bend right, "\supseteq"'] \arrow[r, phantom, "\oplus"] &  i^3H(K_2)\ar[l, bend right, "\supseteq"'] \arrow[r, phantom, "\oplus"] &  i^3H(K_3)\ar[l, bend right, "\supseteq"']. 
\end{tikzcd}
\end{center}
Because $iH(K_1)\subseteq H(K_0)$ and $i$ act as identity on $H(K_0)$, we know $i$ act as inclusion on $iH(K_1)$, hence $iH(K_1)=i^2H(K_1)$. Similarly, we can say $i^n(K_i)$ stabilizes when $n\geq 3$, hence $A_4=A_5=...A_\infty$.
\begin{center}
\begin{tikzcd}
A_4 \arrow[rr, "i"] &  & A_4 \arrow[ld, "j_4"] \\
 & B_4 \arrow[lu, "k_4"] & 
\end{tikzcd}.
\end{center}
Furthermore, since $i: A_4\lrta A_4$ is the inclusion, the map $k_4:B_4\lrta A_4$ must be a zero map, hence the differential $d_4:=j_4k_4=0$ and $B_5=H_{d_4}(B_4)=B_4$. $B-r$ also stabilize after $B_4$. $B_4=B_5=...=B_\infty$.
\begin{center}
\begin{tikzcd}
A_\infty \arrow[rr, "i_\infty:\subseteq"] &  & A_\infty \arrow[ld, "j_\infty"] \\
 & B_\infty \arrow[lu, "k_\infty=0"] & 
\end{tikzcd}.
\end{center}
$k_\infty=0\Lrta$ $B_\infty$ is the quotient of $i_\infty$. In other words, $B_\infty$ is the associated graded complex of the filtration
$$
H(K)=H(K_0)\supseteq iH(K_1)\supseteq iiH(K_2)\supseteq iiiH(K_3).
$$

In general consider a filtration of complex $K$ with differential $D$. 
$$
K=K_0\supseteq K_1\supseteq K_2\supseteq K_3\supseteq...
$$

It induces a sequence in cohomology
$$
H(K)=H(K_0)\overset{i}{\llta} H(K_1)\overset{i}{\llta} H(K_2)\overset{i}{\llta} H(K_3)\overset{i}{\llta}...
$$
Set $F_p:=i^pH(K_p)$ be the image of $H(K_p)$ in $H(K)$. It gives a filtration of $H(K)$
$$
H(K)=F_0\supseteq F_1\supseteq F_2\supseteq F_3\supseteq...
$$
A filtration $K_\bullet$ is of \textbf{length} $l$ if the descending chain terminates after $K_l$. If $K_\bullet$ is of finite length, then $A_r$ and $B_r$ eventually stabilize and $B_\infty$ is the associated graded complex $\oplus F_p/F_{p+1}$ of $F_\bullet H(K)$. 
\begin{dfn}
It is customary to write $E_r$ for $B_r$. A sequence of differential complex $\{E_r,d_r\}$ in which $E_{r+1}=H_{d_r}(E_r)$ is called a \textbf{spectral sequence}. If $E_r$ eventually stabilize, we denote the stationary value $E_\infty$. If $E_\infty\cong Gr_\bullet H$ of some filtered complex $H$. 
\end{dfn}
Now assume $K$ is a graded differential complex $K=\oplus_n K^n$, with filtration $K_\bullet$. Then each graded piece $K^n$ is filtered complex with filtration $K^{n}_p=K^n\cap K_p$.
\begin{thm}
If $K=\oplus_n K^n$ is a graded filtered complex with filtration $\{K_p\}$ and let $H_D(K)$ denote the cohomology of $K$ with a filtration $\{F_p\}$ induced by $\{K_p\}$. Suppose that for each fixed grading index $n$, the filtration $\{K^{n}_p\}$ is of finite length. Then the short exact sequence 
$$
0\lrta \oplus_{p\in\intg} K_{p+1}\lrta \oplus_{p\in\intg} K_{p}\lrta \oplus_{p\in\intg} K_{p}/K_{p+1}\lrta 0
$$
induces a spectral sequence converging to $H_D(K)$.
\end{thm}
\begin{proof}
We have the exact couple
\begin{center}
\begin{tikzcd}
A_r \arrow[rr, "i"] &  & A_r \arrow[ld, "j_r"] \\
 & B_r \arrow[lu, "k_r"] & 
\end{tikzcd},
\end{center}
where $A_r=i^{r-1}H(K_p)$, if $r\geq p$, $i^rH(K_p)=F_p$. (When $r\geq p+1$, the map $i:i^rH(K_p)\lrta i^rH(K_p)$ is an inclusion).

Recall that $k_1$ is the connecting map $k_1:H^{*}(B)\lrta H^{*+1}(A)$. $k_r$ would send $B_r^d\lrta A_r^{n+1}$, while $i,j_r$ would fix $n$.

For a fixed grading index $n$, assume the length of the filtration $\{K_p^n\}$ is $l(n)$. When $r\geq l(n+1)+1$, for every $p$ we have
$$
i^rH^{n+1}(K_p)=F_p^{n+1}
$$
$$
A_r^{n+1}=\oplus_p F_p^{n+1}
$$
and the map
$$
i: i^rH^{n+1}(K_p)\lrta i^rH^{n+1}(K_p)
$$
is inclusion. Hence 
$$
i:A_r^{n+1}\lrta A_r^{n+1}\ \ \ i:F_{p+1}^{n+1}\lrta F_{p}^{n+1}
$$
is injective and 
$$
k_r:B_r^{n}\lrta A_r^{n+1}
$$
is zero map. We have
$$
0\lrta \oplus_p F_{p+1}^{n}\overset{i}{\lrta} \oplus_p F_{p}^{n}\lrta B_\infty^n\lrta 0 
$$

Then we know
$$
B_\infty^n=\oplus_{p\leq l(n)}F^{n}_p/F^n_{p+1}
$$
and 
$$
B_\infty=\oplus_n B_\infty^n=\oplus_p F_p/F_{p+1}=Gr_\bullet H_D(K)
$$

\end{proof}
\subsection{Spectral sequences of double complex}

\section{Model categories}
Model categories were an abstraction of homotopy theory. They especially useful when we only care about topological spaces up to some weak form of equivalence. For example, homotopy equivalence is a prototype of this ``weak equivalence''.
\begin{dfn}
\textbf{weak homotopy equivalences} are map from $X$ to $Y$ that induces isomorphism on each homotopy groups, $\pi_n(X,x)\cong \pi_n(Y,f(x))$. We often denote weak equivalences by $\overset{\sim}{\lrta}$.
\end{dfn}

\begin{dfn}
Let $\calc$ be any category. The arrow category of $\calc,$ denoted $\mathbf{Arr}(\calc)$ is defined to be a category where objects are the arrows of $\calc$ and morphism are commutative squares.
\end{dfn}

\begin{dfn}
Let $\calc$ be any category. An arrow $f\in Arr(\calc)$ is a retract of an arrow $g\in Arr(\calc)$ if it is a retract of an object in $\mathbf{Arr}(\calc)$.

Explicitly, $f$ is a retract of $g$ if we are given a commutative diagram as the following:
$$
\begin{tikzcd}
A \arrow[r] \arrow[rr, "id_A", bend left] \arrow[d, "f"'] & C \arrow[r] \arrow[d, "g"] & A \arrow[d, "f"] \\
B \arrow[r] \arrow[rr, "id_B"', bend right] & D \arrow[r] & B
\end{tikzcd}
$$
\end{dfn}

\begin{dfn}
Let $\calc$ be any category. A \textbf{model structure} on $\calc$ is  the given three full subcategories $W, FIB, COFIB$ of $\mathbf{Arr}(\calc)$ satisfying the following axioms:
\begin{enumerate}[label=\textbf{MC\arabic*}]
\item $\calc$ is (small) complete and (small) cocomplete;
\item if $f,g,h$ are arrows s.t. $fg=h$ and two of them are in $W$, then so is the third;
\item $W,FIB,COFIB$ are closed under retracts;
\item every arrow in $W\cap FIB$ has the RLP with respect to every arrow in $COFIB$ and every arrow in $FIB$ has the RLP with respect to every arrow in $W\cap COFIB$;
\item there are functorial $(W\cap COFIB, FIB)$ and $(COFIB,W\cap FIB)$ factorization in $\calc$.
\end{enumerate}
\end{dfn}

\begin{dfn}
By Axiom \textbf{MC1}, we know a model category have initial object $\emptyset$ and final object $*$. We see an object $A\in \calc$ is cofibrant if $\emptyset\lrta A$ is a cofibration and we say $B\in\calc$ is fibrant if $B\lrta *$ is a fibration.
\end{dfn}

\begin{ex}
The category $Ch_R$ can be given the structure of a model category by defining a map $f: M\lrta N$ to be 
\begin{enumerate}[label=(\roman*)]
\item a weak equivalence if $f$ induces isomorphisms on homology groups,
\item a cofibration if for each $k\geq 0$ the map $f_k:M_k\Lrta N_k$ is a monomorphism with projective $R$-module as its cokernel, and
\item a fibration if for each $k\geq 1$ the map $f_k: M_k\lrta N_k$ is an epimorphism.
\end{enumerate} The initial object in $Ch_R$ is the zero complex. 
The cofibrant object in $Ch_R$ are the projective chain complexes. The homotopy category $Ho(Ch_r)$ is equivalent to the category whose objects are these cofibrant chain complexes and whose morphisms are ordinary chain homotopy classes of maps.
\end{ex}

\begin{lemma}
Let $\calc$ be a model categroy. Then :
\begin{enumerate}[label=(\roman*)]
\item fibrations are exactly those arrows with RLP with respect to all acyclic cofibrations;
\item acyclic fibrations are exactly those arrows with RLP with respect to all cofibrations;
\item  cofibrations are exactly those arrows with LLP with respect to all acyclic fibrations;
\item acyclic cofibrations are exactly those arrows with LLP with respect to all fibrations;
 \end{enumerate}
\end{lemma}
\begin{proof}
We sketch the proof of $(i)$: One inclusion is by definition: every fibration has RLP with every acyclic cofibration. For the reverse inclusion, consider an arrow $f$, by axiom $MC5$, $f$ can be factorized in to $p\circ i$ where $i$ is an acyclic cofibration and $p$ is a fibration. 
$$
\begin{tikzcd}
\bullet \arrow[d, "i"'] \arrow[r, "id"] & \bullet \arrow[d, "f"] \\
\bullet \arrow[r, "p"] \arrow[ru, "h"] & \bullet
\end{tikzcd}
$$
and observe that the diagram 
$$
\begin{tikzcd}
\bullet \arrow[r, "i"] \arrow[d, "f"] & \bullet \arrow[r, "h"] \arrow[d, "p"] & \bullet \arrow[d, "f"] \\
\bullet \arrow[r, "id"'] & \bullet \arrow[r, "id"'] & \bullet
\end{tikzcd}
$$
expresses $f$ as a retract of $p$. This implies that $f$ is a fibration.
\end{proof}

\begin{cor}
Let $\calc$ be a model category. Then 
\begin{enumerate}[label=(\roman*)]
\item FIB is closed under pull back;
\item COFIB is closed under pushout.
\end{enumerate}
\end{cor}
\begin{proof}
Given a fibration $f$ and consider the following diagram
$$
\begin{tikzcd}
\bullet \arrow[d, "j"'] \arrow[r] & * \arrow[r] \arrow[d, "q"'] & \bullet \arrow[d, "f"] \\
\bullet \arrow[r] \arrow[rru, "h'"'] \arrow[ru, "h"] & \bullet \arrow[r, "g"'] & \bullet
\end{tikzcd}
$$
where the right square is the pullback diagram funder $g$. Consider an arbitrary acyclic cofibration $j$, then there exists a $h'$ because $f$ is RLP w.r.t $j$. $h'$ must factor through $h$ because $*$ is a pull back. As a result $q$ has RLP w.r.t every acyclic cofibration. Thus we know $q$ is a fibration by the lemma above.
\end{proof}

\begin{dfn}
Let $\calc$ be a model category. Then A cofibrant approximation to an object $X\in Obj(\calc)$ is a pair $(\tilde{X},i)$ where $\tilde{X}$ is cofibrant and $i:\tilde{X}\lrta X$ is a weak equivalence;

a fibrant approximation to an object $X\in Obj(\calc)$ is a pair $(\hat{X},j)$ where $\hat{X}$ is a fibrant object and $j: X\lrta \hat{X}$ is a weak equivalence;
\end{dfn}

\begin{prop}
\begin{enumerate}[label=(\roman*)]
\item every object $X\in Obj(\calc)$ has a cofibrant approximation $(\tilde{X},i_X)$ where $i_X$ is a trivial fibration;
\item if $(\tilde{X},i_X)$ $(\tilde{X}',i_X
')$ are cofibrant approximations there is a weak equivalence $f:\tilde{X}\lrta \tilde{X}'$;
\item every morphism in $\calc$ has fibrant approximation.
\end{enumerate}
\end{prop}
\begin{proof}
(i) Every morphism to $X$ factorizes as $p\circ q$ where $q$ is a cofibration and $p$ is a acyclic fibration. In particular, the initial object has a morphism ending in $X$, $\emptyset\lrta X$ factorizes as $i_X\circ q$ where $i_X$ is a acyclic fibration and $q:\emptyset\lrta \tilde{X}$ is a  cofibration.  By definition $\tilde{X}$ is cofibrant.

(ii) Consider the diagram
$$
\begin{tikzcd}
\emptyset \arrow[d, "p"'] \arrow[r, "q"] & \tilde{X}' \arrow[d, "i_{X}'"] \\
\tilde{X} \arrow[r, "i_X"] \arrow[ru, "f", dashed] & X
\end{tikzcd}
$$
where $i_X,i_X'$ are acyclic fibrations and $p,q$ are cofibrations. Because each  cofibration has the LLP with acyclic fibrations, there exists a map $f: \tilde{X}\lrta \tilde{X}'$. And this map $f$ is a weak equivalence because. Because $i_X, i_X'$ are weak equivalence and $i_X'f=i_X$, by the 2 of 3 axiom for weak equivalences, $f$ is also a weak equivalence.

(iii): Here we use the notion $(\hat{X},j_X)$ to denote the functorial fibration approximation where $j_X$ is an acyclic cofibration. And we consider $(\hat{Y},j)$ an ordinary fibration approximation of $Y$. Consider the following diagram
$$
\begin{tikzcd}
X \arrow[r, "\alpha"'] \arrow[d, "j_X"'] \arrow[rd] & Y \arrow[d, "j"'] \\
\hat{X} \arrow[r, dashed] \arrow[d, "p_X"'] & \hat{Y} \arrow[ld, "p"] \\
* & 
\end{tikzcd}
$$
where the composition $j\circ \alpha$ gives a map from $X$ to an fibrant object $\hat{Y}$, therefore $X,\hat{X},*, \hat{Y}$ form a  commutative square. Note that $j_X$ is an acyclic cofibration and $p$ is a fibration. There is a lift $h:\hat{X}\lrta \hat{Y}$ because fibration has RLP w.r.p acyclic cofibrations. This lift $h$ gives the fibrant approximation of $\alpha$.
\end{proof}

\begin{lemma}[Ken Brown's Lemma]
Let $\calc$ be a model category and let $\calb$ be a category with subcategory $S\subset \mathbf{Arr}(\calc)$ containing all the identities and satisfies the 2-out-of-3 axiom. If $F:\calc\lrta \calb$ is a functor hat takes acyclic cofibrations between cofibrant objects to elements of $S$, then $F$ takes fibrations between fibrant objects to elements of $S$, then $F$ takes every weak equivalence between fibrant objects to elements of $S$.
\end{lemma}
\begin{rmk}
Basically, this lemmas says a functor sending acyclic (co)fibrations between (co)fibrant objects to $S$ (``weak equivalences in $\calb$''), would send all weak equivalence in $\calc$ to ``weak equivalence in $\calb$'' 
\end{rmk}
\begin{proof}
Let $f: A\lrta B$ be a generic weak equivalence between cofibrant objects between cofibrant objects. By $\mathbf{MC1}$ all colimits exists  in $\calc$, there the coproduct $A\sqcup B\in Obj(\calc)$. By universal property of coproduct there exists a morphism $\lgl f,id_B\rgl: A\sqcup B\lrta B$.
We can factorize this map as
$$
A\sqcup B\overset{q}{\lrta }C\overset{p}{\lrta }B 
$$
with $p$ and acyclic fibration and $q$ a cofibration. Since $A$ and $B$ are cofibrant, stability of cofibrations under pushout implies $i_0, i_1$ are both cofibrations  
$$
\begin{tikzcd}
 & C \arrow[rd, "p"] &  \\
A\sqcup B \arrow[rr, "{\lgl f,id_B\rgl}", bend left] \arrow[ru, "q"] &  & B \arrow[ll, "i_1"] \\
 &  &  \\
A \arrow[uu, "i_0"] \arrow[rruu, "f"] &  & \emptyset \arrow[ll] \arrow[uu]
\end{tikzcd}.
$$
Then $qi_0$ and $qi_1$ are both weak equivalences because of 2-out-of-3 axiom by considering the the following two triangles.
$$
\begin{tikzcd}
C \arrow[rd, "p"] &  & C &  \\
A \arrow[r, "f"] \arrow[u, "q\circ i_0"] & B & A \arrow[u, "q\circ i_0"] \arrow[r, "f"] & B \arrow[lu, "q\circ i_1"].
\end{tikzcd}
$$
By hypothesis, $F(q\circ i_0)$ and $F(q\circ i_0)$ are elements of $S$. Since $F(p\circ q\circ i_1)=F(id_B)$ is in $S$. By 2-out-of-3 hypothesis of $S$, It follows that $F(p)\in S$. Hence $F(f)=F(p)\circ F(q\circ i_0)$ is in $S$.
\end{proof}

\subsection{The homotopy category}

\begin{dfn}
Let $\mathbb{U}\subset\mathbb{V}$ be Grothendieck universe and let $\calc$ be a $\mathbb{U}$-small category. Let $S\subset Arr(\calc)$ be a set of Arrows. A $\mathbb{V}$-\textbf{localization of $\calc$ with respect to $S$} is a $\mathbb{V}$-small category $\calc[S^{-1}]$ together with a functor $F_S:\calc\lrta \calc[S^{-1}]$ such that 
\begin{enumerate}[label=(\roman*)]
\item for all $s\in S$, $F_S(s)$ is an isomorphism;
\item for any other $\mathbb{V}$-small categroy $\cala$ and any functor $\calc\lrta \cala$ such that $G(s)$ is an isomorphism for each $s\in S$, there is a functor $G_s:\calc[S^{-1}]\lrta \cala$ and a natural isomorphism
$$
\eta_G: G_S\circ F_S\cong G
$$
\item for any $\mathbb{V}$-small category $\cala$, the induces functor
$$
F_S^*: Func(\calc[S^{-1}], \cala), \lrta Func(\calc,\cala)
$$
is fully faithful.
\end{enumerate}
\end{dfn}

\begin{thm}
Let $\mathbb{U}$ be a Grothendieck universe and let $\calc$ be a model category. Then there exists a $\mathbb{U}$-localization of $\calc$ with respect to the set of Weak equivalences $W$.
\end{thm}

\begin{dfn}
Let $\calc$ be a category and let $\calc_0$, $\mathbf{W}\subset \calc$ be subcategories. We say that $\calc_0$ is a left (resp. right) deformation retract of $\calc$ with respect to $\mathbf{W}$ if there exits a functor $R:\calc\lrta \calc_0$ and a natural transformation $s: R\lrta Id_\calc$ (resp. $s: Id_\calc\lrta R$) such that:\begin{enumerate}
\item $R$ sends $\mathbf{W}$ into $\mathbf{W}\cap \calc_0$;
\item for every object $C\in Obj(\calc)$, the map $s_C$ is in $\mathbf{W}$;
\item for every object $C_0\in Obj(\calc_0)$, the map $s_{C_0}$ is in $\mathbf{W}\cap \calc_0$.
\end{enumerate}
The pair $(R,s)$ is called a left (resp. right) deformation retraction from $\calc$ to $\calc_0$ with respect to $\mathbf{W}$. When $\mathbf{W}=\calc$, we say  $(R,s)$ is an absolute deformation retraction of $\calc$ to $\calc_0$.
\end{dfn}

\begin{lemma}
Let $\calc$ be a category and let $\calc_0, \mathbb{W}\subset \calc$ be subcategories. Let $R:\calc\lrta \calc_0$ be an absolute left deformation retraction. Assume that for every object $C\in Obj(\calc)$, the map $s_C$ is in $\mathbf{W}$; if $\mathbf{W}$ satisfies the 2-out-of-3 axiom, then $R$ sends $\mathbf{W}$ into $\mathbf{W}\cap \calc_0$. If $\calc_0$ is a full subcategory, then for every $C_0\in Obj(\calc_0)$ the map $s_{C_0}$ is in $\mathbf{W}\cap \calc_0$
\end{lemma}
\section{Fibered category}
\section{Formalization renormalization}
\begin{dfn}
A \textbf{parametrix} for the Laplacian $D$ on a manifold is a symmetric distribution $P$ on $M^2$ such that $(D\otimes 1)P-\delta_M$ is a smooth function on $M^2$, where $\delta_M$ stands for the $\delta$-distribution on the diagonal of $M$.
\end{dfn}
Locally it means
$$
D_x P_{x,y}-\delta_{x,y}
$$
is a smooth function on $M^2$

\end{document}