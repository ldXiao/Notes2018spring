%%%%%%%%%%%%%%%%%%%%%%%%%%%%%%%%%%%%%%%%%
% The Legrand Orange Book
% LaTeX Template
% Version 2.3 (8/8/17)
%
% This template has been downloaded from:
% http://www.LaTeXTemplates.com
%
% Original author:
% Mathias Legrand (legrand.mathias@gmail.com) with modifications by:
% Vel (vel@latextemplates.com)
%
% License:
% CC BY-NC-SA 3.0 (http://creativecommons.org/licenses/by-nc-sa/3.0/)
%
% Compiling this template:
% This template uses biber for its bibliography and makeindex for its index.
% When you first open the template, compile it from the command line with the 
% commands below to make sure your LaTeX distribution is configured correctly:
%
% 1) pdflatex main
% 2) makeindex main.idx -s StyleInd.ist
% 3) biber main
% 4) pdflatex main x 2
%
% After this, when you wish to update the bibliography/index use the appropriate
% command above and make sure to compile with pdflatex several times 
% afterwards to propagate your changes to the document.
%
% This template also uses a number of packages which may need to be
% updated to the newest versions for the template to compile. It is strongly
% recommended you update your LaTeX distribution if you have any
% compilation errors.
%
% Important note:
% Chapter heading images should have a 2:1 width:height ratio,
% e.g. 920px width and 460px height.
%
%%%%%%%%%%%%%%%%%%%%%%%%%%%%%%%%%%%%%%%%%

%----------------------------------------------------------------------------------------
%	PACKAGES AND OTHER DOCUMENT CONFIGURATIONS
%----------------------------------------------------------------------------------------

\documentclass[11pt]{book} % Default font size and left-justified equations

%----------------------------------------------------------------------------------------

\input{structure} % Insert the commands.tex file which contains the majority of the structure behind the template

\begin{document}

%----------------------------------------------------------------------------------------
%	TITLE PAGE
%----------------------------------------------------------------------------------------

\begingroup
\thispagestyle{empty}
\begin{tikzpicture}[remember picture, overlay]
\node[inner sep=0pt] (background) at (current page.center) {\includegraphics[width=\paperwidth]{blue_background}};
\draw (7,-5) node{\Huge\centering\bfseries\sffamily\parbox[c][][t]{\paperwidth}{\centering Algebraic Topology\\[15pt] % Book title
{\Large  An exercise-oriented notes}\\[20pt] % Subtitle
{\huge Vector\_Cat }}}; % Author name
\end{tikzpicture}
\vfill
\endgroup

%----------------------------------------------------------------------------------------
%	COPYRIGHT PAGE
%----------------------------------------------------------------------------------------

\newpage
~\vfill
\thispagestyle{empty}

\noindent Copyright \copyright\ 2013 John Smith\\ % Copyright notice

\noindent \textsc{Published by Publisher}\\ % Publisher

\noindent \textsc{book-website.com}\\ % URL

\noindent Licensed under the Creative Commons Attribution-NonCommercial 3.0 Unported License (the ``License''). You may not use this file except in compliance with the License. You may obtain a copy of the License at \url{http://creativecommons.org/licenses/by-nc/3.0}. Unless required by applicable law or agreed to in writing, software distributed under the License is distributed on an \textsc{``as is'' basis, without warranties or conditions of any kind}, either express or implied. See the License for the specific language governing permissions and limitations under the License.\\ % License information

\noindent \textit{First printing, March 2013} % Printing/edition date

%----------------------------------------------------------------------------------------
%	TABLE OF CONTENTS
%----------------------------------------------------------------------------------------

%\usechapterimagefalse % If you don't want to include a chapter image, use this to toggle images off - it can be enabled later with \usechapterimagetrue

\chapterimage{blue_chap_head.pdf} % Table of contents heading image

\pagestyle{empty} % No headers

\tableofcontents % Print the table of contents itself

\cleardoublepage % Forces the first chapter to start on an odd page so it's on the right

\pagestyle{fancy} % Print headers again

%----------------------------------------------------------------------------------------
%	PART
%----------------------------------------------------------------------------------------

%\part{Homotopy theories}
\chapter{Topological spaces}
\chapter{Fundamental groups}
\chapter{Covering spaces}
\chapter{Elementary homotopy theory}
\section{The mapping cylinder}
\begin{definition}
Given a continuous map $f:X\lrta Y$ of topological spaces, one can define its \textbf{mapping cylinder} as a pushout (fibered coproduct) 
\[
\tiny
\begin{tikzcd}
B &  &  &  \\
 & Z(f) \arrow[lu, "\exists!", dashed] &  & X\times I \arrow[ll] \arrow[ld, "f\times id"'] \arrow[lllu] \\
 &  & Y\times I \arrow[lu, "r"] &  \\
 & Y \arrow[uu] \arrow[ru, "i_0"] \arrow[luuu] &  & X \arrow[ll, "f"] \arrow[uu, "i_0"'],
\end{tikzcd}
\]
Set-theoretically, the mapping cylinder is usually represented as the quotient space $(X\times I \coprod Y)/\sim$, where $f(x)\sim (x,1)$.
We use $Z(f)$ to denote it. (other notations are used including $Mf$, $M_f$ and $\text{Cyl}(f)$.)
\end{definition}
Notice that it is $Z(f)$ rather than $Y\times I$ that plays the role of pushout because the map $r$ is not unique. Our only restriction on $r$ is $r\circ j=id$, where $j: Mf\lrta Y\times I$ is the map that restricts to $f\times id$ on $X\times I$ and restricts to $i_0$ on $Y$.
\begin{remark}
Another equivalent definition is used in tom Dieck.
\end{remark}

In the following, we consider $X\coprod Y$ as subspace of $Z(f)$ via the map $J:J(x)=[(x,0)]$ and $J(y)=[y]$. Then we consider a homotopy commutative diagram
\[
\begin{tikzcd}
X \arrow[r, "f"] \arrow[d, "\alpha"'] & Y \arrow[d, "\beta"] \\
X' \arrow[r, "f'"] & Y',
\end{tikzcd}
\]
where the diagram commutes up to a homotopy $\Psi: f'\circ \alpha\simeq \beta \circ f$.
\section{Double mapping cylinder}
\section{Suspension. Homotopy groups}
In this section we consider the pointed spaces. A map $K: X\times I\lrta Y$ is a pointed homotopy from the constant map to itself iff it sends the subspaces $X\times \pd I\cup \{x\}\times I$ to the base point of $Y$. We define the quotient space 
$$
\Sigma X=X\times I/(X\times \pd I\cup \{x\}\times I)
$$
to be \textbf{suspension} of the pointed space $(X,x)$. (This is also called reduced suspension in some other literatures) A homotopy $K: X\times I\lrta Y$ from the constant map to itself now factors through a pointed map $\overline{K}:\Sigma X\lrta Y$ and homotopies relative to $X\times \pd I$ correspond to homotopies $\Sigma X\times I\lrta Y$. 

This leads us to the homotopy set $[\Sigma X, Y]^0$, which carries a group structure
$$
f+g:(x,t)\mapsto\left\{\begin{aligned}
f(x,2t) & t\leq 1/2\\
g(x,2t-1), & 1/2\leq t
\end{aligned}\right.
$$
Suspension $\Sigma$ is a functor from the category of pointed spaces to itself. If $f: X\lrta Y$ is a pointed map, then $f\times id(I)$ is a compatible with passing to the suspension and induces $\Sigma f: \Sigma X\lrta \Sigma Y, (x,t)\mapsto (f(x),t)$. In this manner the suspension becomes a functor $\Sigma: TOP^0\lrta TOP^0$. This functor is in addition compatible with homotopies: a pointed homotopy $H_t$ induces a pointed homotopy $\Sigma(H_t)$.

We define the $k$-fold suspension by $\Sigma^k X=X\wedge (I^k/\pd I^k)$. Because of the associativity of smash product and 
$$
I^{k+\ell}/\pd I^{k+\ell}\cong I^k/\pd I^k\wedge I^\ell/\pd I^\ell,
$$
 we have
 $$
\Sigma^\ell(\Sigma^k X)\cong\Sigma^{k+\ell}X
 $$

\section{Loop space}
\begin{definition}
The \textbf{loop space} $\Omega Y$ of $Y$ is the subspace of the path space $Y^I$ (with compact-open topology) consisting of the loops in $Y$ with base point $y$
$$
\Omega Y=\{\omega\in Y^I|\omega(0)=\omega(1)=y\}.
$$
\end{definition}

A pointed map $f: Y\lrta Z$ induces a pointed map $\Omega f:\Omega Y\lrta \Omega Z,\omega\mapsto f\circ \omega$. This yields the functor. It is compatible with homotopies $H_t$ yields a pointed homotopy $\Omega H_t$


The most important fact in this section is 
the adjunction between $\Sigma$ and $\Omega$.
$$
[\Sigma X, Y]^0\cong[X,\Omega Y]^0
$$
\section{Groups and cogroups}


\section{The cofibre sequence}

A pointed map $f: (X,*)\lrta (Y,*)$ induces a pointed set map
$$
f^*:[Y,B]^0\lrta [X, B]^0, [\alpha]\mapsto [\alpha f].
$$
The kernel of $f^*$ consists of the classes $[\alpha]$ such that $\alpha f$ is pointed null homotopic. A homotopy set $[Y,B]^0$ is pointed by the constant map.

We say a sequence $A\overset{\alpha}{\lrta} B\overset{\beta}{\lrta} C$ of pointed set maps is \textbf{exact} if $\alpha(A)=\beta^{-1}(*)$. A sequence $U\overset{f}{\lrta} V\overset{g}{\lrta} W$ in $TOP^0$ is called \textbf{h-coexact} if for each  pointed space $B$ the sequence
$$
[U,B]^0\overset{f^*}{\llta}[V,B]^0\overset{g^*}{\llta}[W,B]^0
$$
is exact. If we choose $B=W$ and consider $id(W)\in[W,W]^0$ $f^*g^*([id(W)])=[gf]=[*]$, which means $gf$ is null homotopic.

A pointed homotopy $X\times I\lrta B$ sends $*\times I$ to the base point. Therefore we use the \textbf{pointed cylinder} $XI=X\times I/*\times I$ in $TOP^0$ together with the embeddings $i_t: X\lrta XI, x\mapsto (x,t)$ and the projection $p: XI\lrta X, (x,t)\mapsto x$, and we consider morphisms $XI\lrta Y$ in $TOP^0$ as homotopies in $TOP^0$.

Note that this is the cylinder object in the category of pointed topological spaces.

Similarly, we can define the \textbf{pointed cone} $CX$ over $X$ as $CX=X\times I/X\times 0\cup *\times I$ with base point the identified set. The inclusion $i_1^X=i_1: X\lrta CX, x\mapsto (x,1)$ is an embedding. The maps $h: CX\lrta B$ correspond to the homotopies of the constant map to $hi_1$ (by composition with the projection $XI\lrta CX$.
)

The (pointed) mapping cone of $f$ is defined as $C(f)=CX\vee Y/(x,1)\sim f(x)$, or more formally, via a pushout
$$
\begin{tikzcd}
X \arrow[r, "f"] \arrow[d, "i_1"] & Y \arrow[d, "f_1"] \\
CX \arrow[r, "j"] & C(f).
\end{tikzcd}
$$
We denote the points of $C(f)$ by their representing elements in $X\times I \coprod Y$. The inclusion $Y\subset CX\coprod Y$ induces an embedding $f_1: Y\lrta C(f)$, and $CX\subset CX\coprod Y$ induces $CX\lrta C(f)$. The pushout property says: The pairs $\alpha: Y\lrta B$, $h: CX\lrta B$ with $\alpha f=h i_1$, i.e., the pairs of $\alpha$ and null homotopies of $\alpha$, correspond to maps $\beta:C(f)\lrta B$ with $\beta j=h$ and $\beta f_1=\alpha$.






\chapter{Cofibrations and fibrations}
\section{Fibrations}
\begin{definition}
A continuous map $p: E\lrta B$ is a \textbf{fibration} if it has the \textbf{homotopy lifting property} (HLP); i.e. given the continuous maps $p, G, \tilde{g}$, and the inclusion $Y\times \{0\}\lrta Y\times I$, we can find a continuous map $\tilde{G}: Y\times I\lrta E$ to make the diagram commutes
$$
\begin{tikzcd}
Y\times\{0\} \arrow[d, hook] \arrow[r, "\tilde{g}"] & E \arrow[d, "p"] \\
Y\times I \arrow[r, "G"] \arrow[ru, "\tilde{G}", dashed] & B
\end{tikzcd}
$$
\end{definition}
\begin{exr}
Show that the projection to the first factor $p : B \times F \lrta B$ is a fibration. Show by example that the lifting need not be unique.
\end{exr}
\begin{proof}
Consider the special case of lifting diagram
$$
\begin{tikzcd}
Y\times\{0\} \arrow[d, hook] \arrow[r, "\tilde{g}"] & B\times F \arrow[d, "p"] \\
Y\times I \arrow[r, "G"] \arrow[ru, "\tilde{G}", dashed] & B
\end{tikzcd}
$$
we know $G(y,0)=p\tilde{g}(y)$, where $\tilde{g}: Y\lrta B\times F, y\mapsto (\tilde{g}_1(y), \tilde{g}_2(y))$, which means $G(y,0)=\tilde{g}_{1}(y)$. We can then define $\tilde{G}: Y\times I\lrta B\times F, (y,t)\mapsto (G(y,t), \tilde{g}_2(y))$. This is indeed a well-defined fibration.

It is not unique, for example, assume $F$ to be a vector space, we can alternatively define
$$
\tilde{G}(y,t)=(G(y,t), \tilde{f}_2(y,t))
$$
so that 
$$
\tilde{f}_2(y,0)=\tilde{g}_2(y)
$$
\end{proof}
\begin{theorem}
Let $p: E\lrta B$ be a continuous map. Suppose that $B$ is paracompact and suppose that there exists an open cover $U_\alpha$ of $B$ so that $p: p^{-1}(U_\alpha)\lrta U_\alpha$ is a fibration for each $U_\alpha$. 

Then $p: E\lrta B$ is a fibration.
\end{theorem}
\begin{proof}
Proof can be found at page 51 of JPM.
\end{proof}
\begin{corollary}
If $p: E\lrta B$ is a fiber bundle over a compact space $B$, then it is a fibration.
\end{corollary}
The definition of fiber bundle requires locally trivial and by exercise above, we have proved that trivial fiber bundle is a fibration.

\begin{exr}
Give an example of fibration which is not a fiber bundle.
\end{exr}
\begin{proof}
Consider the the $p: E\lrta B$, where $E=I^2/{1/2}\times I$ and $B=I$ where $p$ is the canonical projection and this can not be a fiber bundle. Because at the point $1/2$, there is no local trivialization. In general it is called homotopy fiber bundle, see this \href{https://math.stackexchange.com/questions/6428/is-there-anywhere-we-use-a-fibration-which-is-not-a-fiber-bundle/9589}{Link}. 

The proof that the counterexample we gave here is indeed a fibration is a simple verification.
\end{proof}

\begin{definition}
If $p: E\lrta B$ and $p': E'\lrta B'$ are  fibrations, then a map of fibrations is a pair of maps $f: B\lrta B'$, $\tilde{f}: E\lrta E'$ so that the diagram commutes.
$$
\begin{tikzcd}
E \arrow[d, "p"] \arrow[r, "\tilde{f}"] & E' \arrow[d, "p'"] \\
B \arrow[r, "f"] & B'
\end{tikzcd}
$$
\end{definition}

\begin{definition}
If $p: E\lrta B$ is a fibration, and $f: x\lrta B$ is a continuous map, define the \textbf{pullback} of $p$ by $f$ to be the map $f^{*}(E)\lrta X$ where
$$
f^*(E)=\{(x,e)\in X\times E|f(x)=p(e)\}\subset X\times E.
$$
and the map $f^*(E)\lrta X$ is the restriction of the projection $X\times E\lrta X$.
\end{definition}
\begin{exr}\label{exr:pullback_of_fibration}
Show that $f^*(E)\lrta X$ is indeed a fibration.
\end{exr}
\begin{proof}
We show this directly by verifying the definition. Consider the diagram
$$
\begin{tikzcd}
Y\times \{0\} \arrow[dd, hook] \arrow[rr, "\tilde{g}"] &  & f^{*}(E) \arrow[rr, "\pi_2"] \arrow[dd, "\pi_1"] &  & E \arrow[dd, "p"] \\
 &  &  &  &  \\
Y\times I \arrow[rr, "G"'] \arrow[rrrruu, "\overline{fG}", dashed] \arrow[rruu, "\tilde{G}", dotted] &  & X \arrow[rr, "f"'] &  & B
\end{tikzcd}.
$$
Because $p: E\lrta B$ is a fibration, by definition, there exists a homotopy $\overline{fG}: Y\times I\lrta E$, which commutes with the outer square.

Recall the universal property of pullback
$$
\begin{tikzcd}
Z \arrow[rdd, "\alpha"', bend right] \arrow[rrd, "\beta", bend left] \arrow[rd, "\exists !\gamma", dashed] &  &  \\
 & f^{*}(E) \arrow[r, "\pi_2"] \arrow[d, "\pi_1"] & E \arrow[d, "p"] \\
 & X \arrow[r, "f"'] & B
\end{tikzcd}
$$
where $\gamma: Z\lrta f^*(E)$ is defined to be 
$z\mapsto (\alpha(z), \beta(z))$. 

Apply this universal property to $\overline{f G}$, we construct $\tilde{G}: Y\times I\lrta f^*(E)$ and
$$
\tilde{G}: (y,t)\mapsto (G(y,t),\overline{fG}(y,t))
$$.

Then it suffices to prove that $\tilde{G}$ indeed commutes with the left square

$\pi_1\circ \tilde{G}=G$\checkmark 

$\tilde{G}(y,0)=(G(y,0),\overline{fG}(y,0))=(\pi_1(\tilde{g},\pi_2\tilde(g))(y)=\tilde{g}(y)$\checkmark
\end{proof}


We have shown before that a fibration is not necessarily a fiber bundle. Nevertheless, a fibration has a well defined fiber up to homotopy. Homotopy lifting property in itself is sufficient to endow a map with the structure of a ``fiber bundle up to homotopy''.

\begin{theorem}
Let $p: E\lrta B$ be a fibration. Assume $B$ is path connected. Then all fibers $E_b=p^{-1}(b)$ are homotopy equivalent. Moreover every path $\alpha: I\lrta B$ defines a homotopy class $\alpha_*$ of homotopy equivalences $E_{\alpha(0)}\lrta E_{\alpha(1)}$ which depends only on the homotopy class of $\alpha$ relative to endpoints, in such a way that multiplication of paths corresponds to composition of homotopy equivalences.

In particular, there exists a well-defined group homomorphism
$$
[\alpha]\mapsto \alpha^{-1}_*
$$
$\pi(B,b_0)\lrta $ Homotopy classes of self-homotopy equivalences of $E_{b_0}$.
\end{theorem}
\begin{remark}
It is more important to understand what the theorem says than under stand the proof. The notion of homotopy classes of self-homotopy equivalences looks quite terrifying. We will dive into detail in the proof.
\end{remark}
\begin{proof}
Let $b_0, b_1\in B$ and let $\alpha$ be a path in $B$ from $b_0$ to $b_1$. The inclusion $E_{b_0}\inj E$ completes to a square diagram, where we define $H(e,t):=\alpha(t)$ And since $p: E\lrta B$ is a fibration, $H$ lifts to $E$.
$$
\begin{tikzcd}
E_{b_0}\times \{0\} \arrow[d] \arrow[r] & E \arrow[d, "P"] \\
E_{b_0}\times I \arrow[r, "H"] \arrow[ru, "\tilde{H}", dashed] & B.
\end{tikzcd}
$$

Notice that the homotopy at time $t=0$, $\tilde{H}_0: E_{b_0}\lrta E$ is just the inclusion of the fiber $E_{b_0}$ in $E$. Furthermore, $p\circ \tilde{H}_t$ is the constant map at $\alpha(t)$, so the homotopy $\tilde{H}$ at time $t$ is a map $\tilde{H}_1: E_{b_0}\lrta p^{-1}(\alpha(t))$. in particular, at time $t=1$, we know $\tilde{H}_1: E_{b_0}\lrta E_{b_1}$.

We will let $\alpha_*:=[\tilde{H}_1]$ denote the homotopy class of this map. Since $\tilde{H}$ is not unique, we need to show that another choice of lift gives a homotopic map. We will in fact show something more general. Suppose $\alpha': I\lrta B$ is another path homotopic to $\alpha$ relative to the end points. Then as before, we obtain a lift $\tilde{H}'$ of $H'$.
$$
\begin{tikzcd}
E_{b_0}\times \{0\} \arrow[d] \arrow[r] & E \arrow[d, "P"] \\
E_{b_0}\times I \arrow[r, "H"] \arrow[ru, "\tilde{H}", dashed] & B.
\end{tikzcd}
$$

\underline{Claim}: $\tilde{H}'_1$ is homotopic to $\tilde{H}_1$.

\textbf{Proof of Claim}. Since $\alpha$ is homotopic relative to end points with $\alpha'$, there exists a map $\Lambda: E_{b_0}\times I\times I\lrta B$ such that
$$
\Lambda(e,s,t)=F(s,t)
$$
where $F(s,t)$ is a homotopy relative to end points from $\alpha$ to $\alpha'$. The solutions $\tilde{H}'$ and $\tilde{H}$ constructed above give a diagram
$$
\begin{tikzcd}
E_{b_0}\times U \arrow[r, "\Gamma"] \arrow[d, hook] & E \arrow[d, "p"] \\
E_{b_0}\times I\times I \arrow[r, "\Lambda"'] & B,
\end{tikzcd}
$$
where $U=I\times \{0,1\}\cup \{0\}\times I$ and
$$
\Gamma(e,s,0)=\tilde{H}(e,s)
$$
$$
\Gamma(e,s,1)=\tilde{H}'(e,s), \text{ and,}
$$
$$
\Gamma(e,0,t)=e.
$$
There exists a homeomorphism $\varphi: I^2\lrta I^2$ such that $\varphi(U)=I\times \{0\}$
\begin{center}
\includegraphics[scale=0.5]{varp}
\end{center}
Thus the diagram 
$$
\begin{tikzcd}
E_{b_0}\times I\times\{0\} \arrow[d, hook] & E_{b_0}\times U \arrow[r, "\Gamma"] \arrow[d, hook] \arrow[l, "id\times \varphi"'] & E \arrow[d, "p"] \\
E_{b_0}\times I\times I \arrow[rru, "\Lambda'", dashed] & E_{b_0}\times I\times I \arrow[r, "\Lambda"'] \arrow[l, "id\times \varphi"] \arrow[ru, "\tilde{\Lambda}", dashed] & B
\end{tikzcd}
$$
has the left two horizontal maps homeomorphisms. Since the homotopy applies to the outside square, there exists a lift $\tilde{\Lambda}:=\Lambda'\circ id\times\varphi$. Thus defined $\tilde{\Lambda}$ is a lift of $\Lambda$ and commutes with the right square.

Therefore we have
$$
\tilde{\Lambda}(e,s,0)=\Gamma(e,s,0)=\tilde{H}(e,s)
$$
$$
\tilde{\Lambda}(e,s,1)=\Gamma(e,s,1)=\tilde{H}'(e,s)
$$

This gives a homotopy from $\tilde{H}$ to $\tilde{H}'$. Restricting to $E_{b_0}\times \{1\}$ we obtain a homotopy from $\tilde{H}_1$ to $\tilde{H}'_1$. Thus the homotopy class $\alpha_*=[\tilde{H}_1]$ depends only on the homotopy class of $\alpha$ relative to end points, establishing the claim. 

Clearly $(\alpha\beta)_*=\beta_*\circ \alpha_*$ if $\beta(0)=\alpha(1)$. In particular, if $\beta=\alpha^{-1}$ then $(const)_*=\beta_*\circ \alpha_*$, where const denote the constant path at $b_0$. But clearly, $(const)_*=[id_{E_{b_0}}]$, therefore $\alpha_*$ is a homotopy equivalence, and since $B$ is path connected, all fibers are homotopy equivalent.

The following exercise completes the proof.
\end{proof}
\begin{exr}
Show that the set of homotopy classes of self-homotopy equivalence of a space $X$ forms a group under composition.
\end{exr}
\begin{proof}
Assume $h_1,h_2: X\lrta X$ are homotopy equivalence from $X$ to $X$ itself. And we denote by $[h_1],[h_2]$ the homotopy classes of them. An obvious choice of multiplication is
$$
[h_1]*[h_2]:=[h_1\circ h_2]
$$
This map is well-defined because composition of homotopic maps are homotopic.
$[id_X]$ is an identity in this group. By definition of homotopy equivalence. Assume $g$ is one homotopy inverse of $h$, we have $gh\simeq id_X$ and $hg\simeq id_X$, therefore, $[g]*[h]=[id_X]$ and $[h]*[g]=[id_X]$.
\end{proof}
\begin{definition}
With a little bit abuse of terminology, we refer to any space in the homotopy class of the space $E_b$ for any $b\in B$ as the fiber of the fibration $p: E\lrta B$.
\end{definition}

Since homotopy equivalences induces isomorphism in homology or cohomology, a fibration with fiber $F$ gives rise to local coefficients systems whose fiber is the homology or cohomology of $F$, as the next corollary asserts.

\begin{corollary}
aaa
\end{corollary}

\section{Path space fibrations}
 An important family of fibrations are the path space fibrations. They will be useful in replacing arbitrary maps by fibrations and then in extending a fibration to a ``fiber sequence''.
 \begin{definition}
Let $(Y,y_0)$ be a base space. The \textbf{path space} $P_{y_0}Y$ is the space of paths in $Y$ stating at $y_0$
$$
P_{y_0}Y=Map((I,0), (Y,y_0))\subset Map(I,Y).
$$
Similarly, the \textbf{loop space} $\Omega_{y_0}Y$ is the space of all loops in $Y$ based at $y_0$
$$
\Omega_{y_0}Y=Map((I,\{0,1\}); (Y,y_0))
$$
Often, we the notion $Y^I:=Map(I,Y)$ to denote the free path space. Let $p: Y^{I}\lrta Y$ be the evaluation at the end point of a path: $p(\alpha)=\alpha(1)$
 \end{definition}
 \begin{exr}
 Let $y_0,y_1$ be two points in a path-connected space $Y$. Prove that $\Omega_{y_0} Y$ and $\Omega_{y_1}Y$ are homotopy equivalent.
 \end{exr}
 \begin{proof}
This is an easy exercise. Because $Y$ is path-connected, we have a loop $\alpha: [0,1]\mapsto Y$ s.t. $\alpha(0)=y_0, \alpha(1)=y_1$. Then we can define a continuous map
$$
\begin{aligned}
f_\alpha: \Omega_{y_0}Y&\lrta \Omega_{y_1}Y\\
 \gamma&\longmapsto \alpha \gamma \alpha^{-1} 
\end{aligned}
$$
where the multiplication is path concatenation.
Similarly, we obtain
$$
\begin{aligned}
g_\alpha: \Omega_{y_1}Y&\lrta \Omega_{y_0}Y\\
 \gamma&\longmapsto \alpha^{-1} \gamma \alpha 
\end{aligned}
$$

Note that the composition of path is associative only up to homotopy.

We know $f_\alpha\circ g_\alpha\simeq id$
 \end{proof}
\begin{theorem}\ 
\begin{enumerate}
	\item The map $p: Y^I\lrta Y$, where $p(\alpha)=\alpha(1)$ is a fibration. Its fiber over $y_0$ is the space of paths which end at $y_0$, a space homeomorphic to $P_{y_0}Y$
	\item The map $p: P_{y_0}Y\lrta Y$ is a fibration. Its fiber over $y_0$ is the loop space $\Omega_{y_0}Y$.
	\item The free path space $Y^I$ is homotopy equivalent to $Y$. The projection $p:Y^I\lrta Y$ is a homotopy equivalence.
	\item The space of paths in $Y$ starting at $y_0$, $P_{y_0}Y$ is contractible.
\end{enumerate}
\end{theorem}
\begin{definition}\label{chap6def:fiber_homotpy}
A \textbf{fiber homotopy} between two morphisms of fibrations is a commutative diagram
$$
\begin{tikzcd}
E\times I \arrow[d, "p\times id"'] \arrow[r, "\tilde{H}"] & E' \arrow[d, "p'"] \\
B\times I \arrow[r, "H"] & B'
\end{tikzcd}
$$
with $H_0=f_0, H_1=f_1, \tilde{H}_0=\tilde{f}_0$ and $\tilde{H}_1=\tilde{f}_1$.
\end{definition}
\section*{Factorizing a continuous map into fibration and homotopy equivalence}

Let $f: X\lrta Y$ be a continuous map. We will replace $X$ by a homotopy equivalent space $P_f$ and obtain a amp $P_f\lrta Y$ which is a fibration.

\begin{definition}
The pullback $P_f=f^*(Y^I)$ of the path space fibration along $f$ is called the mapping path space
$$
\begin{tikzcd}
P_f=f^*(Y^I) \arrow[d] \arrow[r] & Y^I \arrow[d, "q"] \\
X \arrow[r, "f"'] & Y
\end{tikzcd}
$$
where $q(\alpha)=\alpha(0)$.
An element of $P_f$ is a pair $(x,\alpha)$ where $\alpha$ is a path in $Y$ and $x$ is a point in $X$ which maps via $f$ to the starting point of $\alpha$. ($f(x)=\alpha(0)$) The mapping path fibration 
$$
p:P_f\lrta Y
$$
is obtained by evaluating at the end point
$$
p(x,\alpha)=\alpha(1).
$$
\end{definition}
\begin{theorem}\label{thm:factorization_fibration_htp_equiv}
Suppose that $f: X\lrta Y$ is a continuous map.
\begin{enumerate}[label=(\alph*)]
\item There exists a homotopy equivalence $h:X\lrta P_f$ so that the diagram
$$
\begin{tikzcd}
X \arrow[rd, "f"'] \arrow[r, "h"] & P_f \arrow[d, "p"] \\
 & Y
\end{tikzcd}
$$
commutes.
\item The amp $p: P_f\lrta Y$ is a fibration. 
\item IF $f: X\lrta Y$ is a fibration, then $h$ is a fiber homotopy equivalence.
\end{enumerate}
\end{theorem}
\begin{proof}
(a): Let $h: X\lrta P_f$ be the map
$$
x\mapsto (x,const_{f(x)}),
$$
where $const_{f(x)}$  is the constant path at $f(x)$. IN this way, we can have 
$$
p\circ h(x)=const_{f(x)}(1)=f(x)
$$
 the diagram indeed commutes. Then it remains to find the homotopy inverse of $h$. Consider the canonical projection to the first entry
 $p_1: P_f\lrta X, (x,\alpha)\lrta x$. $p_1\circ h=id_X$. For another direction, the homotopy from $h\circ p_1$ to $id_{P_f}$ if given by $F: P_f\times I\lrta P_f$
 $$
 F((x,\alpha),s)
 =(x,\alpha_s)
 $$
 where $\alpha_s(t)=\alpha(st).$ We see therefore $F((x,\alpha),0)=(x, const_{f(x)})=h\circ p_1(x,\alpha)$ and $F((x,\alpha),1)=(x,\alpha)=id_{P_f}(x,\alpha)$.

 (b) we check the homotopy lifting property directly
$$
\begin{tikzcd}
A\times\{0\} \arrow[r, "g"] \arrow[d, "i", hook] & P_f \arrow[d, "p"] \\
A\times I \arrow[r, "H"'] \arrow[ru, "\tilde{H}", dashed] & Y
\end{tikzcd}
$$
$g$ has two components
$$
g(a)=(g_1(a),g_2(a))\in P_f\subset  X\times Y^I
$$
 where $f(g_1(a))=g_2(\alpha)(0)$ by definition of $P_f$. On the other hand, because the outside square commutes, we know $p\circ g(a)=g_2(a)(1)=H(a,0)$. Therefore, there is a path from $f(g_1(a))$ to $g_2(a)(1)=H(a,0)$ and then to $H(a,1)$. This suggest us to define the lift $\tilde{H}=(\tilde{H}_1,\tilde{H}_2)$, where $\tilde{H}_2: A\times I\lrta Y^I$.
 $$
\tilde{H}_1(a,s):= g_1(a)
 $$
 $$
\tilde{H}_2(a,s)(t)=\left\{\begin{aligned}
g_2(a)((1+s) t), &\ \  0\leq t\leq 1/(1+s)\\
H(a,(1+s)t-1), &\ \  1\geq t\geq 1/1/(1+s)
\end{aligned}\right.
 $$
 Thus constructed $\tilde{H}_2$ is continuous and in makes the diagram commutes
 $p\circ \tilde{H}(a,s)=\tilde{H}_2(a,s)(1)=H(a,(1+s)-1)=H(a,s)$. Also $\tilde{H}\circ i(a)=\tilde{H}(a,0)=(g_1(a), g_2(a)(\_))$

 (c): If $f: X\lrta Y$ is already a fibration to start with. Recall the definition~\ref{chap6def:fiber_homotpy} As in the proof of $(a)$, we have proved that $h,p_1$ are homotopy equivalence we also founded the corresponding homotopy. 

 Now it suffices to find another pair of morphism $h: X\lrta P_f$ and $k: P_f\lrta X$ such that $(h, id_Y)$ and $(k,id_Y)$ are morphism of fibrations. And they are fiber homotopy equivalence.

 (1) check $(h,id_Y)$ is morphisms of fibrations. Clearly, the diagram 
 $$
\begin{tikzcd}
X \arrow[d, "f"'] \arrow[r, "h"] & P_f \arrow[d, "p"] \\
Y \arrow[r, "id_Y"'] & Y
\end{tikzcd}
 $$
 commutes.

 However we can not choose $(k=p_1)$ because $f\circ p_1(x,\alpha)=f(x)\neq \alpha(1)$ in general.
 $$
\begin{tikzcd}
X \arrow[d, "f"'] & P_f \arrow[l, "k"'] \arrow[d, "p"] \\
Y & Y \arrow[l, "id_Y"]
\end{tikzcd}
 $$
 The diagram requires $fk(x,\alpha)=\alpha(1)$.

 How to construct $k$?

Consider the test diagram of fibration $f:X\lrta Y$, where $\gamma((x,\alpha),t)=\alpha(t)$.
 $$
\begin{tikzcd}
P_f\times\{0\} \arrow[d, "i"', hook] \arrow[r, "p_1"] & X \arrow[d, "f"] \\
P_f\times I \arrow[r, "\gamma"'] \arrow[ru, "\tilde{\gamma}", dashed] & Y
\end{tikzcd}
 $$
 The outer square indeed commutes and there exists a lift $\tilde{\gamma}$. 

 Now, we define $k:P_f\lrta X$ by $k(x,\alpha):=\tilde{\gamma}((x,\alpha),1)$. In this way $fk(x,\alpha)=f\tilde{\gamma}((x,\alpha),1)=\gamma((x,\alpha),1)=\alpha(1)$. $(k,id_Y)$ is a morphism of fibrations.

 $(2)$ It remains to construct the homotopy $hk\simeq id_{P_f}$ and $kh\simeq id_{X}$

 $$
kh(x)=k(x,const_{f(x)})=\tilde{\gamma}((x,const_{f(x)}),1)
 $$
Also notice $x=\tilde{\gamma}((x, const_{f(x)}),0)$. One obvious choice of homotopy is 
$$
F: X\times I\lrta X
$$
$$
(x,s)\longmapsto \tilde{\gamma}((x,const_{f(x)}),s)
$$

On the other hand
$$
hk(x,\alpha)=h(\tilde{\gamma}(x,\alpha,1))=(\tilde{\gamma}(x,\alpha,1), const_{f(\tilde{\gamma}(x,\alpha,1)})=(\tilde{\gamma}(x,\alpha,1), const_{\gamma(x,\alpha,1)})
$$
where $f\tilde{\gamma}(x,\alpha,1)=\gamma(x,\alpha,1)$. Also note that $(\tilde{\gamma}(x,\alpha,0), \alpha((1-0)t\_)=(x,\alpha(t\_))$. One reasonable choice of homotopy is
$$
G: P_f\times I\lrta P_f
$$
$$
(x,\alpha,s)\mapsto (\tilde{\gamma}(x,\alpha,s),\alpha((1-s)t\_) )
$$
$G(x,\alpha,1)=hk(x,\alpha)$ and $G(x,\alpha,0)=(x,\alpha)$. The pair $(F,G)$ satisfies the definition of fiber homotopy.
\end{proof}
\section{Cofibration}
\begin{definition}
A map $i: A\lrta X$ is called a \textbf{cofibration} or satisfies the \textbf{homotopy extension property} if the following diagram has a solution for arbitrary $Y$.
$$
\begin{tikzcd}
A\times \{0\} \arrow[dd, "i"] \arrow[rr, hook] &  & A\times I \arrow[dd, "i\times id"] \arrow[ld, "h"] \\
 & Y &  \\
X\times\{0\} \arrow[rr, hook] \arrow[ru,"f"] &  & X\times I \arrow[lu, "H",dashed]
\end{tikzcd}
$$
We say there exists a homotopy $H$ extending $h$ with initial condition $f$.
\end{definition}
Cofibration is the dual notion of fibration, using the adjointness of $\square\times I$ and $\square^I$.

Equivalently, the lift problem in the diagram below always has  solution
$$
\begin{tikzcd}
A \arrow[d, "i"'] \arrow[r] & Y^I \arrow[d, "e_0"] \\
X \arrow[r] \arrow[ru, dashed] & Y
\end{tikzcd}
$$
where $e_0(\alpha)=\alpha(0)$. Here $i:A\lrta X$ has the Left lift property with respect to an acyclic fibration $e_0:Y^I\lrta Y$. (we will explain this statement in model category)

Dually, we say a fibration $p:E\lrta B$ has the right lift property with respect to an acyclic cofibration $i_0: Y\times\{0\}\lrta Y\times I$
$$
\begin{tikzcd}
Y\times\{0\} \arrow[d, "i_0",hook] \arrow[r] & E \arrow[d, "p"] \\
Y\times I \arrow[r] \arrow[ru, dashed] & B.
\end{tikzcd}
$$

Most cofibrations we would encounter in this notes are of the form $(X,A)$ where $A$ is a subspace of $X$. We want to find the criterion for the inclusion $i: A\inj X$ to be a cofibration.
\begin{definition}
Let $X$ be compactly generated, $A \subset X$ a subspace. Then $(X,A)$ is called an \textbf{NDR-pair} (neighborhood-deformation-retraction) if there exists continuous maps $u: X\lrta I$ and $h: X\times I\lrta X$ s.t.
\begin{enumerate}
\item $A=u^{-1}(0)$
\item $h(\_, 0)=id_X$
\item $h(a,t)=a,\forall a\in A, t\in [0,1]$
\item $h(x,1)\in A,\forall x\in X$ such that $u(x)<1$
\end{enumerate}
$(X,A)$ is a \textbf{DR-pair} if $u(x) < 1$ for all $x \in X$, in which case $A$ is a deformation retract of $X$. 
\end{definition}

\begin{theorem}\label{thm:steenrod}(Steenrod)
Let $A$ be a closed subspace of $X$. Then the following are equivalent
\begin{enumerate}[label=(\roman*)]
\item $(X,A)$ is an NDR-pair.
\item $(X\times I,X\times \{0\}\cup A\times I)$ is a DR-pair.
\item $X\times \{0\}\cup A\times I$ is a retract of $X\times I$.
\item The inclusion $i: A\inj X$ is a fibration. 
\end{enumerate}
\end{theorem}
\begin{proof}
For proof, see page 45 of JPM.
\end{proof}


\begin{exr}
Prove that a push out of a cofibration is a cofibration. Given $i: A\lrta X$ is a cofibration, assume the square to be a pushout, then the map $B\lrta f_* X$ is a cofibration.
$$
\begin{tikzcd}
A \arrow[r, "f"] \arrow[d, "i"'] & B \arrow[d] \\
X \arrow[r] & f_*X
\end{tikzcd}
$$
\end{exr}
\begin{proof}
This is the dual statement of Exercise~\ref{exr:pullback_of_fibration}. We prove it directily via the diagram.

$$
\begin{tikzcd}
A \arrow[r, "f"] \arrow[d, "i"'] & B \arrow[d] \arrow[r] & Y^I \arrow[d, "e_0"] \\
X \arrow[r] \arrow[rru, "H", dashed] & f_*X \arrow[r] \arrow[ru, "\tilde{H}"', dashed] & Y
\end{tikzcd}
$$
Initially, we assume the right square commutes, which implies the the outside square commutes.
Then by the fact that $i: A\lrta X$ is a fibration, we know there exists $H:X\lrta Y^I$. Then by the fact that $f_*X$ is a pushout, there exists a map $\tilde{H}:f_*X\lrta Y^I$ such that $\tilde{H}\circ [B\lrta f_*H]=B\lrta Y^I$ and $\tilde{H}\circ [X\lrta f_*X]=X\lrta Y^I$. Then it remains to check $e_0\circ \tilde{H}=f_*X\lrta Y$. 

We know $e_0\circ \tilde{H}$ form a commutative diagram
$$
\begin{tikzcd}
A \arrow[r, "f"] \arrow[d, "i"'] & B \arrow[d] \arrow[rrddd] &  &  \\
X \arrow[r] \arrow[rrrdd] & f_*X \arrow[rd, "\tilde{H}"', dashed] &  &  \\
 &  & Y^I \arrow[rd, "e_0"] &  \\
 &  &  & Y
\end{tikzcd}
$$
where the arrows from $X,B$ to $Y$ are just the composition of arrows in the outside square in the initial diagram. Then because the $f_*X$ is a push out  the morphism $f_*X\lrta Y$ is unique. Thus, $e_0\circ \tilde{H}=f_*X\lrta Y$ in the initial diagram. We have verified $\tilde{H}$ indeed form lift.
\end{proof}

\section{Mapping cylinder}
\begin{definition}
A \textbf{mapping cylinder} of map $f: A\lrta X$ is  the space
$$
M_f=\frac{A\times I\coprod X}{(a,1)\sim f(a)}
$$
alternatively, we can define it as  the pushout
$$
\begin{tikzcd}
A\times\{1\} \arrow[r, "f"] \arrow[d, "i_1"', hook] & X \arrow[d] \\
A\times I \arrow[r] & M_f
\end{tikzcd}
$$

A mapping cone $C_f$ is defined to be
$$
C_f=\frac{M_f}{A\times \{0\}}
$$
\end{definition}

There is a dual result to Theorem~\ref{thm:factorization_fibration_htp_equiv}.
\begin{theorem}
Let $f: A\lrta X$ be a continuous map. Let $i: A\lrta M_f$ be the inclusion map $a\mapsto [(a,0)]$
\begin{enumerate}[label=(\arabic*)]
\item there exists a homotopy equivalence $h: M_f\lrta X$ so that the diagram commutes
$$
\begin{tikzcd}
A \arrow[r, "i"] \arrow[rd, "f"'] & M_f \arrow[d, "h"] \\
 & X.
\end{tikzcd}
$$
\item The inclusion $i: A\lrta M_f$ is a cofibration.
\item if $f: A\lrta X$ is already a cofibration, then $h:M_f\lrta X$ is a homotopy equivalence relative to $A$, in particular, $h$ induces homotopy equivalence of cofibers $C_f\lrta X/f(A)$
\end{enumerate}
\end{theorem}
\begin{proof}
(1): Let $h: M_f\lrta X$ be the map $h([a,s])=f(a), h([x])=x$. It is clear that the map $h$ is well-defined. The $h\circ i(a)=h([a,0])=f(a)$ makes the diagram commutes. The homotopy inverse of $h$ is the inclusion $j: X\inj M_f$. $h\circ j=id_X$ and $j\circ h([x])=[x]$ $j\circ h ([a,s])=[f(x)]=[a,1]$. We can find a homotopy $F:M_f\times I\lrta M_f$
$$
F([x],t)=[x]
$$
$$
F([a,s],t)=[a,s+(1-s)t]
$$
This is indeed a well-defined homotopy and $F(\_, 0)=id_X$, $F(\_, 1)=j\circ h(\_)$

(2) We can either choose to prove it via the homotopy extension property or by the Steenrod' s theorem~\ref{thm:steenrod}. (iii$\Lrta $iv). We need to construct a retraction:$M_f\times I\lrta M_f\times \{0\}\cup A\times I$.
\begin{center}
\includegraphics[scale=0.5]{retraction}
\end{center}
Let $r: I\times I\lrta I\times 0\cup 0\times I$ be a retraction so that $r(1\times I)=\{(1,0)\}$. Define $R([a,s],t):=[a,r(s,t)]$ and $R([x],t)=([x],0)$. Thus $i: A\lrta M_f$ is a cofibration.

(3): If $f: A\inj X$ is a cofibration, by Steen rod's theorem there is a retraction
$$
r: X\times I\lrta X\times 1\cup f(A)\times I
$$
and an obvious homeomorphism
$$
q: X\times 1\cup f(A)\times I\lrta M_f.
$$
Define $g: X\lrta M_f$ by $g(x)=q(r(x,0))$. We will show that $g,h$ are homotopy inverse rel $A$. Define the homotopy
$$
H: X\times I\lrta X
$$
as $H:=h\circ r$. Then $H(x,0)=h\circ g(x)$, $H(x,1)=x$, and $H(f(a),t)=f(a)$. Define the homotopy
$$
F:M_f\times I\lrta M_f
$$
as $F([x],t)=q(r(x,t))$ and $F([a,s],t)=q(r(f(a),st))$. Then $F(x,0)=g\circ h(x)$, $F(\_, 1)=id_{M_f}$, and $F(i(a),t)=i(a).$ 
\end{proof}

\section{Sets of homotopy classes of maps}
In this section we will discuss the exact sequence of pointed sets and their behavior under fibrations and cofibrations.
\begin{theorem}
(Basic property of fibrations) Let $p: E\lrta B$ be a fibration, with fiber $F=p^{-1}(b_0)$ and $B$ path-connected. Let $Y$ be any space. Then the sequence of sets
$$
[Y,F]\overset{i_*}{\lrta}[Y,E]\overset{p_*}{\lrta}[Y,B]
$$
is exact.
\end{theorem}
Note that even if $B,E,F, Y$ are ordinary spaces, the set for example $[Y,B]$ is pointed because $B$ is path connected and we can indeed define the notion of exactness for this sequence.
\begin{proof}
Clearly, $p_*(i_*[g])=0$ for a homotopy class $[g]$ of maps from $Y$ to $F$ because $p_*(i_*[g])=p_*([i g])=[pig]=[c_{b_0}]$ where $c_{b_0}$ denote the constant map to $b_0$.

Suppose $f: Y\lrta E$ so that $p_*(f)=const$, i.e. $p\circ f: Y\lrta B$ is null homotopic. Let $G: Y\times I\lrta B$ be a null homotopy, and then let $H: Y\times I\lrta E$ be the lift of $G$ in the following diagram
$$
\begin{tikzcd}
Y\times\{0\} \arrow[d, hook] \arrow[r, "f", dashed] & E \arrow[d, "p"] \\
Y\times I \arrow[r, "G"'] \arrow[ru, "H", dashed] & B
\end{tikzcd}
$$
Then we know $ H(y,0)=f(y)$ and $p\circ H(y,1)=G(y,1)=b_0$, therefore $H(y,1)\in p^{-1}(b_0)$ is a null homotopy from $f$ to $(i_*g)(y):=H(y,1)$ where $g$ is a map from $Y$ to $F$.
\end{proof}

Dually, we have a similar statement for cofibrations.
\begin{theorem}(Basic property of cofibrations). Let $i:A\inj X$ be a cofibration, with cofiber $X/A$. Let $q: X\lrta X/A$ denote the quotient map. Let $Y$ be any path-connected space. Then the sequence of sets
$$
[X/A, Y]\overset{q^*}{\lrta}[X,Y]\overset{i^*}{\lrta}[A,Y]
$$ 
is exact.
\end{theorem}
\begin{proof}
Clearly, $i^*(q^*([g]))=[g\circ q\circ i]=[const]$.

Suppose $f:X\lrta Y$ is a map and suppose that $f|_A: A\lrta Y$ is null-homotopic. Let $h: A\times I\lrta Y$ be a null homotopy. The solution $F$ to the problem
$$
\begin{tikzcd}
X\times \{0\}\cup A\times I \arrow[r, "f\cup h"] \arrow[d, "i"', hook] & Y \\
X\times I \arrow[ru, "F"', dashed] & 
\end{tikzcd}
$$
defines a map $f':X\lrta Y,x\mapsto F(x,1)$, which is homotopic to $f$ and $f'|A$ is a constant map. i.e., $f'(A)=y_0$. Therefore we can define a map $g:X/A\lrta Y$ by $g([x])=f'(x)$. Thus $[f]=[f']=q^*[g]$
\end{proof}

\begin{exr}
Prove the base-point versions of the previous theorems.
\end{exr}
\chapter{Homotopy groups}
\chapter{Stable homotopy. Duality}
\chapter{Cell complexes}
%----------------------------------------------------------------------------------------
%	CHAPTER 1
%----------------------------------------------------------------------------------------


%---------------------------------------------------------------------------------------

%\part{Homologies}

\chapter{Singular homology}
\section{Singular Homology Groups}
\section{The Fundamental Group
}

\section{Homotopy
}
\section{Barycentric Subdivision. Excision}
\section{Weak Equivalences and Homology}
\section{Homology with Coefficients}
\section{The Theorem of Eilenberg and Zilber}
\section{The Homology Product}
\chapter{Homology}
\section{The Axioms of Eilenberg and Steenrod}


\chapter{Homological algebra}
\section{Diagrams}
\section{Exact sequences}
\section{Chain complex}
\section{Cochain complex}
\section{Natural chain maps and homotopies}
\section{Linear algebra of chain complexes}
\begin{definition}
Suppose $(C_\bullet,\pd)$ and $(C'_\bullet,\pd')$ are two non-negative chain complexes. We define the  \textbf{tensor complex} $(C_\bullet\otimes C_\bullet',\Delta)$, where
$$
(C_\bullet\otimes C'_\bullet)_n=\oplus_{i+j=n}C_i\otimes C_j'
$$
and the differential $\Delta$ is defined by 
$$
\Delta(c_i\otimes c'_j)=\pd c_i\otimes c'_j+(-1)^{i}c_i\otimes \pd' c_j
$$
\end{definition}
\begin{definition}
Suppose $f_\bullet:C_\bullet\lrta D_\bullet$ and $g_\bullet: C'_\bullet\lrta D'_\bullet$ are two morphism of chain complexes. Then we can define a chain map
$$
f\otimes g: C_\bullet\otimes C_\bullet'\lrta D_\bullet\otimes D_\bullet'
$$
by 
$$
(f\otimes g)_n=\sum_{i+j=n}f_i\otimes g_j
$$
It is easy to check this is indeed a chain map.
\end{definition}
\begin{exr}
Tensor product is compatible with chain homotopy. Let $s:  f \simeq g: C_\bullet\lrta C_\bullet'$  be a chain homotopy. Then $s\otimes id :f\otimes id  \simeq g \otimes id : C_\bullet\otimes D_\bullet \lrta C_\bullet' \otimes D_\bullet$ is a chain homotopy.
\end{exr}
\begin{proof}
\underline{Know}: $ s\pd_{C}+\pd_{C'} s=f-g$

\underline{Want}: $(s\otimes id_D)\pd_{C\otimes D} +\pd_{C'\otimes D} (s\otimes id_D)=f\otimes id_D-g\otimes id_D$. 

$C\otimes D$ is generated by pure tensors like $c'_n\otimes d_m$, therefore we can  check the formula on element $c_n\otimes d_m\in C_n\otimes D_m$
$$
\begin{aligned}
&(s\otimes id_D)\pd_{C\otimes D}(c_n\otimes d_m)\\
&=(s\otimes id_D)\left(\pd_C c_n\otimes d_m +(-1)^n c_n\otimes\pd_D d_m\right)\\
&=s\circ \pd_C c_n\otimes d_m+(-1)^n s c_n\otimes \pd_D d_m
\end{aligned}
$$
and
$$
\begin{aligned}
&\pd_{C'\otimes D} (s\otimes id_D)(c_n\otimes d_m)\\
&=\pd_{C'\otimes D} (s c_n\otimes d_m)\\
&= \pd_{C'} s c_n \otimes d_m+(-1)^{\deg(sc_n)}sc_n\otimes \pd_D d_m,
\end{aligned}
$$
where $\deg (sc_n)=n-1$. Then we have
$$
\begin{aligned}
&\left(\pd_{C'\otimes D} (s\otimes id_D)+(s\otimes id_D)\pd_{C\otimes D}\right) (c_n\otimes d_m)\\
&=(s\pd_C+\pd_{C'} s)c_n \otimes d_m+0\\
&=(f\otimes id_D-g\otimes id_D)(c_n\otimes d_m)
\end{aligned}
$$
We are done. Also we can generalize this statement to 

 Let $s:  f \simeq g: C\lrta C'$ and  $t:  p \simeq q: D\lrta D'$ be chain homotopies. Then $s\otimes t :f\otimes p  \simeq g \otimes q : C\otimes D \lrta C' \otimes D'$ is a chain homotopy. We easily conclude by $s\otimes id$ and $id\otimes t$ are chain homotopy and composition of chain homotopies is a chain homotopy.
\end{proof}

\begin{exr}\label{chap11exr:free_acyclic_contracting_chain_homotopy}
Let $(C_\bullet,\pd)$ be a free chain complex. Then $C_\bullet$ is acyclic iff it  has contracting chain homotopy
\end{exr}
\begin{proof}
A contracting homotopy means $Q:C_n\lrta C_{n+1}$ s.t. $Q\pd+\pd Q=id$. 

If such $Q$ exists then $H_n(C_\bullet)=0\forall n$. That direction doesn't require $C_\bullet$ to be free.

As for the reverse direction, consider
$$
B_n\subseteq Z_n\subseteq C_n
$$
If we assume $C_\bullet$ is acyclic then
$$
B_n=Z_n,\forall n
$$ 
$$
0\lrta Z_n \overset{i}{\lrta} C_n\overset{\pd}{\lrta}Z_{n-1}\lrta 0
$$

Since $Z_{n-1}$ is free abelian  the sequence splits $\exists r_n:Z_{n-1}\lrta C_n$ s.t. $\pd\circ r_n=id$. Note that $id- r_{n-1}\circ \pd$ has image in $Z_{n-1}$, $c\in C_n$. $\pd(c-r_n\pd c)=\pd c-\pd c=0$

Now define 
$Q_n:C_n\lrta C_{n+1}$ by $Q_{n}=r_n (id-r_{n-1}\circ\pd)$. This works.
$$
\begin{aligned}
\pd Q_n +Q_{n-1}\pd
&=\pd r_n (id -r_{n-1}\pd)+r_{n-1}( id-r_{n-2}\pd )\pd\\
&=id -r_{n-1}\pd+r_{n-1}\pd -r_{n-1}r_{n-2}\pd^2\\
&=id
\end{aligned}
$$
\end{proof}
\begin{definition}
Suppose $f:(C_\bullet,\pd)\lrta (D_\bullet,\pd')$. The \textbf{mapping cone } of $f$ is the chain complex $Cone_\bullet(f),\pd^f$, where $Cone_n(f)=C_{n-1}\otimes D_n$ and 
$\pd^f:Cone_n(f)\lrta Cone_{n-1}(f)$
$$
\pd^f(c,d)=(-\pd c,f c+\pd' d)
$$
$$
\pd^f=
\begin{pmatrix}
&-\pd & 0\\
& f &\pd'
\end{pmatrix}
$$
\end{definition}
\begin{exr}\label{chap11exr:acyclic_mapping_cone_chain_equivalence}
If $f:C_\bullet\lrta D_\bullet$ is a chain map between two free chain complexes and $Cone_\bullet(f)$ is acyclic then prove $f$ is  a chain equivalence.
\end{exr}
\begin{proof}
Note that the definition of mapping cone implies $Cone_\bullet(f)$ to be a free chain complex. Then we can apply Exercise~\ref{chap11exr:free_acyclic_contracting_chain_homotopy} and there is a contracting chain homotopy
$Q$ such that $$
Q\pd^f+\pd^f Q=id
$$
$$
Q=
\begin{pmatrix}
p & g\\
r & -p'
\end{pmatrix}
$$
$$
\begin{pmatrix}
\pd & 0\\
f & -\pd'
\end{pmatrix}
\begin{pmatrix}
p & g\\
r & -p'
\end{pmatrix}
+
\begin{pmatrix}
p & g\\
r & -p'
\end{pmatrix}
\begin{pmatrix}
\pd & 0\\
f & -\pd'
\end{pmatrix}
=\begin{pmatrix}
id & 0\\
0 & id
\end{pmatrix}
$$
$$
\begin{pmatrix}
-\pd p-p\pd +gf & -\pd g+g \pd'\\
* & fg-\pd' p'-p'\pd'
\end{pmatrix}=
\begin{pmatrix}
id & 0\\
0 & id
\end{pmatrix}
$$
Then we know 
$g:D_\bullet \lrta D_\bullet$ is a chain map

$p\pd +\pd p=gf-id$

$p'\pd'+\pd'p'=fg-id$. Thus $f$ is a chain equivalence with inverse $g$.
\end{proof}
\begin{lemma}
Let $f: C_\bullet\lrta D_\bullet$. Then there is a LES
$$
\cdots\lrta H_{n+1}(Cone_\bullet(f))\lrta H_n(C_\bullet)\overset{H_{n}(f)}{\lrta} H_n (D_\bullet)\lrta H_n(Cone_\bullet(f))\lrta \cdots
$$
\end{lemma}
\begin{proof}
Denote by $C^+_\bullet$ the chain complex $C^+_n=C_{n-1}$. There is a SES
$$
0\lrta D_\bullet\overset{i}{\lrta} Cone_\bullet(f)\overset{p}{\lrta} C^+_\bullet\lrta 0
$$
with $i(d)=(0,d)$ and $p (c,d)=c$

Pass to the LES in homology
\[
\begin{tikzcd}
\cdots  \arrow[r] & H_{n+1}(Cone_\bullet(f)) \arrow[r] & H_{n+1}(C^+_\bullet) \arrow[r,"\delta"] \arrow[d, equal] & H_n(D_\bullet) \arrow[r] & H_n(Cone_\bullet(f)) \arrow[r] & \cdots \\
 &  & H_n(C_\bullet) &  &  & 
\end{tikzcd}
\]

It remains to check $\delta=H_n(f)$.


Note if $c$ is a cycle in $C_n$. Then 
$$
\pd^f\circ p^{-1}(c)=(-\pd c, fc)=(0,fc)=i(fc)
$$
$$
\delta:\lgl c\rgl\longmapsto \lgl i^{-1}\pd^fp^{-1}c\rgl=\lgl fc\rgl= H_{n}(f)\lgl c\rgl
$$
\end{proof}
\begin{exr}\label{chap11exr:free_chain_equivalence_isomorphic_homology}
Suppose $f:C_\bullet\lrta D_\bullet$ is  a chain map between the two free chain complex . Then $f$ is a chain equivalence iff 
$$
H_n(f): H_n(C_\bullet)\lrta H_n(D_\bullet)
$$
is an isomorphism for all $n$,
\end{exr}
\begin{proof}
If $f$ is a chain equivalence then $H_n (f)$ is always a isomorphism. This does not require any freeness assumptions and we proved in last semester.

For the converse, if $H_n(f)$ is always an isomorphism, then the LES
$$
\cdots\lrta H_{n+1}(Cone_\bullet(f))\lrta H_n(C_\bullet)\overset{\cong}{\lrta} H_n (D_\bullet)\lrta H_n(Cone_\bullet(f))\lrta \cdots
$$
This implies $H_n(Cone_\bullet (f))=0,\forall n$. Then $Cone_\bullet(f)$ is acyclic, and we can conclude by Exercise~\ref{chap11exr:acyclic_mapping_cone_chain_equivalence}.
\end{proof}
\section{Tor and Ext}
\begin{definition}
Suppose $A$ is an abelian group, A \textbf{Free resolution} is an exact sequence of the form
$$
\cdots\lrta F_2\overset{f_2}{\lrta}F_1\overset{f_1}{\lrta}F_0\overset{f_0}{\lrta}A\lrta 0,
$$
where each $F_i$ is a free abelian group. If moreover $F_i=0,\forall i\geq 2$, we call it \textbf{Short free resolution} 
$$
0\lrta K\lrta F\lrta A\lrta 0
$$
\end{definition}
(We can easily generalize this definition to $R$-modules)
\begin{proposition}
Let $A$ be an abelian group. Then there exists a short free resolution of $A$.
\end{proposition}
\begin{proof}
Let $F$ be the free abelian group generated by all elements in $A$. There is a surjection from $F$ to $A$ by linearly extending the map sending basis element to itself. Let $K$ denote the kernel of this map. $K$ is an abelian subgroup of a free abelian group ($\intg$-module).  A subgroup of a free abelian group is torsion free as a module. $\intg$ is a $PID$. If $R$ is a $PID$, then an  $R$-module is free iff it is torsion free (See Bosch section 4.2). Then we know in particular, $K$ is a free abelian group.
\end{proof}
With this construction, we can define the $\tor$ functor now:
\begin{definition}
Let $A$ be an abelian group. Let $0\rta K\overset{f}{\rta}F\rta A\rta 0$ be a short free resolution of $A$. Given any other abelian group $B$ and apply the functor $\otimes B$ to it. We get an exact sequence
$$
K\otimes B\overset{f\otimes id_B}{\lrta} F\otimes B\lrta A\otimes B\lrta 0
$$
We define 
$$
\tor(A,B):=\ker(f\otimes id_B).
$$
It measure the failure of $\otimes B$ to be left exact.
\tor(A,B) can be more generally defined in the category of $R$-modules, where $R$ is a principal ideal ring, where short free resolution does exist. 
\end{definition}

\begin{exr}\label{chap11exr:tor_functor}
Given $h: A\lrta A'$ a homomorphism of abelian group define the induced morphism
$$
\tor(h, B): \tor (A,B)\lrta \tor(A',B).
$$ 
Use this argument to show that $\tor(A, B)$ is well-defined (independent on the choice of short free resolutions)
\end{exr}
\begin{proof}
Let $0\lrta K\lrta F\lrta A\lrta 0$ and $0\lrta K'\lrta F'\lrta A'\lrta 0$ be two short free resolutions of $A$ and $A'$ respectively, and denote by $C_\bullet$ and $C'_\bullet$ the corresponding chain complexes. 
$$
C_n=\left\{ \begin{aligned}
F, & n=0\\
K, & n=1\\
0, & n\neq 0,1
\end{aligned}
\right.
$$
with the only nontrivial boundary map $\pd: C_1\lrta C_0$ to be $f: K\lrta F$ and $C_\bullet'$ is defined similarly. We have $H_0(C_\bullet)=A$ and $H_0(C_\bullet')=A'$. We can therefore think of $h: A\lrta A'$ as a homomorphism $H_0(C_\bullet)\lrta H_0(C_\bullet')$. Now we can invoke Corollary~\ref{appendixAcor:chain_homotopy_over_H0(f)_free_acyclic}, which tells us there exists a chain map $g_\bullet: C_\bullet\lrta C_\bullet'$ with $H_0(g_\bullet)=h$. 

Tensoring with $B$, we get a chain map $g_\bullet\otimes id_B:C_\bullet\otimes B\lrta C_\bullet'\otimes B$ because $\otimes B$ is a functor. Now pass to the first homology group to get a map
$$
H_1(g_\bullet\otimes id_B): H_1(C_\bullet\otimes B)\lrta H_1(C_\bullet'\otimes B).
$$
However, by definition $H_1(C_\bullet\otimes B)=\tor(A,B)$, we therefore have defined the morphism induced by $\tor(\square,B)$
$$
\tor(h,B):H_1(g_\bullet\otimes id_B)
$$

In order to prove $\tor(A,B)$ is well-defined, we consider the case where $A=A'$, again still by~\ref{appendixAcor:chain_homotopy_over_H0(f)_free_acyclic}, we obtain a chain map $g_\bullet: C_\bullet\lrta C_\bullet'$ with $H_0(g_\bullet)=id_A$ and this chain map $g_\bullet: C_\bullet\lrta C_\bullet$ is a chain equivalence thus induces chain equivalence $g_\bullet\otimes id_B$ and finally we have 
$$
H_1(C_\bullet\otimes B)=H_1(C'_\bullet\otimes B).
$$
Therefore two different short free resolution determines identical $\tor(A,B)$.
\end{proof}
\begin{remark}
The above exercise means basically for each fixed abelian group $B$, $\tor(\square, B):Ab\lrta Ab$ is a covariant functor. In a similar vein, we can also fix the first variable of $\tor$ and define $\tor(A,\square): Ab\lrta Ab$ as a covariant functor. 
\end{remark}
\begin{exr}
Define $\tor(A,h)$ for a given homomorphism $h: B\lrta B'$.
\end{exr}
\begin{proof}
This case is easier, the homomorphism $h: B\lrta B'$ induces a morphism of chain complexes $id_{C_\bullet}\otimes h$ and thus induces the morphism $\tor(A,h):=H_1(id_{C_\bullet}\otimes h):\tor(A,B)\lrta \tor(A,B')$
\end{proof}
\begin{theorem}
(Properties of $\tor$)
\begin{enumerate}[label=(\arabic*)]
\item If either $A$ or $B$ are torsion-free abelian groups then $\tor(A,B)=0$
\item If $T(A)$ denotes the torsion subgroup of $A$ then for any abelian group $B$, one has 
$$
\tor(A,B)\cong \tor(T(A),B).
$$
\item 
If $0\lrta B\lrta B'\lrta B''\lrta 0$ is exact, then for any $A$, there is an exact sequence
$$
\begin{tikzcd}
0 \arrow[r] & \tor(A,B) \arrow[r] & \tor(A,B') \arrow[r] & \tor(A,B'') \arrow[lld] &  \\
 & A\otimes B \arrow[r] & A\otimes B' \arrow[r] & A\otimes B'' \arrow[r] & 0
\end{tikzcd}
$$

If $0\lrta A\lrta A'\lrta A''\lrta 0$ is a short exact sequence, then for any abelian group $B$, there is an exact sequence
$$
\begin{tikzcd}
0 \arrow[r] & \tor(A,B) \arrow[r] & \tor(A',B) \arrow[r] & \tor(A'',B) \arrow[lld] &  \\
 & A\otimes B \arrow[r] & A'\otimes B \arrow[r] & A''\otimes B \arrow[r] & 0
\end{tikzcd}
$$
\item For any two abelian groups $A,B$, $\tor(A,B)\cong \tor(B, A)$.
\item If $B$ is an abelian group and $\{A_\lambda|\lambda\in \Lambda\}$ is a (possibly uncountable ) family of abelian groups then there is an isomorphism
$$
\tor\left(\bigoplus_{\lambda\in\Lambda}A_\lambda, B\right)\cong \bigoplus_{\lambda\in\Lambda}\tor(A_\lambda, B)
$$
and 
$$
\tor\left(B,\bigoplus_{\lambda\in \Lambda}A_\lambda\right)\cong\bigoplus \tor(B,A_\lambda).
$$
\item For any $m\in \mathbb{N}$ and any abelian group $B$,
$$
\tor(\intg/m\intg, B)\cong \{b\in B| mb=0\}
$$
\end{enumerate}
\end{theorem}
\begin{proof}
$(1)$ A torsion free abelian group $B$ is a free $\intg$-module, $\otimes B$ is there fore exact functor, in this case we have $\tor(A,B)=0$ for all $A$. 

On the other hand, if $A$ is free, we can choose a silly short free resolution: $0\lrta 0\lrta A\lrta A\lrta 0$. Then clearly, $\tor(A,B)=0$ for any $B$. We have not prove the case where $A$ is merely torsion free, it will follows directly from $(4)$ and the case $B$ is torsion free. 

Then we skip part $(2)$ and prove the first statement of  $(3)$.
Given an exact sequence $0\lrta B\lrta B'\lrta B''\lrta 0$. let $0\lrta K\lrta F\lrta A\lrta 0$ be short free resolution. Then we know $K\otimes \square$ and $F\otimes \square$ are exact functors. We have a commutative diagram
$$
\begin{tikzcd}
0 \arrow[r] & K\otimes B \arrow[r] \arrow[d] & K\otimes B' \arrow[r] \arrow[d] & K\otimes B'' \arrow[r] \arrow[d] & 0 \\
0 \arrow[r] & F\otimes B \arrow[r] & F\otimes B' \arrow[r] & F\otimes B'' \arrow[r] & 0
\end{tikzcd}
$$
This means we have a short exact sequence of chain complexes
$$
0\lrta C_\bullet\otimes B\lrta C_\bullet\otimes B'\lrta C_\bullet\otimes B''\lrta 0.
$$
It induces a long exact sequence in homology which is just the desired six-term exact sequence in $(3)$. 

The second statement would follow directly from $(4)$ and the first statement of $(3)$.

Then we prove part $(4)$. Given a short free resolution $0\lrta K\lrta F\lrta A\lrta 0$. Because $K,F$ are free, we know $\tor(B,K)=0$ and $\tor(B,F)=0$ from what we have proved in $(1)$. (there is no circular deduction).  Then, from the first statement of $(3)$, we obtain an exact sequence
$$
0\lrta 0\lrta 0\lrta \tor(B,A)\lrta B\otimes K\lrta B\otimes F\lrta B\otimes A\lrta 0.
$$
Also from the definition of $\tor(A,B)$ the bottom row of the next  diagram is exact.
$$
\begin{tikzcd}
0 \arrow[r] & 0 \arrow[d, "\cong"] \arrow[r] & \tor(B,A) \arrow[d] \arrow[r] & B\otimes K \arrow[d, "\cong"] \arrow[r] & B\otimes F \arrow[d, "\cong"] \arrow[r] & B\otimes A \arrow[d, "\cong"] \arrow[r] & 0 \\
0 \arrow[r] & 0 \arrow[r] & \tor(A,B) \arrow[r] & K\otimes B \arrow[r] & F\otimes B \arrow[r] & A\otimes B \arrow[r] & 0
\end{tikzcd}
$$
Then we know from five lemma that $\tor(B,A)\cong \tor(A,B)$

Up to now, we have totally proved $(1)$ and $(3)$ and we come to $(2)$. For any abelian group, if $T(A)$ denote the torsion subgroup, $A/T(A)$ is torsion free. We therefore have $\tor(A/T(A), B)=0$ for any abelian group from $(1)$. Then we apply the second statement of $(3)$  to the short exact sequence $0\lrta T(A)\lrta A\lrta A/T(A)\lrta 0$ and first three terms of the six-term exact sequence is
$$
0\lrta \tor(T(A),B)\lrta\tor(A,B)\lrta 0.
$$

For each $A_\lambda$, we can find a short free resolution $0\lrta K_\lambda\lrta F_\lambda\lrta A_\lambda\lrta 0$ and 
$$
0\lrta \oplus_\lambda K_\lambda\lrta \oplus_\lambda F_\lambda\lrta \oplus_\lambda A_\lambda\lrta 0
$$ is a free resolution of $\oplus_\lambda A_\lambda$. Then
$$
\tor(\oplus_{\lambda\in \Lambda} A_\lambda, B)=\ker\left(\oplus_\lambda K_\lambda\otimes B\lrta \oplus_\lambda F_\lambda\otimes B\right)=\bigoplus_\lambda\ker\left(K\otimes B\lrta F\otimes B\right).
$$
This prove the first statement of $(5)$ and we can prove the second combined with $(4)$.

(6), we can consider the short free resolution $0\lrta \intg\overset{m}{\lrta}\intg\lrta \intg/m\intg\lrta 0$ 
$$
\tor(\intg/m\intg, B)=\ker(B\overset{m}{\lrta} B)
$$

\end{proof}

\begin{definition}
Suppose $A$ is an abelian group and let 
$
0\lrta K \overset{f}{\lrta} F\lrta A\lrta 0
$
be a short free resolution. Take another abelian group $B$ and apply $\hom(\square, B)$, we can find an exact sequence
$$
0\lrta \hom(A, B)\lrta \hom(F,B) \overset{\hom(f,B)}{\lrta} \hom(K,B)
$$
and we define $\ext(A,B):=\coker\hom(f,B)=\hom (K,B)/\im \hom(f,B)$. Thus $\ext (A,B) $ measures the failure for $\hom(\square, B)$ to be right exact.
\end{definition}
Here is a more sophisticated way of viewing $\ext (A,B)$ consider a chain complex $C_1=K$, $C_0=F$, $\pd_1:C_1\lrta C_0=f: K\lrta F$ and all other group zero. $H_0(C_\bullet)=A$. Now apply $\hom(\square, B)$ to a cochain complex $\hom(C_\bullet ,B)$, the definition of $\ext (A,B)$ gives us immediately that
$$
H^1(\hom(C_\bullet,B))=\ext(A,B).
$$
From this it follows that $\ext(\square, B)$ is a contravariant functor and it is well defined (independent of the choice of short free resolution.)
\begin{definition}
An abelian group $D$ is said to be \textbf{divisible} if  for every $b\in D$ and every $n\in \mathbb{N}$ there exists an $a\in D$ s.t., $na=b$
\end{definition}

\begin{theorem}
(Properties of $\ext$)

For a fixed abelian group $A$, $\ext(\square, A)$ is a contravariant functor and $\ext(A,\square)$ is a covariant functor. Moreover,
\begin{enumerate}[label=(\arabic*)]
\item If $A$ is a free group, then $\ext (A,B)=0,\forall B$. If $D$ is a divisible abelian group, then $\ext (A,D)=0\forall A$.
\item If  $A$ is a finitely generated group with torsion subgroup $T(A)$ then $\ext(A,\intg)=T(A)$
\item $0\lrta A\lrta A'\lrta A''\lrta 0$ is exact, then for any $B$, there is an exact sequence
$$
\begin{tikzcd}
0 \arrow[r] & \hom(A'',B) \arrow[r] & \hom(A',B) \arrow[r] & \hom(A,B) \arrow[lld] &  \\
 & \ext(A'',B) \arrow[r] & \ext(A',B) \arrow[r] & \ext(A,B) \arrow[r] & 0
\end{tikzcd}
$$


If $0\lrta B\lrta B'\lrta B''\lrta 0$ is exact, then for any $A$, there is an exact sequence
$$
\begin{tikzcd}
0 \arrow[r] & \hom(A,B) \arrow[r] & \hom(A,B') \arrow[r] & \hom(A,B'') \arrow[lld] &  \\
 & \ext(A,B) \arrow[r] & \ext(A,B') \arrow[r] & \ext(A,B'') \arrow[r] & 0
\end{tikzcd}
$$
\item If $B$ is an abelian group and $\{A_\lambda|\lambda\in \Lambda\}$ is a collection of abelian group then 
$$
\ext\left(\bigoplus_{\lambda\in\Lambda} A_\lambda,B\right)\cong \prod_{\lambda\in \Lambda} \ext(A_\lambda, B)
$$
$$
\ext\left(B, \bigoplus_{\lambda\in \Lambda} A_\lambda\right)\cong \prod_{\lambda\in \Lambda} \ext(B,A_\lambda)
$$
\item For any $m\in \mathbb{N}$ and any $B$
$$
\ext(\intg/ m\intg, B)\cong B/mB
$$
\end{enumerate}
\end{theorem}
\begin{proof}
 $\ext(A,\square)$ and $\ext(\square, A)$ are covariant functor and contravariant functor respectively. The proof is identical to those in~\ref{chap11exr:tor_functor}.

 $(3)-(5)$ can be proved similar to those proof for $\tor$ although we have $\ext(A, B)\neq \ext B,A$ in general.

For the first six-term exact sequence, we can
need an auxiliary fact
$$
\tiny
\begin{tikzcd}
 & 0 & 0 & 0 &  \\
0 \arrow[r] & A \arrow[r, "f"] \arrow[u] & A' \arrow[r, "g"] \arrow[u] & A'' \arrow[r] \arrow[u] & 0 \\
0 \arrow[r] & F \arrow[r, "i_1"] \arrow[u, "u"] & F\oplus F'' \arrow[r, "p_1"] \arrow[u, "u'", dashed] & F'' \arrow[r] \arrow[u, "u''"] & 0 \\
0 \arrow[r] & K \arrow[r, "i_2"] \arrow[u, "v"] & K\oplus K'' \arrow[r, "p_2"] \arrow[u, "v'", dashed] & K'' \arrow[r] \arrow[u, "v''"] & 0 \\
 & 0 \arrow[u] & 0 \arrow[u] & 0 \arrow[u] & 
\end{tikzcd}
$$
Given the first and third column to be short free resolutions and the first row exact, we can construct the middle column to be a short free resolution of $A'$ and each row is exact.

This is guaranteed by the \href{https://en.wikipedia.org/wiki/Horseshoe_lemma}{Horseshoe lemma}

 For the second six-term exact sequence we consider
 $0\lrta K\overset{f}{\lrta} F\lrta 0$ is a chain complex of free abelian group. Free module is a special case of projective module, hence $\hom(K,\square)$ and $\hom(F,\square)$ are exact functors. 

 Then we have the 
 short exact sequence of chain complexes
 $$
0\lrta C_\bullet\lrta C'_\bullet\lrta C''_\bullet\lrta 0.
 $$
 Because each row splits, we also have the SES of  cochain complexes
 $$
0\lrta \hom(C''_\bullet,B)\lrta\hom(C'_\bullet, B)\lrta \hom(C_\bullet, B)\lrta 0
 $$
 and we derive the six-term LES from it.

$$
\begin{tikzcd}
0 \arrow[r] & \hom(K, B) \arrow[r] \arrow[d] & \hom(K, B') \arrow[r] \arrow[d] & \hom(K, B'') \arrow[r] \arrow[d] & 0 \\
0 \arrow[r] & \hom(F, B) \arrow[r] & \hom(F, B') \arrow[r] & \hom(F, B'') \arrow[r] & 0
\end{tikzcd}
$$
which is a short exact sequence of cochain complexes
$$
0\lrta \hom(C_\bullet, B)\lrta \hom(C_\bullet, B')\lrta \hom(C_\bullet, B'')\lrta 0
$$

 (1) If $A$ is a free abelian group, then the free resolution $0\lrta K\lrta F\lrta A\lrta 0$ splits then when applied with functor $\hom(\square, B)$, we still get an exact sequence
 $$
0\lrta \hom(A, B)\lrta \hom(A\oplus K,B) \overset{\hom(f,B)}{\lrta} \hom(K,B)\lrta 0
$$
 whatever $B$ is. $\hom(f,B)(\varphi)=\varphi\circ f$, it is now surjective because for any $v\in \hom(K,B)$, we can choose $0\oplus v\in\hom(A\oplus K, B)$ so that $(0\oplus v)\circ f=v$.

 For the second statement, we \underline{Claim}: when $D$ is divisible, $\hom(\square, D)$ is exact.

 It suffices to prove $\hom(f, D)$ is surjective
$$
0\lrta \hom(A, D)\lrta \hom(F,D) \overset{\hom(f,D)}{\lrta} \hom(K,D)\lrta 0
$$
\end{proof}

\section{Universal coefficients}
\section{Algebraic K\"unneth formula}
In this section
we would prove an algebraic version of K\"unneth formula for free chain complexes. In the next section we would prove Eilenber-Zilber theorem and then derive the general K\"unneth formula as a corollary of the algebraic one. 
\begin{theorem}\label{chap11thm:Algebraic_Kuenneth_formula}
(Algebraic K\"unneth Theorem) Let $(C,\pd)$ and $(D,\pd')$ be two non-negative free complex. Then for every $n\geq 0$, there is a split exact sequence
$$
0\lrta \oplus_{i+j=n}H_i(C_\bullet)\otimes H_j(D_\bullet)\overset{\omega}{\lrta} H_n(C_\bullet\otimes D_\bullet)\lrta \oplus_{k+\ell=n-1}\tor(H_k(C_\bullet),H_\ell(D_\bullet))\lrta 0
$$
where $\omega$ is the map $\langle c_i\rangle\otimes \lgl d_j\rgl\mapsto \lgl c_i\otimes d_j\rgl$. 
Thus there also exists a (non-natural) isomorphism 
$$
H_n(C_\bullet\otimes D_\bullet)\cong \left(\bigoplus_{i+j=n}H_i(C_\bullet)\otimes H_j (D_\bullet)\right)\oplus\left(\bigoplus_{k+\ell=n-1} \tor(H_k(C_\bullet),H_\ell(D_\bullet))\right)
$$
\end{theorem}
\section{Eilenberg-Zilber theorem and K\"unneth formula}
\begin{theorem}
(Eilenberg-Zilber) if $X$ and $Y$ are two topological spaces. There is a nontrivial chain equivalence
$$
\Omega_\bullet: C_\bullet(X\times Y)\lrta C_\bullet(X)\otimes C_\bullet (Y)
$$

which is unique up to chain homotopy.
\end{theorem}
\begin{proof}
$Top\times Top$ is the category of pairs $(X,Y)$ of topological spaces.

We will define two functor from $Top\times Top\lrta Comp$
$$S_\bullet(X,Y)=C_\bullet(X,Y),\ \  T_\bullet (X,Y)=C_\bullet(X)\otimes C_\bullet(Y)$$

For models
$$
\pzm=\{(\Delta^i,\Delta^j), i,j \geq 0\}
$$ 
\underline{Claim}: $S_\bullet$ and $T_\bullet$ are both acyclic in positive degree on $\pzm$ and free with basis contained in $\pzm$

$S_\bullet$, $H_n(S_\bullet(\Delta^i, \Delta^j))=H_n(\Delta^i\times \Delta^j)=0$, $\forall n>0, \forall i,j$ (Acyclic in positive degrees)

$S_i: Top\times Top\lrta Ab$

$S_i(X, Y)=C_i (X\times Y)$\\
\underline{subclaim}: $\{(\Delta^i,\Delta^i)\}$ is a $S_i$-model set and the diagonal map $d_i:\Delta^i\lrta\Delta^i\otimes \Delta^i$  $x\mapsto (x,x)$ gives a model basis.

Indeed, if $(X,Y)$ is any object in $Top\times Top$ and if $
\sigma: \Delta^i\lrta X\times Y
$ is any singular simplex in $S_i(X\times Y)=C_i(X\times Y)$, then we can write 
$\sigma=(\sigma_x,\sigma_y)\circ d_i$,  where $\sigma_x=p_X\circ \sigma$ be the composition of $\sigma$ with $p_X:X\times Y\lrta X$.
$S_i(\tau)(d_i), \tau\in \hom(\Delta^i\times \Delta^i,X\times Y)$ forms a basis of the free abelian group $S_i(X\times Y)=C_i(X\times Y)$.

As for $T_i$, we quote the exercise, for any $(X,Y)\in Top\times Top$, $C_i(X)\otimes C_j(Y)$ is free abelian with ba

 $T_i(X\times Y)=(C_\bullet(X)\otimes C_\bullet(Y))$. $T_i(X, Y)$ is the tensor product of the free groups and thus is free.
$\{(\ell_i,\ell_j)|i+j=n\}$ is a $T_n$-model basis.

The last thing to check is that $T_\bullet(\Delta^i, \Delta^j)$ is acyclic in positive degrees
$$
H_n(C_\bullet(\Delta^i)\otimes C_\bullet(\Delta^j))=0,\forall n>0.
$$
We can not compute this! However we can cheat
$$
H_n(C_\bullet(\Delta^i))=H_n(\Delta^i)=\left\{\begin{matrix}
 \intg & n=0\\
 0 & n\neq 0
\end{matrix}\right.
$$

Consider the chain complex
$$
0\lrta 0\lrta \cdots\lrta 0\lrta \intg\lrta 0\cdots
$$
$C_\bullet(\Delta^i)$ has the same homology as this complex. Thus $C_\bullet(\Delta^i)$ is equivalent to the complex and $C_\bullet(\Delta^j)$ is also chain equivalent to it (By Exercise~\ref{chap11exr:free_chain_equivalence_isomorphic_homology}). $C_\bullet(\Delta^i)\otimes C_\bullet(\Delta^j)$ is chain equivalent to 
$$
0\lrta 0\lrta \cdots\lrta 0\lrta \intg\otimes \intg\lrta 0\cdots
$$
Thus $H_n(C_\bullet(\Delta^i)\otimes C_\bullet(\Delta^j))=H_n(\cdots\lrta0\lrta \intg\otimes \intg\lrta 0\cdots)$. We then know $T_\bullet(\Delta^i,\Delta^j)$ is indeed acyclic in positive degrees. 

We have now verified that the hypotheses of the Acyclic Models Theorem and
its corollary are satisfied.
Define  
$\Theta:H_0\circ S_\bullet\lrta H_0\circ T_\bullet$ is a natural equivalence.
$$
\Theta(X\times Y):H_0(C_\bullet(X\times Y))\lrta H_0(C_\bullet(X)\otimes C_\bullet (Y))
$$
$$
\lgl c_{(x,y)}\rgl\mapsto \lgl c_x\rgl\otimes\lgl c_y\rgl
$$
where $c_{(x,y)}:\Delta^0\lrta (x,y)$ is the constant map to point $(x,y)$. It is indeed a natural transformation
\[
\begin{tikzcd}
H_0(S_\bullet(X\times Y)) \arrow[d, "{H_0(S_\bullet(f,g))}"'] \arrow[r, "\Theta(X\times Y)"] & H_0(T_\bullet(X\times Y)) \arrow[d, "{H_0(T_\bullet(f, g))}"] \\
H_0(S_\bullet(W\times Z)) \arrow[r, "\Theta(W\times Z)"] & H_0(T_\bullet(W\times Z))
\end{tikzcd}
\]
By algebraic K\"unneth formula~\ref{chap11thm:Algebraic_Kuenneth_formula}, we know $H_0(C_\bullet(X\times Y))\cong H_0(C_\bullet(X)\otimes C_\bullet (Y))$ and $\Theta(X,Y)$ is an isomorphism of abelian groups.
The map $ \lgl c_x\rgl\otimes\lgl c_y\rgl\mapsto \lgl c_{(x,y)}\rgl$ gives the inverse of $\Theta(X,Y)$, therefore we know $\Theta$ is a natural equivalence.

By the acyclic models theorem~\ref{apendix:thm:Acyclic_models_theorem}
$$
\Omega_\bullet: S_\bullet\lrta T_\bullet
$$
 is a natural chain equivalence such that $H_0(\Omega_\bullet)=\Theta$

 We therefore find a chain equivalence when apply it to $X\times Y$
 $$
\Omega_\bullet(X,Y):C_\bullet(X\times Y)\lrta C_\bullet(X)\otimes C_\bullet(Y)
 $$
 These two chain complex have isomorphic homologies.
\end{proof}
\begin{corollary}(K\"unneth formula)
As a result, we can apply the algebraic K\"unneth formula here and derive the K\"unneth formula for product of topological spaces.

Then for every $n\geq 0$, there is a split exact sequence
$$
{\scriptstyle
0\lrta \oplus_{i+j=n}H_i(C_\bullet(X))\otimes H_j(C_\bullet(Y))\overset{\omega}{\lrta} H_n(C_\bullet(X)\otimes C_\bullet(Y))\lrta \oplus_{k+\ell=n-1}\tor(H_k(C_\bullet(X)),H_\ell(C_\bullet(Y)))\lrta 0}
$$
where $\omega$ is the map $\langle c_x\rangle\otimes \lgl c_y\rgl\mapsto \lgl c_x\otimes c_y\rgl$. 
Thus there also exists a (non-natural) isomorphism 
$$
\begin{aligned}
H_n(X\times Y)&=H_n(C_\bullet(X\times Y))\\
&\cong H_n(C_\bullet(X)\otimes C_\bullet(Y))\\
&\cong \left(\bigoplus_{i+j=n}H_i(C_\bullet(X))\otimes H_j (C_\bullet(Y))\right)\oplus\left(\bigoplus_{k+\ell=n-1} \tor(H_k(C_\bullet(X)),H_\ell(C_\bullet(Y)))\right)
\end{aligned}
$$

\end{corollary}
\chapter{Cellular homology}
\chapter{Partition of unity in homotopy}
\chapter{Bundles}

\chapter{Manifolds}

\chapter{Homology of manifolds}

\chapter{Cohomology}
\section{Axiomatic approach}
First we state the Eilenberg-Steenord axioms for cohomology. A \textbf{cohomology theory} is a family of contravariant functor $h^n|n\in \intg$ from the category of pair of topological spaces $Top^2$ to the category of $R$-modules $R-Mod$ together with a family of natural transformations $\delta^n|n\in\intg$ s.t. they satisfies the following axioms.
\begin{enumerate}[label=(\arabic*)]
\item \textbf{Homotopy invariance}. Homotopy maps induces same homomorphism.
\item \textbf{Exact sequence}. For each pair $(X,A)$, the sequence 
$$
\cdots\lrta h^{n-1}(A,\emptyset)\overset{\delta}{\lrta} h^n(X,A)\lrta h^n(X,\emptyset)\lrta h^n(A,\emptyset)\overset{\delta}{\lrta}\cdots
$$
is exact and the unspecified arrows are induced by the inclusions.
\item \textbf{Excision.} Let $(X,A)$ be a pair and $U\subset A$ such that $\overline{U}\subset A^\circ$. Then the inclusion $(X\backslash U, A\backslash U)\lrta (X,A)$ induces an isomorphism between $h^n(X,A)\cong h^n(X\backslash U,A\backslash U)$

We say the cohomology theory is \textbf{ordinary} if it in addition satisfies the 4th axiom
\item \textbf{Dimension}. $h^n(pt)=0$ for $n\neq 0$.
\end{enumerate}
Sometimes, we  also refer to the cohomology theories satisfying only the first three axioms as \textbf{generalized cohomologies}.
Notationally , we write $h^n(X,\emptyset )=h^n(X)$.

\begin{exr}
(exact sequence of a triple), prove that given a triple of topological spaces, $(X,A,B)$,  $B\subset A\subset X$. Prove there is a exact sequence of triple
$$
\cdots {\lrta}{h^{n-1}(A,B)}\overset{\delta}{\lrta} h^n(X,A)\lrta h^n(X,B)\lrta h^n(A,B)\overset{\delta}{\lrta}\cdots
$$
all the unspecified arrows are induced by inclusions.
\end{exr}
\begin{proof}
For diagram chasing proof, confer  \cite[Chapter 4, section 8, theorem 5]{spanier1989algebraic}. We will give here a spectral sequence proof here following the \href{https://math.stackexchange.com/questions/1162917/long-exact-sequence-for-a-triple-follows-from-long-exact-sequence-for-a-pair}{StackExchange question}
\end{proof}

\section{Cohomological universal coefficients theorems}

\chapter{Duality}

\chapter{Characteristic classes}

\chapter{Homology and homotopy}

\chapter{Bordism}
%----------------------------------------------------------------------------------------
%  Appendix
%----------------------------------------------------------------------------------------
\appendix
  %\include{Chaps/appendixA}
\chapter{\text{Acyclic models and model categories}}
\section{Acyclic models theorem}
In algebraic topology, the acyclic models theorem can be used to show that two homology theories are isomorphic and usually applying it would great simplify the proof. It can be thought of as a ``universal pattern'' of homology theories.
\begin{definition}Let $\calc$ be a category.
A family of \textbf{models} in $\calc$ is simply an indexed subset $\pzm=\{M_\lambda|\lambda\in \Lambda\}$ of $obj(\calc)$.
\end{definition}

\begin{definition}
Let $\calc$ be a category with family of models $\pzm=\{M_\lambda|\lambda\in \Lambda\}$.  Suppose $T:\calc\lrta Ab$ is a functor. A $T$-\textbf{model set} $\chi$ is a choice of elements $x_\lambda\in T(M_\lambda)$ for each $\lambda$:
$$
\chi=\{x_\lambda\in T(M_\lambda)|\lambda\in \Lambda\}
$$
\end{definition}

\begin{definition}
Let $\calc$ be a category with family of models $\pzm=\{M_\lambda|\lambda\in \Lambda\}$. Suppose $T:\calc\lrta Ab$ is a functor. We say that $T$ is \textbf{free with basis in} $\pzm$ if the following condition holds:
\begin{enumerate}
\item $T(C)$ is a free abelian group $\forall C\in \calc$
\item There is a $T$-model set $\chi=\{x_\lambda\in T(M_\lambda)|\lambda\in \Lambda\}$ s.t.
$$
\{T(f)(x_\lambda)|f\in \hom(M_\lambda,C), \lambda\in \Lambda\}
$$
is a basis for the free abelian group $T(C)$.
\end{enumerate}
We call $\chi$ a \textbf{model basis} for $T$.
\end{definition}
We say $T_\bullet:\calc\lrta Comp$ if free with basis in $\pzm$ if each $T_n$ is free with basis in $\pzm$.

\begin{definition}
$T_\bullet:\calc\lrta Comp$, we say $T_\bullet$ is \textbf{non-negative} if $T_n(C)=0$ for all $n<0$ and $\forall C$. $T_\bullet$ is \textbf{acyclic in the positive degrees on $C$} or $C$ \textbf{is $T_\bullet$-acyclic } if $H_n(T_\bullet(C))=0,\forall n>0$. 
\end{definition}

\begin{example}
Take $\calc=Top$, $\pzm=\{\Delta^n|n\geq 0\}$.  $T_\bullet$ is the singular chain functor.
$$
\calc_\bullet: Top\lrta Comp
$$
$$
X\mapsto C_\bullet(X)
$$
By definition, $T_\bullet$ is free with basis in $\pzm$.
Then $T_\bullet$ is non-negative because
$C_\bullet$ is non-negative, \checkmark. Also, $\Delta^n$ is $T_\bullet$-acyclic
$H_n(C_\bullet(\Delta^i))=H_n(\Delta^i)=0,\forall n>0$\checkmark. (We say $\pzm$ is $T_\bullet$-acyclic)
\end{example}
\begin{theorem}\label{apendix:thm:Acyclic_models_theorem}
Suppose $\calc$ is a category with models $\pzm$. Suppose $T_\bullet, S_\bullet:\calc\lrta Comp$ are two functors such that both $T_\bullet$ and $S_\bullet$ are non-negative. Assume further $T_\bullet$ is free with basis in $\pzm$ and $S_\bullet$ is acyclic in the positive degree on each element $M\in\pzm$.

Suppose 
$$
\Theta: H_0\circ T_\bullet\lrta H_0\circ S_\bullet
$$
 is a natural transformation. $\exists $ a  natural chain morphism $\Psi_\bullet:T_\bullet\lrta S_\bullet$ which is unique up to natural chain homotopy and has $H_0(\Psi_\bullet)=\Theta$.
\end{theorem}
\begin{corollary}
We will be mostly interested in the case where both $S_\bullet$ and $T_\bullet$ are free with basis $\pzm$ and that each model $M\in\pzm$ is both $S_\bullet$-acyclic and $T_\bullet$-acyclic. In this case if $\Theta: H_0\circ T_\bullet\lrta H_0\circ S_\bullet$ is a natural equivalence then every natural chain map $\Phi_\bullet$ inducing $\Theta$ is natural chain equivalence.
\end{corollary}
\begin{corollary}\label{appendixAcor:chain_homotopy_over_H0(f)_free_acyclic}
Suppose $(C_\bullet, \pd)$ and $(D_\bullet, \pd')$ are two non-negative chain complexes. Assume $C_\bullet$ is free and that $D_\bullet$ is acyclic in positive degrees. Then given any homomorphism $h: H_0(C_\bullet)\lrta H_0(D_\bullet)$, there exists a chain map $f_\bullet: C_\bullet\lrta D_\bullet$ over $h$ and this chain map is unique up to chain homotopy.

Moreover, if both non-negative chain complexes $C_\bullet$ and $D_\bullet$ are free and acyclic, and $h: H_0(C_\bullet)\lrta H_0(D_\bullet)$ is an isomorphism, then the chain map $g_\bullet$ is a chain equivalence.
\end{corollary}
Both corollaries can be directly derived, the first from the definition of chain equivalence and in the second, we choose the category with only one object.


To prove Theorem~\ref{apendix:thm:Acyclic_models_theorem}, we need to first quote some two lemmas
\begin{lemma}\label{apendixA:lem:A.1.3}
Let $\calc$ be a category with family of models $\pzm=\{M_\lambda|\lambda\in \Lambda\}$. Assume $S,T:\calc\lrta Ab$ are functors and assume $T$ is free with basis in $\pzm$. Let
$\chi:=\{x_\lambda\in T(M_\lambda)|\lambda\in \Lambda\}$ denote the model basis for $T$. Choose element $y_\lambda\in S(M_\lambda)$ for each $\lambda\in \Lambda$, and set $\Upsilon:=\{y_\lambda\in S(M_\lambda)|\lambda\in\Lambda\}$. Then there exists a unique natural transformation $\Phi:T\lrta S$ such that 
$$
\Phi(M_\lambda)(x_\lambda)=y_\lambda,\forall \lambda\in \Lambda
$$
\end{lemma}
\begin{proof}
Because $T$ is free with basis $\pzm$, we know for each $C\in\calc$, $T(C)$ is free abelian group and
$$
\{T(f)(x_\lambda)|f\in \hom(M_\lambda,C), \lambda\in \Lambda\}
$$
is a basis for the free abelian group $T(C)$. For fixed $\lambda\in \Lambda$ and fixed object $C\in obj(\calc)$, we have a commutative diagram for every morphism $f: M_\lambda\lrta C$
\[
\begin{tikzcd}
T(M_\lambda) \arrow[r, "T(f)"] \arrow[d, "\Phi(M_\lambda)"'] & T(C) \arrow[d, "\Phi(C)"] \\
S(M_\lambda) \arrow[r, "S(f)"'] & S(C)
\end{tikzcd}
\]
We have $\Phi(C)\circ T(f)(x_\lambda)=S(f)(y_\lambda)$. Since $T(f)(x_\lambda)$ forms a basis of $T(C)$, we know $\Phi(C)$ is uniquely determined, therefore $\Phi$ is unique if it exists.

It indeed exists. Fix any object $C\in \calc$, then by assumption $\{T(f)(x_\lambda)\}$ form a basis of $T(C)$ and $\Phi(C)(T(f)(x_\lambda))=S(f)(x_\lambda)$ by the universal property of free abelian group, there exists a unique homomorphism $\Phi(C):T(C)\lrta S(C)$ that restricts to it on basis. 

We have proved each individual $\Phi(C)$ exists and is unique. It only lefts to check that such specified $\Phi$ is indeed a natural transformation
\[
\begin{tikzcd}
T(A) \arrow[r, "T(g)"] \arrow[d, "\Phi(A)"'] & T(B) \arrow[d, "\Phi(B)"] \\
S(A) \arrow[r, "S(g)"'] & S(B)
\end{tikzcd}
\]
Given a typical basis element $T(f)(x_\lambda)$ for some $\lambda\in \Lambda$ and $f\in \hom(M_\lambda,A)$. Then 
$$
S(g)\circ \Phi(A)(T(f)(x_\lambda))=S(g)S(f)y_\lambda=S(g\circ f) y_\lambda.
$$
But also going the other way round:
$$
\Phi(B)\circ T(g) (T(f)(x_\lambda))=\Phi(B)(Tg\circ f)(x_\lambda)=S(g\circ f)(x_\lambda).
$$
Thus  $\Phi$ is indeed a natural transformation.
\end{proof}
\begin{lemma}\label{appendixA:lem:A.1.4}
Let $\calc$ be category with family of models $\pzm$. Suppose given six functors
$T_i,S_i:\calc\lrta Ab$, $i=0,1,2$. together with six natural transformations as pictured below
\[
\begin{tikzcd}
T_2 \arrow[r, "\Phi_2"] & T_1 \arrow[r, "\Phi_1"] \arrow[d, "\Theta_1"] & T_0 \arrow[d, "\Theta_0"] \\
S_2 \arrow[r, "\Psi_2"'] & S_1 \arrow[r, "\Psi_1"'] & S_0
\end{tikzcd}
\]
Assume that 
\begin{enumerate}
	\item For every object $C\in obj(\calc)$, the composition $\Phi_1(C)\circ \Phi_2(C):T_2(C)\lrta T_0(C)$ is the zero homomorphism.
	\item The bottom row is exact on $\pzm$, in the sense that for every model $M\in\pzm$, one has $\im \Psi_2(M)=\ker \Psi_1(M)$.
	\item The diagram commutes for every object $C\in obj(\calc)$.
	\item $T_2$ is free with basis in $M$.
\end{enumerate}
Then there exists a natural transformation $\Gamma:T_2\lrta S_2$ such the the first square commutes for every object of $\calc$.
\[
\begin{tikzcd}
T_2\ar[d, dashed, "\Gamma"] \arrow[r, "\Phi_2"] & T_1 \arrow[r, "\Phi_1"] \arrow[d, "\Theta_1"] & T_0 \arrow[d, "\Theta_0"] \\
S_2 \arrow[r, "\Psi_2"'] & S_1 \arrow[r, "\Psi_1"'] & S_0
\end{tikzcd}
\]
\end{lemma}
\begin{proof}
Let $\chi=\{x_\lambda\in T_2(M_\lambda)|\lambda\in \Lambda\}$ denote a model basis for $T_2$. Then for each $\lambda\in \Lambda$ we have a commutative diagram in $Ab$ such that both top row and bottom row are chain complex and the bottom row is exact. Also $T_2(M_\lambda)$ is free 
\[
\begin{tikzcd}
T_2(M_\lambda)\ar[d, dashed, "\gamma"] \arrow[r, "f_2"] & T_1(M_\lambda) \arrow[r, "f_1"] \arrow[d, "t_1"] & T_0(M_\lambda) \arrow[d, "t_0"] \\
S_2(M_\lambda) \arrow[r, "g_2"'] & S_1(M_\lambda) \arrow[r, "g_1"'] & S_0(M_\lambda)
\end{tikzcd}
\]
$\im t_1\circ f_2\subset \im g_2$ because $g_1\circ t_1\circ f_2=h\circ f_1\circ f_2=0$, hence $\im t_1\circ f_2\subset \ker g_1=\im g_2$. For each $x_\lambda$, there is a $y_\lambda\in S_2(M_\lambda)$ such that $g_2(y_\lambda)=t_1\circ f_2(x_\lambda)$ because $g_2$ is surjective. By universal property of free module, we get a unique morphism $\gamma:T_2(M_\lambda)\lrta S_2(M_\lambda)$ such that $\gamma(x_\lambda)=y_\lambda$. (But because $y_\lambda$ are not unique, $\gamma$ is not the unique morphism that makes the triangle commute)
\[
\begin{tikzcd}
 & T_2(M_\lambda) \arrow[d, "t_1\circ f_2"] \arrow[ld, "\exists \gamma"', dashed] &  \\
S_2(M_\lambda) \arrow[r, "g_2"] & \im g_2 \arrow[r] & 0.
\end{tikzcd}
\]
(It also makes the first square of the previous diagram commutes).
We then know by~\ref{apendixA:lem:A.1.3} there exists a unique natural transformation $\Gamma: T_2\lrta S_2$ such that
$$
\Gamma(M_\lambda)(x_\lambda)=y_\lambda, \forall \lambda\in \Lambda.
$$ 

It remains to check thus constructed $\Gamma$ makes the functor diagram commutes. If we define $z_\lambda:=\Psi_2(M_\lambda)(y_\lambda)=\Theta_1(M_\lambda)(\Phi(M_\lambda)(x_\lambda))$, $\Theta_1\circ\Phi_2$ and $\Psi\circ \Gamma$ are two natural transformations that sends $x_\lambda$ to $z_\lambda$. Then we know the first square of functor diagram commutes by the uniqueness in~\ref{apendixA:lem:A.1.3}.
\end{proof}
Finally we come back to the proof of Theorem~\ref{apendix:thm:Acyclic_models_theorem}.
\begin{proof}{of theorem A.1.1}

1. \underline{Such $\Phi_\bullet$ exists}: 
We need to construct natural transformations $\Phi_n:S_n\lrta T_n$ such that the following diagram commutes.
\[\begin{tikzcd}
\cdots \arrow[r, "\pd"] & T_2 \arrow[r, "\pd"] \arrow[d, "\Phi_2", dashed] & T_1 \arrow[r, "\pd"] \arrow[d, "\Phi_1", dashed] & T_0 \arrow[r] \arrow[d, "\Phi_0", dashed] & H_0(T_\bullet) \arrow[r] \arrow[d, "\Theta"] & 0 \\
\cdots \arrow[r, "\pd'"] & S_2 \arrow[r, "\pd'"] & S_1 \arrow[r, "\pd'"] & S_0 \arrow[r] & H_0(S_\bullet) \arrow[r] & 0
\end{tikzcd}
\]
With Lemma~\ref{appendixA:lem:A.1.4} in hand, we can induct on $n$ to concatenate the ``ladders''.

For $n=0$, we have
\[
\begin{tikzcd}
T_0 \arrow[r] \arrow[d, "\exists \Phi_0"', dashed] & H_0(T_\bullet) \arrow[r] \arrow[d] & 0 \arrow[d] \\
S_0 \arrow[r] & H_0(S_\bullet) \arrow[r] & 0
\end{tikzcd}
\]


If we find $\Phi_i, 0\leq i\leq n$ such that makes the first $n$ ladder commute, we can find a $\Phi_{n+1}$ that makes the diagram commute
\[
\begin{tikzcd}
T_{n+1} \arrow[r] \arrow[d, "\exists \Phi_{n+1}"', dashed] & T_n \arrow[r] \arrow[d, "\Phi_n"] & T_{n-1} \arrow[d, "\Phi_{n-1}"] \\
S_{n+1} \arrow[r] & S_n \arrow[r] & S_{n-1}
\end{tikzcd}
\]

2. \underline{$H_0(\Phi_\bullet)=\Theta$}: This is a direct result of the corresponding diagram of  chain complexes. On each object $C$, $H_0(\Phi_\bullet)(C): H_0(T_\bullet(C))\lrta H_0(S_\bullet(C)):\lgl c\rgl\mapsto \lgl \Phi_0(c)\rgl$
Then we have $H_0(\Phi_\bullet)(C)=\Theta(C)$ because the right most square commutes when applied to object $C$.

3. \underline{$\Phi_\bullet$ is unique up to natural chain homotopy} Suppose now we have two such maps $\Psi_\bullet,\Phi_\bullet: T_\bullet\lrta S_\bullet$. We need to find natural transformation $\Upsilon_n: T_{n}\lrta S_{n+1}$ for $n\geq -1$ such that 
$$
\pd'\Upsilon_n+\Upsilon_{n-1}\pd=\Phi_n-\Psi_n.
$$
Denote the difference $\Phi_n-\Psi_n=:\Xi_n$. We define $\Upsilon_{-1}=0$ and proceed inductively. We have a diagram
\[
\begin{tikzcd}
T_0 \arrow[r] \arrow[d, "\exists\Upsilon_0"', dashed] & T_0 \arrow[d, "\Xi_0"] \arrow[r] & 0 \arrow[d, "0"] \\
S_1 \arrow[r] & S_0 \arrow[r] & H_0(S_\bullet),
\end{tikzcd}
\]
where we have used~\ref{appendixA:lem:A.1.4} again. Inductively, If we have constructed $\Upsilon_{n-1}$, we have the following diagram
\[
\begin{tikzcd}
T_{n+1} \arrow[r, "id"] \arrow[d, "\exists \Upsilon_{n}"', dashed] & T_n \arrow[d, "\Xi_n-\Upsilon_{n-1}\circ\pd"] \arrow[r] & 0 \arrow[d, "0"] \\
S_{n+1} \arrow[r, "\pd'"] & S_n \arrow[r] & S_{n-1}
\end{tikzcd}
\] 
By induction hypothesis, the above diagram commutes because
$$
\begin{aligned}
\pd'(\Xi_n-\Upsilon_{n-1}\pd)\\
&=\pd'\Xi_n-\pd'\Upsilon_{n-1}\pd\\
&=\pd'\Xi_n-(\Xi_{n-1}-\Upsilon_{n-1}\pd)\pd\\
&=\pd'\Xi_n-\Xi_{n-1}\pd\\
&=0.
\end{aligned}
$$
Hence, we can construct $\Upsilon_{n}$ by~\ref{appendixA:lem:A.1.4}.
\end{proof}
\ \\
\ \\
\ \\


\section{Some generalizations}
Recall the comparison theorem in homological algebra.
\begin{theorem}
Let $P_\bullet: \cdots\lrta P_2\lrta P_1\lrta P_0\lrta M\lrta 0$ be a complex with all $P_i$ projective. Let $C_\bullet: \cdots\lrta C_2\lrta C_1\lrta C_0\lrta N\lrta 0$ be an exact complex. Then for any $f\in \hom_R(M,N)$ there exists a map of complexes $f_\bullet: P_\bullet\lrta C_\bullet$ that extends $f$. This lifting $f_\bullet$ is unique up to homotopy.
\end{theorem}
How to generalize the acyclic model theorem in this direction?

How to define ``projective'' functor $S_\bullet: \calc\lrta Comp_R$? with some model $\pzm$ specified?

It should not be the projective object in the functor category. We should define the notion of ``projective with basis''.

Recall one of the equivalent definitions of projective module

A module $P$ is projective iff there exists a set $\{a_i\in P\mid i\in I\}$ and a set $\{f_i\in \hom(P,R)\mid i\in I\}$ such that for every $x\in P$, $f_i(x)$ is only nonzero for finitely many $i$, and $x=\sum_i f_i(x)a_i$. In this case we say $P$ is endowed with a \textbf{projective basis}.

This inspires us to define the notion of ``projective functor with basis''.

\begin{definition}
A functor $S: \calc\lrta Mod_{R}$ is said to be \textbf{projective with basis in $\pzm$} if the following two condition holds
\begin{enumerate}
\item $T(C)$ is projective for all $C\in\calc$.
\item There is a $T$-model set $\chi=\{x_\lambda\in T(M_\lambda)\mid M_\lambda\in \Lambda\}$ s.t.
$$
\{T(g)(x_\lambda)|g\in \hom(M_\lambda,C), \lambda\in \Lambda\}
$$
is projective basis for $T(C)$. i.e.
For each $x\in T(C)$, it can be expressed as
$$
x=\sum_{\lambda\in \Lambda}\sum_{g\in \hom(M_\lambda,C)} f_{g,\lambda}^C(x)T(g)(x_\lambda)
$$
where $\{f_i: T(C)\lrta R\}$ is a fixed set of morphisms of $R$-modules.
A functor $S_\bullet:\cala\lrta Comp_R$, where $Comp_R$ is the category of chain complex of $R$-modules, is said to be \textbf{projective with basis in $\pzm$} if each $S_n$ is projective with basis in $\pzm$. 
\end{enumerate}
\end{definition}

\begin{theorem}[Conjecture]\label{conj}
Suppose $\calc$ is a category with models $\pzm$. Suppose $T_\bullet, S_\bullet:\calc\lrta Comp_R$ are two functors such that both $T_\bullet$ and $S_\bullet$ are non-negative. Assume further $T_\bullet$ is projective with basis in $\pzm$ and $S_\bullet$ is acyclic in the positive degree on each element $M\in\pzm$.

Suppose 
$$
\Theta: H_0\circ T_\bullet\lrta H_0\circ S_\bullet
$$
 is a natural transformation. $\exists $ a  natural chain morphism $\Psi_\bullet:T_\bullet\lrta S_\bullet$ which is unique up to natural chain homotopy and has $H_0(\Psi_\bullet)=\Theta$.
\end{theorem}
\begin{lemma}\label{conj:lemma1}
Let $\calc$ be a category with family of models $\pzm=\{M_\lambda|\lambda\in \Lambda\}$. Assume $S,T:\calc\lrta Mod_R$ are functors and assume $T$ is projective with basis in $\pzm$. Let
$\chi:=\{x_\lambda\in T(M_\lambda)|\lambda\in \Lambda\}$ denote the projective model basis for $T$. 
\textbf{If there exists morphisms of $R$-modules $F_\lambda: T(M_\lambda)\lrta S(M_\lambda)$} 

Then there exists a unique natural transformation $\Phi:T\lrta S$ such that 
$$
\Phi(M_\lambda)=F_\lambda
$$
\end{lemma}
\begin{proof}
Because $T$ is projective with basis $\pzm$, we know for each $C\in\calc$, $T(C)$ is projective $R$-module and
$$
\{T(f)(x_\lambda)|f\in \hom(M_\lambda,C), \lambda\in \Lambda\}
$$
is a projective basis for the projective $R$-modules $T(C)$. For fixed $\lambda\in \Lambda$ and fixed object $C\in obj(\calc)$, we have a commutative diagram for every morphism $f: M_\lambda\lrta C$
\[
\begin{tikzcd}
T(M_\lambda) \arrow[r, "T(f)"] \arrow[d, "\Phi(M_\lambda)=F_\lambda"'] & T(C) \arrow[d, "\Phi(C)"] \\
S(M_\lambda) \arrow[r, "S(f)"'] & S(C)
\end{tikzcd}
\]
We have $\Phi(C)\circ T(f)(x_\lambda)=S(f)(y_\lambda)$. Since $T(f)(x_\lambda)$ forms a projective basis of $T(C)$ in particular the module morphism is uniquely determined by its behavior on generating set.

 we know $\Phi(C)$ is uniquely determined and $C$ is arbitrary, therefore $\Phi$ is unique \textbf{if it exists}.

The only reasonable choice of $\Phi(C)$ should be
$$
\Phi(C)(T(g)(x_\lambda)):=S(g)(F_\lambda(x_\lambda)).
$$
and
$$
\Phi(C)(x)=\sum_{\lambda,g} f_{\lambda,g}^C(x) S(g)(F_\lambda(x_\lambda)).
$$
\textbf{I am not quite sure whether there is circular reasoning in the above definition of $\Phi(C)$}

Finally, it suffices to prove the naturality of $\Phi$. We have proved each individual $\Phi(C)$ exists and is unique. It only lefts to check that such specified $\Phi$ is indeed a natural transformation
\[
\begin{tikzcd}
T(A) \arrow[r, "T(g)"] \arrow[d, "\Phi(A)"'] & T(B) \arrow[d, "\Phi(B)"] \\
S(A) \arrow[r, "S(g)"'] & S(B)
\end{tikzcd}
\]
Given a typical basis element $T(f)(x_\lambda)$ for some $\lambda\in \Lambda$ and $f\in \hom(M_\lambda,A)$. Then 
$$
S(g)\circ \Phi(A)(T(f)(x_\lambda))=S(g)S(f)(F(M_\lambda)(x_\lambda))=S(g\circ f)(F_\lambda(x_\lambda)).
$$
But also going the other way round:
$$
\Phi(B)\circ T(g) (T(f)(x_\lambda))=\Phi(B)(T(g\circ f)(x_\lambda))=S(g\circ f)(F_\lambda(x_\lambda)).
$$
Thus  $\Phi$ is indeed a natural transformation.
\end{proof}

\begin{lemma}\label{conj:lemma2}
Let $\calc$ be category with family of models $\pzm$. Suppose given six functors
$T_i,S_i:\calc\lrta Mod_R$, $i=0,1,2$. together with six natural transformations as pictured below
\[
\begin{tikzcd}
T_2 \arrow[r, "\Phi_2"] & T_1 \arrow[r, "\Phi_1"] \arrow[d, "\Theta_1"] & T_0 \arrow[d, "\Theta_0"] \\
S_2 \arrow[r, "\Psi_2"'] & S_1 \arrow[r, "\Psi_1"'] & S_0
\end{tikzcd}
\]
Assume that 
\begin{enumerate}
	\item For every object $C\in obj(\calc)$, the composition $\Phi_1(C)\circ \Phi_2(C):T_2(C)\lrta T_0(C)$ is the zero homomorphism.
	\item The bottom row is exact on $\pzm$, in the sense that for every model $M\in\pzm$, one has $\im \Psi_2(M)=\ker \Psi_1(M)$.
	\item The diagram commutes for every object $C\in obj(\calc)$.
	\item $T_2$ is projective with basis in $\pzm$.
\end{enumerate}
Then there exists a natural transformation $\Gamma:T_2\lrta S_2$ such the the first square commutes for every object of $\calc$.
\[
\begin{tikzcd}
T_2\ar[d, dashed, "\Gamma"] \arrow[r, "\Phi_2"] & T_1 \arrow[r, "\Phi_1"] \arrow[d, "\Theta_1"] & T_0 \arrow[d, "\Theta_0"] \\
S_2 \arrow[r, "\Psi_2"'] & S_1 \arrow[r, "\Psi_1"'] & S_0
\end{tikzcd}
\]
\end{lemma}
\begin{proof}
Let $\chi=\{x_\lambda\in T_2(M_\lambda)|\lambda\in \Lambda\}$ denote a model basis for $T_2$. Then for each $\lambda\in \Lambda$ we have a commutative diagram in $Mod_R$ such that both top row and bottom row are chain complex and the bottom row is exact. Also $T_2(M_\lambda)$ is projective 
\[
\begin{tikzcd}
T_2(M_\lambda)\ar[d, dashed, "\gamma_\lambda"] \arrow[r, "f_2"] & T_1(M_\lambda) \arrow[r, "f_1"] \arrow[d, "t_1"] & T_0(M_\lambda) \arrow[d, "t_0"] \\
S_2(M_\lambda) \arrow[r, "g_2"'] & S_1(M_\lambda) \arrow[r, "g_1"'] & S_0(M_\lambda)
\end{tikzcd}
\]
$\im t_1\circ f_2\subset \im g_2$ because $g_1\circ t_1\circ f_2=h\circ f_1\circ f_2=0$, hence $\im t_1\circ f_2\subset \ker g_1=\im g_2$. For each $x_\lambda$, there is a $y_\lambda\in S_2(M_\lambda)$ such that $g_2(y_\lambda)=t_1\circ f_2(x_\lambda)$ because $g_2$ is surjective. By universal property of projective module, we get a morphism $\gamma_\lambda:T_2(M_\lambda)\lrta S_2(M_\lambda)$ such that $\gamma_\lambda(x_\lambda)=y_\lambda$. (Even $y_\lambda$ are fixed, $\gamma_\lambda$ is not a unique morphism as shown in the definition of projective modules)
\[
\begin{tikzcd}
 & T_2(M_\lambda) \arrow[d, "t_1\circ f_2"] \arrow[ld, "\exists \gamma"', dashed] &  \\
S_2(M_\lambda) \arrow[r, "g_2"] & \im g_2 \arrow[r] & 0.
\end{tikzcd}
\]
(It also makes the first square of the previous diagram commutes).
We then know by~\ref{conj:lemma1}, when fix $\gamma_\lambda$ there exists a unique natural transformation $\Gamma: T_2\lrta S_2$ such that
$$
\Gamma(M_\lambda)=\gamma_\lambda.
$$ 

It remains to check thus constructed $\Gamma$ makes the functor diagram commutes.
At the level of morphism of modules, we know $\Psi(M_\lambda)\circ \Gamma(M_\lambda)=\Theta_1(M_\lambda)\circ \Phi_2(M_\lambda)$.
 
 Then we know the first square of functor diagram commutes by the uniqueness in~\ref{conj:lemma1}. $\Phi\circ \Gamma=\Theta\circ \Phi_2$.
\end{proof}

\begin{proof}{of theorem A.1.1}

1. \underline{Such $\Phi_\bullet$ exists}: 
We need to construct natural transformations $\Phi_n:S_n\lrta T_n$ such that the following diagram commutes.
\[\begin{tikzcd}
\cdots \arrow[r, "\pd"] & T_2 \arrow[r, "\pd"] \arrow[d, "\Phi_2", dashed] & T_1 \arrow[r, "\pd"] \arrow[d, "\Phi_1", dashed] & T_0 \arrow[r] \arrow[d, "\Phi_0", dashed] & H_0(T_\bullet) \arrow[r] \arrow[d, "\Theta"] & 0 \\
\cdots \arrow[r, "\pd'"] & S_2 \arrow[r, "\pd'"] & S_1 \arrow[r, "\pd'"] & S_0 \arrow[r] & H_0(S_\bullet) \arrow[r] & 0
\end{tikzcd}
\]
With Lemma~\ref{appendixA:lem:A.1.4} in hand, we can induct on $n$ to concatenate the ``ladders''.

For $n=0$, we have
\[
\begin{tikzcd}
T_0 \arrow[r] \arrow[d, "\exists \Phi_0"', dashed] & H_0(T_\bullet) \arrow[r] \arrow[d] & 0 \arrow[d] \\
S_0 \arrow[r] & H_0(S_\bullet) \arrow[r] & 0
\end{tikzcd}
\]


If we find $\Phi_i, 0\leq i\leq n$ such that makes the first $n$ ladder commute, we can find a $\Phi_{n+1}$ that makes the diagram commute
\[
\begin{tikzcd}
T_{n+1} \arrow[r] \arrow[d, "\exists \Phi_{n+1}"', dashed] & T_n \arrow[r] \arrow[d, "\Phi_n"] & T_{n-1} \arrow[d, "\Phi_{n-1}"] \\
S_{n+1} \arrow[r] & S_n \arrow[r] & S_{n-1}
\end{tikzcd}
\]

2. \underline{$H_0(\Phi_\bullet)=\Theta$}: This is a direct result of the corresponding diagram of  chain complexes. On each object $C$, $H_0(\Phi_\bullet)(C): H_0(T_\bullet(C))\lrta H_0(S_\bullet(C)):\lgl c\rgl\mapsto \lgl \Phi_0(c)\rgl$
Then we have $H_0(\Phi_\bullet)(C)=\Theta(C)$ because the right most square commutes when applied to object $C$.

3. \underline{$\Phi_\bullet$ is unique up to natural chain homotopy} Suppose now we have two such maps $\Psi_\bullet,\Phi_\bullet: T_\bullet\lrta S_\bullet$. We need to find natural transformation $\Upsilon_n: T_{n}\lrta S_{n+1}$ for $n\geq -1$ such that 
$$
\pd'\Upsilon_n+\Upsilon_{n-1}\pd=\Phi_n-\Psi_n.
$$
Denote the difference $\Phi_n-\Psi_n=:\Xi_n$. We define $\Upsilon_{-1}=0$ and proceed inductively. We have a diagram
\[
\begin{tikzcd}
T_0 \arrow[r] \arrow[d, "\exists\Upsilon_0"', dashed] & T_0 \arrow[d, "\Xi_0"] \arrow[r] & 0 \arrow[d, "0"] \\
S_1 \arrow[r] & S_0 \arrow[r] & H_0(S_\bullet),
\end{tikzcd}
\]
where we have used~\ref{conj:lemma2} again. Inductively, If we have constructed $\Upsilon_{n-1}$, we have the following diagram
\[
\begin{tikzcd}
T_{n+1} \arrow[r, "id"] \arrow[d, "\exists \Upsilon_{n}"', dashed] & T_n \arrow[d, "\Xi_n-\Upsilon_{n-1}\circ\pd"] \arrow[r] & 0 \arrow[d, "0"] \\
S_{n+1} \arrow[r, "\pd'"] & S_n \arrow[r] & S_{n-1}
\end{tikzcd}
\] 
By induction hypothesis, the above diagram commutes because
$$
\begin{aligned}
\pd'(\Xi_n-\Upsilon_{n-1}\pd)\\
&=\pd'\Xi_n-\pd'\Upsilon_{n-1}\pd\\
&=\pd'\Xi_n-(\Xi_{n-1}-\Upsilon_{n-1}\pd)\pd\\
&=\pd'\Xi_n-\Xi_{n-1}\pd\\
&=0.
\end{aligned}
$$
Hence, we can construct $\Upsilon_{n}$ by~\ref{conj:lemma2}.
\end{proof}


\section{Model categories}
Model categories catch the essence of homotopy. In order to study homotopic invariant, we need machinery to construct (weak) homotopy invariant functors.

\begin{definition}
A \textbf{model category} on a category $\calc$ consists of three distinguished classes of morphisms:\textbf{weak equivalences, fibrations} and \textbf{cofibrations} and two functorial factorizations $(\alpha,\beta)$ and $(\gamma,\delta)$ subject to the following axioms. Each classes of morphisms contain all the identities and are closed under composition of morphisms.
\begin{enumerate}[label=MC\arabic*:]
\item Finite limits and colimits exists in $\calc$
\item If $f,g$ are maps in $\calc$ such that $g\circ f$ is defined and if two of the three maps $f,g,g\circ f$ are weak equivalences, then so is the third.
\item If $f$ is a retract of $g$ and $g$ is a fibration, cofibration or weak equivalence, then so if $f$.
\item Given a commutative diagram of the 
\end{enumerate} 
\end{definition}

Similar with the localization of rings, we can inverse morphisms for some collection of morphisms.
\begin{definition}\textbf{Localization of a category}:
Given a category $\calc$ and some class $W$ of morphisms in $\calc$. (similar to localization of rings, we don't need $W$ to be multiplicatively closed, because the construction will naturally lead to multiplicative closure of $W$). We can define it by universal property: there is a natural localization functor $\calc\lrta W^{-1}\calc$ and given any other category $\mathcal{D}$, a functor $F:\calc\lrta\mathcal{D}$ factors uniquely through $\calc\lrta W^{-1}\calc$ iff $F$ sends every morphism in $W$ to an isomorphism in $\mathcal{D}$
\[
\begin{tikzcd}
\calc \arrow[d, "\iota_W"'] \arrow[r, "F"] & \mathcal{D} \\
W^{-1}\calc \arrow[ru, "\exists!"', dashed] & 
\end{tikzcd}
\]
\end{definition}

\chapter{Derived functors and derived categories}
\section{More on Tor and Ext}
%----------------------------------------------------------------------------------------
%	BIBLIOGRAPHY
%----------------------------------------------------------------------------------------

\chapter*{Bibliography}
\addcontentsline{toc}{chapter}{\textcolor{ocre}{Bibliography}}

%------------------------------------------------

\section*{Articles}
\addcontentsline{toc}{section}{Articles}
\printbibliography[heading=bibempty,type=article]

%------------------------------------------------

\section*{Books}
\addcontentsline{toc}{section}{Books}
\printbibliography[heading=bibempty,type=book]

%----------------------------------------------------------------------------------------
%	INDEX
%----------------------------------------------------------------------------------------

\cleardoublepage
\phantomsection
\setlength{\columnsep}{0.75cm}
\addcontentsline{toc}{chapter}{\textcolor{ocre}{Index}}
\printindex

%----------------------------------------------------------------------------------------

\end{document}
