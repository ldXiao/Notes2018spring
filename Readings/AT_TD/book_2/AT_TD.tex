%%%%%%%%%%%%%%%%%%%%%%%%%%%%%%%%%%%%%%%%%
% The Legrand Orange Book
% LaTeX Template
% Version 2.3 (8/8/17)
%
% This template has been downloaded from:
% http://www.LaTeXTemplates.com
%
% Original author:
% Mathias Legrand (legrand.mathias@gmail.com) with modifications by:
% Vel (vel@latextemplates.com)
%
% License:
% CC BY-NC-SA 3.0 (http://creativecommons.org/licenses/by-nc-sa/3.0/)
%
% Compiling this template:
% This template uses biber for its bibliography and makeindex for its index.
% When you first open the template, compile it from the command line with the 
% commands below to make sure your LaTeX distribution is configured correctly:
%
% 1) pdflatex main
% 2) makeindex main.idx -s StyleInd.ist
% 3) biber main
% 4) pdflatex main x 2
%
% After this, when you wish to update the bibliography/index use the appropriate
% command above and make sure to compile with pdflatex several times 
% afterwards to propagate your changes to the document.
%
% This template also uses a number of packages which may need to be
% updated to the newest versions for the template to compile. It is strongly
% recommended you update your LaTeX distribution if you have any
% compilation errors.
%
% Important note:
% Chapter heading images should have a 2:1 width:height ratio,
% e.g. 920px width and 460px height.
%
%%%%%%%%%%%%%%%%%%%%%%%%%%%%%%%%%%%%%%%%%

%----------------------------------------------------------------------------------------
%	PACKAGES AND OTHER DOCUMENT CONFIGURATIONS
%----------------------------------------------------------------------------------------

\documentclass[11pt]{book} % Default font size and left-justified equations

%----------------------------------------------------------------------------------------

\input{structure} % Insert the commands.tex file which contains the majority of the structure behind the template

\begin{document}

%----------------------------------------------------------------------------------------
%	TITLE PAGE
%----------------------------------------------------------------------------------------

\begingroup
\thispagestyle{empty}
\begin{tikzpicture}[remember picture, overlay]
\node[inner sep=0pt] (background) at (current page.center) {\includegraphics[width=\paperwidth]{blue_background}};
\draw (7,-5) node{\Huge\centering\bfseries\sffamily\parbox[c][][t]{\paperwidth}{\centering Algebraic Topology\\[15pt] % Book title
{\Large  A solution manual by and for stupid student}\\[20pt] % Subtitle
{\huge Vector\_Cat }}}; % Author name
\end{tikzpicture}
\vfill
\endgroup

%----------------------------------------------------------------------------------------
%	COPYRIGHT PAGE
%----------------------------------------------------------------------------------------

\newpage
~\vfill
\thispagestyle{empty}

\noindent Copyright \copyright\ 2013 John Smith\\ % Copyright notice

\noindent \textsc{Published by Publisher}\\ % Publisher

\noindent \textsc{book-website.com}\\ % URL

\noindent Licensed under the Creative Commons Attribution-NonCommercial 3.0 Unported License (the ``License''). You may not use this file except in compliance with the License. You may obtain a copy of the License at \url{http://creativecommons.org/licenses/by-nc/3.0}. Unless required by applicable law or agreed to in writing, software distributed under the License is distributed on an \textsc{``as is'' basis, without warranties or conditions of any kind}, either express or implied. See the License for the specific language governing permissions and limitations under the License.\\ % License information

\noindent \textit{First printing, March 2013} % Printing/edition date

%----------------------------------------------------------------------------------------
%	TABLE OF CONTENTS
%----------------------------------------------------------------------------------------

%\usechapterimagefalse % If you don't want to include a chapter image, use this to toggle images off - it can be enabled later with \usechapterimagetrue

\chapterimage{blue_chap_head.pdf} % Table of contents heading image

\pagestyle{empty} % No headers

\tableofcontents % Print the table of contents itself

\cleardoublepage % Forces the first chapter to start on an odd page so it's on the right

\pagestyle{fancy} % Print headers again

%----------------------------------------------------------------------------------------
%	PART
%----------------------------------------------------------------------------------------

%\part{Homotopy theories}
\chapter{Topological spaces}
\chapter{Fundamental groups}
\chapter{Covering spaces}
\chapter{Elementary homotopy theory}
\section{The mapping cylinder}
\begin{definition}
Given a continuous map $f:X\lrta Y$ of topological spaces, one can define its \textbf{mapping cylinder} as a pushout (fibered coproduct) 
\[
\tiny
\begin{tikzcd}
B &  &  &  \\
 & Z(f) \arrow[lu, "\exists!", dashed] &  & X\times I \arrow[ll] \arrow[ld, "f\times id"'] \arrow[lllu] \\
 &  & Y\times I \arrow[lu, "r"] &  \\
 & Y \arrow[uu] \arrow[ru, "i_0"] \arrow[luuu] &  & X \arrow[ll, "f"] \arrow[uu, "i_0"'],
\end{tikzcd}
\]
Set-theoretically, the mapping cylinder is usually represented as the quotient space $(X\times I \coprod Y)/\sim$, where $f(x)\sim (x,0)$.
We use $Mf$ to denote it. (other notations are used including $Mf$, $M_f$ and $\text{Cyl}(f)$.)
\end{definition}
Notice that it is $Mf$ rather than $Y\times I$ that plays the role of pushout because the map $r$ is not unique. Our only restriction on $r$ is $r\circ j=id$, where $j: Mf\lrta Y\times I$ is the map that restricts to $f\times id$ on $X\times I$ and restricts to $i_0$ on $Y$.
\begin{remark}
Another equivalent definition is used in tom Dieck.
\end{remark}

In the following, we consider $X\coprod Y$ as subspace of $Z(f)$ via the map $J:J(x)=[(x,0)]$ and $J(y)=[y]$. Then we consider a homotopy commutative diagram
\[
\begin{tikzcd}
X \arrow[r, "f"] \arrow[d, "\alpha"'] & Y \arrow[d, "\beta"] \\
X' \arrow[r, "f'"] & Y',
\end{tikzcd}
\]
where the diagram commutes up to a homotopy $\Psi: f'\circ \alpha\simeq \beta \circ f$. 
\chapter{Cofibrations and fibrations}
\chapter{Homotopy groups}
\chapter{Stable homotopy. Daulity}
\chapter{Cell complexes}
%----------------------------------------------------------------------------------------
%	CHAPTER 1
%----------------------------------------------------------------------------------------


%---------------------------------------------------------------------------------------

%\part{Homologies}

\chapter{Singular homology}
\chapter{Homology}
\section{The Axioms of Eilenberg and Steenrod}


\chapter{Homological algebra}
\section{Diagrams}
\section{Exact sequences}
\section{Chain complex}
\section{Cochain complex}
\section{Natural chain maps and homotopies}
\section{Linear algebra of chain complexes}
\begin{exr}
Tensor product is compatible with chain homotopy. Let $s:  f \simeq g: C\lrta C'$  be a chain homotopy. Then $s\otimes id :f\otimes id  \simeq g \otimes id : C\otimes D \lrta C' \otimes D$ is a chain homotopy.
\end{exr}
\begin{proof}
\underline{Know}: $ s\pd_{C}+\pd_{C'} s=f-g$

\underline{Want}: $(s\otimes id_D)\pd_{C\otimes D} +\pd_{C'\otimes D} (s\otimes id_D)=f\otimes id_D-g\otimes id_D$. 

$C\otimes D$ is generated by pure tensors like $c'_n\otimes d_m$, therefore we can  check the formula on element $c_n\otimes d_m\in C_n\otimes D_m$
$$
\begin{aligned}
&(s\otimes id_D)\pd_{C\otimes D}(c_n\otimes d_m)\\
&=(s\otimes id_D)\left(\pd_C c_n\otimes d_m +(-1)^n c_n\otimes\pd_D d_m\right)\\
&=s\circ \pd_C c_n\otimes d_m+(-1)^n s c_n\otimes \pd_D d_m
\end{aligned}
$$
and
$$
\begin{aligned}
&\pd_{C'\otimes D} (s\otimes id_D)(c_n\otimes d_m)\\
&=\pd_{C'\otimes D} (s c_n\otimes d_m)\\
&= \pd_{C'} s c_n \otimes d_m+(-1)^{\deg(sc_n)}sc_n\otimes \pd_D d_m,
\end{aligned}
$$
where $\deg (sc_n)=n-1$. Then we have
$$
\begin{aligned}
&\left(\pd_{C'\otimes D} (s\otimes id_D)+(s\otimes id_D)\pd_{C\otimes D}\right) (c_n\otimes d_m)\\
&=(s\pd_C+\pd_{C'} s)c_n \otimes d_m+0\\
&=(f\otimes id_D-g\otimes id_D)(c_n\otimes d_m)
\end{aligned}
$$
We are done. Also we can generalize this statemnt to 

 Let $s:  f \simeq g: C\lrta C'$ and  $t:  p \simeq q: D\lrta D'$ be chain homotopies. Then $s\otimes t :f\otimes p  \simeq g \otimes q : C\otimes D \lrta C' \otimes D'$ is a chain homotopy. We easily conclude by $s\otimes id$ and $id\otimes s$ are chain homotopy and composition of chain homotopies is a chain homotopy.
\end{proof}
\section{Tor and Ext}
\section{Universal coeffcients}
\section{The K\"unneth Formula}
\chapter{Cellular homology}
\chapter{Partition of unity in homotopy}
\part{More}

%----------------------------------------------------------------------------------------
%	BIBLIOGRAPHY
%----------------------------------------------------------------------------------------

\chapter*{Bibliography}
\addcontentsline{toc}{chapter}{\textcolor{ocre}{Bibliography}}

%------------------------------------------------

\section*{Articles}
\addcontentsline{toc}{section}{Articles}
\printbibliography[heading=bibempty,type=article]

%------------------------------------------------

\section*{Books}
\addcontentsline{toc}{section}{Books}
\printbibliography[heading=bibempty,type=book]

%----------------------------------------------------------------------------------------
%	INDEX
%----------------------------------------------------------------------------------------

\cleardoublepage
\phantomsection
\setlength{\columnsep}{0.75cm}
\addcontentsline{toc}{chapter}{\textcolor{ocre}{Index}}
\printindex

%----------------------------------------------------------------------------------------

\end{document}
