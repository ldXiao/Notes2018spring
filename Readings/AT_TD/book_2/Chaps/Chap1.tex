\chapterimage{orange_cat.jpg}
\chapter{Some category theory}

\section{Motivation}
\begin{exr}
A category in which each morphism is an isomorphism is called a groupoid. (This notion is not important in what we will discuss. The point of this exercise is to give you some practice with categories, by relating them to an object you know well.)
\begin{enumerate}[label=(\alph*)]
\item A perverse definition of a \textit{group} is: a groupoid with one object. Make sense of this.
\item Describe a groupoid that is not a group.
\end{enumerate}
\end{exr}
\begin{proof}
\begin{enumerate}[label=(\alph*)]
\item 
\end{enumerate}
\end{proof}
\section{Categories and functors}

\section{Universal properties determine an object up to unique isomorphism}

\section{Limis and colimits}

\section{Adjoints}

\section{An introduction to Abelian categroies}

\section{Spectral sequences}