\chapter{Some properties of schemes}
\section{Topological properties}
\begin{exr}
Show that $\proj^n_k$ is irreducible.
\end{exr}
\begin{proof}
In this case $S_\bullet=k[x_1,..,x_n]$ and $S_+=(x_1,...,x_n)$.
Assume $\proj^n_k$ can be written as union of two proper closed subsets: $V(I_1)\cup V(I_2)$, where $I_1,I_2$ are two homogeneous ideal in $k[x_1,..,x_n]$, also we require $I_1,I_2$ not contain $S_+$.

$(0)$ is still a homogeneous prime ideal in $k[x_1,..,x_n]$ because it is integral domain and it does not contain the irrelevant ideal. W.l.o.g, assume $V(I_1)\ni [(0)]$, which means $I_1\subset (0)\Lrta I_1=(0)$. But $V(I_1)=V((0))=\Proj S_\bullet=\proj^n_k$, which contradicts the assumption that $V(I_1)$ is a proper closed subset.
\end{proof}

\begin{exr}\label{chap5exr:bijection_closed_generic_point_scheme}
Exercise~\ref{chap3exr:bijection_closed_subsets_points_affine} showed that there is a bijection between irreducible closed subsets and points for affine schemes (the map sending a point $p$ to the closed subset $\overline{\{p\}}$ is a bijection). Show that this is true of schemes in general.
\end{exr}
\begin{proof}
Given a scheme $X$, we want a bijection
$$
\begin{aligned}
X&\lrta \{Z\subset X:Z\text{ closed irreducible}\}\\
z&\longmapsto \overline{\{z\}}.
\end{aligned}
$$ 
We already showed this bijection exists for affine schemes. This map can be naturally extended to general scheme, with the Zariski topology.

We only need to show 
\begin{itemize}
\item $\overline{\{z\}}$ is irreducible in $X$, and
\item $\overline{\{z\}}\neq\overline{\{x\}}$ if $z\neq x$, and
\item Each irreducible closed subset in $X$ is of the form.
\end{itemize}

Each $\{z\}$ is irreducible in $X$, the closure of a irreducible subset is irreducible by~\ref{chap3exr:irreducible_subset_dense_open_irreducible_closure}.

If $\overline{\{z\}}=\overline{\{x\}}$, they are covered by some affine open subsets $U_i$. By definition of closure, each $U_i$ contain $z,x$. $\overline{\{z\}}\cap U_i=\overline{\{z\}}\cap U_i$ are the closure of $z,x$ in the subset topology of $U_i$, they are irreducible in $U_i$ because $\{z\},\{x\}$ are irreducible in $U_i$ with subset topology. Then because the bijection exists in affine schemes, there is a unique generic point corresponding to $\overline{\{z\}}\cap U_i$, $z=x$ in $U_i$, which means $z=x$ in $X$.

Given an irreducible subset $Z\subset X$. Let $U$ be an affine open subset such that $Z\cap U\neq \emptyset$. $Z\cap U$ is an open set in $Z$ with subspace topology on $Z$. By~\ref{chap3exr:irreducible_subset_dense_open_irreducible_closure}, $Z\cap U$ is dense and irreducible in $Z$, with subspace topology. Then it is also irreducible in $U$ with subspace topology on $U$. Hence it corresponds to a unique generic point in $U$. $\exists z\in U\cap Z$ s.t. $Z\cap U=\overline{\{z\}}'$ is the closure of $z$ in $Z\cap U$. However, we already know $Z\cap U$ is dense in $Z$, therefore $\overline{\{z\}}=Z$.

We have established to bijection between the irreducible closed subsets and generic points of the irreducible closed subsets.
\end{proof}


\begin{exr}
Prove that if $X$ is a scheme that has a finite cover $X = \cup_{i=1}^n \spec A_i$ where $A_i$ is Noetherian, then $X$is a Noetherian topological space (We will soon call a scheme with such a cover a \textbf{Noetherian scheme}) Hint: show that a topological space that is a finite union of Noetherian subspaces is itself Noetherian.
\end{exr}
\begin{proof}
We have shown in~\ref{chap3exr:Noetherian_rings_Noetherian_topological_spaces} that affine scheme corresponding to a Noetherian ring is Noetherian topological space.

It suffices to check finite union of Noetherian subspaces $X=\cup_i^nX_i$ is itself Noetherian.

A topological space is called Noetherian if it satisfies the descending chain condition for closed subsets. Given an descending chain of closed subsets in $X$.
$$
Z_1\supset Z_2\supset\cdots\supset Z_k\supset \cdots
$$
$Z_k=\cup_i^n(X_i\cap Z_k)$, $X_i\cap Z_k$ would give a descending chain of closed subsets in the subspace $X_i$ with subspace topology on $X_i$. Each would stabilize after $m_i$. Choose $m=\max\{m_i\}_{1\leq i\leq n}$, the descending chain $\{Z_k\}=\{\cup_i^n(X_i\cap Z_k)\}$ would also stabilize after $m$.
\end{proof}


\begin{exr}\label{chap5exr:quasicompact_scheme_union_affine_subschemes}
Show that a scheme $X$ is quasicompact if and only if it can be written as a finite union of affine open subschemes. (Hence $\proj^n_A$ is quasicompact for any ring $A$.)
\end{exr}
\begin{proof}
One direction is easy, if a scheme $X$ can be written as union of affine open subschemes $X_i$. Given a open cover of $X$, its intersection with affine open subschemes also given an open cover of $X_i$.
 Affine open subscheme is quasicompact, we can select a finite subcover of these intersection covers, and the union gives a finite subcover of the initial cover.

 For the reverse direction, consider a quasicompact scheme $X$. By definition, for each point $x$, there is an affine open subscheme $U_x$ contain $x$. These affine open subscheme give an open cover of $X$. $\cup_{x\in X}U_x$, we can select a finite subcover $\cup_{x\in I}U_x$ of $X$. Then the scheme $X$ is the union $\cup_{x\in I}U_x$. Then $X$ can be written as a finite union of affine open subschemes.

 In particular, $\proj^n_A$ is quasicompact.
\end{proof}

\begin{exr}\label{chap5exr:quasicompact_scheme_have_closed_points}
QUASICOMPACT SCHEMES HAVE CLOSED POINTS. Show that if $X$ is a quasicompact scheme, then every point has a closed point in its closure. Show that every nonempty closed subset of $X$ contains a closed point of $X$. In particular, every nonempty quasicompact scheme has a closed point.
\end{exr}
\begin{proof}
We already showed that $\overline{\{z\}}$ is a irreducible subset in $X$. Hence it suffices to show only every nonempty closed subset of a quasicompact scheme $X$ contains a closed point of $X$. 

Given a closed subset $Z\subset X$. Observe that a point $p\in Z$ is closed in $Z$ iff $p$ is closed in $X$. 

$X$ is a finite union of affine opens $\cup_i^nU_i$ by
~\ref{chap5exr:quasicompact_scheme_union_affine_subschemes}. Consider a finite sub-collection of $\{U_i\}_{1\leq i\leq n}$ that intersects with $Z$, w.l.o.g, we assume it is $\{U_i\}_{1\leq i\leq m}$. Then a point $p\in Z$ is closed iff $p$ is closed in each $\{U_i\}_{1\leq i\leq m}$ that contains $z$.

A closed point in $U_1$ corresponds to maximal ideals in $\spec A_1$. $Z\cap U_1$ is not empty, hence contains at least one $[\scp]$ and $\overline{\{[\scp]\}}\subset Z\cap U_1$. $\scp$ is contained in a maximal ideal $\scm$. $[\scm]\in \overline{\{[\scp]\}}$, which means there is at least one closed point in $Z\cap U_1$.

We can find a closed point (in $U_1$)  $z_1\in Z\cap U_1$. If $z_1$ is also closed in other $U_i$, we are done. 

If not, we can find in $\overline{\{z_1\}}\cap U_2$ a closed point $z_2$ such that $z_2$ is closed in $U_2$. $z_2\notin U_1$ because if $z_2\in U_1$, $\overline{\{z_1\}}\cap U_1\neq \emptyset$, which means $z_1$ is not closed point in $U_1$ in the first place. 

If $z_2$ is closed in all $U_i$ (besides $U_1$ ) that contains it, we are done. If not, assume $z_2$ is not closed point in $U_3$. $\overline{\{z_2\}}\subset \overline{\{z_1\}}$. We can find $z_3\in \overline{\{z_2\}}\cap U_3$, $z_3$ can't lie in $U_1$ or $U_2$, because it would imply $z_1$ not close point in $U_1$, $z_2$ not close point in $U_2$.

We can induct on these process until we find $z_m$ is closed in all $U_i$ that contains it. The process would terminate because we started with a finite cover with affine opens. 

Any closed subset in quasicompact scheme contains a closed point. In particular, a quasicompact scheme contains a closed point and the closure of each point in a quasicompact scheme contains a closed point.
\end{proof}

\begin{exr}\label{chap5exr:characterization_quasiseparatedness}
Show that a scheme is quasiseparated if and only if the intersection of any two affine open subsets is a finite union of affine open subsets.
\end{exr}
\begin{proof}
The $\Lrta$ direction is trivial, affine open subsets are quasicompact, if the scheme is quasiseparated, then the intersection of any two affine open subsets $A,B$ would be quasicompact and by~\ref{chap5exr:quasicompact_scheme_union_affine_subschemes} this intersection should be finite union of affine open subsets in $A\cap B$. Because $A\cap B$ is endowed with subspace topology of $X$, $A\cap B$ is also finite union of affine open subsets in $X$.

For the reverse direction. In a scheme any quasicompact subset is a finite union of affine open subsets, $Y=\cup_i^n U_i$ and $Z=\cup_{j}^m V_j$. Then the intersection of $Y$ and $Z$ should be 
$$
Y\cap Z=\cup_{1\leq i\leq n,1\leq j\leq m} U_i\cap V_j.
$$
If each $U_i\cap V_j$ can be written as finite union of affine open subsets, $Y\cap Z$ can be written as finite union of affine open subsets, which is equivalent to $Y\cap Z$ is quasicompact by~\ref{chap5exr:quasicompact_scheme_union_affine_subschemes}
\end{proof}

\begin{exr}
Show that affine schemes are quasiseparated.
\end{exr}
\begin{proof}
We have to firstly characterize a quasicompact open subset in $\spec A$.
(Do not try to characterize all affine opens in an affine scheme in general, it is difficult. The hint seems to be a trap)

Any open subset $U$, in $\spec A$ can written as a union of distinguished base. If in addition this set is quasicompact, we can write it as a finite union of $D(f_i)$.

$$
Y\cap Z=\cup_i^nD(f_i)\cap \cup_j^mD(g_j)=\cup_{1\leq i\leq n,1\leq j\leq m}D(f_i)\cap D(g_j)=\cup_{1\leq i\leq n,1\leq j\leq m}D(f_ig_j)
$$
which can be written as a finite union of affine opens. Then $Y\cap Z$ is quasicompact by~\ref{chap5exr:quasicompact_scheme_union_affine_subschemes}.

Then intersection of any two quasicompact opens is also quasicompact. Affine scheme is quasiseparated.
\end{proof}

\begin{exr}\label{chap5exr:characterization_quasicompact_And_quasiseparated}
Show that a scheme $X$ is quasicompact and quasiseparated if and only if $X$ can be covered by a finite number of affine open subsets, any two of which have intersection also covered by a finite number of affine open subsets.
\end{exr}
\begin{proof}
$\Lrta$ direction: $X$ can be written as a finite union of affine open subsets because $X$ is quasicompact by~\ref{chap5exr:quasicompact_scheme_union_affine_subschemes}. If in addition $X$ is quasiseparated, each intersection of these affine opens can be written as a finite union of affine open subsets by~\ref{chap5exr:characterization_quasiseparatedness}.

$\Llta$ direction: Assume $X$ can be written as a finite union of affine open subsets such that each pairwise intersection can be written as finite union of affine open subsets. Then $X$ is quasicompact by~\ref{chap5exr:quasicompact_scheme_union_affine_subschemes}. 

The quasiseparatedness part of the reverse direction is tricky. It can be cleanly proved using the diagonal morphism arguments, as shown in the first answer to this
\href{https://math.stackexchange.com/questions/9809/quasiseparated-if-finitely-covered-by-affines-in-appropriate-way}{StackExchange question}. We will give a more topological proof here.

Notice that we have proved affine schemes are quasicompact and quasiseparated. The specified affine opens are quasicompact and quasiseparated in the subspace topology. Given $Z$, $Y$ two quasicompact subsets in $X$. Each is covered by finitely many of specified affine opens.
$$
Z\cap Y=\cup_i(Z\cap Y\cap U_i)=\cup_i(Z\cap U_i)\cap (Y\cap U_i)
$$

\underline{Claim}: $Z\cap U_i$, $Y\cap U_i$ are quasicompact in $U_i$ for all $i$.

If we have the claim, we know $(Z\cap U_i)\cap (Y\cap U_i)$ is quasicompact because $U_i$ is quasiseparated. And a finite union of quasicompact space is always  quasicompact. Hence we are done.

Then the problem reduces to prove the claim:

\underline{proof of the claim}: $Z$ is quasicompact in $X$, therefore can be written as finite union of affine opens $Z=\cup_k Z_k$.
$$
Z=\cup_{k,i}(Z_k\cap U_i)
$$
$Z_k$ is affine open we can choose a distinguished base of $W_{k,i,\alpha}:=D(f_{k,i,\alpha})$, each $W_{k,i,\alpha}$ is quasicompact.

$Z_k\cap U_i$ is an open set in $Z_k$, hence can be written as union of $W_{k,\alpha}$ (not necessarily finite). We have
$$
Z=\cup_{k,i}(Z_k\cap U_i)=\cup_{k,i}\cup_{\alpha\in L_{ki}}W_{k,i,\alpha},
$$
where $L_{ki}$ is not finite in general. But because $Z$ is quasicompact, there is a finite subcover, we can choose a finite subset $F_{ki}$ for each $L_{ki}$ so that
$$
Z=\cup_{k,i}\cup_{\alpha\in F_{ki}}W_{k,i,\alpha}.
$$
(Now $\cup_{\alpha\in F_{ki}}W_{k,\alpha}$ no longer cover $Z_k\cap U_i$ in general)
Intersect it with $U_i$, we get
$$
Z\cap U_1=\cup_{k,i}\cup_{\alpha\in F_{ki}}W_{k,i,\alpha}\cap U_1.
$$
The condition that ``$U_i\cap U_j$ can be written as a finite union of affine opens $\cup_{l\in M_{ij}}V_{ij;l}$ '' would be used in the next step, where $M_{ij}$ is a finite index set depending on $i,j$.
$$
Z\cap U_1=\cup_{k,i}\cup_{\alpha\in F_{ki}}W_{k,i,\alpha}\cap U_1\overset{(*)}{=}\cup_{k,i}\cup_{\alpha\in F_{ki}}W_{k,i,\alpha}\cap U_1\cap U_i=\cup_{k,i}\cup_{\alpha\in F_{ki}}\cup_{l\in M_{1i}} W_{k,i,\alpha}\cap V_{1i;l}.
$$
where $(*)$ holds because each $W_{k.i.\alpha}\subset U_i$. $W_{k,i,\alpha}$ and $ V_{1i;l}$ are quasicompact subset in $U_i$ and their intersection is quasicompact because $U_i$ is quasiseparated. $Z\cap U_1$ is finite union of quasicompact subsets, therefore is compact. We proved the claim there fore the quasiseparatedness part of the reverse direction.
\end{proof}

\begin{exr}\label{chap5exr:Proj_A-scheme_quasicompact_quasiseparated}
Show that all projective $A$-schemes are quasicompact and quasiseparated. 
\end{exr}
\begin{proof}
By definition a projective $A$-scheme $X$ is defined to be $\Proj S_\bullet$, where $S_\bullet$ is a finitely generated graded ring over $A$. A graded ring is finitely generated iff $S_+$ is finitely generated. Assume $S_+=(x_1,...,x_n)$, where each $x_i$ is homogeneous, we can consider an open cover of $X$:
$$
D(x_i).
$$
We already showed that $D(x_i)$ is affine open subset that is isomorphic to $\spec ((S_\bullet)_{x_i})_0$.

$$
D(x_i)\cap D(x_j)=D(x_ix_j).
$$

In summary, $X=\Proj S_\bullet$ can be covered by a finite union of affine open subsets and each intersection of these affine open subsets can be written again as finite union of affine open subsets. Then we can conclude that $X=\Proj S_\bullet$ is quasicompact and quasiseparated by~\ref{chap5exr:characterization_quasicompact_And_quasiseparated}.
\end{proof}

\begin{exr}
Let $X = \spec k[x_1, x_2,...]$, and let $U$ be $X - [\scm]$ where $\scm$ is the maximal ideal $(x_1,x_2,...)$. Take two copies of $X$, glued along $U$ (``affine $\infty$-space with a doubled origin'', see Example 4.4.5 and Exercise~\ref{chap4exr:affine_doubled_origin}  for ``finite-dimensional'' versions). Show that the result is not quasiseparated.
\end{exr}
\begin{proof}
We glue two copies of $\spec k[x_1,x_2,...]$ $X,X'$along the open
$$
U:=\spec k[x_1,x_2,...]-[\scm],
$$
and denote the resulting scheme as $\affn^\infty_{(2)}$.
We showed in Exercise~\ref{chap3exr:affine_scheme_quasicompact}(b) that $U$ is not quasicompact in $\spec k[x_1,x_2,...]$.

Denote the the quotient map $\pi:X\coprod X'\lrta \affn^\infty_{(2)}$.
$\affn^\infty_{(2)}$ is equipped with quotient topology, which make the quotient map continuous. The images $\pi(X)$ and $\pi(X')$ in $\affn^\infty_{(2)}$ are quasicompact $\spec B$ is quasicompact for any ring $B$. And $\pi (U)$ is not quasicompact because $\cup_{i\in\intg}\pi(D(x_i))$ is an open cover of $\pi(U)$ with no finite subcover.

$\pi(X)\cap \pi(X')=\pi(U)$. Intersection of two quasicompact open subsets is not quasicompact, therefore $\affn^\infty_{(2)}$ is not quasiseparated.
\end{proof}

\section{Reducedness and integrality}

\begin{exr}\label{chap5exr:reducedness_stalk_local}
(REDUCEDNESS IS A stalk-local PROPERTY, I.E., CAN BE CHECKED AT STALKS). Show that a scheme is reduced if and only if none of the stalks have nonzero nilpotents. Hence show that if $f$ and $g$ are two functions (global sections of $\calo_X$) on a reduced scheme that agree at all points, then $f = g$.
\end{exr}
\begin{proof}
``$\Lrta$'': A scheme $X$ is reduced  then $\calo_X(U)$ is reduced for all open subsets $U\subset X$. Assume $[(f\in \calo_X(U);U)]\in\calo_{X,p}$ is nilpotent, which means $f|_V^n=0\in\calo_X(V)$ for some open neighborhood $V$ of $p$. But $\calo_X(V)$ is reduced ring, $\Lrta f|_V=0$ and $[(f\in \calo_X(U);U)]=0$, which implies there is no non-zero nilpotents in $\calo_{X,p}$.

``$\Llta$'': Given $f\in\calo_X(U)$, assume $f^n=0$. Then its image to stalks are also nilpotent. The condition that stalks are reduced implies the image of $f$ to each stalk is zero. By~\ref{chap2exr2.4.A}, we know the morphism 
$$
\calo_X(U)\lrta \prod_{p\in U}\calo_{X,p}
$$
is injective, therefore we know $f=0\in\calo_X(U)$.

We have to clarify what it means by ``agree on all points''. It means given any affine open subset $U$, $(U,\calo_X(U))\cong (\spec A,\calo_{\spec A})$, $f|_U\equiv g|_U\mod \scp,\forall [\scp]\in \spec A$. 
$$
f|_U\equiv g|_U\mod \scp
$$
means $f|_U\equiv g|_U$ divers only by a nilpotent. Because $\calo_X(U)$ is reduced, we have $f|_U=g|_U$. $f,g$ agree when restricted to a base of topology, then $f,g$ have same germs on each stalks which means $f=g$.
\end{proof}
\begin{remark}
We should have mentioned  this before. When $f=g+n$, where $n\in\sqrt{(0)}$, they are generally not equal when taking germs
$$
\frac{n}{1}\neq \frac{0}{1}\in A_\scp.
$$
For example, $A=k[x,\epsilon]/(\epsilon^2)$, $f=x^2,g=x^2+\epsilon$. $\epsilon/1\neq 0/1\in A_{(\epsilon)}$.
\end{remark}

\begin{exr}\label{chap5exr:reduced_ring_reduced_affine_scheme}
If $A$ is a reduced ring, show that $\spec A$ is reduced. Show that $\affn^n_k$ and $\proj^n_k$ are reduced.
\end{exr}
\begin{proof}
 
We state a stronger version: 

\textbf{$\spec A$ is reduced scheme iff $A$ is a reduced ring.}

One direction is already contained in the definition of reduced scheme.

For the other direction, we prove an algebraic fact. Any localization of a reduced ring is reduced.

 Given any multiplicative subset $S\subset A$. Assume $a/s\in S^{-1}A$ is nilpotent $(a/s)^n=0$, there exists $t\in S$ s.t. $ta^n=0\in A$, but then we have $(ta)^n=0\in A$ and $A$ being reduced, hence $ta=0$. $\Lrta a/s=0\in S^{-1}A$.

Exercise~\ref{chap5exr:reducedness_stalk_local} also translate to: $A$ is reduced iff $A_\scp$ is reduced for all $[\scp]\in\spec A$. We will give a algebraic proof here. Remember that taking radical is commutative with localization, $\mathfrak{N}(A)_\scp=\mathfrak{N}(A_\scp)$. Also notice that being zero is a local property for $A$-modules, each $A_\scp$ begin reduced implies $\mathfrak{N}(A)$  is zero.

We even have a stronger version: $A$ is reduced iff $A_\scm$ is reduced for all maximal ideals in $A$. (Because the ``being zero'' local property of $A$-module says an $A$ module $M$ is zero iff $M_\scm$ is zero module for all maximal ideals in $A$.)

But note that being reduced can not be checked only on closed points for general schemes, we need quasicompactness, see Exercise~\ref{chap5exr:quasicompact_scheme_reducedness_closed_point} below.


Then $\affn^n_k$ is reduced because $k[x_1,...,x_n]$ is reduced. On the other hand, $\proj^n_k$ is reduced because it can be covered by $\affn^n_k$ and hence each stalk is reduced.
\end{proof}



\begin{exr}
Show that  $\left(k[x, y]/(y^2, xy)\right)_x$  has no nonzero nilpotent elements.
(Possible hint: show that it is isomorphic to another ring, by considering the geometric picture. Exercise~\ref{chap3exr:3.2.L} may give another hint.) Show that the only point of $\spec k[x, y]/(y^2 , xy)$ with a nonreduced stalk is the origin.

\end{exr}
\begin{proof}
Geometrically $\spec \left(k[x, y]/(y^2, xy)\right)_{\bar{x}}=$ the distinguished open $D([x])$ in $\spec k[x, y]/(y^2, xy)$. It is equivalent to the non-vanishing loci of $x$ in $V(y^2,xy)$ in the affine plain, which is the affine line with origin removed. This geometric picture gives us some intuition

\underline{Claim}:
$$
\left(k[x, y]/(y^2, xy)\right)_{\bar{x}}=k[x]_{x},
$$
where $\bar{x}$ is the image of $x$ in the quotient ring. A general element in $k[x, y]/(y^2, xy)$ is of the form
$$
p(\bar{x})+a\bar{y},
$$
where $p$ is a polynomial in $k[x]$, $a$ is a constant in $k$ and $\bar{x}\bar{y}=0$ and $\bar{y}^2=0$.
A general element in $\spec \left(k[x, y]/(y^2, xy)\right)_{\bar{x}}$ is of the form
$$
\frac{p(\bar{x})}{\bar{x}^n}.
$$
$\bar{y}/1=0$ because $\bar{x}\bar{y}=0$.
 There is a well-define ring morphism
$$
\frac{p(\bar{x})}{\bar{x}^n}\mapsto \frac{p(x)}{x^n},
$$
Which is obviously an isomorphism.

$k[x]_x$ is reduced because $k[x]$ is reduced. We then know there is no nonzero nilpotents in $\left(k[x, y]/(y^2, xy)\right)_{\bar{x}}$.

Consider the ring of $\spec \left(k[x, y]/(y^2, xy)\right)$. The prime ideals of this ring corresponds to prime ideals in $k[x,y]$ that contains $(y^2,xy)$. Hence a general prime ideal in this ring is of the form $(\bar{x}-a, \bar{y})$ or $(\bar{y})$. 
When $a\neq 0$, $\bar{x}\notin (\bar{x}-a,\bar{y})$. $\bar{x}$ is invertible in $\left(k[x, y]/(y^2, xy)\right)_{(\bar{x}-a,\bar{y})}$, hence it is further localization of $\left(k[x, y]/(y^2, xy)\right)_{\bar{x}}$ and should be reduced.

When $a=0$, we focus on the ring
$$
\left(k[x, y]/(y^2, xy)\right)_{(\bar{x},\bar{y})}.
$$
The multiplicative set consists of elements like $b(\bar{x})+c\bar{y}$, where $b(\bar{x})$ is a polynomial of $\bar{x}$ and the constant term $b_0\neq 0$. $\bar{y}(b\bar{x}+c\bar{y})^n=b_0^n\bar{y}\neq 0$, therefore $\bar{y}/1\neq 0/1$. And $\bar{y}/1$ is nilpotent.

$\left(k[x, y]/(y^2, xy)\right)_{(\bar{x},\bar{y})}$ contains at least one nonzero nilpotent $\bar{y}/1$, hence is nonreduced.
\end{proof}


\begin{exr}\label{chap5exr:quasicompact_scheme_reducedness_closed_point}
If $X$ is a quasicompact scheme, show that it suffices to check reducedness at closed points. Hint: Do not try to show that reducedness is an open condition (see Remark 5.2.2). Instead show that any nonreduced point has a nonreduced closed point in its closure, using Exercise~\ref{chap5exr:quasicompact_global_section_vanishes_on_point_nilpotent}. (This result is interesting, but we won’t use it.)
\end{exr}
\begin{proof}
Follow the hint, we prove the contrapositive: If a quasicompact scheme is not reduced then at least one of its closed points has non-reduced stalk.

By definition, we know if $X$ is not reduced then at least one of its stalk is not reduced by~\ref{chap5exr:reducedness_stalk_local}. Denote this non-reduced point by $p$, then there is a closed point $z\in \overline{\{p\}}$ by~\ref{chap5exr:quasicompact_scheme_have_closed_points}. (I suppose the hint in the body of this problem is wrong). The stalk $\calo_{X,p}$ is a localization of the stalk $\calo_{X,z}$ at closed point. $\calo_{X,p}$ is not reduced, therefore $\calo_{X,z}$ is also non-reduced.
\end{proof}



\begin{exr}\label{chap5exr:quasicompact_global_section_vanishes_on_point_nilpotent}
Suppose $X$ is quasicompact, and $f$ is a function that vanishes at all points of $X$. Show that there is some $n$ such that $f^n = 0$. Show that this may fail if $X$ is not quasicompact. (This exercise is less important, but shows why we like quasicompactness, and gives a standard pathology when quasicompactness doesn’t hold.)
\end{exr}
\begin{proof}
$f$ is a global section on a quasicompact scheme $X$ that vanishes at each point. 

$X$ is quasicompact and can be written as a finite union of affine open subschemes $\cup_i U_i$, where each $U_i\cong \spec A_i$. We then have $f|_{U_i}\calo_X(U_i)=A_i$. $f|_{U_i}$ vanishes on each point $[\scp]\in \spec A_i$ and hence is an element in the nilpotent $\mathfrak{A_i}$, $\Lrta f_{U_i}^{n_i}=0$ for some $n_i\in\intg$. There are only finitely many such affine opens, we can choose $n:=\max_i\{n_i\}$ and then $f^n|_{U_i}=0\in \calo_X(U_i)$ and there is only one element $0$ to glue to in the global section.

For the counterexample when $X$ is not quasicompact, we follow the hint and consider the union of infinite disjoint union of $\spec A_n$, where $A_n=k[\epsilon]/(\epsilon^{n})$. Each $\spec A_n$ consists of only one point $[(\epsilon)]$. Consider the function $f=\coprod_n \epsilon$. We know $f$ vanishes at each each point but is not nilpotent in the global section.
\end{proof}

\begin{exr}\label{chap5exr:integral=irreducible+reduced}
Show that a scheme $X$ is integral if and only if it is irreducible and reduced. (Thus we picture integral schemes as: ``one piece, no fuzz'')
\end{exr}
\begin{proof}
In a scheme $X$ is integral, $\calo_X(U)$ is integral for all open subsets, hence $\calo_X(U)$ is also reduced because integral domain has no nonzero zero divisors.

An integral scheme should be irreducible. Assume contrarily $X$ is reducible, and can be written as union of two closed subsets $X=Y\cup Z$. Define the complements $U:=X-Y$ and $V=X-Z$, we know $U,V$ are nonempty opens and their have empty intersection. The structure sheaf $\calo_X(U\cup V)=\calo_X(U)\times \calo_X(V)$ which is not integral in general.


For the reverse direction, we would use the hint~\ref{chap3exr:locally_ringed_space_function_value}.

\underline{Claim}: Any open subset $U$ in a irreducible space $X$ is irreducible with the subset topology.

For any other nonempty open subset $U\subset X$. By~\ref{chap3exr:irreducible_subset_dense_open_irreducible_closure}, $U$ is dense in $X$. Assume $U$ is reducible and can be written as $(Y\cap U)\cup (Z\cap U)=(Y\cup Z)\cap U$, where $Y,Z$ are nonempty proper closed subsets in $X$. 
We also know $Y\cup Z\supset U$. Hence $\overline{Y\cup Z}=\overline{Y}\cup \overline{Z}=Y\cup Z\supset \overline{U}=X$, which means $X$ is reducible, contradiction. 

Given $f,g\in\calo_X(U)$ and $fg=0\in\calo_X(U)$. Recall~\ref{chap3exr:locally_ringed_space_function_value}, the set where $f,g$ vanishes is closed. (Notice here vanishes means $f_pg_p\in\scm_p$). $U$  could be covered by the closed set $V$ $f$ vanishes and the closed set $W$ where $g$ vanishes. Then because $X$ is irreducible, we know $W$ or $V$ must be the whole space $X$. Assume $f$ vanishes on every point in $X$, we know $f=0\in \calo_X(U)$ because $U$ is a reduced scheme~\ref{chap5exr:reducedness_stalk_local}.
\end{proof}

\begin{exr}\label{chap5exr:affine_integral_scheme}
Show that an affine scheme $\spec A$ is integral if and only if $A$ integral domain.
\end{exr}
\begin{proof}
The ``only if'' direction is obvious, because $\Gamma(\spec A,\calo_{\spec A})=A$.


Consider the ``if'' direction. We simply have $\spec A$ is irreducible if $A$ is an integral domain by~\ref{chap3exr:Integral_domain_irreducible}. Also an integral domain is always reduced ring, hence $\spec A$ is  reduced by~\ref{chap5exr:reduced_ring_reduced_affine_scheme}. We know $\spec A$ is irreducible and reduced if $A$ is integral domain, hence $\spec A$ is integral by~\ref{chap5exr:integral=irreducible+reduced}.
\end{proof}


\begin{exr}\label{chap5exr:generic_point_integral_scheme_function_field}
Suppose $X$ is an integral scheme. Then $X$ (being irreducible) has a generic point $\eta$. Suppose $\spec A$ is any nonempty affine open subset of $X$. Show that the stalk at $\eta$, $\calo_{X,\eta}$ is naturally identified with $K(A)$, the fraction field of $A$. This is called the \textbf{function field} $K(X)$ of $X$. It can be computed on any nonempty open set of $X$, as any such open set contains the generic point. The reason for the name: we will soon think of this as the field of \textit{rational functions} on $X$
\end{exr}
\begin{proof}
Suppose $X$ is an integral scheme. $X$ is itself irreducible closed and hence corresponds to a unique generic point by~\ref{chap5exr:bijection_closed_generic_point_scheme}. This generic point $\eta$ is contained in any open subset of $X$. Specifically, it is contained in affine open $U:=\spec A$ and corresponds to $[(0)]\in \spec A$, where $A$ is an integral domain. $\calo_{X,\eta}\cong A_{[(0)]}=K(A)$, where localization at the prime ideal $(0)$ is isomorphic to the fraction field $K(A)$.
\end{proof}

\begin{exr}
Suppose $X$ is an integral scheme. Show that the restriction maps $\res_{U,V}: \calo_X(U) \lrta \calo_X(V)$ are inclusions so long as $V\neq \emptyset$. Suppose $\spec A$ is any nonempty affine open subset of $X$ (so $A$ is an integral domain). Show that the natural map $\calo_X(U) \lrta  \calo_{X,\eta} = K(A)$ (where $U$ is any nonempty open subset) is an inclusion.
\end{exr}
\begin{proof}
In fact, the first question relies on the second. Because we have the composition of maps
$$
\calo_X(U)\overset{\res_{U,V}}{\lrta}\calo_X(V)\lrta\calo_{X,\eta},
$$
it suffices to prove $\calo_X(U)\lrta \calo_{X,\eta}$ is injective. ($\res_{U,V}$ has to be injective because its composition with an injection is again an injection.)

\underline{Claim}: $\calo_X(U)\lrta K(X)$ is injective.

Assume a section $f\in\calo_X(U)$ and $f_\eta=0$. \underline{Want}: $f=0\in \calo_X(U)$. It suffices to prove that $f_\eta\Lrta f|_W=0$ for all affine open subsets contained in $U$. 

For an affine open $W=\spec A$, the natural morphism
$$
\calo_X(W)\lrta \calo_{X,\eta}=K(A)
$$
as shown in~\ref{chap5exr:generic_point_integral_scheme_function_field}. This map must be the canonical inclusion because the stalk at $[(0)]$ is a further localization of any $A_f=\calo_{\spec  A}(D(f))$.

We have prove both questions.
\end{proof}

\section{Properties of schemes that can be checked ``affine-locally''}
\begin{exr}
Show that locally Noetherian schemes are quasiseparated.
\end{exr}
\begin{proof}
In a locally Noetherian scheme, each affine open is isomorphic to $\spec A$ for some Noetherian ring $A$. By~\ref{chap5exr:characterization_quasiseparatedness}, we can check whether any intersection of two affine open can be written as finite union of affine opens.

Given $U=\spec A$ and $V=\spec B$ for $A,B$ Noetherian rings, consider then intersection $U\cap V$.  By proposition 5.3.1, we know $U\cap V$ can be written as union of open sets such that are simultaneously distinguished open subschemes in $\spec A$ and $\spec B$. But notice by~\ref{chap3exr:open_subset_Noetherian_space_quasicompact} that open subset of Noetherian space is quasicompact, we can cover $U\cap V$ with finitely many simultaneous distinguished opens, which are affine opens.

We don't need Prop 5.3.1 to prove this statement but it is more clear to see which of the affine opens are selected.
\end{proof}


\begin{exr}\label{chap5exr:Noetherian_scheme_finite_irreducible_compoenents}
Show that a Noetherian scheme has a finite number of irreducible components. (Hint: Proposition 3.6.15.) Show that a Noetherian scheme has a finite number of connected components, each a finite union of irreducible components.
\end{exr}
\begin{proof}
\underline{Claim}: The underline topological space of a Noetherian scheme is Noetherian.

Note that a Noetherian scheme is locally Noetherian and quasicompact, we know a Noetherian scheme $X$ can be covered by finitely many Noetherian affine subschemes.

\underline{subclaim}: A finite union of Noetherian subspace is also Noetherian.

We can prove the subclaim directly by checking the descending chain condition on closed subsets. Given a chain $V_0\supset V_1\supset \cdots$, each $V_i$ can be covered by finitely many $U_j$, where $U_j$ is Noetherian subsapce. It renders a descending chain of closed subset in $U_j$ with subset topology, hence will stabilize after $n_j$. Choose $N=\max\{n_j\}$, we have $V_n=\cup_n (V_N\cap U_j)$ would stabilize when $n\geq N$.


Then in particular, we know a Noetherian scheme is Noetherian topological space and by Prop 3.6.15, we know it has only finitely many irreducible components.


Exercise~\ref{chap3exr:topology_irreducible_components_connected_components} already showed that connected components in topological space $X$ are unions of irreducible components of $X$. But in the case of Noetherian scheme, there are only finitely many irreducible components, hence $X$ is the union of finitely many connected components and each connected component is union of finitely many irreducible components.
\end{proof}

\begin{exr}\label{chap5exr:Noetherian_scheme_integral_stalk}
Show that a Noetherian scheme $X$ is integral if and only if $X$ is nonempty and connected and all stalks $\calo_{X,p}$ are integral domains. Thus in ``good situations'', integrality is the union of local (stalks are integral domains) and global (connected) conditions.
\end{exr}
\begin{proof}
We follow the hint and recall Exercise~\ref{chap5exr:integral=irreducible+reduced} says integral $=$ irreducible $+$ reduced. Note that ``being reduced'' is stalk-local~\ref{chap5exr:reducedness_stalk_local}. 

$\Lrta$ direction is easy:

``Irreducible'' alone guarantees that $X$ is connected. And also note that localization of integral domain is also integral domain because $\frac{a}{s}\frac{b}{t}=\frac{0}{1}\Lrta \exists r\in S, rab=0\Lrta ab=0$.

We want to prove the reverse direction under Noetherian hypothesis.

Each stalk being integral domain alone implies each stalk is reduced, hence the scheme $X$ is reduced.

\underline{Want}: under Noetherian hypothesis, connected $+$ stalks being integral domain $\Lrta$ $X$ is irreducible.

$X$ is Noetherian, we know each connected component is union of finitely many irreducible components. We know $X$ is  connected and hence $X$ is the union of finitely many irreducible components, and some of the pair-wise intersections of these irreducible components are nonempty to make sure their union is  connected.

Assume $Y,Z$ are two of these irreducible components and their intersection is non-empty. $p\in Y\cap Z$. \underline{Claim}: $\calo_{X,p}$ is not integral domain.

Choose an affine open $U\ni p$, $U=\spec A$ is the union of $U\cap Y$ and $U\cap Z$, $U$ is not irreducible and $A$ is not integral domain~\ref{chap3exr:Integral_domain_irreducible}. Assume $U\cap Y=V(\scq_1)$ and $U\cap Z=V(\scq_2)$, where $\scq_1,\scq_2$ are two prime ideals in $A$, they should be minimal prime because there is a bijection~\ref{chap3exr:irreducible_components_minimal_pirmes_bij} between the minimal primes and irreducible components in $\spec A$. $V(\scq_1)\cap V(\scq_2)=V(\scq_1+\scq_2)$. $p=[\scp]\in V(\scq_1+\scq_2)\Lrta \scq_1+\scq_2\subset \scp$. The stalk at $p$, is $\calo_{X,p}=A_\scp$. Then there are at least two minimal prime ideals in $A_\scp$, therefore $A_\scp$ can not be integral domain.

In short, we have
\textbf{the minimal primes in $\calo_{X,p}$ is in one to one correspondence with the irreducible components that pass through $p$}.

The only possibility is that $X$ is irreducible. Combined with our proof that $X$ is reduced, we know $X$ is integral scheme.
\end{proof}

\begin{exr}\label{chap5exr:quasiprojective=>finite_type;Open_subschem_of_Noetherian_projective_scheme_is_of_finite_type}\ 
\begin{enumerate}[label=(\alph*)]
\item (quasiprojective implies finite type)
If $X$ is a quasiprojective $A$-scheme (Definition 4.5.9), show that $X$ is of finite type over $A$. If $A$ is furthermore assumed to be Noetherian, show that $X$ is a Noetherian scheme, and hence has a finite number of irreducible components.
\item Suppose $U$ is an open subscheme of a projective $A$-scheme. Show that $U$ is locally of finite type over $A$. If $A$ is Noetherian, show that $U$ is quasicompact, and hence quasiprojective over $A$, and hence by (a) of finite type over $A$. Show this need not be true if $A$ is not Noetherian. Better: give an example of an open subscheme of a projective $A$-scheme that is not quasicompact, necessarily for some non-Noetherian $A$.
\end{enumerate}
\end{exr}
\begin{proof}\ 
\begin{enumerate}[label=(\alph*)]
\item a quasiprojective $A$-scheme $X$ is a quasicompact open subscheme of projective $A$-scheme $Y$.

A projective $A$-scheme is of the form $Y:=\Proj S_\bullet$, where $S_\bullet$ is finitely generated graded ring over $A$. It is locally of finite type over $A$. Because it is covered by affine open sets $\spec ((S_\bullet)_f)_0$, where $(S_\bullet)_f$ is finitely generated over $A$ because it is generated by $\{S_\bullet, 1/f\}$ hence its degree zero piece is finitely generated over $A$.

Recall the Affine Communication Lemma~5.3.2 and also check that ``being finite over $A$'' is indeed an affine-local property. Then $Y=\cup_{i\in I}\spec B_i$ and $\spec B_i$ is of finite generated algebra over $A$, $\Lrta$ any affine open $U\subset X$ is of finite type over $A$. In particular $X$ can be cover covered by affine open $\spec_{j\in J} B_j$, where $B_j$ is finitely generated $A$-algebra, therefore $X$ is of locally finite type over $A$. In addition $X$ is quasicompact, we know $X$ is of finite type over $A$.

If $A$ is furthermore assumed to be Noetherian, then each affine open of $X$ is isomorphic to $\spec B_i$, where $B_i$  is Noetherian because $B_i$ is finitely generated over a Noetherian ring $A$, hence is also Noetherian by Hilbert basis theorem. Then it has finite number of irreducible components by~\ref{chap5exr:Noetherian_scheme_finite_irreducible_compoenents}.

\item For same reasons, an open subscheme in a scheme of locally of finite type over $A$ is locally of finite type over $A$, therefore any open subscheme $U$ of a projective $A$-scheme $Y$ is locally of finite type over $A$.

If $A$ is Noetherian, we know $U$ is locally Noetherian. Also notice that a projective $A$-scheme is quasicompact and quasiseparated by~\ref{chap5exr:Proj_A-scheme_quasicompact_quasiseparated}. 

\textbf{A projective $A$-scheme over a Noetherian ring $A$ is a Noetherian scheme.} 


By~\ref{chap3exr:open_subset_Noetherian_space_quasicompact}, an open subset of a Noetherian topological space is quasicompact. We know $U$ is quasicompact and locally Noetherian, hence is Noetherian. $U$ is also quasiprojective and of finite type over $A$.

As for the counterexample. Consider the silly example 4.5.11, $\Proj A[T]\cong\spec A$, where $x_0=T$ is of degree $1$. Also recall~\ref{chap3exr:affine_scheme_quasicompact}, $\spec A$ can have non-quasicompact open subscheme.
\end{enumerate}
\end{proof}
\begin{remark}\label{chap5rmk:open_subscheme_Noetherian}
The part of (locally)Noetherian and quasicompactness does not depend on (Projective), hence we have in general:

\textbf{
Any open or closed subscheme of a (locally) Noetherian scheme is
(locally) Noetherian.}
\end{remark}

\begin{exr}\ 
\begin{enumerate}[label=(\alph*)]
\item
Showthat $\spec k[x_1,...,x_n]/I$ is an affine $k$-variety if and only if $I \subset k[x_1,...,x_n]$ is a radical ideal.
\item Suppose $ I \subset  k[x_0,...,x_n]$ is a radical graded ideal. Show that $\Proj k[x_0,...,x_n]/I$ is a projective $k$-variety. (Caution: The example of $ I = (x_0^2, x_0x_1,...,x_0x_n)$ shows that $\Proj k[x_0,...,x_n]/I$ can be a projective $k$-variety without $I$ being radical.)
\end{enumerate}
\end{exr}
\begin{proof}
\begin{enumerate}[label=(\alph*)]
\item $X:=\spec k[x_1,...,x_n]/I$ is of finite type over $k$.

$k$ is Noetherian, $\spec k[x_1,...,x_n]/I$ is a projective $k$-scheme. We know open subscheme of a projective $A$-scheme over a Noetherian ring is of finite type over $A$ by Exercise~\ref{chap5exr:quasiprojective=>finite_type;Open_subschem_of_Noetherian_projective_scheme_is_of_finite_type}.

\underline{Want}:
$\spec k[x_1,...,x_n]/I$ is reduced iff $I$ is radical ideal.

Recall~\ref{chap5exr:reduced_ring_reduced_affine_scheme}, an affine scheme $\spec A$ is reduced iff $A$ is reduced.
The nilradical of $k[x_1,...,x_n]/I$ is just $\sqrt{I}$. We know $k[x_1,...,x_n]/I$ is reduced iff $I$ is radical.
\item We define projective $k$-variety to be reduced $k$-scheme. $\Proj k[x_1,...,x_n]/I$ is automatically of finite type over $k$ by~\ref{chap5exr:quasiprojective=>finite_type;Open_subschem_of_Noetherian_projective_scheme_is_of_finite_type}. 

We know $I$ is radical homogeneous ideal, then $S_\bullet:=k[x_1,...,x_n]/I$ is a reduced graded ring, hence $((S_\bullet)_f)_0$ is reduced and so is any further localizations. We have checked any stalks of $\Proj k[x_1,...,x_n]/I$ is reduced ring, hence it is a reduced projective $k$-scheme, i.e., a projective $k$-variety.
\end{enumerate}
\end{proof}

\begin{exr}
Show that a point of a locally finite type $k$-scheme is a closed point if and only if the residue field of the stalk of the structure sheaf at that point is a finite extension of $k$. Show that the closed points are dense on such a scheme (even though it needn't be quasicompact, cf. Exercise~\ref{chap5exr:quasicompact_scheme_have_closed_points}).
\end{exr}
\begin{proof}
We have a stronger version, for a locally finite type $k$-scheme $X$, the followings are equivalent:
\begin{enumerate}[label=(\alph*)]
	\item the point $p\in X$ is closed point
	\item the field extension $k\inj K(\calo_{X,p})$  is finite
	\item the field extension $k\inj K(\calo_{X,p})$ is algebraic
\end{enumerate}
$p$ is contained in an affine open subset $U=\spec A$ of $X$, where $A$ is a $k$-algebra and it corresponds to a maximal ideal $\scm$ in $A$. By Hilbert Nullstellensatz, $K(\calo_{X,p})\cong A/\scm$ is a finite field extension of $k$.

A finite field extension is automatically an algebraic field extension. These proves $(a)\Lrta (b),(c)$.

For $(c)\Lrta (a)$, $p$ is a point in $\spec A$, and $\scp$ is the corresponding prime ideal. $K(\calo_{X,p})\cong Frac(A/\scp)$ is algebraic over $k$, (It is integral over $k$, hence the integral domain $A/\scp$ is integral over $k$.) Recall a lemma in commutative algebra, Lemma 1.9 in~\cite{gortz2010algebraic}. Or one can use directly \cite[\href{http://stacks.math.columbia.edu/tag/00GS}{Lemma 00GS}]{stacks-project}

$B$ is integral over $A$, and both of them are integral domains. Then
${A}$ is a field $\Llrta$ ${B}$ is a field.

We know then $A/\scp$ is a field, therefore $\scp$ is a maximal ideal ideal, which means $p$ is a closed point.


\underline{Claim}:Let $k$ be a field. Let $X$ be a $k$-scheme locally of finite type. Let $U$ be a non-empty affine open subset of $X$. Then any closed point of $U$ is a closed point of $X$.

In an affine open subset $U\subset X$, the residue field at a closed point $p$ is finite field extension of $k$, which means it is also a closed point in $X$.

Hence, there is a closed point in each  affine open, therefore the closed point is dense in $X$.
\end{proof}

\begin{exr}
Suppose $X$ is a reduced, finite type $\cplx$-scheme. Define the corresponding complex analytic prevariety $X_{an}$. (The definition of an analytic prevariety is the same as the definition of a variety without the Hausdorff condition.) 
\end{exr}
\begin{proof}
It would be a long story, \textbf{To Be Added}. The major reference should be Chapter$4-5$ of ~\cite{neeman2007algebraic}
\end{proof}


\begin{exr}
Finish the proof of Proposition 5.3.3(a).
\end{exr}
\begin{proof} $A=(f_1,...,f_n)$.
It remains to show that, given a strictly increasing chain of ideals $I_1\subsetneq I_2\subsetneq \cdots$ in $A$, we can construct a strictly increasing chain of ideals in $A_{f_i}$ by
$$
I_{i,1}\subset I_{i,2}\subset \cdots,
$$
where $I_{i,j}=I_j\otimes_A A_{f_i}$. It suffices to show that for each $j$,
$$
I_{i,j}\subsetneq I_{i,j+1},
$$
for some $i$. (Notice that this does mean there is a strictly increasing chain is some $A_{f_i}$)
But it means there is  at least a chain $I_{i,1}\subset I_{i,2}\subset \cdots,$ not stabilize in the Noetherian ring $A_{f_i}$, contradiction.

Each $I_{i,j}$ could be interpreted as localization of the $A$-module $I_j$ at $f_i$ by~\ref{chap1exr:localization_extension_of_scalar}

\underline{Claim}:
$A=(f_1,...,f_n)$. Consider an $A$-module $M$, it is zero iff $M_{f_i}=0$ for all $f_i$. 

We only need the part: $\forall f_i, M_{f_i}=0\Lrta M=0$. Assume $M_{f_i}=0$ for all $i$, choose $m\in M$, exists $n_i\geq 0$ s.t., $f_i^n m=0$. On the other hand $1=\sum_{i=1}^nr_i f_i$, choose $N\geq \sum_{i=1}^n n_i+1$. $1\cdot m=1^Nm=(\sum_i r_i f_i)^N=\sum_i a_i f_i^{n_i}m=0$. Hence $M=0$.

Or equivalently, we can prove it by observing that each $M_{\scp}$ is further localization of some $M_{f_i}$. All $M_{f_i}=0$ implies all $M_\scp=0$, which in return implies $M=0$~\cite[\href{http://stacks.math.columbia.edu/tag/00HN}{Lemma 00HN}]{stacks-project}.

We apply the above claim to $I_{j+1}/I_j\neq0$, therefore we have for each $j$,
$$
I_{i,j+1}/I_{i,j}\neq 0
$$
for some $i$.
\end{proof}

\begin{exr} In text, we want to prove the proposition 

``Given $A=(f_1,...,f_n)$, then $A$ is a finitely generated $B$-algebra iff each $A_{f_i}$ is a finitely generated $B$-algebra.''

with some ``partition of unity argument''.
Make this argument precise.
\end{exr}
\begin{proof}
One direction is clear, $\spec A_{f_i}$ is generated by $A,1/f_i$. If $A$ is a finitely generated $B$-algebra, so is $A_{f_i}$.

$A$ is generated by $f_i$. $1=\sum_i c_i f_i$ for $c_i\in A$. $A_{f_i}$ is finitely generated as $B$-algebra and we assume its generators are $\{r_{ij}/f_i^{k_j}\}$. 

\underline{Claim:}$\{f_i\}\cup \{c_i\}\cup \{r_{ij}\}$ generate $A$ as a $B$-algebra.

$r$ is an element in $A$ and restricts to $r/1\in A_{f_i}$, which could be written as
$$
\sum_j b_{ij} \frac{r_{ij}}{f_{i}^{k_{ij}}}=\frac{\sum_j b_{ij}r_{ij}f_i^{K_{ij}} }{f_i^{Q_{i}}}=:\frac{P_i(r_{ij},f_i)}{g_i},
$$
where  we define $P_{i}:=\sum_j b_{ij}r_{ij}f_i^{K_{ij}}\in B[r_{ij},f_i]$ and $g_i:=f_i^{Q_i}$ for simplicity. 

$P_i/g_i$ and $P_j/g_j$ would agree on $A_{g_ig_j}$, which means
$$
(g_i g_j)^{m_{ij}}(P_i g_j-P_j g_i)=0.
$$
By taking $m=\max\{m_{ij}\}$, (the indexing set is finite), we can simplify the notation
$$
(g_i g_j)^{m}(P_i g_j-P_j g_i)=0,
$$
for all $i,j$.

Let $T_i:=g_i^m P_i$ and $h_j:=g_j^{m+1}$, we have
$$
T_i h_j-T_j h_i=0.
$$

Now, because $D(f_i^n)=D(f_i)$, we have $\spec A=\cup_i D(h_i)$ and this implies $1=\sum_i s_i h_i$, where each $s_i$ can be written as a polynomial of $c_i,f_i$.
Define 
$$
r':=\sum_i s_i T_i.
$$
$r'$ restricts to each $T_i/h_i$, because 
$$
r'h_j=\sum_i {s_i T_i}h_j=\sum_i s_i h_i T_j=T_j.
$$
In other words, it restricts to each 
$$
\sum_j b_{ij} \frac{r_{ij}}{f_{i}^{k_{ij}}}
$$
and by the identity axiom $r=r'$, which can be expressed as a polynomial in $B[\{f_i\}\cup \{c_i\}\cup \{r_{ij}\}]$

\end{proof}


\section{Normality and factoriality}
\begin{exr}
Show that integrally closed domains behave well under localization: if $A$ is an integrally closed domain, and $S$ is a multiplicative subset not containing $0$, show that $S^{-1}A$ is an integrally closed domain.
\end{exr}
\begin{proof}
Assume $x\in K(A)$ is integral over $S^{-1}A$. It satisfies a monic polynomial
$$
x^n+\frac{a_{n-1}}{s_{n-1}}x^{n-1}+\cdots +\frac{a_{0}}{s_0}=0.
$$
Define $s:=\prod_i s_i$ and if multiply the above equation by $s^n$, we get a polynomial of $sx$ with coefficients in $A$. Because $A$ is integral closed, we know $sx\in A$, hence $x\in S^{-1}A$, which means $S^{-1}A$ is integral closed.
\end{proof}


\begin{exr}
Show that a Noetherian scheme is normal if and only if it is the finite disjoint union of integral Noetherian normal schemes.
\end{exr}
\begin{proof}
$\Lrta$ direction: A Noetherian scheme $X$ is normal at least implies each stalks of this scheme is integral. Recall each connected component is open subscheme and open subscheme of Noetherian scheme is Noetherian (Remark~\ref{chap5rmk:open_subscheme_Noetherian}).

 By~\ref{chap5exr:Noetherian_scheme_integral_stalk}, each connected component of $X_i$ is a integral scheme. Also by~\ref{chap5exr:Noetherian_scheme_finite_irreducible_compoenents}, a Noetherian scheme has only finitely many connected components.

Then we know $X$ is a disjoint finite union of integral Noetherian normal schemes.

$\Llta$ direction is clear. Because normality is stalk-local, we know a finite union of integral Noetherian normal schemes is normal and it is Noetherian because it can be covered by finite Noetherian affine opens. 
\end{proof}

\begin{exr}
If $A$ is an integral domain, show that $A = \cap A_\scm$, where the intersection runs over all maximal ideals of $A$.
\end{exr}
\begin{proof}
If we interpret them as subsets in $K(A)$, we have clearly
$$
A\subset \cap A_\scm.
$$
As for the reverse direction, we claim:
$$
K(A)\backslash A\subset K(A)\backslash \cap A_\scm.
$$
If nonempty, pick $s\in K(A)\backslash A$, we can construct the ideal of denominators of $s$
$$
I_s:=\{r\in A:rs\in A\}
$$
$I_s\neq A$ because $1\notin I_s$. Then $I_s\subset \scm$ for some maximal ideal. $s\notin A_\scm$.
\end{proof}

\begin{exr}\label{chap5exr:5.4.D}
One might naively hope from experience with unique factorization domains that the ideal of denominators is principal. This is not true. As a counterexample, consider our new friend $A = k[w, x, y, z]/(wz - xy)$ (which we first met in Example 4.4.12, and which we will later recognize as the cone over the quadric surface), and $w/y = x/z \in K(A)$. Show that the ideal of denominators of this element of $K(A)$ is $(y, z)$.
\end{exr}
\begin{proof}
Denote the element $s:=w/y=x/z$. we easily check that $(y,z)\subset I_s$.

Assume $t\in I_s$, we have 
$$
tw/y=tx/z\in A,
$$
hence $tw=ya$. Then $t$ can not contain monomials in $w,x$. Furthermore, $t$ can not contain terms with only $w,x$ because $w^nx^m$ can not be a multiple of $y$. As a result, $t\in (y,z)$.
\end{proof}

\begin{exr}\label{chap5exr:localization_of_UFD}
Show that any nonzero localization of a unique factorization domain is a unique factorization domain.
\end{exr}
\begin{proof}
Recall the definition of a UFD. A UFD is an integral domain in which every non-zero element $x$ of $R$ can be written as a product of irreducible elements and a unit and this product is unique up to order of irreducible elements and units. 

An equivalent characterization is ``an integral domain $R$ in which every non-zero element can be written as a product of unit and  prime elements of $R$''.

We prove the statement based on the second characterization. $S\subset R$ is a multiplicative set and let $T$ be the set of all prime elements that divides an element of $S$, and let $M$ be the set of all prime element not in $T$.

\underline{Claim}: $p\in T$ iff image of $p$ in $S^{-1}R$ is a unit.

Indeed, if $p\in T$, then there exists $s\in S$, such that $p|s$; Let $x\in R$ with $px=s$. Then
$$
\frac{p}{1}\cdot\frac{x}{s}=\frac{ps}{s}\cdot \frac{x}{s}=1_{S^{-1}R}.
$$

For the ``if'' part, consider $p/1$ a unit, there is an inverse $v/s$, $\exists t\in S$ s.t., $pvt=st$, $pv=s$ because $0\notin S$ and $R$ is integral domain.

\underline{Claim}: if $p\in M$, then $p/1\in S^{-1}R$ is prime.

Assume $p/1|(a/s)(b/t)$, $(p c)/(r)=(ab)/(st)$, because $R$ is iintegral domain, $pcst=ab r$. Then we know $p|ab$ or $p|r$ because $p$ is prime, but we assumed $p\nmid s\forall s\in S$. Then we have $p|a$ or $p|b$ because $p$ is prime, which means $p/1|a/s$ or $p/1|b/t$ in return.

Now given $a=up_1^{b_1}\cdots p_r^{b_r}q_1^{c_1}\cdots q_t^{c_t}$ be a prime factorization of $a$, where $p_i\in T$ and $q_j\in M$. We have
$$
\frac{a}{s}=\frac{u}{s}\left(\frac{p_1s}{s}\right)^{b_1}\cdots \left(\frac{p_rs}{s}\right)^{b_r}\left(\frac{q_1s}{s}\right)^{c_1}\cdots \left(\frac{q_ts}{s}\right)^{c_t}.
$$
Then we know each $a/s\in S^{-1}R$ can be factorized into product of a unit and prime elements.

This proof is a improved version of~\href{https://math.stackexchange.com/q/140596}{This Answer}, in that we don't need to verify the uniqueness of factorization.
\end{proof}

\begin{exr}
Show that unique factorization domains are integrally closed. Hence factorial schemes are normal, and if $A$ is a unique factorization domain, then $\spec A$ is normal. 
\end{exr}
\begin{proof} Assme $R$ is a UFD, denote the fraction field by $K(R)$.
Let $x\in K(R)^{\times}$, say $x=r/s$, $gcd(x,s)=1$, $r,s\in R, s\neq 0$. Suppose $\exists a_1,..,a_n\in R$ s.t.,
$$
x^n+a_1 x^{n-1}+...+a_n=0.
$$
Then after multiplying it by $s^n$, set
$$
r^n=-(a_1 r^{n-1}s+a_2 r^{n-2}s^2+...+a_n s^n)
$$
$\Lrta s|r^n$, $gcd(r^n,s)=1,\Lrta s\in R^\times\Lrta x\in R$.

Then we have ``all stalks of a scheme $X$ are UFD'' $\Lrta$ ``all stalks of a scheme $X$ are normal'', which means factorial schemes are normal. 

If $A$ is a UFD,  each stalk  is a localization of $A$ is  therefore UFD by~\ref{chap5exr:localization_of_UFD} . $\spec A$ is factorial $\Lrta$ $\spec A$ is normal.
\end{proof}

\begin{exr}
Show that the following schemes are normal: $\affn^n_k$ , $\proj_k^n$ , $\spec \intg$. (As usual, k is a field. Although it is true that if A is integrally closed then A[x] is as well, this is not an easy fact, so do not use it here.)
\end{exr}
\begin{proof}
$\spec \intg$ is normal because $\intg$ is UFD.

$\affn^n_k=\spec k[x_1,..,x_n]$, where the polynomial ring over a field $k$ is UFD.

$\proj^n_k=\Proj k[x_0,...,x_n]$ can covered by $k[x_{0/i},...,x_{n/i}](x_{i/i}-1)$, which are UFDs, there fore we know each stalk is UFD. Then we know $\proj^n_k$ is factorial and hence normal.
\end{proof}

\begin{exr}\label{chap5exr:handy_characterization_normal_scheme}
Suppose $A$ is a unique factorization domain with $2$ invertible, and $z^2 - f$ is irreducible in $A[z]$.
\begin{enumerate}[label=(\alph*)]
\item Show that if $f \in A$ has no repeated prime factors, then $\spec A[z]/(z^2 - f)$ is normal. Hint: $B := A[z]/(z^2 - f)$ is an integral domain, as $(z^2 - f)$ is prime in $A[z]$. Suppose we have monic $F(T)\in B[T]$ so that $F(T) = 0$ has a root $\alpha$ in $ K(B)\backslash \ K(A).$ Then by replacing $F(T)$ by $\bar{F}(T)F(T)$, we can assume $F(T) \in A[T]$. Also, $\alpha = g+hz$ where $g,h \in K(A).$ Now $\alpha$ is the root of $Q(T) = 0$ for monic $Q(T) = T^2 - 2gT + (g^2 - h^2f) \in K(A)[T]$, so we can factor $F(T) = P(T)Q(T)$ in $K(A)[T]$. By Gauss’s lemma, $2g, g^2 - h^2f \in A$. Say $g = r/2$, $h = s/t$ ($s$ and $t$ have no common factors, $r, s, t \in A$). Then $g^2 - h^2f = (r^2t^2 - 4s^2f)/4t^2$. Then $t$ is invertible.
\item Show that if $f \in A$ has repeated prime factors, then $\spec A[z]/(z^2 - f)$ is not normal.
\end{enumerate}
\end{exr}
\begin{proof}
\begin{enumerate}[label=(\alph*)]
\item In fact, the hint already gives the detailed solution. We only write here some further explanations.

$A[z]/(z^2-f)$ is integral domain because $z^2-f$ is irreducible and hence is prime element in UFD.

$B$ is equivalent to formally adjoining a pair of roots $\pm \sqrt{f}$. We can take the ``conjugate'' of a monic $F(T)$ be mapping the coefficients $a+b\sqrt{f}$ to $a-b\sqrt{f}$ and we denote the conjugate polynomial $\bar{F}(T)$. Then $\bar{F}(T)F(T)\in A[T]$.

Assume $\alpha\in K(B)\backslash K(A)$ is a root of $F(T)=0$ hence a root of $\bar{F}(T)F(T)=0$,  We replace $F(T)$ by $\bar{F}(T)F(T)$ from now on. \underline{Want}: $\alpha$ is contained in $B$.

The root $\alpha$ can be written as $g+hz$ or equivalently ($g+h\sqrt{f}$), where $g,h\in K(A)$.

Then $\alpha$ is the root of $(T-g-h\sqrt{f})(T-g+h\sqrt{f})=T^2-2gT+g^2-h^2f\in K(A)[T]$, so we can factor $F(T)$ as $P(T)Q(T)\in K(A)[T]$. Recall Gauss' lemma for UFD:

\textbf{Given $A$ is a UFD, and $K(A)$ its fraction field. $F(T)\in A[T]$, if it factors in $K(A)[T]$, then it factors in $A[T]$.} 

Then we know $2g, g^2-h^2f\in A$. Because $2$ is invertible in $A$, we know $g\in A$ hence $h^2f\in A$. Assume $h=r/s$ and $gcd(r,s)=1$, we have $r^2f/s^2\in A$, because $f$ is square free, the only possibility is $s\in A^\times$. Then we know $h\in A$, therefore $\alpha=g+hz=g+h\sqrt{f}\in B$.

We have proved that $B=A[z]/(z^2-f)$ is integral closed, therefore $\spec A[z]/(z^2-f)$ is normal.
\item Assume $f\in A$ has repeated prime factors. 

\underline{Want}: $B:=A[z]/(z^2-f)$ and $B_\scp$ is not integral closed for some $\scp\in \spec B$. By Proposition 5.4.2, it is equivalent to prove $B$ is not integral closed.

\underline{Want}:$F(T)\in B[T]$, find an root of $F(T)=0$, $\beta\in K(B)\backslash K(A)$ s.t., $\beta\notin B$.

Like in $(a)$, we can restrict to the case $F(T)\in A[T]$. Assume $p$ is the repeated prime element in $f$, i.e, $p^2|f$.

We can choose the polynomial $F(T):=T^2-1/p^2 f$, $\sqrt{f}/p$ is a root of $F(T)$ in $K(B)$ and is not an element in $B$.
\end{enumerate}
\end{proof}

\begin{exr}\label{chap5exr:normal_schemes_examples}
Show that the following schemes are normal:
\begin{enumerate}[label=(\alph*)]
\item  $\spec\intg[x]/(x^2-n)$ where $n$ is a square-free integer congruent to $3$ modulo $4$. Caution: the hypotheses of Exercise~\ref{chap5exr:handy_characterization_normal_scheme} do not apply, so you will have to do this directly. (Your argument may also show the result when $3$ is replaced by $2$. A similar argument shows that $\intg[(1+\sqrt{n})/2]$ is integrally closed if $n \equiv 1 (\mod 4)$ is square-free.)
\item
 $\spec k[x_1,...,x_n]/(x^2_1+x_2^2+\cdots+x^2_m)$ where $\char k\neq 2, n\geq m\geq 3$.
\item $\spec k[w, x, y, z]/(wz - xy)$ where $\char\ k\neq  2$. This is our cone over a quadric surface example from Example 4.4.12 and Exercise~\ref{chap5exr:5.4.D}. Hint: Exercise~\ref{chap5exr:quadratic_forms_diagonalization} may help. (The result also holds for $\char k = 2$, but don’t worry about this.)
\end{enumerate}
\end{exr}
\begin{proof}
\begin{enumerate}[label=(\alph*)]
\item $\intg[x]/(x^2-n)\cong B:=\intg[\sqrt{n}]$. Assume a monic polynomial $P(T)\in B[T]$, we can take the conjugate $\bar{P}(T)$ and $\bar{P}(T)P(T)\in \intg[T]$. W.l.o.g, we can set $P(T)\in \intg[T]$. Assume $\alpha=g+h\sqrt{n}\in K(B)\backslash \ratl$ is a root of $P(T)=0$, where $g,h\in \ratl$. $\alpha$ is also a root of $G(T)= T^2 - 2gT + (g^2 - h^2n) \in \ratl[T]$. $P(T)=R(T)G(T)$.

Again, by Gauss's lemma, we have $G(T)\in \intg[T]$.
$$
2g\in \intg
$$
and
$$
g^2-h^2n\in \intg
$$
Assume $g=k/2$ and $h=r/s$, $k,r,s\in\intg$, we have
$$
\frac{k^2}{4}-\frac{r^2n}{s^2}\in \intg
$$
$n=4e+3$ and $k=2q$ or $k=2q+1$. The later case is ruled out. If $k=2q+1$,  the equation reduces to
$$
\frac{1}{4}-\frac{r^2n}{s^2}=m\in \intg
$$
Then we know $r^2 n=s^2/4 +ms^2$ and
$gcd(r,s)=1$ and $n$ is square-free, we know $4|s^2$, assume $s^2=4t^2$, therefore
$$
3r^2\equiv t^2\ (\mod 4), 
$$
where $gcd(r,t)=1$. We know $r,t$ are coprime and have same parity. $r$ and $t$ can't be both even number and have to be both odd. Assume $r=2u+1$ and $t=2v+1$, then the equation reduces to
$$
3\equiv 1\ (\mod 4).
$$
We get the contradiction. As a result, $k=2q$ and $r^2n/s^2\in\intg$, which means $s=1$. $g,h\in \intg$ $\beta=g+h\sqrt{f}\in B$.

The identical argument works when $n$ congruent to $2$ modulo $4$.

\item $\char k\neq 2$, then $2$ is invertible in $k$. For example, in the case
$k[x_1,x_2,x_3]/(x_1^2+x_2^2+x_3^2)$, regard $k[x_1,x_2,x_3]$ as $A[x_1]$, where $A=k[x_2,x_3]$, $A$ is a UFD, $2$ is invertible in $A$, $x_2^2+x_3^2$ is square-free in $A$, and $x_1^2+(x_2^2+x_3^2)$ is irreducible in $A[x_1]$. We can now use~\ref{chap5exr:handy_characterization_normal_scheme} to prove that $k[x_1,x_2,x_3]/(x_1^2+x_2^2+x_3^2)$ is integrally closed. The identical argument can be used to prove for arbitrary $n\geq m\geq 3$.
\item We can diagonalize the quadratic form $wz-xy$.
$$
wz-xy=\left(\frac{w+z}{2}\right)^2-\left(\frac{w-z}{2}\right)^2-\left(\frac{x+y}{2}\right)^2+\left(\frac{x-y}{2}\right)^2.
$$
Then it reduces to a special case of $(b)$ above.

\end{enumerate}
\end{proof}
\begin{exr}\label{chap5exr:quadratic_forms_diagonalization}
Suppose $k$ is an algebraically closed field of characteristic not $2$. (The hypothesis that $k$ is algebraically closed is not necessary, so feel free to deal with this more general case.)
\begin{enumerate}[label=(\alph*)]
\item Show that any quadratic form in $n$ variables can be ``diagonalized'' by changing coordinates to be a sum of at most $n$ squares.
\item Show that the number of squares appearing depends only on the quadratic. For example, $x^2 + y^2 + z^2$ cannot be written as a sum of two squares.
\end{enumerate}
\end{exr}
\begin{proof}
In linear algebra courses, we have discussed the diagonalization of real symmetric matrices and Hermitian matrices. This exercise is a simple generalization to fields with non-2 characteristics.
\begin{enumerate}[label=(\alph*)]
\item Consider a quadratic form 
$$
Q_{n}(\mathbf{x}):=\sum_{1\leq i,j\leq n}a_{ij}x_ix_j,
$$
with coefficients in $k$.
Follow the hints, we can induct on $n$. We firstly prove for $n=2$,
$$
Q_2(\mathbf{x})=a_{11}x_1^2+2a_{12}x_1x_2+a_{22}x_2^2.
$$
Assume $a_{11},a_{22}=0$ and $2a_{12}\neq 0$ (which means $a_{12}\neq 0$ because $\char k\neq 2$), we have $Q_{2}(\mathbf{x})=a_{12}(x_1+x_2)^2-a_{12}x_1^2-a_{12}x_2^2$. Else, one of $a_{11},a_{22}\neq 0$, w.l.o.g, $a_{11}\neq 0$, we get $Q_2(\mathbf{x})=a_{11}(x_1-a_{12}a_{11}^{-1}x_2)^2+(a_{22}-a_{12}^2/a_{11})x_2^2$.

Assume now all $Q_n$ can be diagonalized, consider a quadratic form $Q_{n+1}$, it can be written as 
$$
Q_{n+1}(\mathbf{x},x_{n+1})=P_{n}(\mathbf{x})+2c_{1,n+1}x_1x_{n+1}+...+2c_{n,n+1}x_nx_{n+1}+c_{n+1,n+1}x_{n+1}^2.
$$
Up to reordering, we require $c_{n+1,n+1}\neq 0$.
By induction hypothesis, we can write $P_n(\mathbf{x})$ as $\sum_{i=1}^n b_{ii}y_i^2$, with with possibility all $b_{ii}=0$ and $\mathbf{y}=A^{-1}\mathbf{x}$, where $A$ is a linear transforamtion.
$$
\begin{aligned}
Q_{n+1}(\mathbf{x},x_{n+1})&=\mathbf{y}^{T}B\mathbf{y}+2\sum_{i=1}^n\sum_{j=1}^n A_{ij}c_{i,n+1}y_jx_{n+1}+c_{n+1,n+1}x_{n+1}^2\\
&=\mathbf{y}^{T}B\mathbf{y}+2\sum_{j=1}^n \tilde{c}_{j,n+1}y_jx_{n+1}+c_{n+1,n+1}x_{n+1}^2,
\end{aligned}
$$ 
where $\tilde{c}_{j,n+1}:=\sum_{i=1}^nA_{ij} c_{i,n+1}$.
$$
\begin{aligned}
Q_{n+1}(\mathbf{x},x_{n+1})
&=c_{n+1,n+1}\left(x_{n+1}^2+\sum_{j=1}^n\frac{\tilde{c}_{j,n+1}}{c_{n+1,n+1}}y_j\right)^2+\sum_{i=1}^n\left(b_{ii}-\frac{c_{j,n+1}^2}{c_{n+1,n+1}}\right) y_i^2.
\end{aligned}
$$ 
There are at most $n+1$ terms.
\item Given a quadratic form $Q(\mathbf{x})=\mathbf{x}^{T}M\mathbf{x}$, under a transformation of basis, $\tilde{\mathbf{x}}=A \mathbf{x}$, we can write $Q(\tilde{\mathbf{x}})=\tilde{\mathbf{x}}^T \tilde{M} \tilde{\mathbf{x}}$, where 
$$
\tilde{M}=A^{T}M A.
$$
Assume $\mathbf{v}$ is in the null space of $M$, then $A^{-1}v$ is  in the null space of $\tilde{M}$, because $A$ is a non-degenerate matrix, we know the dimension of  null space is invariant under the change of basis. Hence, the rank of matrix $M$ is preserved under the base change. For quadratic forms, the rank is just the number of independent square terms.
\end{enumerate}
\end{proof}

\begin{exr}
(RINGS CAN BE INTEGRALLY CLOSED BUT NOT UNIQUE FACTORIZATION DOMAINS, ARITHMETIC VERSION). Show that $\intg[\sqrt{-5}]$ is integrally closed but not a unique factorization domain.
\end{exr}
\begin{proof}
This is a classical counter-example.

 $A:=\intg[\sqrt{-5}]$ is integrally closed by~\ref{chap5exr:normal_schemes_examples}(a) because 
 $$-5\equiv 3\mod 4.
 $$

 It is not unique factorization domain because for example $6=2\times 3$ and $6=(1-\sqrt{-5})(1+\sqrt{-5})$.
\end{proof}

\begin{exr}
(RINGS CAN BE INTEGRALLY CLOSED BUT NOT UNIQUE FACTORIZATION DOMAINS, GEOMETRIC VERSION). Suppose $\char k \neq 2$. Let $A = k[w, x, y, z]/(wz - xy)$, so $\spec A$ is the cone over the smooth quadric surface (cf. Exercises 4.4.12 and~\ref{chap5exr:5.4.D}).
\begin{enumerate}[label=(\alph*)]
\item Show that $A$ is integrally closed. (Hint: Exercises~\ref{chap5exr:normal_schemes_examples}(c) and~\ref{chap5exr:quadratic_forms_diagonalization})
\item Show that $A$ is not a unique factorization domain. (Clearly $wz = xy$. But why are $w, x, y$, and $z$ irreducible? Hint: $A$ is a graded integral domain. Show that if a homogeneous element factors, the factors must be homogeneous.)
\end{enumerate}
\end{exr}
\begin{proof}
For (a), we can diagonalize the quadratic form $wz-xy$.
$$
wz-xy=\left(\frac{w+z}{2}\right)^2-\left(\frac{w-z}{2}\right)^2-\left(\frac{x+y}{2}\right)^2+\left(\frac{x-y}{2}\right)^2.
$$
We know $A\cong k[x_1,x_2,x_3,x_4]/(x_1^2-x_2^2-x_3^2+x_4^2)$, then we know it is integrally closed by~\ref{chap5exr:normal_schemes_examples}(b).

As for $(b)$, we clearly have $wz=xy$. Then it suffices to prove $w,x,y,z$ are irreducible.  

Indeed, we can regard $A$ as a graded integral domain because it is a quotient of the polynomial ring by a homogeneous prime ideal. 

\underline{Claim}: If a homogeneous element factors, the factors must be homogeneous. 

An element $f$ is of homogeneous degree $n$ iff each monomial of $f$ is of degree $n$.

Now assume $f=gh$, where $g,h$ are not homogeneous. Assume $g=g_{0}+g_1$, where $g_0$ is the highest degree term in $g$ and $g_1$ is the rest. The degree of monomials in $g_0$ is strictly higher than those in $g_1$. We have $f=g_0 h+g_1 h$, but it is impossible for the two summands to have same highest degrees.

Then we know $x$ is irreducible, because the only degree 
$1$ homogeneous terms are $x,y,z,w$ and we won't have the relation $w=\lambda x$. 
\end{proof}

\begin{exr}
Suppose $A$ is a $k$-algebra, and $l/k$ is a finite extension of fields. (Most likely your proof will not use finiteness; this hypothesis is included to avoid distraction by infinite-dimensional vector spaces.) Show that if $A\otimes_k l$ is a normal integral domain, then $A$ is a normal integral domain as well.
\end{exr}
\begin{proof}
Here the notion ``normal integral domain'' is the same as ``integrally closed''. $A$ can be identified as a subring of $A\otimes_k l$, therefore $A\otimes_k l$ is an integral domain implies that $A$ being an integral domain. 

The proof is sketched in the hint, we just follow it step by step.

 Fix a $k$-basis for $l$, $b_1=1,...,b_d$. There is a dual basis $v_1=1,...,v_d$ such that $trace_{l/k}(v_i b_j)=\delta_{ij}$. We can extend the trace to $Tr:A\otimes_k l\lrta A: a\otimes x\mapsto a\cdot trace_{l/k}(x)$. The map $Tr$ is nondegenerate $Tr(a\otimes x)=0\Lrta a=0$ or $x=0$.

 \underline{Claim}: $1\otimes b_1,...,1\otimes b_d$ forms a free $A$-basis for $A\otimes_k l$. 

 Assume an element $a\otimes x\in A\otimes_k l$, where $x=\sum_i x_i b_i$. We have 
 $a\otimes x=a\otimes \sum_i x_i b_i$ and it equals to $\sum_i (x_i a) (1\otimes b_i)$. It indeed forms an generating set of  $A$-module $A\otimes_k l$. And we can show that $A\otimes_k l$ is a free $A$-module with this basis. Assume $\sum_i r_i(1\otimes b_i)=0$, $\sum_i r_i\otimes b_i=0$. Consider $Tr((1\otimes v_j)\cdot (\sum_i r_i\otimes b_i))=r_j=0$. We have proved that $A\otimes_k l$ is a free $A$-module with the specified basis.

 \underline{Claim}: The diagram commutes and the composition map $A\lrta K(A)\otimes_k l$ is injective.
\[
\begin{tikzcd}
A \arrow[r] \arrow[d] & K(A) \arrow[d] \\
A\otimes_k l \arrow[r] & K(A)\otimes_k l
\end{tikzcd}
\]
$A\lrta K(A)$ is injective and $l$ is a flat $k$-module. Hence the composition map is injective.

\underline{Claim}: $K(A)\otimes_k l\cong K(A\otimes_k  l)$.

We know $A\otimes_k l\subset K(A)\otimes_k l\subset K(A\otimes_k l)$, then it suffices to prove only $K(A)\otimes_k l$ is a field.

A general term in $K(A)\otimes_k l$ is of the form $\frac{a}{c}\otimes x$, $1\otimes b_i$ gives a $K(A)$-basis of $K(A)\otimes_k l$. (it comes from our first claim). By~\ref{exr:chap3_fin-dim_k_domain_is_field}, a finite $k$-algebra which is an integral domain must be a field. We therefore have $K(A)\otimes_k l$ is a  field. 

\underline{Claim}: $K(A)\cap A\otimes_k l=A$, where we regard both $K(A)$ and $A\otimes_k l$ as subrings of $K(A\otimes_k l)$.

$A\subset K(A)\cap A\otimes_k l$ is automatically true. $K(A)\otimes_k l$ is a $K(A)$-vector space with basis $1\otimes b_i$. The subspace $K(A)\cong K(A)\otimes_k 1$ is generated by $1\otimes 1$. $K(A)\cap A\otimes_k l$ is an $A$-submodule of $A\otimes_k l$ generated by $1\otimes 1$, therefore is contained in $A$.

Now we have finished our preparation. Given $P(T)$ a monic polynomial in $A[T]$. We can regard $P(T)$ as an element in $K(A\otimes_k l)[T]$. By assumption, $A\otimes_k l$ is normal integral closed. The root of $P(T)=0$ in $K(A\otimes_k l)$, $\alpha\in A\otimes_k l$. Assume further more $\alpha$ is a root in $K(A)$, we then have $\alpha\in K(A)\cap A\otimes_k l=A$. Whence, we can conclude that $A$ is normal integral domain. 

\end{proof}
\begin{exr}
(UFD-NESS IS NOT AFFINE-LOCAL). Let $A = (\ratl[x, y]_{x^2+y^2})_0$ denote the homogeneous degree $0$ part of the ring $\ratl[x,y]_{x^2 +y^2}$ . In other words, it consists of quotients $f(x, y)/(x^2 + y^2)^n$ , where $f$ has pure degree $2n$. Show that the distinguished open sets $D\left( \frac{x^2}{x^2+y^2}\right)$ and $D\left( \frac{y^2}{x^2+y^2}\right)$ cover $\spec A$. Show that $A_{\frac{x^2}{x^2+y^2}}$ and $A_{\frac{y^2}{x^2+y^2}}$
are unique factorization domains. Finally, show that A is not a unique factorization domain.
\end{exr}
\begin{proof}
By~\ref{exr:open_cover_distinguished}, we know 
$ \frac{x^2}{x^2+y^2}+ \frac{y^2}{x^2+y^2}=1\Lrta D\left( \frac{x^2}{x^2+y^2}\right)\cup D\left( \frac{y^2}{x^2+y^2}\right)=\spec A$.

A general element of $A_{\frac{x^2}{x^2+y^2}}$ is of the form
$$
\frac{\frac{f(x,y)}{(x^2+y^2)^n}}{\frac{x^{2m}}{(x^2+y^2)^m}}=\frac{f(x,y)/x^{2n}}{(x^2+y^2)^{n-m} (x^{2})^{m-n}}=\frac{g(t)}{(1+t^2)^{n-m}}
$$
where $t=\frac{y}{x}$ and $g(t)=f(x,y)/x^{2n}$. Hence we know both $A_{x^2/(x^2+y^2)}$ and $A_{y^2/(x^2+y^2)}$ are isomorphic to $\ratl[t]_{t^2+1}$, which is a localization of UFD, by~\ref{chap5exr:localization_of_UFD}, they are also UFDs.

Finally, $A$ is not a UFD, for example
$$
\left(\frac{xy}{x^2+y^2}\right)^2=\frac{x^2}{x^2+y^2}\cdot \frac{y^2}{x^2+y^2},
$$
where both 
$$
\frac{xy}{x^2+y^2}
$$
and
$$
\frac{x^2}{x^2+y^2}
$$
are irreducible because $x/(x^2+y^2),y/(x^2+y^2)\notin A$ because they are not of degree $0$.
\end{proof}
\section{The crucial points of a scheme: Associated points and primes}
\begin{exr}
Suppose $f$ is a function on $\spec k[x,y]/(y^2,xy)$(i.e.,$f\in k[x,y]/(y^2,xy)$). Show that $\supp f$ is either the empty set, or the origin, or the entire space.
\end{exr}
\begin{proof}Denote the ring $k[x,y]/(xy, y^2)=:A$
$f$ is a global section on $\spec k[x,y]/(xy, y^2)$. By definition 
$$
\supp(f):=\{p\in X=\spec A:f_p\neq 0\text{ in } \calo_{X,p}\}
$$
$\supp f$ is the point $[\scp]$ where $f/1\neq 0$ in $A_\scp$. $f/1\neq 0$ means $fs\neq 0\forall s\in A-\scp$. 

A general element $f$ in $k[x, y]/(y^2, xy)$ is of the form
$$
q(\bar{x})+a\bar{y},
$$
where $q$ is a polynomial in $k[X]$, $a$ is a constant in $k$ and $\bar{x}\bar{y}=0$ and $\bar{y}^2=0$. 

1. Assume $q=0$ and $a=0$, then $f=0$. In this trivial case the support of $f$ is empty set because $fs=0$ for whatever $s\in A$.

2. In the case $f=q(\bar{x})+a \bar{y}$, $q\neq 0,a\neq 0$ 
$$
\bar{y}^2f=0
$$
which implies $\bar{y}^2\in\scp$ for $[\scp]\in \supp f$. But in fact, all the prime ideals in $A$  contains $\bar{y}$. 

\underline{Claim}: if $f=q(\bar{x})+a\bar{y}$ with $q\neq 0, a\neq 0$, then $\supp f=\spec A$.

Indeed the point of $\spec A$ is of the form $(\bar{x}-b,\bar{y}) $ or $(\bar{y})$, for any $\scp_b=(\bar{x}-b,\bar{y})$, the elements in multiplicative subset $S_b=A-\scp_b$ are of  the form $g(\bar{x})+c\bar{y}$ where $g(b)\neq 0$. $(g(\bar{x})+c\bar{y})(q(\bar{x})+a\bar{y})\neq 0$. For $\scp=(\bar{y})$, the multiplicative set are of the form $g(\bar{x})$ and $g(\bar{x})(q(\bar{x})+c\bar{y})\neq 0$.

3. Assume $f$ is $a\bar{y}$, $f$ is not supported on $(\bar{y})$ because $\bar{x}\in A-(\bar{y})$. $f$ is neither supported over $(\bar{x}-b,\bar{y})$ since $\bar{x}\in A-(\bar{x}-b,\bar{y})$.

$f$ is only supported on the origin $(\bar{x},\bar{y})$, where the multiplicative set are just the constants.

We have discussed all the possible cases thus finished the proof.
\end{proof}

\begin{exr}\label{chap5exr:integral_associated_point_generic_point}(ASSUMING \textbf{(A)})
Suppose $A$ is an integral domain. Show that the generic point $[0]$ is the only associated point of $\spec A$
\end{exr}
\begin{proof}
\textbf{Confusion: when we talk about the associated point of $\spec A$, does it mean we choose $M=A$?}
 
We use the definition \textbf{(A)} in page 167 in the case $M=A$ and always assume $A$ to be Noetherian when discussing associated points.

Given $A$  a Noetherian integral domain, $(\spec A,\calo_{\spec A})$ is a Noetherian integral scheme by~\ref{chap5exr:affine_integral_scheme}, which means it is irreducible and reduced by~\ref{chap5exr:integral=irreducible+reduced}. 

The global sections of $\spec A,\calo_{\spec A}$ are just elements in $A$. Now each section $a$ is supported on the whole scheme $\spec A$. The only irreducible component is $\spec A$ itself. Hence the generic point is the only associated point of $\spec A$.
\end{proof}

\begin{exr}(ASSUMING \textbf{(A)})
Show that if $A$ is reduced, $\spec A$ has no embedded points.
\end{exr}
\begin{proof}
What we want to show is 

``the associated points of a scheme $\spec A$ for a reduced ring $A$ are generic points of the irreducible components of $\spec A$''

Follow the hint.

(i): In the case where $A$ is integral domain, we know the only associated point is the generic point of $\spec A$ by Exercise~\ref{chap5exr:integral_associated_point_generic_point}.

(ii) For a general reduced ring $A$, we \underline{Claim}:

If $f$ is a nonzero function on a reduced affine scheme,  then $\supp f = \overline{D(f)}$

Indeed, $D(f)=\{[\scq]\in\spec A|f\notin \scq\}$, $[\scp]\in D(f)\Llrta$ $f\in A-\scp$, $f/1\neq 0\in A_\scp$ because $A-\scp$ is multiplicatively closed and $A-\scp$ does not contain $0$. Then we know $D(f)\subset \supp f$, therefore $\overline{D(f)}\subset\supp f$ because $\supp f$ is closed. 
For the reverse inclusion, consider $[\scp]\in\supp f$ $\Lrta fs\neq 0, \forall s\in A-\scp$. Suppose $[\scp]\in D(g)$ for some $g\in A$, $g\notin \scp$($g\in A-\scp$), $D(f)\cap D(g)=D(fg)$. Then we know $fg\neq 0$. $fq$ is not nilpotent because $A$ is a reduced ring, therefore $D(fg)\neq \emptyset$. Useful exercises are~\ref{chap3exr:3.5.D} and~\ref{chap3exr:3.5.F}

\underline{Claim}: $\overline{D(f)}$ is the union of irreducible component that intersects $D(f)$. 

By Proposition 3.6.15, we know in Noetherian topological space, a non-empty closed set can be expressed uniquely as a finite union of irreducible closed subsets.

 Apply it to $\overline{D(f)}$, it can be expressed as union of $\cup_i D_i$. What we need to prove is that the irreducible closed subset $D_i$ are in fact a irreducible components $X_i$ in $\spec A$. 
Assume by contradiction, $D_i\subsetneq Z_i\subset X_i$ is not a maximal irreducible subset, where $Z_i$ is a minimal irreducible closed subset that contains $D_i$. (i.e. there is no intermediate irreducible closed subset between $D_i$ and $Z_i$). By~\ref{chap3exr:bijection_closed_subsets_points_affine}, there is a reverse inclusion of generic point $\scp_d\supsetneq \scp_z$, where $\overline{\scp_d}=D_i$ and $\overline{\scp_z}=Z_i$. (This also means $\scp_d$ is one of  associated primes of $A$ corresponding to $f$)

 Consider the ring $A/\scp_z$, $\scp_d/\scp_z$ is a non-zero prime ideal in the quotient ring. 

The canonical projection $A\lrta A/\scp_z$ induces a continuous map $\spec A/\scp_z\lrta \spec A$, which can be identified as the inclusion of a closed subset.

We know $\overline{\scp_d/\scp_z}=\overline{\scp_d}\cap \spec A/\scp_z$. On the other hand
$\overline{\scp_d}\subset \overline{D(f)}$, $\overline{\scp_d/\scp_z}=\overline{\scp_d}\cap \spec A/\scp_z\cap \overline{D(f)}=\overline{\scp_d}\cap \overline{D([f])}$, where $[f]$ is the equivalence class of $f$ in $A/\scp_z$. Every subspace of Noetherian space is also Noetherian by~\ref{chap3exr:open_subset_Noetherian_space_quasicompact}. We know $\overline{D([f])}$ is a subspace of Noetherian space $\overline{D(f)}$. Still by Proposition 3.6.15, both $\overline{D([f])}$ and $\overline{D(f)}$
can expressed uniquely as a union of their irreducible components. Also because each irreducible closed subset is contained in an irreducible component. 

We then know
\underline{Lemma:}
\textbf{ the irreducible component $Y_i$ of a subspace $Y$ in Noetherian space $X$ is the intersection of $Y\cap X_i$ with the irreducible component of $X$}.

As a result, $\overline{\scp_d/\scp_z}$ is the irreducible component of $\overline{D([f])}=\supp [f]$.

$\scp_d/\scp_z$ is an associated point of $A/\scp_z$, where $A/\scp_z$ is an integral domain, we then apply $(i)$. $\scp_d/\scp_z$ can not be an embedded point ($\scp_d/\scp_z=0$) contradicting to our setting $\scp_d\subsetneq \scp_z$.

Now we are done.

 

\end{proof}

\begin{exr}
(ASSUMING \textbf{(A)}). Suppose $m \in M$. Show that $\supp m$ is the closure of those associated points of $M$ where $m$ has non-zero germ.(Hint: $\supp m$ is a closed set containing the points described, and thus their closure. Why does it contain no other points?)
\end{exr}
\begin{proof}
Recall what we mean by the support of a section $m$
$$
\supp(m)=\{\scp\in X:m_{\scp}\neq 0\}.
$$
By definition in $\mathbf{(A)}$, associated points are the generic points of irreducible components of $\supp(n)$ for some $n$. In particular the associated described in the problem is contained in $\supp(m)$ hence contains the closure of those associated point.

Denote the set of associated point where $m$ has nonzero germ by $S$. We want to prove $\supp(m)-\overline{S}=\emptyset$. Choose a point $\scq\in \supp(m)$, want $U\ni\scq\Lrta U\cap S\neq \emptyset$. Consider $D(f)\ni \scq\Lrta$ $D(f)\cap \supp(m)\neq \emptyset$. $D(f)$ intersects with as least one irreducible component $W$ of $\supp(m)$ by definition $(A)$, it corresponds to an associated point $\scp$ and $\overline{\scp}=W$. We know $D(f)\cap W\neq \emptyset,\Lrta D(f)\ni \scp$. $\Lrta D(f)\cap S\neq \emptyset$.
\end{proof}
\begin{exr}(Assuming $\mathbf{(A)}$ and $\mathbf{(B)}$)
Show that the locus on $\spec A$ of points $[\scp]$ where $\calo_{\spec A,[\scp]} = A_\scp$ is nonreduced is the closure of those associated points of $\spec A$ whose stalks are nonreduced. (Hint: why do points in the closure of these associated points all have nonreduced stalks? Why can’t any other point have a nonreduced stalk?)
\end{exr}
\begin{proof}
Follow the hint, we first show that the points in the closure of those specified associated point all have nonreduced stalk.

Let $\{\scn_i\}$ be the points with nonreduced stalks and let $\{\scp_j\}$ be the associated points with nonreduced stalks. What we want to show is basically the equality
$$
\overline{\cup_j [\scp_j]}=\{[\scn_i]\}
$$
By $\mathbf{(B)}$, the closure of finite union is the finite union of closures.
$$
\cup_j \overline{[\scp_j]}\overset{?}{=}\{[\scn_i]\}
$$

The containment $\supset$ follows from the following fact: if $f\in A_\scn$ is a nilpotent, consider $\supp(f)$.
$0\neq f\in Nil(A_\scn)$ means $f$ is contained in all prime ideals in $\scn$ and also $\scn\in \supp f$.

Choose $\scq$ to the associated point corresponding to the irreducible component containing $[\scn]$. Then $f/1$ is nilpotent in $A_\scq$. This is because $\scq\subset \scn$ and $$f\in Nil(A_\scn)=\cap_{prime\ \sca\in \scn}\sca\subset \cap_{prime\ \scb\in \scq}\scb=Nil(A_\scq).$$

The reverse inclusion is easier because assume
$\sca\in \cup\overline{[\scp_i]}$, this means some $A_\scp$ is a further localization of $A_\sca$, where $A_\scp$ is nonreduced. We then know $A_\sca$ is nonreduced because we have shown before that localization of a reduced ring is reduced in Exercise~\ref{chap5exr:reduced_ring_reduced_affine_scheme}.
\end{proof}