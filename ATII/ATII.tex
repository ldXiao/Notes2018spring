\documentclass[11pt]{article}
\usepackage{amssymb}
\usepackage{latexsym}
\usepackage{amsmath}
\usepackage{amsthm}
\usepackage{stmaryrd}
\usepackage{fancyhdr}
\pagestyle{headings}
\usepackage{dsfont}
\usepackage{pifont}
\usepackage{mathtools}
\usepackage{natbib}
\usepackage{tikz-cd}
\usepackage{pgfplots}
\usepackage{enumitem} 
\usepackage{hyperref}
\usepackage{geometry}
\geometry{left=4cm,right=4cm}
\pgfplotsset{every axis/.append style={
                    axis x line=middle,    % put the x axis in the middle
                    axis y line=middle,    % put the y axis in the middle
                    axis line style={<->}, % arrows on the axis
                    xlabel={$x$},          % default put x on x-axis
                    ylabel={$y$},          % default put y on y-axis
                    ticks=none,
                    }}
%\usepackage[urw-garamond]{mathdesign}
%\usepackage{cmbright}
%\usepackage{concmath}
%\usepackage{sansmathfonts}
%\renewcommand*\familydefault{\sfdefault} %% Only if the base font of the document is to be sans serif

%\usepackage{pdfrender,xcolor,scrpage2}
%\pdfrender{StrokeColor=black,TextRenderingMode=2,LineWidth=1pt}
\tikzset{
  subseteq/.style={
    draw=none,
    edge node={node [sloped, allow upside down, auto=false]{$\subseteq$}}},
  Subseteq/.style={
    draw=none,
    every to/.append style={
      edge node={node [sloped, allow upside down, auto=false]{$\subseteq$}}}
    },
    Subsetneq/.style={
    draw=none,
    every to/.append style={
      edge node={node [sloped, allow upside down, auto=false]{$\subsetneq$}}}
    },
  Supseteq/.style={
    draw=none,
    every to/.append style={
      edge node={node [sloped, allow upside down, auto=false]{$\supseteq$}}}
  }
}

\hypersetup{
    colorlinks,
    citecolor=blue,
    filecolor=blue,
    linkcolor=blue,
    urlcolor=blue
}
\newtheorem{thm}{Theorem}[section]
\newtheorem{prop}[thm]{Proposition}
\newtheorem{lemma}[thm]{Lemma}
\newtheorem{cor}[thm]{Corollary}
\newtheorem{dfn}[thm]{Definition}
\newtheorem{axiom}[thm]{Axiom}
\newtheorem{rmk}[thm]{Remark}
\newtheorem{rmkt}[thm]{Remark by TeXer}
\newtheorem{ex}[thm]{Example}
\newtheorem{nex}[thm]{Non-example}
\newtheorem{exercise}[thm]{Exercise}
\newtheorem{question}[thm]{Question}
\newtheorem{problem}[thm]{Problem}
\newtheorem{dfn/thm}[thm]{Definition/Theorem}
\renewcommand{\baselinestretch}{1.1}
\renewcommand{\hom}{\text{ Hom}}
\newcommand{\im}{\text{ im}}
\newcommand{\coker}{\text{ coker}}
\newcommand{\tor}{\text{ Tor}}
\newcommand{\ext}{\text{ Ext}}
\newcommand{\affn}{\mathbb A}
\newcommand{\proj}{\mathbb P}
\newcommand{\reals}{\mathbb R}
\newcommand{\cplx}{\mathbb C}
\newcommand{\intg}{\mathbb Z}
\newcommand{\bbf}{\mathbb F}
\newcommand{\ratl}{\mathbb Q}
\newcommand{\torus}{\mathbb T}
\newcommand{\sca}{{\mathfrak a}}
\newcommand{\scb}{{\mathfrak b}}
\newcommand{\scc}{{\mathfrak c}}
\newcommand{\scm}{{\mathfrak m}}
\newcommand{\scn}{{\mathfrak n}}
\newcommand{\scp}{{\mathfrak p}}
\newcommand{\scq}{\mathfrak q}
\newcommand{\frakg}{{\mathfrak g}}
\newcommand{\frakd}{{\mathfrak d}}
\newcommand{\pd}{\partial}
\newcommand{\calf}{{\cal F}}
\newcommand{\calg}{{\cal G}}
\newcommand{\cala}{{\cal A}}
\newcommand{\calb}{{\cal B}}
\newcommand{\calc}{{\cal C}}
\newcommand{\cale}{{\cal E}}
\newcommand{\cali}{{\cal I}}
\newcommand{\call}{{\cal L}}
\newcommand{\caln}{{\cal N}}
\newcommand{\calo}{{\cal O}}
\newcommand{\calr}{{\cal R}}
\newcommand{\mathbold}{\bf}
\newcommand{\cinf}{C^{\infty}}
\newcommand{\row}[2]{#1_1,\dots ,#1_{#2}}
\newcommand{\dbyd}[2]{{\partial #1\over\partial #2}}
\newcommand{\Space}{{\bf Space}}
\newcommand{\alg}{{\mathbold Alg}}
\newcommand{\notsubset}{\not \subset}
\newcommand{\notsupset}{\not \supset}
\newcommand{\pois}{{\mathbold Pois}}
\newcommand{\pitilde}{\tilde{\pi}}
\newcommand{\rta}{\rightarrow}
\newcommand{\Lrta}{\Longrightarrow}
\newcommand{\lrta}{\longrightarrow}
\newcommand{\llrta}{\longleftrightarrow}
\newcommand{\Llta}{\Longleftarrow}
\newcommand{\Llrta}{\Longleftrightarrow}
\newcommand{\lgl}{\langle}
\newcommand{\rgl}{\rangle}
\newcommand{\inj}{\hookrightarrow}
\newcommand{\surj}{\twoheadrightarrow}
\newcommand{\cmark}{\ding{51}}%
\newcommand{\xmark}{\ding{55}}%
\newcommand{\downmapsto}{\rotatebox[origin=c]{-90}{$\scriptstyle\mapsto$}\mkern2mu}
\renewcommand{\qedsymbol}{$\square$}
\bibliographystyle{plain}
\title{\bf Summary for Algebraic Topology II}
\author{Notes by Lin-Da Xiao}
\date{2018 ETH} %\thanks{Research partially supported by NSF Grant DMS-96-25122 and the Miller Institute for Basic Research in Science.}
\begin{document}
\maketitle
\tableofcontents
\newpage
\section{Feb 21th: Tor functor}
\begin{dfn}
Suppose $A$ is an abelian group, A \textbf{Free resolution} is an exact sequence of the form
$$
\cdots\lrta F_2\overset{f_2}{\lrta}F_1\overset{f_1}{\lrta}F_0\overset{f_0}{\lrta}A\lrta 0,
$$
where each $F_i$ is a free abelian group. If moreover $F_i=0,\forall i\geq 2$, we call it \textbf{Short free resolution} 
$$
0\lrta K\lrta F\lrta A\lrta 0
$$
\end{dfn}
(We can easily generalize this definition to $R$-modules)
\begin{prop}
Let $A$ be an abelian group. Then there exists a short free resolution of $A$.
\end{prop}
\begin{proof}
Let $F$ be the free abelian group generated by all elements in $A$. There is a surjection from $F$ to $A$ by linearly extending the map sending basis element to itself. Let $K$ denote the kernel of this map. $K$ is an abelian subgroup of a free abelian group ($\intg$-module).  A subgroup of a free abelian group is torsion free as a module. $\intg$ is a $PID$. If $R$ is a $PID$, then an  $R$-module is free iff it is torsion free (See Bosch section 4.2). Then we know in particular, $K$ is a free abelian group.
\end{proof}
With this construction, we can define the $\tor$ functor now:
\begin{dfn}
Let $A$ be an abelian group. Let $0\rta K\overset{f}{\rta}F\rta A\rta 0$ be a short free resolution of $A$. Given any other abelian group $B$, we define 
$$
\tor(A,B):=\ker(f\otimes id_B)
$$
\tor(A,B)
\end{dfn}

This definition is independent on the choice of short free resolution.

\section{Feb 28th:}

\underline{Question}: Given $X, Y$ what is the cohomology of $X\times Y$?

\underline{Answer}:
$$
H_n(X\times Y)\cong \bigoplus_{i+j=n}H_i(X)\otimes H_j (Y)+\bigoplus_{k+\ell=n-1} \tor(H_k,(X),H_\ell(Y)
$$
We will discuss Elenberg-Zilber theorem along this line the next lecture.

Today, we will prove the Algebraic Kueneth Theorem
\begin{dfn}
Suppose $(C_\bullet,\pd)$ and $(C'_\bullet,\pd')$ are two non-negative chain complexes. We define the  \textbf{tensor complex} $(C_\bullet\otimes C_\bullet',\Delta)$, where
$$
(C_\bullet\otimes C'_\bullet)_n=\oplus_{i+j=n}C_i\otimes C_j'
$$
and the differential $\Delta$ is defined by 
$$
\Delta(c_i\otimes c'_j)=\pd c_i\otimes c'_j+(-1)^{i}c_i\otimes \pd' c_j
$$
\end{dfn}
First, note that $\Delta(c_i\otimes c'_j)$ does indeed belong to $(C_\bullet\otimes C_\bullet')_{n-1}$. The reason for $(-1)^i$ is to make $\Delta^2=0$.
$C_\bullet\otimes C'_\bullet$ is another  non-negative chain complex.
\begin{dfn}
Suppose $f_\bullet:C_\bullet\lrta D_\bullet$ and $g_\bullet: C'_\bullet\lrta D'_\bullet$ are two morphism of chain complexes. Then we can define a chain map
$$
f\otimes g: C\otimes C'\lrta D\otimes D'
$$
by 
$$
(f\otimes g)_n=\sum_{i+j=n}f_i\otimes g_j
$$
It is easy to check this is indeed a chain map.
\end{dfn}
\begin{lemma}
If $f':C\lrta C'$  and $g':D\lrta D'$ are two more chain maps with $f$ homotopic to $f'$ and $g$ homotopic to $g'$. Then $f'\otimes g'$ is homotopic to $f\otimes g$.
\end{lemma}
\begin{thm}
(Algebraic Kuenneth Theorem) Let $(C,\pd)$ and $(D,\pd')$ be two non-negative free complex. Then for every $n\geq 0$, there is a split exact sequence
$$
0\lrta \oplus_{i+j=n}H_i(C)\otimes H_j(D)\lrta H_N(C\otimes D)\lrta \oplus_{k+\ell=n-1}\tor(H_k(C),H_\ell(D))\lrta 0
$$
where $\omega$ is the map $\langle c_i\rangle\otimes \lgl d_j\rgl\mapsto \lgl c_i\otimes d_j\rgl$. 
Thus there also exists a (non-natural) isomorphism 
$$
H_n(C\times D)\cong \oplus_{i+j=n}H_i(C)\otimes H_j (D)+\oplus_{k+\ell=n-1} \tor(H_k,(C),H_\ell(D)
$$
\end{thm}
The proof requires two auxiliary results.
\begin{prop}
Let $(E_\bullet,0)$ be a non-negative chain complex with all differential zero and $(D_\bullet,\pd)$ be any non-negative chain complex. Given $i\geq 0$, let $D^i_\bullet$ denote the chain complex where $D^i_n=D_{n-i}$ and the boundary map 
$$
D^i_n\lrta D^i_{n-1}
$$
is just the map: $D_{n-i}\lrta D_{n-i-1}$.

Then
$$
H_n(E_\bullet\otimes D_\bullet)\cong \bigoplus_{i\geq 0} H_n(E_i\otimes D^i_\bullet)
$$
\end{prop}
\begin{proof}(of the Proposition)
Since $E_\bullet$ has no differentials
$$
\begin{aligned}
\Delta(e_i\otimes d_{n-i})&=(-1)^{i} e_i \otimes \pd d_{n-i}\\
&=(-1)^i (id_E\otimes \pd)[e_i\otimes d_{n-i}]
\end{aligned}
$$
$$
\begin{aligned}
H_n(E_\bullet\otimes D_\bullet)&=\frac{ker\Delta}{im \Delta}\\
&=\bigoplus_{i\geq 0}\frac{ker(id_E\otimes \pd|_{D_{n-i}})}{im(id_E\otimes \pd|_{D_{n-i+1}})}\\
&=\bigoplus_{i\geq 0} H_n(E_i\otimes D^i_\bullet)
\end{aligned}
$$
\end{proof}

\begin{proof}(of Theorem) We will prove it in three steps:

Let's use the same notation as we did in the proof of the universal coefficient theorem.
$B_n\subset Z_n\subset C_n$. $(Z_\bullet,0)$ and $(B^+_\bullet,0)$  are chain complexes with no differentials, where $B^+_n= B_{n-1}$. $(H_\bullet,0)$ be the chain complex.
$i:Z_n\inj C_n$, $j:B_n\inj Z_n$, $d: C_n\lrta B_{n-1}$, where $d$ is the just the differential $\pd$ of $C_\bullet$ and we use $p$ to denote the projection $Z_n\surj H_n$. Then we have two short exact sequence of chain complexes
$$
0\lrta Z_\bullet\overset{i_\bullet}{\lrta}C_\bullet\overset{D_\bullet}{\lrta} B^+_\bullet\lrta 0
$$
$$
0\lrta B_\bullet\overset{j_\bullet}{\lrta}Z_\bullet\overset{p_\bullet}{\lrta} H_\bullet\lrta 0.
$$
We tensor it with $D_\bullet$.
$$
0\lrta Z_\bullet\otimes D_\bullet\overset{i_\bullet}{\lrta}C_\bullet\otimes D_\bullet\overset{D_\bullet}{\lrta} B^+_\bullet\otimes D_\bullet\lrta 0
$$
$$
0\lrta B_\bullet\otimes D_\bullet\overset{j_\bullet}{\lrta}Z_\bullet\otimes D_\bullet\overset{p_\bullet}{\lrta} H_\bullet\otimes D_\bullet\lrta 0.
$$
They are again short exact sequence of chain complexes because $D$ is free Abelian group thus flat module.
\[
\begin{tikzcd}
0 \arrow[r] & Z_n \arrow[r, "i", bend left] & C_n \arrow[l, "r", bend left] \arrow[r, "d"] & B_{n-1} \arrow[r] & 0
\end{tikzcd}
\]
This sequence splits as $B_{n-1}$ is free abelian. Thus $\exists$ a map $r:C_n\lrta Z_n$ such that $r|_{Z_n}$ is the identity
$r_\bullet:C_\bullet \lrta Z_\bullet$. 

Denote by $\mu$ the composition $p\circ r: C_\bullet \lrta H$.

Claim: $\mu$ is a chain map from $(C_\bullet,\pd)\lrta (H_\bullet,0)$. Take $c\in C_{n+1}$ and check it commutes
$$
\mu\circ \pd c=\mu\pd c=p\circ r\pd c=\lgl \pd c\rgl=0
$$
and
$0\circ \mu c=0$

Step 2: Define
$\varphi=H_n(\mu\otimes id)$. $H_n(C_\bullet\otimes D_\bullet)\lrta H_n(H_\bullet\otimes D_\bullet)$.

\underline{Claim}: $\varphi$ is an isomorphism.

It suffices to prove the diagram commutes and conclude by five lemma.
\[
\tiny
\begin{tikzcd}
H_{n+1}(B^+_\bullet\otimes D_\bullet) \arrow[d, "id"] \arrow[r, "\delta"] & H_n(Z_\bullet\otimes D_\bullet) \arrow[d, "id"] \arrow[r] & H_n(C_\bullet\otimes D_\bullet) \arrow[d, "\varphi"] \arrow[r] & H_n(B^+_\bullet\otimes D_\bullet) \arrow[d, "id"] \arrow[r, "\delta" description] & H_{n-1}(Z_\bullet\otimes D_\bullet) \arrow[d, "id"] \\
H_n(B_\bullet\otimes D_\bullet) \arrow[r] & H_n(Z_\bullet\otimes D_\bullet) \arrow[r] & H_n(H_\bullet\otimes D_\bullet) \arrow[r, "\delta'"] & H_{n-1}(B_\bullet\otimes D_\bullet) \arrow[r] & H_{n-1}(Z_\bullet\otimes D_\bullet)
\end{tikzcd}
\]

Step 3: We complete the proof
$$
\begin{aligned}
H_n(C_\bullet \otimes \otimes D_\bullet)&\cong H_n(H_\bullet\otimes D_\bullet)\\
&\cong \bigoplus_{i\geq 0} H_n(H_i(C_\bullet)\otimes D_\bullet^i)
\end{aligned}
$$
By the universal coefficient theorem, there is a split exact sequence
$$
0\lrta H_i(C_\bullet)\otimes H_n(D^i_\bullet)\lrta H_n(H_i(C_\bullet)\otimes D^i_\bullet)\lrta \tor(H_i (C_\bullet),H_{n-1}(D^i_\bullet))\lrta 0
$$
If we get rid of the notation $D^i_\bullet$.
$$0\lrta H_i(C_\bullet)\otimes H_n(D^i_\bullet)\lrta H_n(H_i(C_\bullet)\otimes D^i_\bullet)\lrta \tor(H_i (C_\bullet),H_{n-1-i}(D_\bullet))\lrta 0
$$
Take the direct sum over $i$ and use the fact that 


\end{proof}


\section{Mar 2nd: Eilenberg-Zilber}

\begin{thm}
(Eilenberg-Zilber) if $X$ and $Y$ are two topological spaces. There is a nontrivial chain equivalence
$$
\Omega_\bullet: C_\bullet(X\times Y)\lrta C_\bullet(X)\otimes C_\bullet (Y)
$$

which is unique up to chain homotopy
\end{thm}

Digression on chain equivalences
\begin{lemma}
Let $(C_\bullet,\pd)$ be a free chain complex. Then $C_\bullet$ is acyclic iff it  has contracting chain homotopy
\end{lemma}
\begin{proof}
A contracting homotopy means $Q:C_n\lrta C_{n+1}$ s.t. $Q\pd+\pd Q=id$. 

If such $Q$ exists then $H_n(C_\bullet)=0\forall n$. That direction doesn't require $C_\bullet$ to be free
$$
B_n\subseteq Z_n\subseteq C_n
$$
If we assume $C_\bullet$ is acyclic then
$$
B_n=Z_n,\forall n
$$ 
$$
0\lrta Z_n \overset{i}{\lrta} C_n\overset{\pd}{\lrta}Z_{n_1}\lrta 0
$$

Since $Z_{n-1}$ is free abelian  the sequence splits $\exists r_n:Z_{n-1}\lrta C_n$ s.t. $\pd\circ r_n=id$. Note that $id- r_{n-1}\circ \pd$ jas image in $Z_{n-1}$, $c\in C_n$. $\pd(c-r_n\pd c)=\pd c-\pd c=0$

Now define 
$Q_n:C_n\lrta C_{n+1}$ by $Q_{n}=r_n (id-r_{n-1}\circ\pd)$. This works.
$$
\begin{aligned}
\pd Q_n +Q_{n-1}\pd
&=\pd r_n (id -r_{n-1}\pd)+r_{n-1}( id-r_{n-2}\pd )\pd\\
&=id -r_{n-1}\pd+r_{n-1}\pd -r_{n-1}r_{n-2}\pd^2\\
&=0
\end{aligned}
$$
\end{proof}
\begin{dfn}
Suppose $f:(C_\bullet,\pd)\lrta (D_\bullet,\pd')$. The \textbf{mapping cone } of $f$ is the chain complex $Cone_\bullet(f),\pd^f$, where $Cone_n(f)=C_{n-1}\otimes D_n$ and 
$\pd^f:Cone_n(f)\lrta Cone_{n-1}(f)$
$$
\pd^f(c,d)=(-\pd c,f c+\pd' d)
$$
$$
\pd^f=
\begin{pmatrix}
&-\pd & 0\\
& f &\pd'
\end{pmatrix}
$$
\end{dfn}

Note if $C_\bullet$ and $D_\bullet$ are free chain complex, so is the cone.

\begin{lemma}
If $f:C_\bullet\lrta D_\bullet$ is a chain map between two free chain complexes and $Cone_\bullet(f)$ is acyclic then $f$ is  a chain equivalence.
\end{lemma}
\begin{proof}
If $Cone_\bullet(f)$ is acyclic, there exists $Q$ s.t.
$$
Q\pd^f+\pd^f Q=id
$$
$$
Q=
\begin{pmatrix*}
p & g\\
r & -p'
\end{pmatrix*}
$$
$$
\begin{pmatrix*}
\pd & 0\\
f & -\pd'
\end{pmatrix*}
\begin{pmatrix*}
p & g\\
r & -p'
\end{pmatrix*}
+
\begin{pmatrix*}
p & g\\
r & -p'
\end{pmatrix*}
\begin{pmatrix*}
\pd & 0\\
f & -\pd'
\end{pmatrix*}
=\begin{pmatrix*}
id & 0\\
0 & id
\end{pmatrix*}
$$
$$
\begin{pmatrix*}
-\pd p-p\pd +gf & -\pd g+g \pd'\\
* & fg-\pd' p'-p'\pd'
\end{pmatrix*}
\begin{pmatrix*}
id & 0\\
0 & id
\end{pmatrix*}
$$
Then we know 
$g:D_\bullet \lrta D_\bullet$ is a chain map

$p\pd +\pd p=gf-id$

$p'\pd'+\pd'p=fg-id$. Thus $f$ is a chain equivalence with inverse $g$.
\end{proof}

\begin{lemma}
Let $f: C_\bullet\lrta D_\bullet$. Then there is a LES
$$
\cdots\lrta H_{n+1}(Cone_\bullet(f))\lrta H_n(C_\bullet)\overset{H_{n}(f)}{\lrta} H_n (D_\bullet)\lrta H_n(Cone_\bullet(f))\lrta \cdots
$$
\end{lemma}
\begin{proof}
Denote by $C^+_\bullet$ the chain complex $C^+_n=C_{n-1}$. There is a SES
$$
0\lrta D_\bullet\overset{i}{\lrta} Cone_\bullet(f)\overset{p}{\lrta} C^+_\bullet\lrta 0
$$
with $i(d)=(0,d)$ and $p (c,d)=c$

Pass to the LES in homology
\[
\begin{tikzcd}
\cdots  \arrow[r] & H_{n+1}(Cone_\bullet(f)) \arrow[r] & H_{n+1}(C^+_\bullet) \arrow[r] \arrow[d, equal] & H_n(D_\bullet) \arrow[r] & H_n(Cone_\bullet(f)) \arrow[r] & \cdots \\
 &  & H_n(C_\bullet) &  &  & 
\end{tikzcd}
\]

It remains to check $\delta=H_n(f)$.


Note if $c$ is a cycle in $C_n$. Then 
$$
\pd^f\circ p^{-1}(c)=(-\pd c, fc)=(0,fc)=i(fc)
$$
$$
\delta:\lgl c\rgl\longmapsto \lgl i^{-1}\pd^fp^{-1}c\rgl=\lgl fc\rgl= H_{n}(f)\lgl c\rgl
$$
\end{proof}

\begin{prop}
Suppose $F:C_\bullet\lrta D_\bullet$ is  a chain map between the two free chain complex . Then $F$ is a chain equivalence iff 
$$
H_n(f): H_n(C_\bullet)\lrta H_n(D_\bullet)
$$
is an isomorphism for all $n$,
\end{prop}
\begin{proof}
If $f$ is a chain equivalence then $H_n (f)$ is always a isomorphism. This does not require any freeness assumptions and we proved in last semester.

For the converse, if $H_n(f)$ is always an isomorphism, then the LES
$$
\cdots\lrta H_{n+1}(Cone_\bullet(f))\lrta H_n(C_\bullet)\overset{\cong}{\lrta} H_n (D_\bullet)\lrta H_n(Cone_\bullet(f))\lrta \cdots
$$
This implies $H_n(Cone_\bullet (f))=0,\forall n$. Then $Cone_\bullet(f)$ is acyclic, and we can conclude by the previous lemma.
\end{proof}

\underline{Recap on Acyclic models.}

\begin{dfn}
Suppose $\calc$ is a category and $T_\bullet:\calc\lrta Comp$ is a functor. A family of \textbf{models} in $\calc$ is simply a subset of $obj(\calc)$

Fix $n\in \intg$ and consider  $T_n:\calc\lrta Ab$
$$
T_n(\calc)=(T_\bullet(\calc))_{n \text{th group}}
$$

A $T_n$ model set $\chi$ is simply a choice of element $x_\lambda\in T_n(M_\lambda)$ for each $\lambda$ $\mathcal{M}=\{M_\lambda|\lambda\in \Lambda\}$

We say that the model is free if the following condition holds.
\begin{enumerate}
\item $T_n(C)$ is a free abelian group $\forall C\in \calc$
\item There is a $T_n$-model set $\{x_\lambda|\lambda\in \Lambda\}$ s..t
$$
\{T_n(f)()x_\lambda|f\in Hom(M_\lambda,C), \lambda\in \Lambda\}
$$
is a basis for the free abelian group $T_n(C)$.
\end{enumerate}
$f: M_\lambda\lrta C$ is a  morphism in $\calc$ $T_n (f): T(M_\lambda)\lrta T_n(C)$ is a  homomorphism between two abelian groups. $T(M_\lambda)\in T_n(f)(x_\lambda)$ does indeed belong to $T_n(C)$. A baissi for $T_n(C)$ is obtained by letting $f$ run over all of $Hom(M_\lambda, C)$ and letting $\lambda$ run over $\Lambda$. 

We say $T_\bullet:\calc\lrta Comp$ if free with basis in $\mathcal{M}$ if each $T_n$ is free with basis in $\mathcal{M}$
\end{dfn}

\begin{dfn}
$T_\bullet\calc\lrta Comp$, we say $T_\bullet$ is\textbf{non-negative} if $T_n(C)=0$ forall $n<0$ and $\forall C$. $T_\bullet$ is \textbf{acyclic in the positive degrees on $C$} if $H_n(T_\bullet(C))=0,\forall n>0$. 
\end{dfn}
Suppose $T_\bullet C\lrta Comp$. $H_0\circ T_\bullet \calc\lrta Ab$.
\begin{thm}
Suppose $\calc$ is a categroy with omdels $\mathcal{M}$. Supose $S_\bullet, T_\bullet:\calc\lrta Comp$ are 2 functors such that $S$ and $T$ are non-negative and acyclic in positive degree on every model, and 
both $S$ and $T$ are free with basis in $\mathcal{M}$.

Suppose 
$$
\Theta: H_0\circ S_\bullet\lrta H_0\circ T_\bullet
$$
 is a natural equivalence. $\exists $ a  natural cahin equivalence $\Psi_\bullet:S_\bullet\lrta T_\bullet$ which isn unique up to chain homotopy and has $H_0(\Psi_\bullet)=\Theta$
\end{thm}
\begin{ex}
Take $\calc=Top$, $\mathcal{M}=\{\Delta^n|n\geq 0\}$.  $T$ is the singular chain functor.
$$
\calc_\bullet: Top\lrta Comp
$$
$$
X\mapsto C_\bullet(X)
$$
$C_\bullet$ is non-negative, \checkmark.
$H_n(C_\bullet(\Delta^i))=H_n(\Delta^i)=$.


\end{ex}
\underline{Claim}: $C_n$ is free with basis in $\Delta^n$

Choose an element $x\in C_n(\Delta^n)$. Take $x$ to be the identity map $\Delta^n\lrta \Delta^n$, write this as $\ell_n:\Delta^n\lrta \Delta^n$. Think of the identity map as an element of $C_n(\Delta^n)$ if $\sigma$ is any $n-$simplex in any topological space
$C_n(\sigma)(\ell_n)=\sigma\circ \ell_n=\sigma$

$\{C_n(\sigma)(\ell_n)|\sigma: \Delta^n\lrta X\}$ is basis for the free abelian group $C_n(X)$.

Eilenberg-Zilber
$Top\times Top$ is the category of pairs $(X,Y)$ of topological spaces.

We will define two functor from $Top\times Top\lrta Comp$
$S_\bullet(X,Y)=C_\bullet(X,Y)$. $T_\bullet (X,Y)=C_\bullet(X)\otimes C_\bullet(Y)$

For models
$$
\mathcal{M}=\{(\Delta^i,\Delta^j), i,j \geq 0\}
$$ 
\underline{Claim}: $S$ and $T$ are both acyclic in positive degree on $\mathcal{M}$ and free with basis in $\mathcal{M}$

$S_\bullet$, $H_n(S_\bullet(\Delta^i, \Delta^j))=H_n(\Delta^i\times \Delta^j)=0$, $\forall n>0, \forall i,j$

$S_i: Top\times Top\lrta Ab$

$S_i(X, Y)=C_i (X\times Y)$

\underline{Claim}: $\{(\Delta^i,\Delta^i)\}$ is a $S_i$-model set and a basis is $d_i:\Delta^i\otimes \Delta^i$ the diagonal map $x\mapsto (x,x)$ gives a basis
$$
\sigma: \Delta^i\lrta X\times Y
$$
we can write 
$\sigma=(\sigma_x,\sigma_y)\circ d_i$,  where $\sigma_x=p_X\circ \sigma$ be the composition of $\sigma$ with $p_X:X\times Y\lrta X$.

$\sigma=S_i(\sigma)(d_i)$ so that
$\{s_i(\sigma )(d_i\|\sigma: \Delta^i\lrta X\times Y\}$ is a basis of the free abelian group $C_i(X\times Y)$. $T_i(X\times Y)=(C_\bullet(X)\otimes C_\bullet(Y))$. $T_i(X, Y)$ is the tensor product of the free groups and so is free.
$\{(\ell_i,\ell_j)|i+j=n\}$ is a $T_n$-model basis.

The last thing to check is that $T_\bullet(\Delta^i, \Delta^j)$ is acyclic in positive degrees
$$
H_n(C_\bullet(\Delta^i)\otimes C_\bullet(\Delta^j))=0,\forall n>0.
$$
We can not compute this! However we can cheat
$$
H_n(C_\bullet(\Delta^i))=H_n(\Delta^i)=\left\{\begin{matrix*}
 \intg & n=0\\
 0 & n\neq 0
\end{matrix*}\right.
$$

Consider the chain complex
$$
0\lrta 0\lrta \cdots\lrta 0\lrta \intg\lrta 0\cdots
$$
$C_\bullet(\Delta^i)$ has the same homology as this complex. Thus $C_\bullet(\Delta^i)$ is equivalenct to the complex and $C_\bullet(\Delta^j)$ is also chain equivalent to it. $C_\bullet(\Delta^i)\otimes C_\bullet(\Delta^j)$ is chain equivalent to 
$$
0\lrta 0\lrta \cdots\lrta 0\lrta \intg\otimes \intg\lrta 0\cdots
$$
Thus $H_n(C_\bullet(\Delta^i)\otimes C_\bullet(\Delta^j))=H_n(0\lrta \intg\otimes \intg\lrta 0\cdots)$

\underline{Want}: 
$\Theta:H_0\circ S_\bullet\lrta H_0\circ T_\bullet$ is a natural equivalence.
$$
(x,y)\mapsto x\otimes y
$$
$$
H_0(C_\bullet(X\times Y))\lrta H_0(C_\bullet(X)\otimes C_\bullet (Y))
$$
By the Acylic model theorem
$$
\Omega_\bullet: S_\bullet\lrta T_\bullet
$$
 is a natural chain equivalence
 $$
\Omega_\bullet:C_\bullet(X\times Y)\lrta C_\bullet(X)\otimes C_\bullet(Y)
 $$

\begin{cor}
Kueneth formula.
Let $X$ and $Y$  be otpological spaces then for $n\geq0$

There is a split exact sequence
\end{cor}
\section{Mar 7th: Cochain complexes and cohomology}

\section{Mar 9th: Universal coefficient theorem for cohomology}
\begin{dfn}
Suppose $A$ is an abelian group and let 
$$
0\lrta K \overset{f}{\lrta} F\lrta A\lrta 0
$$
is a short free resolution. Take another abelian group and apply $\hom(\square, B)$, we can find an exact sequence
$$
0\lrta \hom(A, B)\lrta \hom(F,B) \overset{\hom(f,B)}{\lrta} \hom(K,B)
$$
and we define $\ext(A,B):=\coker\hom(f,B)=\hom (K,B)/\im \hom(f,B)$. Thus $\ext (A,B) $measures the failure for $\hom(\square, B)$ to be right exact.
\end{dfn}
Here is a more sophisticated way of viewing $\ext (A,B)$ consider a chain complex $C_1=K$, $C_0=F$, $\pd_1:C_1\lrta C_0=f: K\lrta F$ and all other group zero. $H_0(C_\bullet)=A$. Now apply $\hom(\square, B)$ to a cochain complex $\hom(C_\bullet ,B)$, the definition of $\ext (A,B)$ gives us immediately that
$$
H^1(\hom(C_\bullet,A))=\ext(A,B).
$$
From this it follows that $\ext(\square, B)$ is a contravariant functor and it is well defined (independent of the choice of short free resolution.)
\begin{dfn}
An abelian group $D$ is said to be \textbf{divisible} if  for every $b\in D$ and every $n\in \mathbb{N}$ there exists an $a\in D$ s.t., $na=b$
\end{dfn}

\begin{thm}
(Properties of $\ext$)

For a fixed abelian group $A$, $\ext(\square, A)$ is a contravariant functor and $\ext(A,\square)$ is a covariant funcotr. Moreover,
\begin{enumerate}[label=(\arabic*)]
\item If $F$ is a free group, then $\ext (F,B)=0,\forall B$. If $D$ is a divisible abelian group, then $\ext (A,D)=0\forall A$.
\item If  $A$ is a finitely generated group with torsion subgroup $T(A)$ then $\ext(A,\intg)=T(A)$
\item $0\lrta A\lrta A'\lrta A''\lrta 0$ is exact, then for any $B$, there is an exact sequence
$$
\begin{tikzcd}
0 \arrow[r] & \hom(A'',B) \arrow[r] & \hom(A',B) \arrow[r] & \hom(A,B) \arrow[lld] &  \\
 & \ext(A'',B) \arrow[r] & \ext(A',B) \arrow[r] & \ext(A,B) \arrow[r] & 0
\end{tikzcd}
$$


If $0\lrta B\lrta B'\lrta B''\lrta 0$ is exact, then for any $A$, there is an exact sequence
$$
\begin{tikzcd}
0 \arrow[r] & \hom(A,B) \arrow[r] & \hom(A,B') \arrow[r] & \hom(A,B'') \arrow[lld] &  \\
 & \ext(A,B) \arrow[r] & \ext(A,B') \arrow[r] & \ext(A,B'') \arrow[r] & 0
\end{tikzcd}
$$
\item If $B$ is an abelian group and $\{A_\lambda|\lambda\in \Lambda\}$ is a collection of abelian group then 
$$
\ext\left(\bigoplus_{\lambda\in\Lambda} A_\lambda,B\right)\cong \prod_{\lambda\in \Lambda} \ext(A_\lambda, B)
$$
$$
\ext\left(B, \prod_\lambda A_\lambda\right)\cong \prod_{\lambda\in \Lambda} \ext(B,A_\lambda)
$$
\item For any $m\in \mathbb{N}$ and any $B$
$$
\ext(\intg_m, B)\cong B/mB
$$
\end{enumerate}
\end{thm}
\begin{proof}
Every thin g apart from $(1)$ and $(2)$ all follow identically to the corresponding statements about $\tor$. These two new statements are left as exercise. 
\end{proof}

\subsection*{Three universal coefficient theorems}

Let $C_\bullet$ be a chain complex and $A$ be an abelian group, form $\hom(C_\bullet, A)$ as a cochain complex.

we define a natural chain map $\zeta$ as follows
$$\zeta:H^n(\hom(C_\bullet,A))\lrta \hom(H_n(C_\bullet ),A)$$
$$
\zeta\lgl \gamma\rgl\lgl c\rgl =\gamma(c)\in A
$$
We need to check this is well defined.

Suppose $\gamma,\gamma'$ to be cocycles s.t. $\lgl\gamma\rgl=\lgl\gamma'\rgl$, $c,c'$ to be cycles s.t. $\lgl c\rgl=\lgl c'\rgl$.

\underline{Claim}: $\gamma(c)=\gamma'(c')$.

$$
\begin{aligned}
 \gamma'&=\gamma+ d \delta\\
c'&=c+\pd a\\
\gamma'(c')&= \gamma(c)+ d\delta( c)+\gamma(\pd a)+d \delta(\pd a)\\
&= \gamma(c)+ \delta(\pd c)+d\gamma(a)+d \delta(\pd a)\\
&=\gamma(c)
\end{aligned}
$$
\begin{thm}
(The dual universal coefficients theorem)

Let $(C_\bullet,\pd)$ be a free chain complex and let $A$ be an abelian group. Then for every $n$ there is a split exact sequence
$$
0\lrta \ext(H_{n-1}(C_\bullet, A))\lrta H^n(\hom(C_\bullet, A))\overset{\zeta}{\lrta}\hom(H_n(C_\bullet)m A)\lrta 0,
$$
where $\zeta$ is the map defined above.
$$
H^n(\hom(C_\bullet , A))\cong \hom(H_n(C_\bullet), A)\oplus\ext (H_n{C_\bullet}, A). 
$$
This specialize to $C_\bullet=C_\bullet(X)$  for $X$ a topological space
\end{thm}
\begin{proof}
Go through the proof of universal coefficient theorem and 
\begin{enumerate}[label=(\alph*)]
\item erase $\square \otimes A$ and write $\hom(\square, A)$ 
\item erase $\tor $ and write $\ext$
\item reverse arrow when needed.
\end{enumerate}
\end{proof}

\begin{dfn}
A topological space $X$ is of \textbf{finite type} is $H_n(X)$ is finitely generated for all $n$. This covers all spaces we have looked at so far.
\end{dfn}
\begin{cor}
Let $X$  be of finite type  Denote  by $T_n(X)$ the torsion subgroup of $H_n(X)$. Then $\forall n\geq 0$
$$
H^n(X)\cong H_n(X)/T_n(X)\oplus T_{n-1}(X)
$$ 
\begin{proof}
For any finitely generated group  $A$, $\hom(A,\intg)\cong A/T(A)$[in problem sheet]
$\ext(H_{n-1}(X),\intg)\cong T_{n-1}(X)$ by property $(2)$ of Ext theorem.
\end{proof}
If $C_\bullet$ and $D_\bullet$ be two chain complex there is a natural chain map to male $\hom(C_\bullet, D_\bullet)$ to a cochain complex, and one could mimic what we did in Lecture 26 to obtain an algebraic  Kuenneth lemma. But it is useless, because there is no analogous Eilenberg-Zilber theorem for cochain complex.
\end{cor}

\begin{prop}
Let $X$ be a topological space of finite type. Then there exists a free non-negative chain complex $(E_\bullet,\epsilon )$ such that $(C_\bullet(X),\pd)\cong (E_\bullet,\epsilon)$ and such that each $E_n$ is finitely generated.
\end{prop}
\begin{proof}
Let $p:Z_n(X)\lrta H_n(X)$, $c\mapsto \lgl c\rgl$. Since $H_n(X)$ is finitely generated, $\exists$ finitely generated subgroup $F_n\subset Z_n(X)$ s.t., $p|_{F_n}$  is surjective. Note $F_n$ is free (As $Z_n(X)$ is free). $F_n'=\ker p|_{F_n}:F_n\lrta H_n(X)$, 
$$
E_n=F_n\oplus F_n'
$$
$E_n$ is indeed free and finitely generated.
$\epsilon: E_n\lrta E_{n-1} $ $(c,c')\mapsto (c',0)$, then obviously, $\epsilon^2=0$, $(E_\bullet,\epsilon)$ is a chain complex.
\begin{equation}\tag{*}
H_n(E_\bullet)=\frac{ker \epsilon :E_n\lrta E_{n-1}}{\im \epsilon E_{n+1}\lrta E_n}=\frac{F_n}{F_{n-1}'}=H_n(X).
\end{equation}

Now let us build a chain map $$
f:(E_\bullet, \epsilon)\lrta (C_\bullet(X),\pd)
$$
Since $F_n'$ is free abelian, there exists a homomorphism
$$
g: F_n'\lrta C_{n+1}(X)
$$
s.t. $\pd g(c')=c',\forall c'\in F_n'$. (Lemma 22.3) in ATI notes.

Define 
$$
f: E_n\lrta C_n(X)
$$
$$
(c,c')\longmapsto c+ g(c')
$$
\underline{Claim}: $f\circ \epsilon=\pd \circ f$
$$
f\epsilon (c,c')=f(c',0)=c'
$$
$\pd f(c,c')=\pd c+\pd g(c)$. But $F_n\subset Z_n(X)$ by assumption so $\pd f(c,c')=c'=f\epsilon(c,c')$. Thus $f$ is a chain map and we get an induced map.
$$
H_n(f): H_n(E_\bullet)\lrta H_n(X)
$$
which is isomorphism by $(*)$, then $f$ is a chain equivalence because $E_n$ is free.

\end{proof}

\begin{lemma}
Let $E_\bullet$ be a chian complex s.t. each $E_n$ is finitely generated, and let $A$ be an abelian group. Then there is an isomorphism of cochain complexes
$$
\hom(E_\bullet, \intg)\otimes A\cong \hom(E_\bullet, A)
$$
\end{lemma}
\begin{proof}
$$
h: \hom(E_n,\intg)\lrta \hom(E_n,A)
$$
$$
\gamma\otimes a\longmapsto [c\mapsto \gamma(c)a]
$$
$E_n\ni h(\gamma\otimes a)(c)=\gamma(c)\cdot a$, where $\gamma(c)\in \intg$ and this is multiplication in $A$. This is clearly a chain map  but hwy is it an isomorphism?

induct on the rank of $E_n$ if rank $E_n=1$ then $E_n\cong \intg$ and $\intg\otimes A\cong A$. 

For the inductive steps, we just use that both $\otimes $ and $\hom$ respects $(B\oplus B')$
\end{proof}


\begin{thm}
(Cohomological universal coefficients theorem) Let $X $ be a topological space of finite type and let $A$ be an abelian group. Then $\forall n\geq 0$, there are split exact sequences.
$$
0\lrta H^n(X)\otimes A\lrta H^n(X,A)\lrta \tor(H^{n+1},A)\lrta 0
$$
\end{thm}
\begin{proof}
By the proposition, $\exists $ a free and finitely generated chain complex $E_\bullet$ which is chain equivalent to $C_\bullet (X)$ . Set $E^\bullet=\hom(E_\bullet.\intg)$. Then $E^\bullet$ is free, finitely generated cochina complex, thus by the UCT, there is a split short exact sequence.
$$
0\lrta H^n(E^\bullet)\otimes A\lrta H^n(E^\bullet\otimes A)\lrta \tor(H^{n+1}(E^\bullet), A)\lrta 0
$$
$H^n(E^\bullet)=H^n(\hom(
E_\bullet ,\intg))\cong H^n(\hom(C_n(X),\intg))=H^n(X)$ 

$E^\bullet\otimes A=\hom(E,\intg)\otimes A=\hom(E_\bullet,A)\cong \hom(C_\bullet(X),A)$
\end{proof}

\begin{thm}
(Kuenneth Formula for cohomology)

Let $X$ and $Y$ be topological spaces of finite type. Then $\forall n\geq 0$, $\exists $a split short exact sequence.
$$
0\lrta \bigoplus_{i+j=n}H^i(X)\otimes H^j(Y)\lrta H^n(X\times Y)\lrta \bigoplus_{k+\ell=n+1}\tor(H^k(X),H^\ell(Y))\lrta 0
$$
\end{thm}
\begin{proof}
Let $E_\bullet, F_\bullet$ be two finitely generated free chain complexes equivalent to $C_\bullet(X), C_\bullet(Y)$ respectively.
\end{proof}

\section{Mat 16th: Ring structure on cohomology.}

\begin{dfn}
A ring $R$ is an abelian group with a additional  operation called ``multiplication'' which is associative and distributive over the abelian group structure. There is a multiplicative identity.
\end{dfn}

\begin{rmk}
If $1=0$, we have $R=\{0\}$ is the zero ring.
\end{rmk}

we omit a lot of stuffs on ring theory here and only state a lemma that we will use later
\begin{lemma}
Suppose $R$ is a graded ring and $I$ is a homogeneous graded ideal. then quotient ring is again a graded ring. 
$$
R/I=\bigoplus_n(R^n+I)/I
$$
\end{lemma}

\begin{dfn}
Let $R$ be a ring. Let $X$ be a topological space. 
$H^*(X,R)$ is the total cohomology
$$
\bigoplus_{n\geq 0} H^n(X;R)
$$
\end{dfn}

\underline{Goal}: If $R$ is  commutative, we will show $H^*(X;R) $ is a graded ring (NOT necessarily  commutative.)

Suppose $A$ is an abelian group. How to make $A$ a ring?

If $R$ is a ring, then $\hom(A,R)$ is naturally a ring.
$f,g\in\hom(A,R)$, $fg$ is defined as $(fg)(a)=f(a)g(a)\forall a\in A$. We will use this to endow $H^*(X;R)$ with a ring structure.

Let's recall the face maps from last semester. The face maps 
$$
\begin{aligned}
\epsilon^n_i:&\Delta^{n-1}\lrta \Delta^n\\
&(s_0,s_1..,s_{n-2})\mapsto (s_0,...s_{i-1}, 0,s_{i},...,s_{n-2})
\end{aligned}
$$
maps the standard $n-1$ simplex onto the $i$-th face of the standard $n$-simplex.

\begin{dfn}
Let $0\leq i\leq n$, the ith \textbf{front face}
$$
\begin{aligned}
F_i^n:\Delta^i&\lrta \Delta^n\\
(s_0,...,s_{i-1})\mapsto (s_0,...,s_{i-1},0,0,..,0)
\end{aligned}
$$
the ith \textbf{back face}
$$
\begin{aligned}
B_i^n:\Delta^i&\lrta \Delta^n\\
(s_0,...,s_{i-1})\mapsto (0,0,..,0,s_0,...,s_{i-1})
\end{aligned}
$$
\end{dfn}

\begin{lemma}
\ \begin{enumerate}
 \item $\epsilon^{n+1}_0=B^{n+1})n$ $\epsilon^{n+1}_{n+1}=F^{n+1}_n$
 \item $B^n_{m+k}\circ B^{m+k}_k=B^n_k$, $F^n_{m+k}\circ F^{m+k}_k=F^n_k$
 \item 
 $$
\epsilon^{n+1}_{i}\circ F_m^n=\left\{\begin{aligned}
&F^{n+1}_{m+1}\circ \epsilon^{m+1}_i \ \  i\leq m\\
&F^{n+1}_m\ \ \   i\geq m+1
\end{aligned}\right.
 $$

\end{enumerate}
\end{lemma}


 Let $\alpha\in C^n(H,R)=\hom(C_n(X),R)$, $\beta\in C^m(X,R)=\hom(C_m(X),R)$. 
 Define $\alpha\smile \beta$ the cup product of $\alpha$ and $\beta$ to the elemetnof $C^{n+m}(X,R)$ defined by 
 $$
\alpha\smile\beta(\sigma)=\alpha(\sigma\circ F^n)\beta(\sigma\circ B_m)
 $$

 $F_n:\Delta^n\lrta \Delta^{n+m}, \sigma\circ F_n: \Delta^n\lrta \Delta^{n+m}\lrta X$ is a singular $n$-simplex. $\alpha(\sigma\circ F_n)$ is a well-defined element of $R$. Similarly, $\beta(\sigma\circ B_m)$ is an element in $R$ . Then $\alpha(\sigma\circ F_m)\beta(\sigma\circ B_m)$ is an element of $R$. it is then clear that  $\alpha\smile\beta$ extends by linearity to define an element of $\hom(C_{n+m}(X),R)=C^{n+m}(X,R)$

\begin{prop}
Let $R$ be commutative. Then $C^*(X;R)=\bigoplus_{n\geq 0}C^n(X,R)$ is a graded ring under the cup product
\end{prop}
\begin{proof}
Let $\alpha\in C^n$, $\beta,\gamma\in  C^m$.

\underline{Distributive}: $\alpha\smile(\beta+\gamma)=\alpha\smile \beta+\alpha\smile\gamma$. Take $\sigma: \Delta^{n+m}\lrta X$ and
$$
\begin{aligned}
(\alpha\smile(\beta+\gamma))(\sigma)&=\alpha(\sigma\circ F_n)[(\beta+\gamma)(\sigma\circ B_m)]\\
&=\alpha(\sigma\circ F_n)[\beta(\sigma\circ B_m)+\gamma(\sigma\circ B_m)]\\
&=\alpha(\sigma\circ F)\beta(\sigma\circ B_m)+\alpha(\sigma\circ F)\gamma(\sigma\circ B_m)\\
&=\alpha\smile\beta(\sigma)+\alpha\smile \gamma(\sigma)
\end{aligned}$$
The same argument shows $(\beta+\gamma)\smile \alpha=\beta\smile\alpha+\gamma\smile\alpha$

\underline{Associativity} take $\alpha\in C^n,\beta\in C^m,\gamma\in C^p$ and $\sigma:\Delta^{n+m+p}\lrta X$.
$$
(\alpha\smile \beta)\smile \gamma(\sigma)=\alpha(\sigma\circ F_{n+m}\circ F_n)\beta(\sigma\circ F_{n+m}\circ F_{n})\cdot\gamma(\sigma\circ B_p)
$$
$$
\alpha\smile (\beta\smile \gamma)( \sigma)=\alpha(\sigma\circ  F_n)\beta(\sigma\circ B_{m+p}\circ F_{m})\cdot\gamma(\sigma\circ B_{m+p}\circ B_p)
$$
By the face relation lemma the above two equal.

\underline{identity}: Define $\nu(x)=1_R,\forall x\in X$
\end{proof}

how does this ring structure behave with respect to continuous map? Take
$$
f:X\lrta Y
$$
$$f_\#:C_\bullet(X)\lrta C_\bullet(Y)$$
$$
f^\#:C^\bullet(X,R)\lrta C^\bullet(X,R)
$$
Claim: $f^\#(\alpha\smile \beta)=f^\#(\alpha)\smile f^\#(\beta)$
$$
\begin{aligned}
f^\#(\alpha\smile \beta)(\sigma)&=\alpha\smile \beta(f_\#\sigma)\\
&=(\alpha\smile \beta)(f\circ \sigma)\\
&=\alpha(f_\#(\sigma\circ F_n))\beta(f_\#(\sigma\circ B_m))\\
&=f^\#(\alpha)\smile f^\#(\beta)(\sigma)
\end{aligned}
$$
\begin{cor}
There is a contravariant functor
$$
C^*(\square,R): TOP\lrta Gr-Rings.
$$
The ring structure is not very helpful, because it does not descend to the homotopy category.
\end{cor}

We will see now that $\smile$ induces an operation on cohomology
$$
\lgl\alpha\rgl\smile\lgl\beta\rgl=\lgl\alpha\smile\beta\rgl
$$
and induces a ring structue on $H^*(X;R)$ that does indeed respect homotopy.


\begin{thm}(Ring structure on cohomology)
If $R$ is a commutative ring, then 
$H^*(\square, R): h-Top\lrta GrRings$ is a well-defined functor.
\end{thm}

\begin{prop}
$d(\alpha\smile\beta)=d\alpha\smile \beta+(-1)^n\alpha\smile d\beta$
\end{prop}
\begin{proof}
$$
\begin{aligned}
d(\alpha\smile \beta)(\sigma)&=(\alpha\smile\beta)(\pd\sigma)\\
&=\sum_{i=0}^{n+m+1}(-1)^i(\alpha\smile \beta)(\sigma\circ \epsilon_i)\\
&=\sum_{i=0}^{n}(-1)^i
\alpha(\sigma\circ \epsilon_i\circ F_n)\beta(\sigma\circ \epsilon\circ B_m)+\sum_{i=n+1}^{m+n+1}(-1)^i\alpha(\sigma\circ\epsilon_i\circ F_n)\beta(\sigma\circ \epsilon_i\circ B_m)
\end{aligned}
$$
the first sum becomes $(d\alpha\smile \beta)(\sigma)$ and the second sum is $(-1)^n(\alpha\smile d\beta)(\sigma)$
\end{proof}

Assuming the proposition , lets prove the theorem

\begin{proof}
$$
Z^*=\bigoplus Z^n(X,R)
$$
$$
B^*=\bigoplus B^n(X,R)
$$
If $\alpha,\beta\in Z^*$ then 
$$
d(\alpha\smile \beta)=d\alpha\smile \beta+(-1)^{?}\alpha\smile d\beta=0
$$
Thus $Z^*$ is a graded subring of $C^*$ if $\alpha\in Z^n$ and $\beta\in B^m$ say $\beta=d \gamma$
then
$$
\alpha\smile \beta=\alpha\smile d\gamma=\pm \alpha\smile \gamma+\alpha \smile d\gamma=\pm d(\alpha\smile \gamma)
$$
thus $\alpha\smile \beta\in B^{n+m}$, similarly $\beta\smile \alpha\in B^{n+m}$. $B^*$ is a homogeneous two sided ideal of $C^*$.

Thus by ht elemma at the start of the lecture,
$$
H^*(X;R)=Z^*/B^*
$$
has a graded ring structure, where
$$
\lgl\alpha\rgl\smile\lgl\beta\rgl=\lgl\alpha\smile\beta\rgl
$$.

If $f:X\lrta Y$
$$
H^*(f):\sum_{n\geq 0}H^n(f):\bigoplus H^n(Y;R)\lrta \
$$
\end{proof}

\end{document}
