\documentclass[11pt]{article}
\usepackage{amssymb}
\usepackage{latexsym}
\usepackage{amsmath}
\usepackage{amsthm}
\usepackage{stmaryrd}
\usepackage{fancyhdr}
\pagestyle{headings}
\usepackage{dsfont}
\usepackage{pifont}
\usepackage{mathtools}
\usepackage{natbib}
\usepackage{tikz-cd}
\usepackage{pgfplots}
\usepackage{enumitem} 
\usepackage{hyperref}
\usepackage{geometry}
\geometry{left=4cm,right=4cm}
\pgfplotsset{every axis/.append style={
                    axis x line=middle,    % put the x axis in the middle
                    axis y line=middle,    % put the y axis in the middle
                    axis line style={<->}, % arrows on the axis
                    xlabel={$x$},          % default put x on x-axis
                    ylabel={$y$},          % default put y on y-axis
                    ticks=none,
                    }}
%\usepackage[urw-garamond]{mathdesign}
%\usepackage{cmbright}
%\usepackage{concmath}
%\usepackage{sansmathfonts}
%\renewcommand*\familydefault{\sfdefault} %% Only if the base font of the document is to be sans serif

%\usepackage{pdfrender,xcolor,scrpage2}
%\pdfrender{StrokeColor=black,TextRenderingMode=2,LineWidth=1pt}
\tikzset{
  subseteq/.style={
    draw=none,
    edge node={node [sloped, allow upside down, auto=false]{$\subseteq$}}},
  Subseteq/.style={
    draw=none,
    every to/.append style={
      edge node={node [sloped, allow upside down, auto=false]{$\subseteq$}}}
    },
    Subsetneq/.style={
    draw=none,
    every to/.append style={
      edge node={node [sloped, allow upside down, auto=false]{$\subsetneq$}}}
    },
  Supseteq/.style={
    draw=none,
    every to/.append style={
      edge node={node [sloped, allow upside down, auto=false]{$\supseteq$}}}
  }
}

\hypersetup{
    colorlinks,
    citecolor=blue,
    filecolor=blue,
    linkcolor=blue,
    urlcolor=blue
}
\newtheorem{thm}{Theorem}[section]
\newtheorem{prop}[thm]{Proposition}
\newtheorem{lemma}[thm]{Lemma}
\newtheorem{cor}[thm]{Corollary}
\newtheorem{dfn}[thm]{Definition}
\newtheorem{axiom}[thm]{Axiom}
\newtheorem{rmk}[thm]{Remark}
\newtheorem{rmkt}[thm]{Remark by TeXer}
\newtheorem{ex}[thm]{Example}
\newtheorem{nex}[thm]{Non-example}
\newtheorem{exercise}[thm]{Exercise}
\newtheorem{question}[thm]{Question}
\newtheorem{problem}[thm]{Problem}
\newtheorem{dfn/thm}[thm]{Definition/Theorem}
\renewcommand{\baselinestretch}{1.1}
\renewcommand{\hom}{\text{ Hom}}
\newcommand{\tor}{\text{ Tor}}
\newcommand{\affn}{\mathbb A}
\newcommand{\proj}{\mathbb P}
\newcommand{\reals}{\mathbb R}
\newcommand{\cplx}{\mathbb C}
\newcommand{\intg}{\mathbb Z}
\newcommand{\bbf}{\mathbb F}
\newcommand{\ratl}{\mathbb Q}
\newcommand{\torus}{\mathbb T}
\newcommand{\sca}{{\mathfrak a}}
\newcommand{\scb}{{\mathfrak b}}
\newcommand{\scc}{{\mathfrak c}}
\newcommand{\scm}{{\mathfrak m}}
\newcommand{\scn}{{\mathfrak n}}
\newcommand{\scp}{{\mathfrak p}}
\newcommand{\scq}{\mathfrak q}
\newcommand{\frakg}{{\mathfrak g}}
\newcommand{\frakd}{{\mathfrak d}}
\newcommand{\pd}{\partial}
\newcommand{\calf}{{\cal F}}
\newcommand{\calg}{{\cal G}}
\newcommand{\cala}{{\cal A}}
\newcommand{\calb}{{\cal B}}
\newcommand{\calc}{{\cal C}}
\newcommand{\cale}{{\cal E}}
\newcommand{\cali}{{\cal I}}
\newcommand{\call}{{\cal L}}
\newcommand{\caln}{{\cal N}}
\newcommand{\calo}{{\cal O}}
\newcommand{\calr}{{\cal R}}
\newcommand{\mathbold}{\bf}
\newcommand{\cinf}{C^{\infty}}
\newcommand{\row}[2]{#1_1,\dots ,#1_{#2}}
\newcommand{\dbyd}[2]{{\partial #1\over\partial #2}}
\newcommand{\Space}{{\bf Space}}
\newcommand{\alg}{{\mathbold Alg}}
\newcommand{\notsubset}{\not \subset}
\newcommand{\notsupset}{\not \supset}
\newcommand{\pois}{{\mathbold Pois}}
\newcommand{\pitilde}{\tilde{\pi}}
\newcommand{\rta}{\rightarrow}
\newcommand{\Lrta}{\Longrightarrow}
\newcommand{\lrta}{\longrightarrow}
\newcommand{\llrta}{\longleftrightarrow}
\newcommand{\Llta}{\Longleftarrow}
\newcommand{\Llrta}{\Longleftrightarrow}
\newcommand{\lgl}{\langle}
\newcommand{\rgl}{\rangle}
\newcommand{\inj}{\hookrightarrow}
\newcommand{\surj}{\twoheadrightarrow}
\newcommand{\cmark}{\ding{51}}%
\newcommand{\xmark}{\ding{55}}%
\newcommand{\downmapsto}{\rotatebox[origin=c]{-90}{$\scriptstyle\mapsto$}\mkern2mu}
\renewcommand{\qedsymbol}{$\square$}
\bibliographystyle{plain}
\title{\bf Summary for Algebraic Topology II}
\author{Notes by Lin-Da Xiao}
\date{2018 ETH} %\thanks{Research partially supported by NSF Grant DMS-96-25122 and the Miller Institute for Basic Research in Science.}
\begin{document}
\maketitle
\tableofcontents
\newpage
\section{21th Feb: Tor functor}
\begin{dfn}
Suppose $A$ is an abelian group, A \textbf{Free resolution} is an exact sequence of the form
$$
\cdots\lrta F_2\overset{f_2}{\lrta}F_1\overset{f_1}{\lrta}F_0\overset{f_0}{\lrta}A\lrta 0,
$$
where each $F_i$ is a free abelian group. If moreover $F_i=0,\forall i\geq 2$, we call it \textbf{Short free resolution} 
$$
0\lrta K\lrta F\lrta A\lrta 0
$$
\end{dfn}
(We can easily generalize this definition to $R$-modules)
\begin{prop}
Let $A$ be an abelian group. Then there exists a short free resolution of $A$.
\end{prop}
\begin{proof}
Let $F$ be the free abelian group generated by all elements in $A$. There is a surjection from $F$ to $A$ by linearly extending the map sending basis element to itself. Let $K$ denote the kernel of this map. $K$ is an abelian subgroup of a free abelian group ($\intg$-module).  A subgroup of a free abelian group is torsion free as a module. $\intg$ is a $PID$. If $R$ is a $PID$, then an  $R$-module is free iff it is torsion free (See Bosch section 4.2). Then we know in particular, $K$ is a free abelian group.
\end{proof}
With this construction, we can define the $\tor$ functor now:
\begin{dfn}
Let $A$ be an abelian group. Let $0\rta K\overset{f}{\rta}F\rta A\rta 0$ be a short free resolution of $A$. Given any other abelian group $B$, we define 
$$
\tor(A,B):=\ker(f\otimes id_B)
$$
\tor(A,B)
\end{dfn}

This definition is independent on the choice of short free resolution.

\section{28th Feb:}

\underline{Question}: Given $X, Y$ what is the cohomology of $X\times Y$?

\underline{Answer}:
$$
H_n(X\times Y)\cong \bigoplus_{i+j=n}H_i(X)\otimes H_j (Y)+\bigoplus_{k+\ell=n-1} \tor(H_k,(X),H_\ell(Y)
$$
We will discuss Elenberg-Zilber theorem along this line the next lecture.

Today, we will prove the Algebraic Kueneth Theorem
\begin{dfn}
Suppose $(C_\bullet,\pd)$ and $(C'_\bullet,\pd')$ are two non-negative chain complexes. We define the  \textbf{tensor complex} $(C_\bullet\otimes C_\bullet',\Delta)$, where
$$
(C_\bullet\otimes C'_\bullet)_n=\oplus_{i+j=n}C_i\otimes C_j'
$$
and the differential $\Delta$ is defined by 
$$
\Delta(c_i\otimes c'_j)=\pd c_i\otimes c'_j+(-1)^{i}c_i\otimes \pd' c_j
$$
\end{dfn}
First, note that $\Delta(c_i\otimes c'_j)$ does indeed belong to $(C_\bullet\otimes C_\bullet')_{n-1}$. The reason for $(-1)^i$ is to make $\Delta^2=0$.
$C_\bullet\otimes C'_\bullet$ is another  non-negative chain complex.
\begin{dfn}
Suppose $f_\bullet:C_\bullet\lrta D_\bullet$ and $g_\bullet: C'_\bullet\lrta D'_\bullet$ are two morphism of chain complexes. Then we can define a chain map
$$
f\otimes g: C\otimes C'\lrta D\otimes D'
$$
by 
$$
(f\otimes g)_n=\sum_{i+j=n}f_i\otimes g_j
$$
It is easy to check this is indeed a chain map.
\end{dfn}
\begin{lemma}
If $f':C\lrta C'$  and $g':D\lrta D'$ are two more chain maps with $f$ homotopic to $f'$ and $g$ homotopic to $g'$. Then $f'\otimes g'$ is homotopic to $f\otimes g$.
\end{lemma}
\begin{thm}
(Algebraic Kuenneth Theorem) Let $(C,\pd)$ and $(D,\pd')$ be two non-negative free complex. Then for every $n\geq 0$, there is a split exact sequence
$$
0\lrta \oplus_{i+j=n}H_i(C)\otimes H_j(D)\lrta H_N(C\otimes D)\lrta \oplus_{k+\ell=n-1}\tor(H_k(C),H_\ell(D))\lrta 0
$$
where $\omega$ is the map $\langle c_i\rangle\otimes \lgl d_j\rgl\mapsto \lgl c_i\otimes d_j\rgl$. 
Thus there also exists a (non-natural) isomorphism 
$$
H_n(C\times D)\cong \oplus_{i+j=n}H_i(C)\otimes H_j (D)+\oplus_{k+\ell=n-1} \tor(H_k,(C),H_\ell(D)
$$
\end{thm}
The proof requires two auxiliary results.
\begin{prop}
Let $(E_\bullet,0)$ be a non-negative chain complex with all differential zero and $(D_\bullet,\pd)$ be any non-negative chain complex. Given $i\geq 0$, let $D^i_\bullet$ denote the chain complex where $D^i_n=D_{n-i}$ and the boundary map 
$$
D^i_n\lrta D^i_{n-1}
$$
is just the map: $D_{n-i}\lrta D_{n-i-1}$.

Then
$$
H_n(E_\bullet\otimes D_\bullet)\cong \bigoplus_{i\geq 0} H_n(E_i\otimes D^i_\bullet)
$$
\end{prop}
\begin{proof}(of the Proposition)
Since $E_\bullet$ has no differentials
$$
\begin{aligned}
\Delta(e_i\otimes d_{n-i})&=(-1)^{i} e_i \otimes \pd d_{n-i}\\
&=(-1)^i (id_E\otimes \pd)[e_i\otimes d_{n-i}]
\end{aligned}
$$
$$
\begin{aligned}
H_n(E_\bullet\otimes D_\bullet)&=\frac{ker\Delta}{im \Delta}\\
&=\bigoplus_{i\geq 0}\frac{ker(id_E\otimes \pd|_{D_{n-i}})}{im(id_E\otimes \pd|_{D_{n-i+1}})}\\
&=\bigoplus_{i\geq 0} H_n(E_i\otimes D^i_\bullet)
\end{aligned}
$$
\end{proof}

\begin{proof}(of Theorem) We will prove it in three steps:

Let's use the same notation as we did in the proof of the universal coefficient theorem.
$B_n\subset Z_n\subset C_n$. $(Z_\bullet,0)$ and $(B^+_\bullet,0)$  are chain complexes with no differentials, where $B^+_n= B_{n-1}$. $(H_\bullet,0)$ be the chain complex.
$i:Z_n\inj C_n$, $j:B_n\inj Z_n$, $d: C_n\lrta B_{n-1}$, where $d$ is the just the differential $\pd$ of $C_\bullet$ and we use $p$ to denote the projection $Z_n\surj H_n$. Then we have two short exact sequence of chain complexes
$$
0\lrta Z_\bullet\overset{i_\bullet}{\lrta}C_\bullet\overset{D_\bullet}{\lrta} B^+_\bullet\lrta 0
$$
$$
0\lrta B_\bullet\overset{j_\bullet}{\lrta}Z_\bullet\overset{p_\bullet}{\lrta} H_\bullet\lrta 0.
$$
We tensor it with $D_\bullet$.
$$
0\lrta Z_\bullet\otimes D_\bullet\overset{i_\bullet}{\lrta}C_\bullet\otimes D_\bullet\overset{D_\bullet}{\lrta} B^+_\bullet\otimes D_\bullet\lrta 0
$$
$$
0\lrta B_\bullet\otimes D_\bullet\overset{j_\bullet}{\lrta}Z_\bullet\otimes D_\bullet\overset{p_\bullet}{\lrta} H_\bullet\otimes D_\bullet\lrta 0.
$$
They are again short exact sequence of chain complexes because $D$ is free Abelian group thus flat module.
\[
\begin{tikzcd}
0 \arrow[r] & Z_n \arrow[r, "i", bend left] & C_n \arrow[l, "r", bend left] \arrow[r, "d"] & B_{n-1} \arrow[r] & 0
\end{tikzcd}
\]
This sequence splits as $B_{n-1}$ is free abelian. Thus $\exists$ a map $r:C_n\lrta Z_n$ such that $r|_{Z_n}$ is the identity
$r_\bullet:C_\bullet \lrta Z_\bullet$. 

Denote by $\mu$ the composition $p\circ r: C_\bullet \lrta H$.

Claim: $\mu$ is a chain map from $(C_\bullet,\pd)\lrta (H_\bullet,0)$. Take $c\in C_{n+1}$ and check it commutes
$$
\mu\circ \pd c=\mu\pd c=p\circ r\pd c=\lgl \pd c\rgl=0
$$
and
$0\circ \mu c=0$

Step 2: Define
$\varphi=H_n(\mu\otimes id)$. $H_n(C_\bullet\otimes D_\bullet)\lrta H_n(H_\bullet\otimes D_\bullet)$.

\underline{Claim}: $\varphi$ is an isomorphism.

It suffices to prove the diagram commutes and conclude by five lemma.
\[
\tiny
\begin{tikzcd}
H_{n+1}(B^+_\bullet\otimes D_\bullet) \arrow[d, "id"] \arrow[r, "\delta"] & H_n(Z_\bullet\otimes D_\bullet) \arrow[d, "id"] \arrow[r] & H_n(C_\bullet\otimes D_\bullet) \arrow[d, "\varphi"] \arrow[r] & H_n(B^+_\bullet\otimes D_\bullet) \arrow[d, "id"] \arrow[r, "\delta" description] & H_{n-1}(Z_\bullet\otimes D_\bullet) \arrow[d, "id"] \\
H_n(B_\bullet\otimes D_\bullet) \arrow[r] & H_n(Z_\bullet\otimes D_\bullet) \arrow[r] & H_n(H_\bullet\otimes D_\bullet) \arrow[r, "\delta'"] & H_{n-1}(B_\bullet\otimes D_\bullet) \arrow[r] & H_{n-1}(Z_\bullet\otimes D_\bullet)
\end{tikzcd}
\]

Step 3: We complete the proof
$$
\begin{aligned}
H_n(C_\bullet \otimes \otimes D_\bullet)&\cong H_n(H_\bullet\otimes D_\bullet)\\
&\cong \bigoplus_{i\geq 0} H_n(H_i(C_\bullet)\otimes D_\bullet^i)
\end{aligned}
$$
By the universal coefficient theorem, there is a split exact sequence
$$
0\lrta H_i(C_\bullet)\otimes H_n(D^i_\bullet)\lrta H_n(H_i(C_\bullet)\otimes D^i_\bullet)\lrta \tor(H_i (C_\bullet),H_{n-1}(D^i_\bullet))\lrta 0
$$
If we get rid of the notation $D^i_\bullet$.
$$0\lrta H_i(C_\bullet)\otimes H_n(D^i_\bullet)\lrta H_n(H_i(C_\bullet)\otimes D^i_\bullet)\lrta \tor(H_i (C_\bullet),H_{n-1-i}(D_\bullet))\lrta 0
$$
Take the direct sum over $i$ and use the fact that 


\end{proof}


\section{2nd Mar: Eilenberg-Zilber}

\begin{thm}
(Eilenberg-Zilber) if $X$ and $Y$ are two topological spaces. There is a nontrivial chain equivalence
$$
\Omega_\bullet: C_\bullet(X\times Y)\lrta C_\bullet(X)\otimes C_\bullet (Y)
$$

which is unique up to chain homotopy
\end{thm}

Digression on chain equivalences
\begin{lemma}
Let $(C_\bullet,\pd)$ be a free chain complex. Then $C_\bullet$ is acyclic iff it  has contracting chain homotopy
\end{lemma}
\begin{proof}
A contracting homotopy means $Q:C_n\lrta C_{n+1}$ s.t. $Q\pd+\pd Q=id$. 

If such $Q$ exists then $H_n(C_\bullet)=0\forall n$. That direction doesn't require $C_\bullet$ to be free
$$
B_n\subseteq Z_n\subseteq C_n
$$
If we assume $C_\bullet$ is acyclic then
$$
B_n=Z_n,\forall n
$$ 
$$
0\lrta Z_n \overset{i}{\lrta} C_n\overset{\pd}{\lrta}Z_{n_1}\lrta 0
$$

Since $Z_{n-1}$ is free abelian  the sequence splits $\exists r_n:Z_{n-1}\lrta C_n$ s.t. $\pd\circ r_n=id$. Note that $id- r_{n-1}\circ \pd$ jas image in $Z_{n-1}$, $c\in C_n$. $\pd(c-r_n\pd c)=\pd c-\pd c=0$

Now define 
$Q_n:C_n\lrta C_{n+1}$ by $Q_{n}=r_n (id-r_{n-1}\circ\pd)$. This works.
$$
\begin{aligned}
\pd Q_n +Q_{n-1}\pd
&=\pd r_n (id -r_{n-1}\pd)+r_{n-1}( id-r_{n-2}\pd )\pd\\
&=id -r_{n-1}\pd+r_{n-1}\pd -r_{n-1}r_{n-2}\pd^2\\
&=0
\end{aligned}
$$
\end{proof}
\begin{dfn}
Suppose $f:(C_\bullet,\pd)\lrta (D_\bullet,\pd')$. The \textbf{mapping cone } of $f$ is the chain complex $Cone_\bullet(f),\pd^f$, where $Cone_n(f)=C_{n-1}\otimes D_n$ and 
$\pd^f:Cone_n(f)\lrta Cone_{n-1}(f)$
$$
\pd^f(c,d)=(-\pd c,f c+\pd' d)
$$
$$
\pd^f=
\begin{pmatrix}
&-\pd & 0\\
& f &\pd'
\end{pmatrix}
$$
\end{dfn}

Note if $C_\bullet$ and $D_\bullet$ are free chain complex, so is the cone.

\begin{lemma}
If $f:C_\bullet\lrta D_\bullet$ is a chain map between two free chain complexes and $Cone_\bullet(f)$ is acyclic then $f$ is  a chain equivalence.
\end{lemma}
\begin{proof}
If $Cone_\bullet(f)$ is acyclic, there exists $Q$ s.t.
$$
Q\pd^f+\pd^f Q=id
$$
$$
Q=
\begin{pmatrix*}
p & g\\
r & -p'
\end{pmatrix*}
$$
$$
\begin{pmatrix*}
\pd & 0\\
f & -\pd'
\end{pmatrix*}
\begin{pmatrix*}
p & g\\
r & -p'
\end{pmatrix*}
+
\begin{pmatrix*}
p & g\\
r & -p'
\end{pmatrix*}
\begin{pmatrix*}
\pd & 0\\
f & -\pd'
\end{pmatrix*}
=\begin{pmatrix*}
id & 0\\
0 & id
\end{pmatrix*}
$$
$$
\begin{pmatrix*}
-\pd p-p\pd +gf & -\pd g+g \pd'\\
* & fg-\pd' p'-p'\pd'
\end{pmatrix*}
\begin{pmatrix*}
id & 0\\
0 & id
\end{pmatrix*}
$$
Then we know 
$g:D_\bullet \lrta D_\bullet$ is a chain map

$p\pd +\pd p=gf-id$

$p'\pd'+\pd'p=fg-id$. Thus $f$ is a chain equivalence with inverse $g$.
\end{proof}

\begin{lemma}
Let $f: C_\bullet\lrta D_\bullet$. Then there is a LES
$$
\cdots\lrta H_{n+1}(Cone_\bullet(f))\lrta H_n(C_\bullet)\overset{H_{n}(f)}{\lrta} H_n (D_\bullet)\lrta H_n(Cone_\bullet(f))\lrta \cdots
$$
\end{lemma}
\begin{proof}
Denote by $C^+_\bullet$ the chain complex $C^+_n=C_{n-1}$. There is a SES
$$
0\lrta D_\bullet\overset{i}{\lrta} Cone_\bullet(f)\overset{p}{\lrta} C^+_\bullet\lrta 0
$$
with $i(d)=(0,d)$ and $p (c,d)=c$

Pass to the LES in homology
\[
\begin{tikzcd}
\cdots  \arrow[r] & H_{n+1}(Cone_\bullet(f)) \arrow[r] & H_{n+1}(C^+_\bullet) \arrow[r] \arrow[d, equal] & H_n(D_\bullet) \arrow[r] & H_n(Cone_\bullet(f)) \arrow[r] & \cdots \\
 &  & H_n(C_\bullet) &  &  & 
\end{tikzcd}
\]

It remains to check $\delta=H_n(f)$.


Note if $c$ is a cycle in $C_n$. Then 
$$
\pd^f\circ p^{-1}(c)=(-\pd c, fc)=(0,fc)=i(fc)
$$
$$
\delta:\lgl c\rgl\longmapsto \lgl i^{-1}\pd^fp^{-1}c\rgl=\lgl fc\rgl= H_{n}(f)\lgl c\rgl
$$
\end{proof}

\begin{prop}
Suppose $F:C_\bullet\lrta D_\bullet$ is  a chain map between the two free chain complex . Then $F$ is a chain equivalence iff 
$$
H_n(f): H_n(C_\bullet)\lrta H_n(D_\bullet)
$$
is an isomorphism for all $n$,
\end{prop}
\begin{proof}
If $f$ is a chain equivalence then $H_n (f)$ is always a isomorphism. This does not require any freeness assumptions and we proved in last semester.

For the converse, if $H_n(f)$ is always an isomorphism, then the LES
$$
\cdots\lrta H_{n+1}(Cone_\bullet(f))\lrta H_n(C_\bullet)\overset{\cong}{\lrta} H_n (D_\bullet)\lrta H_n(Cone_\bullet(f))\lrta \cdots
$$
This implies $H_n(Cone_\bullet (f))=0,\forall n$. Then $Cone_\bullet(f)$ is acyclic, and we can conclude by the previous lemma.
\end{proof}

\underline{Recap on Acyclic models.}

\begin{dfn}
Suppose $\calc$ is a category and $T_\bullet:\calc\lrta Comp$ is a functor. A family of \textbf{models} in $\calc$ is simply a subset of $obj(\calc)$

Fix $n\in \intg$ and consider  $T_n:\calc\lrta Ab$
$$
T_n(\calc)=(T_\bullet(\calc))_{n \text{th group}}
$$

A $T_n$ model set $\chi$ is simply a choice of element $x_\lambda\in T_n(M_\lambda)$ for each $\lambda$ $\mathcal{M}=\{M_\lambda|\lambda\in \Lambda\}$

We say that the model is free if the following condition holds.
\begin{enumerate}
\item $T_n(C)$ is a free abelian group $\forall C\in \calc$
\item There is a $T_n$-model set $\{x_\lambda|\lambda\in \Lambda\}$ s..t
$$
\{T_n(f)()x_\lambda|f\in Hom(M_\lambda,C), \lambda\in \Lambda\}
$$
is a basis for the free abelian group $T_n(C)$.
\end{enumerate}
$f: M_\lambda\lrta C$ is a  morphism in $\calc$ $T_n (f): T(M_\lambda)\lrta T_n(C)$ is a  homomorphism between two abelian groups. $T(M_\lambda)\in T_n(f)(x_\lambda)$ does indeed belong to $T_n(C)$. A baissi for $T_n(C)$ is obtained by letting $f$ run over all of $Hom(M_\lambda, C)$ and letting $\lambda$ run over $\Lambda$. 

We say $T_\bullet:\calc\lrta Comp$ if free with basis in $\mathcal{M}$ if each $T_n$ is free with basis in $\mathcal{M}$
\end{dfn}

\begin{dfn}
$T_\bullet\calc\lrta Comp$, we say $T_\bullet$ is\textbf{non-negative} if $T_n(C)=0$ forall $n<0$ and $\forall C$. $T_\bullet$ is \textbf{acyclic in the positive degrees on $C$} if $H_n(T_\bullet(C))=0,\forall n>0$. 
\end{dfn}
Suppose $T_\bullet C\lrta Comp$. $H_0\circ T_\bullet \calc\lrta Ab$.
\begin{thm}
Suppose $\calc$ is a categroy with omdels $\mathcal{M}$. Supose $S_\bullet, T_\bullet:\calc\lrta Comp$ are 2 functors such that $S$ and $T$ are non-negative and acyclic in positive degree on every model, and 
both $S$ and $T$ are free with basis in $\mathcal{M}$.

Suppose 
$$
\Theta: H_0\circ S_\bullet\lrta H_0\circ T_\bullet
$$
 is a natural equivalence. $\exists $ a  natural cahin equivalence $\Psi_\bullet:S_\bullet\lrta T_\bullet$ which isn unique up to chain homotopy and has $H_0(\Psi_\bullet)=\Theta$
\end{thm}
\begin{ex}
Take $\calc=Top$, $\mathcal{M}=\{\Delta^n|n\geq 0\}$.  $T$ is the singular chain functor.
$$
\calc_\bullet: Top\lrta Comp
$$
$$
X\mapsto C_\bullet(X)
$$
$C_\bullet$ is non-negative, \checkmark.
$H_n(C_\bullet(\Delta^i))=H_n(\Delta^i)=$.


\end{ex}
\underline{Claim}: $C_n$ is free with basis in $\Delta^n$

Choose an element $x\in C_n(\Delta^n)$. Take $x$ to be the identity map $\Delta^n\lrta \Delta^n$, write this as $\ell_n:\Delta^n\lrta \Delta^n$. Think of the identity map as an element of $C_n(\Delta^n)$ if $\sigma$ is any $n-$simplex in any topological space
$C_n(\sigma)(\ell_n)=\sigma\circ \ell_n=\sigma$

$\{C_n(\sigma)(\ell_n)|\sigma: \Delta^n\lrta X\}$ is basis for the free abelian group $C_n(X)$.

Eilenberg-Zilber
$Top\times Top$ is the category of pairs $(X,Y)$ of topological spaces.

We will define two functor from $Top\times Top\lrta Comp$
$S_\bullet(X,Y)=C_\bullet(X,Y)$. $T_\bullet (X,Y)=C_\bullet(X)\otimes C_\bullet(Y)$

For models
$$
\mathcal{M}=\{(\Delta^i,\Delta^j), i,j \geq 0\}
$$ 
\underline{Claim}: $S$ and $T$ are both acyclic in positive degree on $\mathcal{M}$ and free with basis in $\mathcal{M}$

$S_\bullet$, $H_n(S_\bullet(\Delta^i, \Delta^j))=H_n(\Delta^i\times \Delta^j)=0$, $\forall n>0, \forall i,j$

$S_i: Top\times Top\lrta Ab$

$S_i(X, Y)=C_i (X\times Y)$

\underline{Claim}: $\{(\Delta^i,\Delta^i)\}$ is a $S_i$-model set and a basis is $d_i:\Delta^i\otimes \Delta^i$ the diagonal map $x\mapsto (x,x)$ gives a basis
$$
\sigma: \Delta^i\lrta X\times Y
$$
we can write 
$\sigma=(\sigma_x,\sigma_y)\circ d_i$,  where $\sigma_x=p_X\circ \sigma$ be the composition of $\sigma$ with $p_X:X\times Y\lrta X$.

$\sigma=S_i(\sigma)(d_i)$ so that
$\{s_i(\sigma )(d_i\|\sigma: \Delta^i\lrta X\times Y\}$ is a basis of the free abelian group $C_i(X\times Y)$. $T_i(X\times Y)=(C_\bullet(X)\otimes C_\bullet(Y))$. $T_i(X, Y)$ is the tensor product of the free groups and so is free.
$\{(\ell_i,\ell_j)|i+j=n\}$ is a $T_n$-model basis.

The last thing to check is that $T_\bullet(\Delta^i, \Delta^j)$ is acyclic in positive degrees
$$
H_n(C_\bullet(\Delta^i)\otimes C_\bullet(\Delta^j))=0,\forall n>0.
$$
We can not compute this! However we can cheat
$$
H_n(C_\bullet(\Delta^i))=H_n(\Delta^i)=\left\{\begin{matrix*}
 \intg & n=0\\
 0 & n\neq 0
\end{matrix*}\right.
$$

Consider the chain complex
$$
0\lrta 0\lrta \cdots\lrta 0\lrta \intg\lrta 0\cdots
$$
$C_\bullet(\Delta^i)$ has the same homology as this complex. Thus $C_\bullet(\Delta^i)$ is equivalenct to the complex and $C_\bullet(\Delta^j)$ is also chain equivalent to it. $C_\bullet(\Delta^i)\otimes C_\bullet(\Delta^j)$ is chain equivalent to 
$$
0\lrta 0\lrta \cdots\lrta 0\lrta \intg\otimes \intg\lrta 0\cdots
$$
Thus $H_n(C_\bullet(\Delta^i)\otimes C_\bullet(\Delta^j))=H_n(0\lrta \intg\otimes \intg\lrta 0\cdots)$

\underline{Want}: 
$\Theta:H_0\circ S_\bullet\lrta H_0\circ T_\bullet$ is a natural equivalence.
$$
(x,y)\mapsto x\otimes y
$$
$$
H_0(C_\bullet(X\times Y))\lrta H_0(C_\bullet(X)\otimes C_\bullet (Y))
$$
By the Acylic model theorem
$$
\Omega_\bullet: S_\bullet\lrta T_\bullet
$$
 is a natural chain equivalence
 $$
\Omega_\bullet:C_\bullet(X\times Y)\lrta C_\bullet(X)\otimes C_\bullet(Y)
 $$

\begin{cor}
Kueneth formula.
Let $X$ and $Y$  be otpological spaces then for $n\geq0$

There is a split exact sequence
\end{cor}



\end{document}
